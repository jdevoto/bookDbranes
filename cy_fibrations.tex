
\vspace{250pt}

The first part of this chapter is devoted to the introduction of some
basic notions from category theory. Additive and pseudo-abelian
categories are needed in the next chapter to study maximal Cardy
fibrations; we bundled all the definitions in this chapter just for
convenience.

%%%%%%%%%%%%%%%%%%%%%%%%%%%%%%%
\section{Calabi-Yau Categories}
\label{sheaf_boundary_conditions}

Let $R$ be a commutative ring with unit. A category ${\bf X}$ is said
to be \emph{enriched over the category of $R$-modules} if for
arbitrary objects $a,b\in {\bf X}$, $\opnm{Hom}_{\bf X}(a,b)$ is an
$R$-module and the composition map is $R$-bilinear. In particular, if
$R=\ent$, we say that ${\bf X}$ is enriched over the category of
abelian groups.

Recall also that an object $a\in {\bf X}$ is said to be \emph{initial}
(respectively \emph{terminal}) if for each $b\in {\bf X}$,
there exists a unique arrow $a\to b$ (respectively $b\to a$).

\begin{defi}\label{linear_cats}
  Let ${\bf X}$ be a category and $R$ a commutative, unital ring. Then
  ${\bf X}$ is called
  \begin{enumerate}

  \item an \emph{$R$-linear category} if it is enriched
    over the category of $R$-modules;

  \item an \emph{additive category} if it is
    $\ent$-linear, has an initial object $0$ and for each pair of
    objects $a,b\in {\bf X}$ there exists a sum $a\oplus b\in {\bf
      X}$;

  \item a \emph{pseudo abelian category} if it is additive
    and given any object $a\in {\bf X}$, for each idempotent $\sigma
    :a\rightarrow a$ (i.e. $\sigma^2=\sigma$) there exists an object
    $\opnm{Ker} \sigma \in {\bf X}$, called the \emph{kernel of}
    $\sigma$, such that the canonical arrow
    \begin{equation}\label{canonical_arrow}
      \opnm{Ker} \sigma \oplus \opnm{Ker} (1_a -\sigma )\longrightarrow a
    \end{equation}
    is an isomorphism;

  \item a \emph{Calabi-Yau category (over $R$)} ({\sc cy} category for
    short) if it is $R$-linear, the objects $\operatorname{Hom}_{{\bf X}}(a,a)$ 
		are finitely generated, projective $R$-modules and, for each object $a\in {\bf X}$, 
		there exists a linear form
$$\theta_a:\operatorname{Hom}_{{\bf X}}(a,a)\longrightarrow R$$
such that the composite
\begin{equation}\label{pair}
  \operatorname{Hom}_{{\bf X}}(a,b)\otimes_R
  \operatorname{Hom}_{{\bf X}}(b,a)\longrightarrow 
  \operatorname{Hom}_{{\bf X}}(a,a)\stackrel{\theta_a}{\longrightarrow }R
\end{equation}
is a perfect pairing (the first arrow is the composition map $\sigma
\otimes \tau \mapsto \tau \sigma$) and, given arbitrary arrows $\sigma
:a\to b$ and $\tau :b\to a$, the equality
$$\theta_a(\tau \sigma )=\theta_b(\sigma \tau )$$
holds.
\end{enumerate}
\end{defi}

Let us add some more comments on the previous definitions. For
details, the reader is referred to 
\cite{maclane:_catwm, kn:grothendieck_sga2, freyd:abelian, costello07:_tcft_cy}.

{\sc Additive Categories.} Recall that in a category ${\bf X}$, a
\emph{zero object} $0\in {\bf X}$ is an object which is both initial
and terminal. The sum operation $\oplus$ is usually called a
\emph{biproduct} and, given objects $a_1,a_2\in {\bf X}$, there exist
projection $\pr_k:a_1\oplus a_2\to a_k$ and inclusion morphisms
$i_k:a_k\to a_1\oplus a_2$ ($k=1,2$) enjoying the following
properties:
\begin{itemize}
\item $a_1\oplus a_2$ (together with the projections $\pr_1$ and
  $\pr_2$) is a product.
\item $a_1\oplus a_2$ (together with the inclusions $i_1$ and $i_2$)
  is also a coproduct.
\item $\pr_ki_k=1_{a_k}$ ($k=1,2$).
\item $\pr_li_k=0$ for $l\neq k$, where $0$ is the zero object of the
  abelian group $\opnm{Hom}_{\bf X}(a_k,a_l)$.
\end{itemize}
Schematically, the biproduct structure for $a_1\oplus a_2$ is given by
the following diagrams:
\begin{equation}\label{biproduct}
  \xymatrix{ & b \ar[dr] \ar[dl] \ar[d] &  & & b & \\
    a_1 & a_1\oplus a_2 \ar[l]^-{\pr_1} \ar[r]_-{\pr_2} & a_2 & a_1 \ar[r]_-{i_1} \ar[ur] & a_1\oplus a_2 \ar[u] & a_2 ,\ar[l]^-{i_2} \ar[lu]}
\end{equation}
where the diagonal maps are given and the vertical arrows are uniquely
determined by the other morphisms (the diagram on the left corresponds
to the product structure and the one on the right to the
coproduct). In this setting, a morphism $\sigma :a_1\oplus a_2\to
b_1\oplus b_2$ can be represented as a matrix
$$\sigma =\left (\begin{smallmatrix} \sigma_{11} & \sigma_{21} \\ \sigma_{12} & \sigma_{22} \end{smallmatrix} \right ),$$
where $\sigma_{ij}:a_i\to b_j$.

Some examples: in the category of sets, there is no zero object (the
empty set is the initial object while any singleton is terminal); the
product is the cartesian product, and the coproduct is the disjoint
union; in the category of vector spaces over a field, the direct sum
is both a product and a coproduct; moreover, this category is
additive, the zero object being the trivial vector space. In the
category of groups, the zero object is the trivial group, but there is
no biproduct: the product is the direct product, and the coproduct is
the free product.

{\sc Pseudo-Abelian Categories.} Let $\sigma :a\to b$ be a morphism in
an additive category ${\bf X}$; the \emph{kernel} $K(\sigma )$
\emph{of} $\sigma$ is a pair $(\opnm{Ker}\sigma ,k)$, where
$\opnm{Ker}\sigma$ is an object of ${\bf X}$ and $k=k_{\sigma
}:\opnm{Ker}\sigma \to a$ an arrow such that $\sigma k=0\in
\opnm{Hom}_{\bf X}(\opnm{Ker}\sigma ,b)$. Moreover, $\opnm{Ker}\sigma$
is the ``biggest'' object with this property, in the sense that if
$k':K'\to a$ is another arrow such that $\sigma k'=0$, then there
exists a unique morphism $i:K'\to \opnm{Ker}\sigma$ such that
$k'=ki$. In a pseudo-abelian category ${\bf X}$, every idempotent
$\sigma :a\to a$ has a kernel; as $1_a-\sigma$ is also idempotent,
then $\opnm{Ker}(1_a-\sigma)$ is also defined; the canonical arrow
\eqref{canonical_arrow} is the unique map $\opnm{Ker}\sigma \oplus
\opnm{Ker}(1_a-\sigma )\to a$ which makes the diagram
$$
\xymatrix{
  & a & \\
  \opnm{Ker}\sigma \ar[ur]^{k_{\sigma}} \ar[r]_-{i_1} &
  \opnm{Ker}\sigma \oplus \opnm{Ker}(1_a-\sigma ) \ar[u] &
  \opnm{Ker}(1_a-\sigma ), \ar[ul]_{k_{1_a -\sigma }} \ar[l]^-{i_2} }
$$
commutative.

An important example of a pseudo-abelian category is the category
$\tsf{Vect}(M)$ of vector bundles over a manifold $M$; see section
\ref{bundles_operations}).

Another term used to describe this situation is to say that the
idempotent $\sigma$ \emph{splits}. In fact, the definition of
pseudo-abelian category given here restricts to additive categories,
but the notion of idempotent splitting can be given in an arbitrary
category. Moreover, given any category ${\bf X}$ in which idempotents
do not split, a new category $\widehat{\bf X}$, called the
\emph{idempotent completion}, \emph{Karoubi envelope} or \emph{Cauchy
  completion of} ${\bf X}$ can be constructed in a way such that
\begin{itemize}
\item the category ${\bf X}$ embeds naturally in $\widehat{\bf X}$ and
\item every idempotent in $\widehat{\bf X}$ splits.
\end{itemize}
We sketch the construction of the category $\widetilde{{\bf X}}$: Its
objects are pairs $(a,\sigma )$, where $\sigma :a\to a$ is an
idempotent map. A morphism $(a,\sigma )\to (b,\tau )$ is an arrow
$f:a\to b$ in ${\bf X}$ such that $f\sigma =f=\tau f$, and composition
is the same as the one in ${\bf X}$; the identity arrow of an object
$(a,\sigma )$ is $\sigma$. The embedding ${\bf X}\to \widehat{\bf X}$
is given by the assignment $a\mapsto (a,1_a)$. In the
additive-category setting, the objects $(a,\sigma )$ and
$(a,1_a-\sigma )$ should be interpreted as $\opnm{Ker}\sigma $ and
$\opnm{Ker}(1_a-\sigma )$ (they are in fact kernels in the additive
category $\widehat{{\bf X}}$), and the isomorphism $(a,1_a)\to
(a,\sigma )\oplus (a,1_a-\sigma )$ is given by the matrix $\left
  (\begin{smallmatrix} \sigma & 1_a-\sigma \\ \end{smallmatrix} \right
)$, with inverse $\left (\begin{smallmatrix} \sigma \\ 1_a-\sigma
    \\ \end{smallmatrix} \right )$. If $\sigma :a\to a$ is an
idempotent map ${\bf X}$, then we can view it in $\widehat{\bf X}$ as
an arrow
$$\sigma :(a,\sigma )\oplus (a,1_a-\sigma )\longrightarrow (a,\sigma )\oplus (a,1_a-\sigma ),$$
and hence as a matrix $\left (\begin{smallmatrix} \sigma_{11} &
    \sigma_{21} \\ \sigma_{12} & \sigma_{22} \end{smallmatrix} \right
)$. As the composite maps $\sigma (1_a-\sigma )$ and $(1_a-\sigma
)\sigma$ are both equal to 0, then $\sigma = \left
  (\begin{smallmatrix} 0 & 0 \\ 0 & 1_a \end{smallmatrix} \right )$.

For details, the reader is referred to
\cite{borceux:_cauchy_completion, natan:_envelope} and
references therein.

{\sc Calabi-Yau Categories.} The notion of Calabi-Yau category comes
from physics. In fact, in one of the aforementioned references,
K. Costello shows that $A_\infty$ Calabi-Yau categories classify
open-closed topological conformal field theories. In a {\sc cy} category,
for each object $a\in {\bf X}$, the existence of a trace $\theta_a$
implies that the hom-set $\operatorname{End}_{{\bf
    X}}(a)=\operatorname{Hom}_{{\bf X}}(a,a)$ is a Frobenius
$R$-algebra. Equivalently, as the pairing \eqref{pair} is
non-degenerate, we have that the $R$-module $\operatorname{Hom}_{{\bf
    X}}(b,a)$ is canonically isomorphic to the dual module
$\operatorname{Hom}_{{\bf X}}(a,b)^*$. A {\sc cy} category
\emph{in the sense of Moore and Segal} is a {\sc cy} category which
satisfies the conditions listed in section \ref{data}; in other words,
it is a {\sc cy} category which models an open-closed topological field
theory.

We can generalize these notions to fibred categories over a manifold
$M$.

\begin{defi}
  Let $\scr{R}$ be a sheaf of commutative rings with unit. A presheaf
  of categories $\scr{B}$ over $M$ is said to be
  \emph{$\scr{R}$-linear} iff for every open subset $U\subset M$, the
  category $\scr{B}(U)$ is $\scr{R}(U)$-linear and all the structures
  are compatible with pullbacks. Presheaves of additive,
  pseudo-abelian and of $\scr{R}$-linear Calabi-Yau categories are
  defined analogously.
\end{defi}

Note that if $\scr{B}$ is an $\scr{R}$-linear Calabi-Yau category over
$M$, then for each open subset $U\subset M$ and each object $a\in
\scr{B}(U)$, we have that $\operatorname{Hom}_{\scr{B}(U)}(a,a)$ is a
Frobenius $\scr{R}(U)$-algebra. As for $\scr{R}$-modules, this
statement can be generalized by saying that the presheaf
$\underline{\opnm{Hom}}_U(a,a)$ is a Frobenius
$\scr{R}|_U$-algebra.

\begin{obs}
  We use the term \emph{presheaf of categories} as a synonym for
  \emph{fibred category}.
\end{obs}


%%%%%%%%%%%%%%%%%%%%%%%%%%%%%%%%%%%%%%%%
\section{Calabi-Yau and Cardy Fibrations}

In \cite{moore_segal1}, Moore and Segal define a model for an
open-closed topological field theory of dimension 2. An account of
these results was given in section \ref{sec_octfts} of chapter
\ref{fsfts}. Theorem \ref{theorem_3} provides an algebraic
characterization of a maximal category of boundary conditions, which turns
out to be (non-canonically) equivalent to the category of finite-rank, complex vector bundles over the spectrum of the Frobenius algebra $A$.

Moore and Segal's construction can be regarded as a theory over a one-point space, say $\{x\}$. By replacing
\begin{itemize}
\item $\{x\}$ by an $F$-manifold $M$,\footnote{In fact, broadly speaking, we shall need only to consider manifolds $M$ such that $T_xM$ is a Frobenius algebra and such that for each $x$, a frame of idempotent, orthogonal sections exists.}
\item the closed algebra $C$ by the tangent bundle $TM$ (i.e. over each point $x\in M$, the closed algebra is the fibre $T_xM$) and
\item the spectrum of the algebra $A$ by the spectral cover of $M$
\end{itemize}
we shall obtain not just a category but a sheaf of categories which has relations which 2-vector bundles, as Segal conjectured. 


%%%%%%%%%%%%%%%%%%%%%%%%%%%%%%%%%%
\subsection{Calabi-Yau Fibrations}

From now on, we shall work with a ringed space $(M,\scr{O}_M)$ with the following properties:

\begin{itemize}
\item $TM$ is a bundle of algebras, i.e. $M$ is a manifold with multiplication and $\scr{O}_M$ is the usual structure sheaf (i.e. the sheaf of smooth functions in case $M$ is a smooth manifold; in particular, note that $\scr{O}_{M,x}$ is a local ring for each $x\in M$).
\item There exists a linear form $\theta :\Gamma(TM)\to \scr{O}_M$ making each fibre $T_xM$ a commutative Frobenius $\comp$-algebra.
\item $M$ is massive; i.e. each tangent space $T_xM$ is semisimple. In particular, for each $x\in M$ there exists a neighborhood $U\ni x$ and a frame of sections $\{e_1,\dots ,e_n\}$ defined over $U$ such that $e_ie_j=\delta_{ij}e_i$ and $\sum_i e_i=1$. In this case, we shall also say that $U$ is semisimple.
\end{itemize}

For simplicity, we shall refer to such a space as a \emph{semisimple manifold with multiplication} or just \emph{massive/semisimple manifold}. The reader should be aware that this name hides all the properties listed before.

Let $M$ denote a semisimple manifold with multiplication, with
structure sheaf $\scr{O}=\scr{O}_M$ and let $\scr{B}$ be an
$\scr{O}$-linear {\sc cy} category over $M$. For objects $a,b\in
\scr{B}(U)$, let us denote by $\Gamma_{ab}$ the presheaf
$\underline{\operatorname{Hom}}_U(a,b)$ over $U$ given by
\begin{equation}\label{sheaf_maps}
  V\longmapsto \operatorname{Hom}_{\scr{B}(V)}(a|_V,b|_V).
\end{equation}
By definition of {\sc cy} category, we have that $\Gamma_{aa}$ is a
Frobenius $\scr{O}_U$-algebra for each $a\in
\scr{B}(U)$. We shall denote the linear form corresponding to $\Gamma_{aa}$
by $\theta_a$.


\begin{notation}
  Recall that if the base manifold is clear, we shall supress the
  subscript of the structure sheaf when taking local sections;
  e.g. instead of using the notation $\scr{O}_M(U)$ for $U\subset M$,
  we will only write $\scr{O}(U)$; and the restriction $\scr{O}_M|_U$
  shall be denoted $\scr{O}_U$. The same considerations are applied to
  the tangent sheaf $\scr{T}_M$ of a manifold $M$.
\end{notation}

We now turn to the relevant definitions.

\begin{defi}\label{cy_fibration}
  A \emph{Calabi-Yau ({\sc cy}) fibration over a semisimple manifold $M$} is
  a pair $(\scr{B},\mathfrak{U})$ (the open cover shall be omitted
  form the notation), where $\scr{B}$ is a {\sc cy} category over $M$ and
  $\mathfrak{U}=\{U_\alpha \}$ is an open cover of $M$, subject to the
  following conditions:
  \begin{enumerate}
  \item Each $U_\alpha \in \mathfrak{U}$ is semisimple.
  \item $\scr{B}$ is a stack.\footnote{In particular, the presheaf
      \eqref{sheaf_maps} is a sheaf.}
  \item Given any $U_\alpha\in \mathfrak{U}$ and objects $a,b\in
    \scr{B}(U_\alpha)$, the sheaf $\Gamma_{ab}$ is a locally-free
    $\scr{O}_{U_\alpha}$-module of finite rank. Objects of $\scr{B}(U)$ are called \emph{labels},
    \emph{boundary conditions} or \emph{D-branes} over $U$.
	%\item For each $a\in \scr{B}(U)$, the locally free module $\Gamma_{aa}$ is a symmetric Frobenius 
	%algebra with linear form $\theta_a$.
  \item For each $U_\alpha \in \mathfrak{U}$ and each object $a\in
    \scr{B}(U_\alpha )$, we have transition (sheaf) homomorphisms
    \begin{displaymath}
      \iota_a:\scr{T}_{U_\alpha}\longrightarrow \Gamma_{aa} 
      \quad , \quad \iota^a:\Gamma_{aa}\longrightarrow \scr{T}_{U_\alpha}.     
    \end{displaymath}
The previous data is subject to the following conditions:

\begin{enumerate}
\item $\iota_a$ is a morphism of $\scr{O}_{U_\alpha}$-algebras
  (preserves multiplication and unit) and $\iota^a$ is an
  $\scr{O}_{U_\alpha}$-linear map.\footnote{In particular, $\iota_a$
    provides $\Gamma_{aa}$ with a $\scr{T}_{U_\alpha}$-algebra
    structure.}
\item $\iota_a$ is central: given $X\in \scr{T}(V)$ and $\sigma \in
  \Gamma_{ab}(V)$, we have
  \begin{equation}\label{centrality}
    \sigma \iota_a(X)=\iota_b(X)\sigma
  \end{equation}
  in $\Gamma_{ab}(V)$, for each $V\subset U_\alpha$.
\item There is an adjoint relation between $\iota_a$ and $\iota^a$
  given by
  \begin{equation}\label{adjoint}
    \theta (\iota^a(\sigma )X)=\theta_a(\sigma \iota_a(X)),
  \end{equation}
  for each $X\in \scr{T}_{U_\alpha}$ and $\sigma \in
  \Gamma_{aa}$.\footnote{Recall that, given a sheaf $\scr{S}$ over
    some space $M$, the notation $x\in \scr{S}$ means that $x\in
    \scr{S}(U)$ for some arbitrary open subset $U\subset M$.}
\end{enumerate}
\end{enumerate}
\end{defi}

\begin{obs}
  For some technical considerations (see definition \ref{maximal}), we
  will assume that our {\sc cy} fibrations $\scr{B}$ verify that for each
  open subset $U\subset M$, the skeleton $\opnm{sk}\scr{B}(U)$ of the
  category $\scr{B}(U)$ is a set.
\end{obs}


%%%%%%%%%%%%%%%%%%%%%%%%%%%%%
\subsection{Cardy Fibrations}

For $U_\alpha \in \mathfrak{U}$ open and $a,b\in \scr{B}(U_\alpha )$,
pick a local basis $\{\sigma_i\}$ of $\Gamma_{ab}$ and let
$\{\sigma^i\}$ be a basis of $\Gamma_{ab}^*$ dual to
$\{\sigma_i\}$. Define the map $\pi^a_b:\Gamma_{aa}\to \Gamma_{bb}$ by
$$\pi^a_b(\sigma )=\sum_i\sigma_i\sigma \sigma^i.$$
Some comments are in place: the sequence of maps
\begin{equation}\label{duality}
  \Gamma_{ba}\otimes \Gamma_{ab}\longrightarrow \Gamma_{bb}\stackrel{\theta_a}{\longrightarrow}\scr{O}_U
\end{equation}
induces a duality isomorphism
$\Gamma_{ba}\stackrel{\cong}{\longrightarrow}\Gamma_{ab}^*$. The dual
basis in the definition of $\pi_b^a$ is in fact the preimage of the
dual basis of $\{\sigma_i\}$ under this isomorphism. Another key
observation is stated in the following

\begin{proposition}\label{cardy_well_def}
  The map $\pi^a_b$ does not depend on the chosen (local) basis.
\end{proposition}
\begin{proof}
  As $\Gamma_{aa}$, $\Gamma_{bb}$ and $\Gamma_{ba}$ are locally-free,
  we can pick an open cover $\mathfrak{U}_\alpha$ of $U_\alpha$ such
  that $\Gamma_{aa}|_V\cong \scr{O}^{n_{a}}$, $\Gamma_{ba}|_V\cong
  \scr{O}^{n_{ba}}$, etc. for each $V\in \mathfrak{U}_\alpha$. Pick
  then a basis $B=\{e_1,\dots ,e_{n_{ba}}\}$ for
  $\Gamma_{ba}|_V$.\footnote{By a \emph{basis} we mean a system of
    linearly independent generators $e_1,\dots ,e_{n_{ba}}\in
    \Gamma_{ba}(V)$ such that $\{e_1|_W,\dots ,e_{n_{ba}}|_W\}$ is
    also linearly independent and generates $\Gamma_{ba}(W)$ for each
    $W\subset V$. For instance, let $u_1,\cdots ,u_{n_{ba}}\in
    \scr{O}(V)$ be units; then, if $e_i=(0,\dots ,0,1,0,\dots ,0)$,
    the sections $u_1e_1,\dots ,u_{n_{ab}}e_{n_{ba}}$ form a basis.}
  Let $B'=\{e^1,\dots ,e^{n_{ba}}\}$ be the corresponding dual basis
  for $\Gamma_{ba}^*$. Then, in terms of this basis we have $\pi_b^a
  (\sigma )=\sum_ie_i\sigma e^i$. Let $D=\{f_1,\dots ,f_{n_{ba}}\}$ be
  another basis over $V$ with dual basis $D'$. We then have
$$f_i=\sum_j\lambda_{ij}e_j \quad \text{and} \quad f^i=\sum_j\mu^{ij}e^j.$$
Replacing these linear combinations in the equality
$\delta_{ij}=f^i(f_j)$ we obtain
$$\delta_{ij}=\sum_k\mu^{ik}\lambda_{jk}.$$
If $A:=(\lambda_{ij})$ and $B:=(\mu^{ij})$ then the previous equality
implies that $AB^t=I$ or, equivalently, $A^tB=I$, which in terms of
the coefficients is expressed by
$\delta_{ij}=\sum_k\lambda_{ki}\mu^{kj}$. We now compute
$$
\begin{aligned}
  \sum_if_i\sigma f^i &= \sum_i\Bigl (\sum_j\lambda_{ij}e_j\Bigr )\sigma \Bigl (\sum_k\mu^{ik}e^k\Bigr ) \\
  &= \sum_{j,k}\Bigl (\sum_i\lambda_{ij}\mu^{ik}\Bigr )e_j\sigma e^k \\
  &=\sum_{j,k}\delta_{jk}e_j\sigma e^k \\
  &=\sum_je_j\sigma e^j, \\
\end{aligned}
$$
as desired.
\end{proof}

Then, when defining $\pi_b^a$ locally on each $V$, we have that, by
the previous computation, these expressions coincide over non-empty
overlaps, and thus can be glued together to obtain a morphism over
$U_\alpha \in \mathfrak{U}$
$$\pi_b^a :\Gamma_{aa}\longrightarrow \Gamma_{bb}.$$
This final layer of structure is included in the following

\begin{defi}\label{cardy_fib}
  A Calabi-Yau fibration $\scr{B}$ is called a \emph{Cardy fibration}
  if the following condition, called the \emph{Cardy
    condition}, holds for each open subset $U_\alpha \in
  \mathfrak{U}$: For $a,b\in \scr{B}(U_\alpha )$,
$$\pi^a_b=\iota_b\iota^a.$$
In other words, the following triangle
$$
\xymatrix{
  \Gamma_{aa} \ar[dr]_{\iota^a}\ar[rr]^{\pi^a_b} & & \Gamma_{bb} \\
  & \scr{T}_U \ar[ur]_{\iota_b} & }
$$
should commute.
\end{defi}

We shall deal with Cardy fibrations all along.

\begin{defi}
  A Cardy fibration $\scr{B}$ is said to be \emph{trivializable} if
  and only if conditions (3), (4)a-c in definition \ref{cy_fibration}
  and the Cardy condition hold also for any open subset of each
  $U_\alpha \in \mathfrak{U}$.
\end{defi}

A characterization of a certain kind of \emph{trivializable} Cardy
fibrations shall be given in the next chapter.

% Hasta aqui 2013-09-30
%%%%%%%%%%%%%%%%%%%%%%%%%%%%%%%%%%%%%%%
\subsection{Global Objects}

We shall now deduce some further structure enjoyed by globally defined
boundary conditions. These properties are needed in chapter \ref{dbtvb}.

We first note that for a proper open subset $U$ of $M$ ($U\neq
U_\alpha$ for each $\alpha$), and objects $a,b\in \scr{B}(U)$, the
sheaves $\Gamma_{ab}$ need not be locally free. But this situation is
slightly different when considering $U=M$.

Take global objects $a,b\in \scr{B}(M)$; hence, $a_\alpha
:=a|_{U_\alpha}, b_\alpha :=b|_{U_\alpha}\in \scr{B}(U_\alpha )$ and
$\Gamma_{a_\alpha b_\alpha }$ is a locally free $\scr{O}_{U_\alpha
}$-module, which in turn implies that $\Gamma_{ab}$ is a locally free
$\scr{O}$-module.

We also have transition homomorphisms
$$\iota_{a_\alpha}:\scr{T}_{U_\alpha }\longrightarrow \Gamma_{a_\alpha a_\alpha }\quad , \quad \iota^{a_\alpha}:\Gamma_{a_\alpha a_\alpha }\longrightarrow \scr{T}_{U_\alpha }.$$
Pick now an open subset $U_\beta \in \mathfrak{U}$ such that
$U_{\alpha \beta }\neq \emptyset$ and let $a_\beta
:=a|_{U_\beta}$. For $U_{\alpha \beta }$, as $\Gamma_{a_\alpha
  a_\alpha }(U_{\alpha \beta })=\Gamma_{a_\beta a_\beta }(U_{\alpha
  \beta })=\Gamma_{aa}(U_{\alpha \beta})$, we have maps
$$
\iota_{a_\alpha ,U_{\alpha \beta}},\iota_{a_\beta ,U_{\alpha \beta}} :
\scr{T}(U_{\alpha \beta })\longrightarrow \Gamma_{aa}(U_{\alpha \beta
}),
$$
which we also shall denote by $\iota_{a_\alpha }$ and $\iota_{a_\beta
}$ for notation's sake.

Let now $X\in \scr{T}(U_{\alpha \beta })$ and let $\sigma \in
\Gamma_{aa}$. The centrality condition \eqref{centrality} implies that
over $U_{\alpha \beta }$ the equality
$$\sigma |_{U_{\alpha \beta}} \iota_{a_\alpha} (X) = \iota_{a_\beta} (X)\sigma |_{U_{\alpha \beta }}$$
holds. Taking $\sigma =1_a$ we conclude that the morphisms
$\iota_{a_\alpha }|_{U_{\alpha \beta }}$ and $\iota_{a_\beta
}|_{U_{\alpha \beta }}$ are equal, and hence can be glued into a
global algebra homomorphism
$$\iota_a:\scr{T}_M\longrightarrow \Gamma_{aa}.$$

An analogous conclusion can be derived for the other transition map;
for this we use tha adjoint relation \eqref{adjoint}. First note that
the restrictions of the linear forms $\theta_{a_\alpha}$ and
$\theta_{a_\beta }$ to $U_{\alpha \beta}$ are the same, as they are
both equal to the restriction $\theta_a|_{U_{\alpha \beta}}$. Then,
using this fact together with the adjoint relation over $U_{\alpha
  \beta }$ we obtain
$$\theta (\iota^{a_\alpha }(\sigma )X)=\theta_{a_\alpha}(\sigma \iota_{a_\alpha}(X))=\theta_{a_\beta}(\sigma \iota_{a_\beta}(X))=\theta (\iota^{a_\beta }(\sigma )X)$$
for each vector field $X:U_{\alpha \beta }\to TM$ and each section
$\sigma \in \Gamma_{aa}(U_{\alpha \beta})$. Hence, the equality
$$\theta \left ((\iota^{a_\alpha }(\sigma )-\iota^{a_\beta}(\sigma ))X\right )=0$$
holds for each $X$ and $\sigma$. As $\theta$ is non degenerate, we can
then conclude that the morphisms $\iota^{a_\alpha }|_{U_{\alpha \beta
  }}$ and $\iota^{a_\beta }|_{U_{\alpha \beta }}$ are equal, thus
obtaining a global map
$$\iota^a:\Gamma_{aa}\longrightarrow \scr{T}_M.$$

A similar procedure shows that the map $\pi^a_b$ exists also for
global objects $a,b\in \scr{B}(M)$. Moreover, the verification of the
centrality condition, adjoint relation and Cardy condition for these
``new'' maps can be deduced with no difficulties from the local
versions.


\clearpage

{\small
%%%%%%%%%%%%%%%%%%%%%%%%%%%%%%%%%%%%%%%%%%%%
%%%%%%%%%%%%%%%%%%%%%%%%%%%%%%%%%%%%%%%%%%%%
\section{Resumen del Cap\'itulo \ref{cfib}}

En este cap\'itulo se definen los objetos que componen el n\'ucleo de este trabajo, los cuales, a grandes rasgos, son b\'asicamente familias de teor\'ias topol\'ogicas de campos, indexadas por una variedad con multiplicaci\'on particular.


%%%%%%%%%%%%%%%%%%%%%%%%%%%%%%%%%%%%%%%%%%%
\subsection{Categor\'ias de Calabi-Yau}

Para lo que sigue ser\'a necesario introducir ciertos tipos de categor\'ias. Daso un anillo conmutativo $R$ con unidad, diremos que una categor\'ia ${\bf X}$ es
\begin{itemize}
\item \emph{$R$-lineal} si est\'a enriquecida sobre la categor\'ia de $R$-modulos;
\item \emph{aditiva} si es $\ent$-lineal , tiene un objeto inicial $0$ y para cada par de objetos $a,b\in {\bf X}$ se tiene definida una suma $a\oplus b\in {\bf X}$;
\item \emph{pseudo-abeliana} si es aditiva y para cada objeto $a\in {\bf X}$ y cada idempotente $\sigma :a\to a$ existe un objeto $\opnm{Ker}\sigma \in {\bf X}$ (el \emph{n\'ucleo} de $\sigma$) tal que la aplicaci\'on can\'onica
$$ \opnm{Ker} \sigma \oplus \opnm{Ker} (1_a -\sigma )\longrightarrow a$$
es un isomorfismo;
\item una \emph{categor\'ia de Calabi-Yau} (abreviado {\sc cy}) si es $R$-lineal, los $R$-m\'odulos $\opnm{Hom}_{\bf X}(a,a)$ son finitamente generados y proyectivos y para cada objeto $a\in {\bf X}$ se tiene una forma lineal
$$\theta_a:\opnm{Hom}_{\bf X}(a,a)\longrightarrow R$$
tal que la composici\'on
$$ \operatorname{Hom}_{{\bf X}}(a,b)\otimes_R
  \operatorname{Hom}_{{\bf X}}(b,a)\longrightarrow 
  \operatorname{Hom}_{{\bf X}}(a,a)\stackrel{\theta_a}{\longrightarrow }R$$
es una forma bilineal no degenerada.
\end{itemize}

Las categor\'ias que nos interesan se construyen a partir de las anteriores, b\'asicamente considerando categor\'ias fibradas.

%%%%%%%%%%%%%%%%%%%%%%%%%%%%%%%%%%%%%%
\subsection{Fibraciones de Calabi-Yau}

En lo que sigue, $M$ ser\'a una variedad con multiplicaci\'on con las siguiente propiedades:

\begin{itemize}
\item Se tiene una forma lineal $\theta :\Gamma(TM)=:\scr{T}_M\to \scr{O}_M$ que hace a cada espacio tangente $T_xM$ una $\comp$-\'algebra de Frobenius, siendo $\scr{O}_M$ el haz estructural usual (por ejemplo, el haz de funciones suaves en caso que $M$ sea una variedad $C^\infty$; en particular, $\scr{O}_{M,x}$ es un anillo local para cada $x\in M$).
\item $M$ es masiva; es decir, $T_xM$ es semisimple para cada $x$.
\end{itemize}
\medskip

{\bf Definici\'on.} Una \emph{fibraci\'on de Calabi-Yau} ({\sc cy}) sobre una variedad semisimple $M$ es una par $(\scr{B},\mathfrak{U})$ formado por una categor\'ia de {\sc cy} $\scr{B}$ sobre $M$ y un cubrimiento abierto $\mathfrak{U}=\{U_\alpha \}$ sujetos a las siguientes condiciones:
\begin{enumerate}
\item Cada $U_\alpha$ es un abierto semisimple; es decir, existe sobre $U$ una base de secciones idempotentes ortogonales $\{e_1,\dots ,e_n\}$ tales que $\sum_ie_i=1$.
\item $\scr{B}$ es un stack.
\item Dado $U_\alpha \in \mathfrak{U}$ y $a,b \in \scr{B}(U_\alpha )$, el haz de morfismos $a\to b$, que notamos $\Gamma_{ab}$, es un $\scr{O}_{U_\alpha }$-m\'odulo localmente libre de rango finito. Los objetos de $\scr{B}(U)$ se llamar\'an \emph{condiciones de borde o $D$-branas sobre $U$}.
\item Para cada $U_\alpha \in \mathfrak{U}$ y cada $a\in \scr{B}(U_\alpha )$ se tienen morfismos de transici\'on $\iota_a:\scr{T}_{U_\alpha }\to \Gamma_{aa}$, $\iota^a:\Gamma_{aa}\to \scr{T}_{U_\alpha}$.
\end{enumerate}
Lo anterior sujeto a las siguientes condiciones:
\begin{enumerate}[(a)]
\item $\iota_a$ es un morfismo de \'algebras e $\iota^a$ es $\scr{O}_{U_\alpha }$-lineal.
\item $\iota_a$ es central: dado un campo local $X$ sobre $V\subset U_\alpha$ y $\sigma \in \Gamma_{ab}(V)$, se tiene $\sigma \iota_a(X)=\iota_b(X)\sigma$ en $\Gamma_{ab}(V)$.
\item Se tiene una relaci\'on de adjunci\'on entre $\iota_a$ e $\iota^a$ dada por $\theta (\iota^a(\sigma )X)=\theta_a(\sigma \iota_a(X))$ para cada campo $X$ y cada $\sigma :a\to a$.\footnote{Dado un haz $\scr{S}$, digamos de conjuntos para fijar ideas, la notaci\'on $x\in \scr{S}$ indica $x\in \scr{U}$ para un abierto arbitrario $U$.}
\end{enumerate}

%%%%%%%%%%%%%%%%%%%%%%%%%%%%%%%%%
\subsection{Fibraciones de Cardy}

Dado $U_\alpha \in \mathfrak{U}$ y $a,b \in \scr{B}(U_\alpha )$, sea $\{\sigma_i\}$ una base local arbitraria de $\Gamma_{ab}$ y sea $\{\sigma^i\}$ su dual. Se define un mapa $\pi^a_b:\Gamma_{aa}\to \Gamma_{bb}$ por la ecuaci\'on
$$\pi^a_b(\sigma )=\sum_i\sigma_i\sigma \sigma^i.$$
Un comentario sobre esta definici\'on: se tiene un isomorfismo $\Gamma_{ba}\to \Gamma_{ab}^*$ inducido por la forma bilineal $\Gamma_{ba}\otimes \Gamma_{ab}\to \Gamma_{bb}\to \scr{O}_U$; la base dual a la que nos referimos est\'a en realidad formada por las preimagenes de $\sigma^i$ bajo el isomorfismo anterior. Mas a\'un, una demostraci\'on elemental muestra que el mapa $\pi^a_b$ no depende de la base elegida.

Definimos a continuaci\'on los objetos que estudiaremos en detalle en lo que resta del trabajo.
\medskip

{\bf Definici\'on.} Una fibraci\'on de {\sc cy} se dice una \emph{fibraci\'on de Cardy} si la siguiente ecuaci\'on, llamada \emph{condici\'on de Cardy}, se verifica en cada $U_\alpha \in \mathfrak{U}$: $\pi^a_b=\iota_b\iota^a$.
\medskip

{\bf Observaci\'on.} Es importante hacer notar (y lo usaremos mas adelante), que los morfismos $\iota_a,\iota^a$ y $\pi^a_b$ existen tambi\'en sobre $M$; es decir, si $a,b\in \scr{B}(M)$, podemos entonces considerar las restricciones $a|_{U_\alpha}$ y $b|_{U_\alpha}$ y tambi\'en los morfismos $\iota_{a|_{U_\alpha}}$, $\iota_{b|_{U_\alpha}}$ y $\pi^{a|_{U_\alpha}}_{b|_{U_\alpha}}$. Dadas las propiedades que verifican los morfismos locales, podemos pegar estos mapas en mapas globales $\iota_a$, $\iota^a$, $\pi^a_b$.









}








%%% Local Variables: 
%%% mode: latex
%%% TeX-master: "master"
%%% End: 
