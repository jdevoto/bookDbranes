%\usepackage{amsfonts,amsmath,palatino,enumerate,mathpazo}
\documentclass[12pt, a4paper, reqno]{amsbook}
\usepackage{amssymb}
%Sans Serif Font
\renewcommand{\sfdefault}{pr4}
\usepackage{textcomp}


%%%%%%%%%%%%
% PAQUETES %
%%%%%%%%%%%%
\usepackage[all]{xy}
\usepackage{euscript} 
\usepackage{layout}
\usepackage{graphicx}
\usepackage{longtable}
\usepackage{enumerate}
\usepackage{mathrsfs}
\usepackage{bbm}
\usepackage{pstricks}
\usepackage{graphicx}
\usepackage{epsfig}
\usepackage[dvips]{geometry}
\usepackage{color}
\usepackage{amsfonts}
\usepackage{framed}
\usepackage{fancyhdr}
\usepackage[multiple]{footmisc}    % Para multiples footnotes (separados por comas).
																	 % Modo de uso:
																	 % \footnote{.....}%
																	 % \footnote{.....}%

\newcommand{\makenote}[1]{\marginpar{\raggedright\sffamily\scriptsize#1}}

\makeatletter
\renewcommand{\section}{\@startsection{section}{1}{\z@}{-30pt}{15pt}{\Large\bfseries\sffamily}}
\renewcommand{\subsection}{\@startsection{subsection}{2}{\z@}{-20pt}{10pt}{\large\bfseries\sffamily}}
\renewcommand{\subsubsection}{\@startsection{subsection}{2}{\z@}{-15pt}{10pt}{\bfseries\sffamily}}
\renewcommand{\paragraph}{\@startsection{paragraph}{4}{\z@}{15pt}{-15pt}{\bfseries\sffamily}}

% Teoremas Tipo (si se hace en castellano)

\newtheorem{proposition}{Proposition}[section]
\newtheorem{theorem}[proposition]{Theorem}
\newtheorem{cor}[proposition]{Corollary}
\newtheorem{lemma}[proposition]{Lemma}
\newtheorem{pyd}[proposition]{Proposition \& Definition}

\theoremstyle{definition}
%\theorembodyfont{\rmfamily}

\newtheorem{defi}[proposition]{Definition}
\newtheorem{ej}[proposition]{Example}
\newtheorem{obs}[proposition]{Remark}
\newtheorem{notation}[proposition]{Notation}


%%% Comandos

%% Numeros

\newcommand{\natu}{\mathbbm{N}} 
\newcommand{\ent}{\mathbbm{Z}}
\newcommand{\rat}{\mathbbm{Q}}
\newcommand{\re}{\mathbbm{R}} 
\newcommand{\comp}{\mathbbm{C}}
\newcommand{\field}{\mathbbm{F}}
\newcommand{\vect}{\text{Vect}}
\newcommand{\rproj}{\mathbbm{RP}}
\newcommand{\cproj}{\mathbbm{CP}}

\newcommand{\tsf}[1]{\mathsf{#1}}
\newcommand{\fp}[3]{ #1 \times_{#3} #2}
\newcommand{\bit}[1]{\textit{\textbf{#1}}}
\newcommand{\bitu}[2]{\bit{#1}^{\bit{#2}}}
\newcommand{\bitd}[2]{\bit{#1}_{\bit{#2}}}
\newcommand{\scr}{\mathscr}
\newcommand{\mbm}{\mathbbm}
\newcommand{\equi}{\text{Eq}(\tsf{Vect})} 
\newcommand{\tcal}[1]{\!
  \text{\textcalligra{#1}}\, }
\newcommand{\eus}[1]{\EuScript{#1}}
\newcommand{\opnm}{\operatorname}
\newcommand{\coh}[1]{\operatorname{H}^{#1}}
\newcommand{\rcoh}[1]{\widetilde{\operatorname{H}}^{#1}}
\newcommand{\ccoh}[1]{\check{\operatorname{H}}^{#1}}
\newcommand{\hgy}[1]{\operatorname{H}_{#1}}
\newcommand{\rhgy}[1]{\widetilde{\operatorname{H}}_{#1}}
\newcommand{\pr}{\operatorname{pr}}


\renewcommand\qedsymbol{Q.E.D.}
\renewcommand{\contentsname}{Contents \& Introduction}

%%%%%%%%%%%%%%%%%%%%%%%%%%%%%%%%%%%%%%%%%%
%Jorge Define
%%%%%%%%%%%%%%%%%%%%%%%%%%%%%%%%%%%%%%%%
\newcommand{\mr}[1]{\mathscr{#1}} 
\renewcommand{\top}{\textsf{Top}}
\newcommand{\Top}{\mr{T}op}
\newcommand{\obj}{\text{obj}}
\newcommand{\kk}{\mathscr{K}}

%%%%%%%%%%%%%%%%%%%%%%%%%%%%%%%%%%%%%%%%%%%%%%%%%%%%%

\hyphenation{vec-to-ria-les va-rie-da-des si-guien-tes pro-pie-da-des res-pues-tas}

%%%%%%%%%%%%%%%%%%%%%%%%%%%%%%%%%%%%%%%%%%%%%%%%%%%%%



\title{D-Branes and Two Vectos bundles}
\author{Anibal Amoreo \and Jorge A. Devoto}
\date{version: \today}
\makeindex
\begin{document}

\maketitle

\bigskip
\bigskip


\chapter*{Introduction}
\addcontentsline{toc}{chapter}{\string Introduction}
\input{intro.tex}


% Chapter 1
\chapter{Bundles and Sheaves}\label{bs}

\vspace{250pt}

In this section we shall deal first with vector bundles and two variants of them: their categorical analogues (or at least one kind of possible categorical analogue),which are usually called 2-vector bundles, and bundles with twisted cocycles. We will first introduce some basic terminology and facts about vector bundles of finite rank which are necessary for subsequent sections. We shall also give a brief account of sheaves, locally free modules and ringed spaces. 

Though we usually reference to smooth manifolds and (complex) vector bundles over them, constructions in this chapter, unless stated to the contrary, can also be applied to complex manifolds and (holomorphic) vector bundles.

%%%%%%%%%%%%%%%%%%%%%%%%
%%%%%%%%%%%%%%%%%%%%%%%%
\section{Vector Bundles}
\label{vector_bundles}

A \emph{vector bundle} over a smooth manifold $M$ consists of the
following data:
\begin{enumerate}
\item A manifold $E$, called the \emph{total space}, and a
  (surjective) map $\pi :E\to M$, called the \emph{projection};
\item a $\comp$-vector space structure on each fibre
  $E_x:=\pi^{-1}(\{x\})$;
\item an open cover $\mathfrak{U}=\{U_i\}_{i\in I}$ of $M$ and, for
  each $i\in I$, a fibre-preserving
  diffeomorphism
  $$h_i:E|_{U_i}:=\pi^{-1}(U_i)\stackrel{\cong}{\longrightarrow}U_i\times
  \comp^n$$ for each $U_i\in \mathfrak{U}$ such that
  \begin{enumerate}
  \item the restriction $h_{i,x}:E_x\to \comp^n$ of $h_i$ to the fibre
    $E_x$ is a $\comp$-linear isomorphism for each $x\in U_i$ and
  \item for each pair of indices $i,j\in I$ such that the intersection
    $U_{ij}:=U_i\cap U_j$ is non-empty, the map
    $g_{ij}:U_i\cap U_j\to \operatorname{GL}(n,\comp )$ defined by
$$h_ih_j^{-1}:(U_i\cap U_j)\times \comp^n\longrightarrow(U_i\cap U_j)\times \comp^n$$
$$(x,z)\longmapsto (x,g_{ij}(x)z)$$
is smooth.
\end{enumerate}
\end{enumerate}

If $M$ is connected, then the assignment $x\mapsto \dim E_x$ is
constant and is called the \emph{rank} of the vector bundle. Vector
bundles of rank equal to one are called \emph{line bundles}.

In the previous definition, the isomorphism $h_i$ is called a
\emph{local trivialization}, and the open cover $\mathfrak{U}$, a
\emph{trivializing cover}; the reference to the word ``trivial'' in
this context refers to product bundles $M\times \comp^n$ (see
definition \ref{trivial_bundle} below). In general, vector bundles are
only locally equivalent to such products.

Despite all the spaces and maps involved in this definition, we will
usually denote a vector bundle just by specifying its total space.

\begin{ej}
  Some important examples of vector bundles closely associated to a
  manifold $M$ include the (real) tangent $TM$ and cotangent bundles
  $T^*M$; their fibres over a point $x\in M$ are given by the tangent
  space $T_xM$ and the dual (cotangent) space $(T_xM)^*=:T^*_xM$
  respectively.
\end{ej}

The proof of the next assertion follows immediately from the
definition of the maps $g_{ij}$.

\begin{pyd}\label{cocycle}
  The family of maps $\{g_{ij}\}$ satisfy the so-called \emph{cocycle
    conditions}:
  \begin{enumerate}
  \item $g_{ii}=1$,
  \item $g_{ji}=g_{ij}^{-1}$ and
  \item $g_{ij}g_{jk}=g_{ik}$ on triple overlaps
    $U_{ijk}=U_i\cap U_j\cap U_k$.
  \end{enumerate}
  In general, any family of maps
  $\{g_{ij}:U_{ij}\to \opnm{GL}_n(\comp )\}$ satisfying the previous
  three conditions is called a \emph{cocycle}.
\end{pyd}

Before proving some important properties of cocycles, let us discuss
about bundle morphisms.

Let $f:N\to M$ be a map and let $F$ and $E$ be vector bundles over $N$
and $M$ respectively. A \emph{homomorphism} over $f$ is a
fibre-preserving map $\phi:F\to E$ which is $\comp$-linear over each
point; that is, the following square
$$\xymatrix{
  F\ar[r]^\phi \ar[d] & E \ar[d] \\
  N \ar[r]^f & M,}
$$
where the vertical arrows are the corresponding projections, is
commutative, and the restriction
$$\phi_x:E_x\longrightarrow F_{f(x)}$$
is a linear map between the vector spaces $E_x$ and $F_{f(x)}$. A
particular and important case is when $N=M$ and $f$ is the identity
map. We can define the category $\tsf{Vect}(M)$ of vector bundles over
$M$; objects are vector bundles of finite rank and arrows are given by
homomorphisms over the identity map $M\to M$. Given bundles $E$ and
$F$ over a space $M$, the set of bundle morphisms in the category
$\tsf{Vect}(M)$ will be denoted by $\opnm{Hom}_M(E,F)$.

\begin{defi}\label{trivial_bundle}
  The product bundle $M\times \comp^n$ is called the \emph{trivial
    vector bundle of rank $n$ over $M$}. If $E$ is a vector bundle
  over $M$ for which there exists a bundle isomorphism
  $E\to M\times \comp^n$, then $E$ is called \emph{trivializable}.
\end{defi}

By definition, every vector bundle is then locally isomorphic to a
trivial bundle, i.e every vector bundle is locally trivial.

\begin{notation}
  In ocassions when confussion is unlikely to occur, we will denote
  the trivial vector bundle $M\times \comp^n$ just by $\comp^n$.
\end{notation}

The following theorem shows that cocycles comprises all data to
completely describe a vector bundle.

\begin{theorem}\label{construct_bundles}
  Let $\mathfrak{U}=\{U_i\}$ be an open cover of $M$ and let
  $\{g_{ij}:U_{ij}\to \operatorname{GL}_n(\comp )\}$ be a
  cocycle. Then, there exists a unique, up to isomorphism, vector
  bundle $E$ with cocycle $\{g_{ij}\}$ and local trivializations
  $E|_{U_i}\stackrel{\cong}{\longrightarrow}U_i\times \comp^n$.
\end{theorem}

In other words, an open cover together with a cocycle let us define a
vector bundle in an essentially unique way.

\begin{proof}
  Define
$$E=\bigsqcup_iU_i\times \comp^n / \sim ,$$
with the quotient topology, where the equivalence relation is defined
in the following way: $(i,(x,z))\sim (j,(y,w))$ if and only if
$x=y\in U_{ij}$ and $w=g_{ij}(x)^{-1}(z)$. Denoting by $[i,x,z]$ the
equivalence class of the pair $(i,(x,z))$, the fibre over $x\in M$ is
the set $\{[i,x,z]\; | \; z\in \comp^n\}$, the vector space structure
is given by the relation
$$\lambda [i,x,z]+[i,x,w]=[i,x,\lambda z+w]$$
and the projection $E\to M$ is $[i,x,z]\mapsto x$. Local
trivializations
$U_i\times \comp^n\stackrel{\cong}{\longrightarrow}E|_{U_i}$ are given
by $(x,z)\mapsto [i,x,z]$.
\end{proof}

The following result shows the relationship between cocycles of
isomorphic bundles.

\begin{proposition}\label{cocycles_iso}
  Let $E$ and $F$ be vector bundles of rank $n$ over $M$ with cocycles
  $\{g_{ij}\}$ and $\{f_{ij}\}$ respectively (we are assuming that the
  same open cover $\{U_i\}$ trivializes $E$ as well as $F$). Then $E$
  and $F$ are isomorphic (that is, there exists a map $E\to F$ with an
  inverse $F\to E$, both preserving fibres) if and only if there
  exists a family of maps $\{g_i:U_i\to \operatorname{GL}_n(\comp )\}$
  such that
$$f_{ij}=g_ig_{ij}g_j^{-1}$$
over each non-empty overlap $U_{ij}$.
\end{proposition}
\begin{proof}
  First note that we can assume that the same open cover trivializes
  both $E$ and $F$: if $E$ is trivial over each $U\in \mathfrak{U}$
  and $F$ over each $V\in \mathfrak{V}$, then $E$ and $F$ are trivial
  over the elements of the cover
  $\mathfrak{U}\cap \mathfrak{V}:=\{U\cap V\}$.

  Assume now that we have local trivializations
$$E|_{U_i}\stackrel{h^E_i}{\longrightarrow}U_i\times \comp^n \stackrel{h^F_i}{\longleftarrow} F|_{U_i}$$
and let $\phi :E\to F$ be an isomorphism. For each index $i$, the
following (commutative) diagram of bundles over $U_i$
$$
\xymatrix{
  E|_{U_i}\ar[rr]^\phi \ar[dr]_{h_i^E} & & F|_{U_i} \ar[dl]^{h_i^F} \\
  & U_i\times \comp^n & }
$$
lets us define maps $g_i:U_i\to \operatorname{GL}_n(\comp )$,
$$g_i(x)(z):=\opnm{pr}_2\left (h_i^F\phi (h_i^E)^{-1}(x,z)\right ),$$
satisfying $f_{ij}=g_ig_{ij}g_j^{-1}$.

Conversely, given the family $\{g_i\}$, we can define a bundle
isomorphism $\phi :E\to F$ by patching the maps
$$
\xymatrix{ E|_{U_i}\ar[r]^-{h_i^E} & U_i\times \comp^n\ar[r]^{1\times
    g_i} & U_i\times \comp^n \ar[r]^-{(h_i^F)^{-1}}& F|_{U_i}}.$$
\end{proof}


%%%%%%%%%%%%%%%%%%%%%%%
\subsection{Operations}\label{bundles_operations}

The usual operations between vector spaces, like for example tensor
product and direct sum (among many others) can also be defined for
vector bundles. We will now describe some of these operations. For a
general construction and further details, the interested reader may
consult \cite{atiyah:_k}.

Let $\pi :E\to M$ and $\tau :F\to M$ be two vector bundles over $M$ of
rank $n$ and $k$ respectively, $\mathfrak{U}=\{U_i\}$ a trivializing
cover for both bundles and let $\{g_{ij}\}$ and $\{f_{ij}\}$ be
cocycles for $E$ and $F$ respectively.

\begin{enumerate}
\item \emph{Pullback}. Given a mapping $f:N\to M$, we can define the
  pullback bundle $f^*E$ over $N$ by
$$f^*E=\{(y,u)\in N\times E \; | \; f(y)=\pi (u)\},$$
with the projection $(y,u)\mapsto y$. The fibre over $y\in N$ is then
given by $E_{f(y)}$. Moreover, $f^*E$ has the same rank as $E$ and the
cover $\{f^{-1}(U_i)\}$ trivializes $f^*E$. Cocycles for $f^*E$ are
given by the maps
$f^*g_{ij}:f^{-1}(U_i)\cap f^{-1}(U_j)\to \operatorname{GL}_n(\comp
)$,
$$f^*g_{ij}(y)=g_{ij}(f(y)).$$
When $U\subset M$, the pullback along the inclusion $U\to M$ is
denoted $E|_U$ and called the \emph{restriction} of $E$ to $U$.

\item \emph{External Direct Sum}. Let $N$ be another manifold and
  consider a vector bundle $\rho :D\to N$ of rank $r$. We define a
  vector bundle $\pi \times \rho : E\boxplus D\to M\times N$ over
  $M\times N$, called the external direct sum, in the following way:
  over a point $(x,y)\in M\times N$, the fibre $(E\boxplus D)_{(x,y)}$
  is given by the external direct sum $E_x\oplus F_y$. If
  $\mathfrak{V}=\{V_s\}$ is a trivializing cover for $D$ and
  $h_i:E|_{U_i}\to U_i\times \comp^n$ and
  $h'_s:D|_{V_s}\to V_s\times \comp^r$ are local trivializations for
  $E$ and $D$ respectively, then the map
  $\overline{h}_{is}:(E\boxplus D)|_{U_i\times V_s}\to (U_i\times
  V_s)\times (\comp^n\oplus \comp^r)$ defined by the composite
$$\xymatrix{
  (E\boxplus D)|_{U_i\times V_s}\ar[rr]^-{h_i\times h'_s} &&
  (U_i\times \comp^n)\times (V_s\times \comp^r)\ar[r]^-{\cong} &
  (U_i\times V_s)\times (\comp^n\oplus \comp^r)}$$ is a local
trivialization for $E\boxplus D$ (and
$\mathfrak{U}\times \mathfrak{V}:=\{U_i\times V_s\}$ is a trivializing
cover). If $\{k_{sl}:V_{sl}\to \opnm{GL}_r(\comp )\}$ is a cocycle for
$D$, then the maps
$$g_{ij}\times k_{sl}:U_{ij}\times V_{sl}\longrightarrow \opnm{GL}_{n+r}(\comp )$$
given by $(g_{ij}\times k_{sl})(x,y)=g_{ij}(x)\times k_{sl}(y)$ define
a cocycle for $E\boxplus D$.

\item \emph{Whitney (Direct) Sum}. Let $\Delta :M\to M\times M$ be the
  diagonal map. The pullback bundle $\Delta^*(E\boxplus F)$ is called
  the Whitney or direct sum and is denoted by $E\oplus F$. The fibre
  over a point $x\in M$ is given by the direct sum $E_x\oplus F_x$,
  and the family of maps
  $\{h_{ij}:U_{ij}\to \opnm{GL}_{n+k}(\comp )\}$ given by
$$h_{ij}=
\begin{pmatrix}
  g_{ij} & 0 \\
  0 & f_{ij} \\
\end{pmatrix}
$$
is a cocycle for $E\oplus F$.

\item \emph{Dual Bundle}. We now consider the bundles
  $U_i\times (\comp^n)^*$ and the cocycle given by the maps
  $g_{ij}^*:U_{ij}\to \operatorname{GL}((\comp^n)^*)$,
$$g^*_{ij}(x)(A)=Ag_{ij}(x)^t.$$
In this way we obtain a bundle $E^*$ such that $(E^*)_x\cong E_x^*$.

\item \emph{Tensor Product}. To define the tensor product
  $E\otimes F$, we consider
  $U_i\times (\comp^n\otimes \comp^k)\cong U_i\times \comp^{nk}$ and
  cocycle given by
$$h_{ij}=g_{ij}\otimes f_{ij}.$$

For a real vector bundle $E$ over $M$, the tensor product
$E\otimes (M\times \comp )$ is called the \emph{complexification} of
the bundle $E$ and is usually denoted $E_\comp$.

\begin{obs}
  From now on, any bundle associated to a real, smooth manifold $M$
  (e.g. its tangent, cotangent bundles) shall be considered
  complexified; the subscript ``$\comp$'' will be supressed from the
  notations.
\end{obs}

\item \emph{Homomorphisms}. To define this bundle we consider the
  trivial bundles
$$U_i\times \operatorname{Hom}(\comp^n,\comp^k)\cong U_i\times \operatorname{M}_{k\times n}(\comp )$$
with cocycle given by the maps
$h_{ij}:U_{ij}\to \operatorname{GL}(\operatorname{M}_{k\times n}(\comp
))$,
$$h_{ij}(x)(A)=f_{ij}(x)Ag_{ij}(x).$$
We thus obtain a bundle which fibre over $x$ is isomorphic to the
vector space $\operatorname{Hom}_\comp (E_x,F_x)$, and we denote it by
$\operatorname{Hom}(E,F)$. If $F=E$ and $h:E|_U\to U\times \comp^n$ is
a local trivialization for $E$, then $h$ induces a local
trivialization
$\overline{h}:\opnm{End}(E)|_U:=\opnm{Hom}(E,E)|_U\to U\times
\opnm{M}_n(\comp)$ in the following way: if $\phi_x:E_x\to E_x$
belongs to the fibre $\opnm{End}(E)_x=\opnm{End}(E_x)$, then
$$\overline{h}(\phi_x)=(x,\overline{\phi}_x),$$
where $\overline{\phi}_x:\comp^n\to \comp^n$ is
$\overline{\phi}_x(z)=h_x\phi_xh_x^{-1}(x,z)$ and
$h_x=h|_{E_x}:E_x\to \{x\}\times \comp^n$. In particular, as this
trivialization is multiplicative, this shows that $\opnm{End}(E)$ is
in fact a bundle of matrix algebras.

As in linear algebra, we have the following relation between the
bundles $\operatorname{Hom}(E,F)$, $E\otimes F$ and $E^*$.

\begin{proposition}
  There exists a canonical bundle isomorphism
$$E^*\otimes F\stackrel{\cong}{\longrightarrow}\operatorname{Hom}(E,F).$$
\end{proposition}

In particular, if $F=\comp$ is the trivial line bundle, then the
previous result provides an isomorphism
$E^*\cong \operatorname{Hom}(E,\comp )$.

\begin{proof}
  The map $E^*\otimes F \to \operatorname{Hom}(E,F)$ given by the
  assignment
$$\phi \otimes v \longmapsto (\phi_e:u\mapsto \phi (u)v)$$
is a linear isomorphism.
\end{proof}

\item \emph{Kernels and Images}. Let $\phi :E\to F$ be a homomophism
  of bundles over $M$. Let $\opnm{Ker}\phi$ be the space over $M$
  given by
$$\opnm{Ker}\phi =\bigsqcup_{x\in M}\opnm{Ker}\phi_x,$$
with the obvious projection $\pi =\pr_1:(x,e)\mapsto x$. Then, in
general, $\pr_1:\opnm{Ker}\phi \to M$ fails to be locally trivial, as
the function $x\mapsto \dim \opnm{Ker}\phi_x$ may not be locally
constant (for example, fix a proper subspace $S\subset \comp^n$ and
consider the trivial vector bundle $E:=[0,1]\times \comp^n$ over the
unit interval. Let $\phi :E\to E$ be the map given by
$\phi (t,z)=(t,(1-t)p_S(z)+tz)$, where $p_S$ is the orthogonal
projection of $\comp^n$ onto $S$. Then, if $t>0$, we have that
$\opnm{Ker}\phi_t$ is trivial; but for $t=0$ we have that $\phi_0=p_S$
and thus $\dim \opnm{Ker}\phi_0 >0$).  A bundle morphism
$\phi :E\to F$ is called \emph{strict} if and only the map
$x\mapsto \dim \opnm{Ker}\phi_x$ (or, equivalently, the map
$x\mapsto \dim \opnm{Im}\phi_x$) is locally constant. In that case,
$\pi :\opnm{Ker}\phi \to M$ and
$$\opnm{Im}\phi :=\bigsqcup_{x\in M}\opnm{Im}\phi_x\longrightarrow M$$
are vector bundles \cite{atiyah:_k}.

An important particular case of strict homomorphisms is given by the
idempotent maps; if $\sigma :E\to E$ is a bundle homomorphism such
that $\sigma^2=\sigma$, then $\sigma$ is strict and there exists a
decomposition of $E$ as a sum
$$E=\opnm{Ker}\sigma \oplus \opnm{Ker}(1_E-\sigma),$$
where $1_E$ is the identity map of $E$.
\end{enumerate}



%%%%%%%%%%%%%%%%%%%%%%%
\section{Sheaves}

Sheaves over a manifold $M$ lets us discover many properties  of $M$ by studying objects defined locally on $M$; i.e. over open subsets $U\subset M$. A typical example of this procedure is found in elementary complex analysis: if one wish to study some compact complex manifold $M$ by dealing with maps $M\to \comp$, then one finds out (by Liouiville's theorem) that the only maps available are the constant ones, and thus the only way to obtain a descent supply of maps is to work over open subsets of $M$. 

We will introduce the notion of presheaf and sheaf and recall some useful results about them. In the next section these concepts will be applied when we define sections of vector bundles. For the missing proofs and further details on these topics the reader may consult \cite{tennison:_sheaf}, \cite{kn:kscha}, \cite{kn:warner}, \cite{kn:gortz_wed}.

For a topological space $M$, the category $\tsf{Op}(M)$ is defined in the following way: its objects are open subsets $U\subset M$ and morphisms $V\to U$ are inclusions.

\begin{defi}
A \emph{presheaf of sets} is a functor $\scr{P}:\tsf{Op}(M)^{\circ}\to \tsf{Sets}$.
\end{defi}

In other words, a presheaf of sets, or $\tsf{Sets}$-valued presheaf, assigns to each open subset $U$ of $M$ a set $\scr{P}(U)$ and to each inclusion $i:V\subset U$ a map $i^*:\scr{P}(U)\to \scr{P}(V)$, usually called \emph{restriction}. This terminology is better understood by considering the following

\begin{ej}
Given sets $A,B$, let us denote by $B^A$ the set of maps $A\to B$. Let $M$ be a topological space and $X$ an arbitrary set. For objects $U\in \tsf{Op}(M)$ define
$$\scr{P}(U):=X^U.$$
If $i:V\subset U$ is an inclusion, let $i^*:\scr{P}(U)\to \scr{P}(V)$ be the restriction map
$$i^*(f)=f|_V.$$
Then $\scr{P}$ is a presheaf of sets over $M$.
\end{ej}

Considering only sets as values for presheaves is restrictive and, as we shall see, many situations involve categories with more structure, for example the category of topological spaces, the category of groups, the category of modules over a ring, to name a few. In fact, the previous definition of presheaf can be rewritten \emph{mutatis mutandis} for an arbitrary category ${\bf X}$ instead of the category of sets $\tsf{Sets}$.

\begin{defi}
If ${\bf X}$ is a category, an \emph{${\bf X}$-valued presheaf} over $M$ is a functor $\scr{P}:\tsf{Op}(M)^{\circ}\to {\bf X}$.
\end{defi}

Let $V\to U$ be a map in the category $\tsf{Op}(M)$ (i.e. an inclusion $V\subset U$). Applying $\scr{P}$ we obtain a map $\scr{P}(U)\to \scr{P}(V)$ in ${\bf X}$ which is called the \emph{restriction} map. Given $\sigma \in \scr{P}(U)$, its image by this restriction map is denoted by $\sigma |_V$. Objects of $\scr{P}(U)$ are usually called \emph{sections} over $U$. If we denote the inclusion map $V\subset U$ by $i$, then $\scr{P}(i)$ will be briefly denoted by $i^*$.

\begin{notation}
To simplify notation when the open subset is clear from the context, the restriction $\sigma|_V$ will also be denoted by $\sigma$.
\end{notation}

\begin{defi}\label{def_sheaf}
A presheaf $\scr{S}$ over $M$ is called a \emph{sheaf} if the following conditions hold:
\begin{enumerate}
\item Assume $U\subset M$ is open and $\{U_i\}$ is an open cover of $U$. Suppose that $\sigma ,\tau \in \scr{S}(U)$ are sections such that $\sigma |_{U_i}=\tau |_{U_i}$ for each $i$. Then, $\sigma =\tau$.
\item Let $U$ and $\{U_i\}$ be as in the previous item and $\sigma_i \in \scr{S}(U_i)$ for each $i$. If $\sigma_i|_{U_{ij}}=\sigma_j|_{U_{ij}}$, then there exists a section $\sigma \in \scr{S}(U)$ such that $\sigma |_{U_i}=\sigma_i$.
\end{enumerate}
\end{defi}

Note that the first item in the previous definition implies that the section of the second one is unique.

\begin{notation}
Given a sheaf $\scr{S}$ over some space $M$, we will use the notation $\sigma \in \scr{S}$ to denote a section over an arbitrary (not specified) open subset of $M$.
\end{notation}

A \emph{morphism} between ${\bf X}$-valued (pre)sheaves $\scr{S},\scr{T}$ (both over the same base $M$) is a natural transformation $\eta :\scr{S}\to \scr{T}$; that is, $\eta$ is a family of maps in ${\bf X}$
$$\eta_U:\scr{S}(U)\longrightarrow \scr{T}(U) \quad (U\in \tsf{Op}(M))$$
in the category ${\bf X}$ such that the square
\begin{equation}\label{nat_transf_sheaf}
\xymatrix{
\scr{S}(U) \ar[r]^{\eta_U} \ar[d]_{i^*} & \scr{T}(U) \ar[d]^{i^*} \\
\scr{S}(V) \ar[r]_{\eta_V} & \scr{T}(V) }
\end{equation}
commutes for any $V$ and $U$ with inclusion map $i:V\subset U$.

A morphism of ${\bf X}$-valued presheaves $\eta :\scr{P}\to \scr{Q}$ over $M$ is said to be an \emph{isomorphism} if there exists another morphism $\eta^{-1}:\scr{Q}\to \scr{P}$ such that the composite maps $\eta \eta^{-1}$ and $\eta^{-1}\eta$ are equal to the respective identities. This is equivalent to saying that $\eta_U:\scr{P}(U)\to \scr{Q}(U)$ is an isomorphism in ${\bf X}$ for each open subset $U\subset M$.

\begin{obs}
Note that we only defined the notion of isomorphism for presheaves. This is because a \emph{sheaf} homomorphism $\eta :\scr{S}\to \scr{T}$ may be surjective even if $\eta$ is not surjective as a \emph{presheaf} homomorphism (that is, for $\eta$ to be an \emph{sheaf} isomorphism not all the maps $\eta_U:\scr{S}(U)\to \scr{T}(U)$ need to be surjective). The reason behind this fact is that the image of a sheaf homomorphism need not be a sheaf. See examples \ref{image_1}, \ref{image_2} and definition \ref{def_ker_image}.
\end{obs}

\begin{lemma}
If $\scr{S}$ is a sheaf, $\scr{T}$ a presheaf and $\eta :\scr{S}\to \scr{T}$ is an isomorphism of presheaves, then $\scr{T}$ is also a sheaf
\end{lemma}
\begin{proof}
The result is obtained by pulling back to $\scr{S}$; let $\{U_i\}$ be an open cover of some subset $U\subset M$ and let $\sigma ,\tau \in \scr{T}(U)$. Consider now the restrictions $\sigma |_{U_i}$ and $\tau |_{U_i}$ and suppose that $\sigma |_{U_i}=\tau |_{U_i}$. We can now take these sections back to $\scr{S}(U_i)$ via $\eta^{-1}_{U_i}$, obtaining $\eta^{-1}_U(\sigma )=\eta^{-1}_U(\tau )$ and hence $\sigma =\eta$. The pasting condition is proved analogously.
\end{proof}

Having defined morphisms, we now have the categories $\tsf{PSh}_{\bf X}(M)$ and $\tsf{Sh}_{\bf X}(M)$ of ${\bf X}$-valued presheaves and sheaves over $M$, respectively. Given (pre)sheaves $\scr{S}$ and $\scr{T}$ over $M$, $\opnm{Hom}_M(\scr{S},\scr{T})$ will denote the set of (pre)sheaf homomorphisms $\scr{S}\to \scr{T}$.

\begin{obs}
From now on, the category ${\bf X}$ will be taken to be the category of sets, groups, modules or algebras. We will also supress the subscript ${\bf X}$ in the notation of the categories of sheaves and presheaves, as it is always sufficiently clear from the context.
\end{obs} 

\begin{defi}
Let $\scr{S}$ be a (pre)sheaf over a space $M$.
\begin{itemize}
\item A \emph{sub(pre)sheaf} of $\scr{S}$ is a (pre)sheaf $\scr{T}$ such that for each $U\in \tsf{Op}(M)$, $\scr{T}(U)$ is a subset of $\scr{S}(U)$ (or a subgroup, subring, submodule, etc) and the restriction maps are induced from the ones in $\scr{S}$. 
\item If $U\subset M$ is an open subset, the \emph{restriction} $\scr{S}|_U$ of $\scr{S}$ to $U$ is the (pre)sheaf obtained by evaluating $\scr{S}$ in open subsets of $U$.
\item Given (pre)sheaves $\scr{F}$ and $\scr{G}$ over $M$, we can now define the presheaf $\underline{\opnm{Hom}}(\scr{F},\scr{G})$ in the following way: for an open subset $U\subset M$,
$$\underline{\opnm{Hom}}(\scr{F},\scr{G})(U):=\opnm{Hom}_U(\scr{F}|_U,\scr{G}|_U).$$
The arrow corresponding to the inclusion $V\subset U$ is also the restriction. This construction is well-behaved in the category of sheaves, in the sense that $\underline{\opnm{Hom}}(\scr{F},\scr{G})$ is a sheaf if $\scr{F}$ and $\scr{G}$ are.
\end{itemize}
\end{defi}

To introduce the following concepts, assume that $\eta :\scr{S}\to \scr{T}$ is a morphism of sheaves of groups over a space $M$. The \emph{kernel} of $\eta$ is the presheaf defined by the assignment
$$U\longmapsto \opnm{Ker}\eta_U.$$
Likewise, we define the presheaf $I_{\eta}$ by
$$U\longmapsto \opnm{Im}\eta_x.$$
By following the definition of sheaf it can be proved directly that the kernel of a morphism of sheaves is in fact a sheaf; in particular, note that, as $\eta$ is a natural transformation, kernels are preserved by restrictions; that is, if $\sigma \in \opnm{Ker}\eta_U\subset \scr{S}(U)$ and $V\subset U$, the commutativity of the square \eqref{nat_transf_sheaf} forces $i^*(\sigma )=\sigma|_V$ to be in the kernel of $\eta_V$.

The image $I_{\eta}$ is generally just a subpresheaf of $\scr{T}$; it does not behave as nicely as the kernel (see example \ref{image_1} below).

\begin{ej}
Let $\scr{P}$ be the presheaf over a space $M$ which assigns an open subset $U$ the space of constant maps $U\to \re$. Then $\scr{P}$ is a sheaf if and only if the space $M$ is connected. Examples of sheaves on a topological space are the sheaf of continuous maps, the sheaf of locally constant functions; if the base space happens to be a smooth manifold, then we also have the sheaves of smooth maps, differential forms, vector fields, etc.
\end{ej}

\begin{ej}
Given a presheaf of groups or modules $\scr{S}$ over $M$, the sheaf $0$ is defined by assigning the trivial group or modules to each open subset of $M$.
\end{ej}

\begin{ej}
Let $X$ be an object of some category ${\bf X}$ with terminal object $1$ and let $x_0\in M$ be fixed; the skyscraper sheaf $\scr{S}^{(x_0)}_X:\tsf{Op}(M)^\circ \to {\bf X}$ is defined in the following way:
$$
\scr{S}^{(x_0)}_X(U)=
\begin{cases}
X & \text{if $x\in U$} \\
1 & \text{if $x\not \in U$.} \\
\end{cases}
$$
The name ``skyscraper'' comes from the fact that the only stalk distinct from $1$ is the one over $x_0$ which is equal to $X$.
\end{ej}

\begin{ej}\label{bounded}
Let $M=\re$ and, for an open subset $U\subset M$, let $B(U)$ be the space of bounded mappings $U\to \re$. Consider the open interval $(0,1)$ and let $U_i:=\left (\frac{1}{i+1},1 \right )$; then $U=\bigcup_{i\geqslant 1}U_i$. Consider the maps $f_i:U_i\to \re$ given by $f_i(x)=\frac{1}{x}$. We then have that $f_i\in B(U_i)$, but these maps cannot be glued together to provide a bounded map $(0,1)\to \re$ which restriction to each $U_i$ is $f_i$. Thus, $U\mapsto B(U)$ is not a sheaf.
\end{ej}

\begin{ej}\label{image_1}
Let $\scr{O}$ denote the sheaf over $\comp^{\times}=\comp \setminus \{0\}$ of holomorphic maps $f:U\to \comp$ ($U\subset \comp^{\times}$ open) and let $\scr{O}^{\times}$ be the sheaf over $\comp^{\times}$ of invertible holomorphic maps; i.e. maps $g:U\to \comp^{\times}$. Define the exponential map $\opnm{exp}:\scr{O}\to \scr{O}^{\times}$ by $\opnm{exp}_U(u)=e^u$. Let $U_1$ and $U_2$ be the open subsets of $\comp^{\times}$ defined by $U_1:=\comp^{\times}\setminus \re_{\geqslant 0}$ and $U_2:=\comp^{\times}\setminus \re_{\leqslant0}$, where $\re_{\geqslant 0}$ (respectively $\re_{\leqslant 0}$) denotes the set of complex numbers with imaginary part equal to zero and nonnegative (respectively nonpositive) real part. We then have that $U_1\cup U_2=\comp^{\times}$. Let now $\varphi :\comp^\times \to \comp$ be any holomorphic map and denote by $u_1$ and $u_2$ the restrictions of $\varphi$ to $U_1$ and $U_2$ respectively. Let now $f_1=e^{u_1}\in \opnm{Im}\; \opnm{exp}_{U_1}$ and $f_2=e^{u_2}\in \opnm{Im}\; \opnm{exp}_{U_2}$. As $U_1$ and $U_2$ are simply-connected, the maps $u_1$ and $u_2$ are indeed well-defined holomorphic maps and given by $u_1=\log f_1$, $u_2=\log f_2$ (fixing a branch of the logarithm). Moreover, these maps coincide on the intersection $U_1\cap U_2$, which is given by the (disjoint) union of the upper and lower half-planes. But it is clear that these maps $f_1$ and $f_2$ cannot be glued together into a holomorphic map $f:\comp^{\times}\to \comp^{\times}$ such that $h=e^w$ (if so, then $w$ should be the logarithm $\log h$, but it is not a section in $\scr{O}^{\times}(\comp^{\times})$ as it is not even continuous on the whole punctured plane).
\end{ej}

We can remedy the situation described in examples \ref{bounded} and \ref{image_1} by constructing a sheaf $\scr{P}^+$ from the presheaf $\scr{P}$ in a universal way (to be specified soon). Moreover, if $\scr{P}$ is in fact a sheaf, then $\scr{P}^+$ shall be canonically isomorphic to $\scr{P}$.

Before going into the next section, we give the following

\begin{defi}
A morphism of presheaves $\eta :\scr{P}\to \scr{Q}$ over $M$ is said to be a \emph{monomorphism} if $\opnm{Ker}\eta =0$; a more general statement which also includes presheaves of sets is that $\eta$ is a monomorphism if $\eta_U$ is injective for each open subset $U\subset M$. Likewise, $\eta$ is said to be an \emph{epimorphism} if $I_{\eta}=\scr{Q}$. The map $\eta$ is an \emph{isomorphism} if it is both a monomorphism and an epimorphism.
\end{defi}

\begin{obs}
Note that the previous definition applies for an ${\bf X}$-valued presheaf in as much the notions of injectivity and surjectivity, as usually defined, make sense in ${\bf X}$. We adopt this definition because all sheaves arn presheaves we consider takes values in categories in which this notions apply. A definition with a wider range may be given using the right and left cancellation properties. Though we shall not use them, these are included in several results of section \ref{section_isomorphisms}.
\end{obs}

By the previous definition, the morphism of presheaves $\eta$ is an isomorphism if and only if $\eta_U:\scr{P}(U)\to \scr{Q}(U)$ is an isomorphism for each open subset $U$.


%%%%%%%%%%%%%%%%%%%%%%%%%%%%%%%%%%%%%%
\subsection{Stalks and Sheafification}

The process of turning a presheaf into a sheaf (sheafification) is mainly based on considering stalks, which we define and discuss next.

\begin{defi}
Given a (pre)sheaf $\scr{S}$ over a space $M$, the \emph{stalk} $\scr{S}_x$ of $\scr{S}$ over $x\in M$ is given by
$$\scr{S}_x:=\underset{U\ni x}{\operatorname{colim}}\; \scr{S}(U);$$
objects of $\scr{S}_x$ are called \emph{germs} (of sections).
\end{defi}

To be more explicit, $\scr{S}_x$ is given by taking the disjoint union $\bigsqcup_{U\ni x}\scr{S}(U)$ modulo the equivalence relation given by $(U,\sigma )\sim (V,\tau )$ if and only if there exists a neighborhood $W$ of $x$, $W\subset U\cap V$, such that $\sigma|_W=\tau|_W$.

\begin{notation}
The germ of a section $\sigma$ will be denoted by $\sigma_x$; if the reference to the open subset over which $\sigma$ is defined is needed, we will denote $\sigma_x$ by the symbol $[U,\sigma ]_x$. On the other hand, if the reference to the point $x$ is clear from the context, to ease the notation we will abuse and also use $\sigma$ to denote the germ of $\sigma$ at $x$.
\end{notation}

The assignment $\scr{S}\mapsto \scr{S}_x$ is functorial, and for each $U\ni x$, we have a canonical projection $\scr{S}(U)\to \scr{S}_x$; moreover, each (pre)sheaf homomorphism $\eta:\scr{S}\to \scr{T}$ gives rise to a morphism of stalks $\eta_x:\scr{S}_x\to \scr{T}_x$ such that the diagram
$$
\xymatrix{\scr{S}(U)\ar[r]^{\eta_U} \ar[d] & \scr{T}(U) \ar[d] \\
\scr{S}_x \ar[r]_{\eta_x} & \scr{T}_x}
$$
commutes (vertical arrows are projections). If $\scr{S},\scr{T}$ are (pre)sheaves of modules, algebras, rings, etc then so is $\eta_x$ for each $x$: for example, assume that $\scr{R}$ is a sheaf of rings and fix a point $x$ in the base space. Given points $[U,\sigma ],[V,\tau ]\in \scr{R}_x$ (with $x\in U\cap V$),  the product which makes $\scr{R}_x$ a ring (and the projection $\scr{R}(U)\to \scr{R}_x$ a ring homomorphism for each $U\ni x$) is given by $[U,\sigma ][V,\tau ]:=[U\cap V,\sigma \tau ]$, where the product $\sigma \tau$ on the right hand side is taken over $U\cap V$.

Consider now a presheaf $\scr{P}$ over $M$; we can associate to $\scr{P}$ a sheaf $\scr{P}^+$ preserving stalks. For this, we will first introduce another representation for sheaves, as a topological space over $M$.

By a ``topological space over $M$'' we mean a space $E$ together with a continuous map $E\to M$. A morphism between spaces $E\to M$, $F\to M$ over $M$ is a continuous map such that
$$
\xymatrix{
E \ar[rr] \ar[dr] & & F \ar[dl] \\
& M & }
$$
commutes. The category thus obtained is denoted by $\tsf{Top}(M)$. If $E$ is a space over $M$ with map $\pi :E\to M$, a \emph{section} of $E$ is a continuous map $\sigma :M\to E$ such that $\pi \sigma =\opnm{id}_M$. The symbol $\Gamma (E)$ will denote the space of sections $M\to E$. If $U\subset M$, we can consider local sections $U\to E$; sections of $E$ over $U$ will be denoted $\Gamma_E(U)$.

Given the presheaf $\scr{P}$ over $M$, consider the disjoint union of the stalks
$$e(\scr{P}):=\bigsqcup_{x\in M}\scr{P}_x,$$
together with the canonical projection $(x,\sigma_x)\mapsto x$ onto $M$. We now define a topology that makes this projection a local homeomorphism: let $U\subset M$ be open and let $\sigma \in \scr{P}(U)$. Define a map $\sigma^+:U\to e(\scr{P})$ by the formula
$$\sigma^+(x)=\sigma_x.$$
As $\sigma^+(x)\in \scr{P}_x$, the map $\sigma^+$ is called a \emph{section} of the space $e(\scr{P})$ over $M$. We now declare $\{\sigma^+(U)\; | \; U\subset M \; \text{open}\}$ to be a basis for the topology of $e(\scr{S})$. This topological space is called the \emph{\'etale space} of $\scr{P}$. A couple of remarks on this spaces are relevant
\begin{itemize}
\item The topology on $e(\scr{S})$ is not usually ``nice'': it is typically non-Hausdorff and
\item the projection $e(\scr{P})\to M$ is a local homeomorphism.
\end{itemize}

If $E=e(\scr{P})$ is the \'etale space of a presheaf $\scr{P}$, we denote the space of sections of $e(\scr{P})$ over $U$ by $\Gamma_{\scr{P}}(U)$. For the topology defined on $e(\scr{P})$, a section $\sigma^+ :U\to e(\scr{P})$ is continuous at $x\in U$ if and only if there exists a neighborhood $V$ of $x$ in $U$ and a section $\sigma \in \scr{P}(V)$ such that
$$\pi_y(\sigma )=\sigma^+(y)$$
for each $y\in V$, where $\pi_x:\scr{P}(V)\to \scr{P}_x$ is the canonical projection.

It will be useful also to describe the inverse construction; that is, how to obtain a sheaf from a space over $M$. For this, we need to find out for which spaces over $M$, their spaces of sections are sheaves. For a complete discussion the reader is referred to \cite{tennison:_sheaf}.

\begin{proposition}\label{loc_homeo_sheaf}
If $E\to M$ is a surjective local homeomorphism, then $\Gamma_E$ is a sheaf.
\end{proposition}

\begin{obs}
Note that if the fibres of the space $E$ over $M$ are, for example, groups, then $\Gamma_E$ will be a sheaf of groups.
\end{obs}

\begin{proposition}
The assignments $e:\scr{P}\mapsto e(\scr{P})$ and $\Gamma:E\mapsto \Gamma_E$ defines functors from the category $\tsf{PSh}(M)$ of presheaves over $M$ to the category $\tsf{Top}(M)$ of spaces over $M$ and from the category of spaces over $M$ to the category $\tsf{Sh}(M)$ of sheaves over $M$, respectively. Moreover, if $\scr{S}$ is a sheaf, the correspondence $\scr{S}\mapsto e(\scr{S})$ defines and equivalence between the category of sheaves of sets over $M$ and the category of surjective local homeomorphisms with base $M$.\footnote{For sheaves with more algebraic structure, for example sheaves of groups, modules, etc, the fibres of the local homeomorphisms defining these sheaves should of course be groups, modules, etc.}
\end{proposition}

Now define
$$\scr{P}^+(U)=\{\sigma :U\to e(\scr{P})\; | \; \sigma \; \text{is continuous and $\sigma^+(x)\in \scr{P}_x$ for each $x\in U$}\}.$$
That is, $\scr{P}^+$ is the image of $\scr{P}$ by the composite $\Gamma e$,
$$\scr{P}^+=\Gamma (e(\scr{P}))=\Gamma_{\scr{P}},$$
and then the correspondence $\scr{P}\mapsto \scr{P}^+$ defines a functor $\tsf{PSh}(M)\to \tsf{PSh}(M)$. The crucial fact is that $\Gamma$ produces a sheaf if we evaluate it in \'etale spaces of presheaves.

Note also that this construction provides a natural map of presheaves $\eta :\scr{P}\to \scr{P}^+$ defined by $\eta (\sigma )=\sigma^+$.

We summarize some important properties of these constructions in the following result, which proof can also be found in \cite{tennison:_sheaf}.

\begin{theorem}\label{sheafification}
The following properties hold:
\begin{enumerate}
\item If $E$ is an \'etale space of a presheaf $\scr{P}$, then $e(\Gamma_E)$ is isomorphic to $E$ as (\'etale) spaces over $M$.
\item The assignment $\scr{P}\mapsto \scr{P}^+$ defines a functor from the category of presheaves over $M$ to the category of sheaves over $M$.
\item The map $\eta$ induces isomorphisms $\scr{P}_x\stackrel{\cong}{\longrightarrow}\scr{P}_x^+$ for each $x\in M$.
\item A presheaf $\scr{S}$ is isomorphic to $\scr{S}^+$ if and only if $\scr{S}$ is a sheaf (and the isomorphism is the natural map $\eta$).
\item Let $\scr{P}$ be a presheaf and $\scr{S}$ a sheaf, both over $M$. Then, any morphism of presheaves $\phi :\scr{P}\to \scr{S}$ factors uniquely through the natural map $\eta :\scr{P}\to \scr{P}^+$:
$$
\xymatrix{
\scr{P} \ar[r]^{\phi} \ar[d]_{\eta} & \scr{S} \\
\scr{P}^+ \ar[ur]_{\exists ! } & 
}$$
\end{enumerate}
\end{theorem}

The sheaf $\scr{P}^+$ is called the \emph{associated sheaf} or \emph{sheafification} of the presheaf $\scr{P}$.



%%%%%%%%%%%%%%%%%%%%%%%%%%%%%%%%%%%%%%%%%%%
\subsection{Isomorphisms in $\tsf{Sh}(M)$.}
\label{section_isomorphisms}

In this section we shall discuss some important notions regarding morphisms of sheaves, for example the precise notion of surjectivity. We note first that and equality $\scr{S}=\scr{T}$ of sheaves means that, for each open subset $U\subset M$, $\scr{S}(U)=\scr{T}(U)$.

\begin{defi}\label{def_ker_image}
Let $\eta :\scr{S}\to \scr{T}$ be a morphism of sheaves of groups (or rings, modules, etc).
\begin{itemize}
\item The \emph{kernel} of $\eta$, denoted $\opnm{Ker}\eta$, is the sheaf given by $U\mapsto \opnm{Ker}\eta_U$.
\item The \emph{image} of $\eta$, denoted $\opnm{Im}\eta$, is defined as $\opnm{Im}\eta :=I_{\eta}^+$.
\end{itemize}
\end{defi}

\begin{lemma}\label{subsheaf_im}
For a morphism of sheaves $\eta :\scr{S}\to \scr{T}$ over $M$, the sheaf $\opnm{Im}\eta$ can be identified with a subsheaf of $\scr{T}$.
\end{lemma}
\begin{proof}
The conclusion of the lemma is immediate representing sections of the sheaf $\scr{T}$ as maps $\sigma :U\to \bigsqcup_{x\in U}\scr{T}_x$.
\end{proof}

\begin{defi}
A sheaf homomorphism $\eta :\scr{S}\to \scr{T}$ is said to be
\begin{itemize}
\item a \emph{monomorphism} or an \emph{injective morphism} if $\opnm{Ker}\eta=0$.
%\item an \emph{epimorphism} or a \emph{surjective morphism} if and only $\eta$ verifies the following cancellation property: for any sheaf $\scr{Q}$ and morphisms $\phi ,\theta :\scr{T}\to \scr{Q}$ such that $\phi \eta =\theta \eta$, then $\phi =\theta$.
\item an \emph{epimorphism} or a \emph{surjective morphism} if $\opnm{Im}\eta=\scr{T}$ (this last equality relies on lemma \ref{subsheaf_im}).
\item an \emph{isomorphism} if it is both a monomorphism and an epimorphism.
\end{itemize}
\end{defi}

\begin{obs}
For a morphism of sheaves of sets $\eta :\scr{S}\to \scr{T}$, injectivity can be defined by asking the maps $\eta_U$ to be injective for each $U$.
\end{obs}

The next lemma shows that injectivity and surjectivity are preserved when passing to the stalks.

\begin{lemma}\label{stalks_ker_im}
For each $x\in M$ we have
$$(\opnm{Ker}\eta )_x=\opnm{Ker}\eta_x \quad , \quad (\opnm{Im}\eta )_x=\opnm{Im}\eta_x \quad \text{and} \quad I_{\eta ,x}=\opnm{Im}_{\eta_x}.$$
\end{lemma}
\begin{proof}
That the class $[U,\sigma ]_x$ belongs to $(\opnm{Ker}\eta )_x$ is equivalent to saying that $\eta_U(\sigma )=0$, and then
$$\eta_x[U,\sigma]_x=[U,\eta_U(\sigma )]_x=0.$$
Now, $[U,\sigma ]_x\in \opnm{Ker}\eta_x$ if and only if there exists a neighborhood $V\subset U$ of $x$ such that $\eta_U(\sigma )|_V=\eta_V(\sigma|_V)=0$ (the first equality by naturality of $\eta$). But then $[V,\sigma|_V]_x\in (\opnm{Ker}\eta)_x$.

For the second equality, first note that $(\opnm{Im}\eta)_x=I^+_{\eta ,x}=I_{\eta ,x}$, as the sheafification functor preserves stalks. Thus, we only need to prove the equality $I_{\eta ,x}=\opnm{Im}\eta_x$ which can be done in exactly the same fashion as for the previous equality.
\end{proof}

Let us point out the following fact: assume that $\scr{S}$ and $\scr{T}$ are sheaves such that $\scr{S}_x\cong \scr{T}_x$ for each $x$ in the base space. Then the conclusion that the sheaves $\scr{S}$ and $\scr{T}$ are isomorphic is generally not true (locally-free sheaves are a good example; see section \ref{modules}).

\begin{lemma}
Let $\eta :\scr{S}\to \scr{T}$ be a sheaf homomorphism. If $\eta$ is an isomorphism in the category $\tsf{PSh}(M)$, then it is also an isomorphism in the category $\tsf{Sh}(M)$.
\end{lemma}
\begin{proof}
The notion of injectivity for morphisms in the category of presheaves is the same as the one for arrows in the category of sheaves. If $\eta$ is an epimorphism viewed in the category of presheaves, then for each open subset $U$, $\eta_U$ is a surjective map. We need to show that $\opnm{Im}\eta =\scr{T}$.

First note that by lemma \ref{stalks_ker_im}, the map $\eta_x:\scr{S}_x\to \scr{T}_x$ is surjective. Let now $\sigma \in \scr{T}(U)$; this object is a continuous section $\sigma :U\to \bigsqcup_{x\in U}\scr{T}_x$. But $\scr{T}_x=I_{\eta ,x}=I^+_{\eta ,x}$. The lemma is proved.
\end{proof}

The converse to the previous statement is false, as the next example shows.

\begin{ej}\label{image_2}
Consider the exponential map $\opnm{exp}:\scr{O}\to \scr{O}^{\times}$ of example \ref{image_1}. This map is a surjective sheaf homomorphism which is not surjective as a morphism of presheaves: if $w:\comp^\times \to \comp^{\times}$ is a holomorphic map, then the equation $e^u=w$ does not have a solution in $\scr{O}(\comp^{\times})$.
\end{ej}

Let us finish this discussion by recalling and introducing some useful characterizations for mono, epi and isomorphisms of (pre)sheaves. For details, the reader may consult again the comprehensive exposition given in \cite{tennison:_sheaf}.

\begin{theorem}
For a morphism of presheaves $\eta :\scr{P}\to \scr{Q}$ the following conditions are equivalent:
\begin{enumerate}
\item $\eta$ is injective; that is $\opnm{Ker}\eta=0$.
\item For each open subset $U\subset M$, the map $\eta_U:\scr{P}(U)\to \scr{Q}(U)$ is inyective.
\item If $\scr{S}$ is any presheaf and $\phi ,\theta :\scr{S}\to \scr{P}$ are two morphisms of presheaves such that $\eta \phi =\eta \theta$, then $\phi =\theta$.
\end{enumerate}
\end{theorem}

\begin{obs}
The conditions enumerated in the previous theorem imply that for each $x\in M$ the morphism $\eta_x:\scr{P}_x\to \scr{Q}_x$ is injective. But in order to add this property into the list of equivalent conditions the presheaves must be \emph{separated}. We will not define this notion here, but the reader may consult the aforementioned reference.
\end{obs}

\begin{theorem}
For a morphism of sheaves $\eta :\scr{S}\to \scr{T}$ the following conditions are equivalent:
\begin{enumerate}
\item $\eta$ is injective; that is $\opnm{Ker}\eta =0$.
\item For each open subset $U\subset M$, the map $\eta_U:\scr{S}(U)\to \scr{T}(U)$ is inyective.
\item For each $x\in M$, the map $\eta_x:\scr{P}_x\to \scr{Q}_x$ is injective.
\item If $\scr{R}$ is any sheaf and $\phi ,\theta :\scr{R}\to \scr{S}$ are two morphisms of presheaves such that $\eta \phi =\eta \theta$, then $\phi =\theta$.
\end{enumerate}
\end{theorem}

For surjective morphisms of presheaves we have the following

\begin{theorem}\label{presheaf_epi}
For a morphism of presheaves $\eta :\scr{P}\to \scr{Q}$ the following conditions are equivalent:
\begin{enumerate}
\item $\eta$ is surjective; that is, $I_{\eta}=\scr{T}$. 
\item For each open subset $U\subset M$, $\eta_U$ is surjective.
\item For any presheaf $\scr{R}$ and morphisms $\phi ,\theta :\scr{Q}\to \scr{R}$ such that $\phi \eta =\theta \eta$, then $\phi =\theta$.
\end{enumerate}
\end{theorem}

For sheaves we have an analogous result.

\begin{theorem}
For a morphism of sheaves $\eta :\scr{S}\to \scr{T}$ the following conditions are equivalent:
\begin{enumerate}
\item $\eta$ is surjective, that is $I^+_{\eta }=\scr{T}$.
\item For each point $x\in M$, the map $\eta_x:\scr{S}_x\to \scr{T}_x$ is surjective.
\item For any sheaf $\scr{R}$ and morphisms $\phi ,\theta :\scr{T}\to \scr{R}$ such that $\phi \eta =\theta \eta$, then $\phi =\theta$.
\end{enumerate}
Moreover, any of the conditions of theorem \ref{presheaf_epi} implies these ones.
\end{theorem}

We will now combine these facts into the notion of isomorphism, which we define first; we omit the words ``sheaf'' and ``presheaf'' just because the same definition applies to both of them.

\begin{theorem}
For a morphism of presheaves $\eta :\scr{P}\to \scr{Q}$ the following conditions are equivalent:
\begin{enumerate}
\item $\eta$ is an isomorphism.
\item For each open subset $U\subset M$, $\eta_U$ is a bijection.
\item $\eta$ is a monomorphism and an epimorphism.
\end{enumerate}
\end{theorem}

For sheaves we have:

\begin{theorem}\label{stalk_iso}
For a morphism of sheaves $\eta :\scr{S}\to \scr{T}$ the following conditions are equivalent:
\begin{enumerate}
\item $\eta$ is an isomorphism.
\item For each $x\in M$, $\eta_x:\scr{S}_x\to \scr{T}_x$ is an isomorphism.
\end{enumerate}
\end{theorem}
\begin{proof}
The ``only if'' part follows immediately from lemma \ref{stalks_ker_im}.

For the ``if'' part, assume that each stalk map is an isomorphism and let $U\subset M$ be any open subset. Let $\sigma,\tau \in \scr{S}(U)$ be sections such that $\eta_U(\sigma )=\eta_U(\tau )$. This implies that $\sigma_x=\tau_x$ for each $x\in U$. We can then find a collection $\{W_x\}_{x\in U}$ of open subsets such that $x\in W_x$ and $\sigma|_{W_x}=\tau|_{W_x}$. As $U=\bigcup_{x\in U}W_x$ the equality $\sigma=\tau$ follows from the definition of sheaf, by gluing the restrictions $\sigma|_{W_x}$ and $\tau|_{W_x}$.

If $\tau \in \scr{T}(U)$, then for each $x\in U$ we have a unique element $\sigma_x\in \scr{S}_x$ such that $\eta_x(\sigma_x)=\tau_x$. Assume that $\sigma^{(x)}\in \scr{S}(U_x)$ is a section with germ equal to $\sigma_x$, where $U_x\subset U$ is a neighborhood of $x$. Then, as the germs $\eta_x(\sigma_x)=\eta_{U_x}(\sigma^{(x)})_x$ and $\tau_x$ coincide, there exists a neighborhood $W_x\subset U_x$ of $x$ such that $\eta_{W_x}(\sigma^{(x)}|_{W_x})=\tau|_{W_x}$. We will now check that the sections $\sigma^{(x)}$ can be glued together into a section $\sigma \in \scr{S}(U)$ such that $\eta_U(\sigma )=\tau$. Let $x,x'\in U$ be such that $W_{xx'}:=W_x\cap W_{x'}\neq \emptyset$. Then $\eta_{W_{xx'}}(\sigma^{(x)}|_{W_{xx'}})=\tau|_{W_{xx'}}=\eta_{W_{xx'}}(\sigma^{(x')}|_{W_{xx'}})$. Then, for each $y\in W_{xx'}$,
$$\eta_y(\sigma^{(x)}_y)=\eta_y(\sigma^{(x')}_y).$$
The last equality and the injectivity of $\eta_y$ implies that $\sigma^{(x)}_y=\sigma^{(x')}_y$ and then we can find a neighborhood $Z=Z^{(y)}\subset W_{xx'}$ of $y$ such that $\sigma^{(x)}|_{Z^{(y)}}=\sigma^{(x')}|_{Z^{(y)}}$. As $\scr{S}$ is a sheaf, this is equivalent to the equality $\sigma^{(x)}|_{W_{xx'}}=\sigma^{(x')}|_{W_{xx'}}$ and this, again by glueing properties of sheaves, to the existence of a section $\sigma \in \scr{S}(U)$ such that $\sigma|_{U_x}=\sigma^{(x)}$ for each $x$ and $\eta_U(\sigma )=\tau$, as desired.
\end{proof}


%%%%%%%%%%%%%%%%%%%%%%%%%%%%%%%%%%%%%%%%
\subsection{Direct and Inverse Image}

Assume that $f:M\to N$ is a continuous map. In this section we will describe how to construct a sheaf over $N$ from a sheaf over $M$ and viceversa.

Let us start first with a sheaf $\scr{S}$ over $M$. Define the presheaf $f_*\scr{S}$ over $N$ in the following way: given $V\in \tsf{Op}(N)$, $(f_*\scr{S})(V)=\scr{S}(f^{-1}(V))$. If $i:W\to V$ is an inclusion and $\sigma \in (f_*\scr{S})(V)$, then $f^{-1}(W)\subset f^{-1}(V)$ and $i^*(\sigma )=\sigma |_{f^{-1}(W)}$. The proof that this presheaf is in fact a sheaf follows inmediately from the definition of sheaf. Moreover, a sheaf homomorphism $\eta :\scr{S}\to \scr{T}$ induces a morphism $f_*\eta :f_*\scr{S}\to f_*\scr{T}$ by defining $(f_*\eta )_V=\eta_{f^{-1}(V)}$. We thus obtain a functor
$$f_*:\tsf{Sh}(M)\longrightarrow \tsf{Sh}(N)$$
which is called the \emph{direct image functor}. The sheaf $f_*\scr{S}$ is called the \emph{direct image of $\scr{S}$ by $f$}. Note that this construction is well suited for sets, abelian groups, rings, algebras and modules (this last case is treated separatedly).

The stalks of the direct image sheaves are easy to compute in some particular cases, as the following result shows.

\begin{proposition}\label{direct_covering}
Let $f:M\to N$ be an $n$-sheeted covering map and let $y\in N$. Then
$$(f_*\scr{S})_y\cong  \scr{S}_{x_1}\times \cdots \times \scr{S}_{x_n},$$
where $f^{-1}(y)=\{x_1,\dots ,x_n\}$.
\end{proposition}
\begin{proof}
Let $y\in N$ an assume that $V\ni y$ is a neighborhood such that $f^{-1}(V)=\bigsqcup_{i=1}^nU_i$ and $f|_{U_i}:U_i\cong V$. As $(f_*\scr{S})(U)=\scr{S}(f^{-1}(V))=\scr{S}\left (\bigsqcup_iU_i\right )$, a section $\sigma \in (f_*\scr{S})(V)$ can be represented as a continuous map $\sigma :U\to \bigsqcup_{x\in U}\scr{M}_x$, where $U:=\bigsqcup_iU_i$, and this is equivalent to having $n$ sections $\sigma_i:U_i\to \bigsqcup_{x\in U_i}\scr{M}_x$. From these facts we can define a map $(f_*\scr{S})_y\to \scr{S}_{x_1}\times \cdots \times \scr{S}_{x_n}$,
$$[V,\sigma ]_y\longmapsto ([U_1,\sigma_1]_{x_1},\dots ,[U_k,\sigma_k]_{x_n})$$
which is the desired isomorphism.
\end{proof}

In particular, if $f^{-1}(U)=\bigsqcup_iU_i$ and $f|_{U_i}:U_i\cong U$, by the previous result we have an isomorphism
$$(f_*\scr{S})|_U\cong \prod_i\scr{S}|_{U_i}.$$

\begin{obs}
The previous proof shows that this result remains valid for sheaves of abelian groups and rings, by replacing $\times$ with the direct sum $\oplus$.
\end{obs}

The other construction we will deal with starts with a sheaf over $N$ and provides a sheaf over $M$ (just as the pullback construction for bundles). So let $\scr{T}$ be a sheaf over $N$, which can be taken to be a sheaf of sets, abelian groups, modules, etc. We will now define the sheaf $f^{-1}\scr{T}$, usually called the \emph{topological inverse image of $\scr{T}$ by $f$}. If one tries to define this sheaf in the same way as the direct image, that is, by defining $(f^{-1}\scr{T})(U)$ as $\scr{T}(f(U))$, then a problem arises, as $f(U)$ need not be an open subset. This drawback makes the definition of the inverse image much more complicated than the one for the direct image. We need to consider, not $f(U)$, but a colimit taken over neighborhoods of it. That is, we consider the correspondence
\begin{equation}\label{presheaf_inv}
U\longmapsto \underset{V\supset f(U)}{\opnm{colim}}\scr{T}(V).
\end{equation}
But this correspondence is just a presheaf, and not generally a sheaf. The topological inverse image $f^{-1}\scr{T}$ is then defined as the sheafification of this presheaf.

The construction of the inverse image can also be given in terms of \'etale spaces, which provide a better way to handle it. We will now describe it briefly, refering the reader again to \cite{tennison:_sheaf} to take care of details.

If $\scr{T}$ is a sheaf over $N$, consider its \'etale space $e(\scr{T})$. We thus have a diagram of topological spaces and continuous maps
$$
\xymatrix{
  & e(\scr{T}) \ar[d] \\
M \ar[r]^f & N,}
$$
where the vertical arrow is the projection. Let $E$ be defined as the pullback of $e(\scr{T})$ along $f$,
$$E:=f^*e(\scr{T})=\{(x,\sigma_x)\in M\times e(\scr{T})\; | \; x\in M\}$$
with the induced product topology. The pullback $E$ together with the projection $(x,\sigma_x)\mapsto x$ defines a space over $M$ which is a local homeomorphism. By \ref{loc_homeo_sheaf}, $\Gamma_E$ defines a sheaf, which turns out to be isomorphic to the inverse image.

From the previous discussion the next result is immediate.

\begin{proposition}
We have an isomorphism $(f^{-1}\scr{T})_x\cong \scr{T}_{f(x)}$.
\end{proposition}

Hence, one can easily deduce that the inverse image of a sheaf of abelian groups or rings is also a sheaf of abelian groups or rings.

In some particular cases, the inverse image sheaf admits a simpler form.

\begin{proposition}
Let $f:M\to N$ be an open map (e.g. a covering map). Then the assignment $U\mapsto \scr{T}(f(U))$ is a sheaf over $M$ isomorphic to the inverse image.
\end{proposition}
\begin{proof}
Verification of the sheaf conditions for $U\mapsto \scr{T}(f(U))$ is obtained by following the definition \ref{def_sheaf}. The other statement follows from the fact that the presheaf \eqref{presheaf_inv} is in fact a sheaf as $f(U)$ is open for each open $U\subset M$.
\end{proof}

The next result states the important relation between the direct and inverse image.

\begin{theorem}{\rm (\cite{tennison:_sheaf}, Theorem 3.7.13.)}\label{adjunction}
The functor $f^{-1}$ is left adjoint to $f_*$.
\end{theorem}

In other words, given sheaves $\scr{S}$ and $\scr{T}$ over $M$ and $N$ respectively, we have a bijection
$$F:\opnm{Hom}_M(f^{-1}\scr{T},\scr{S})\stackrel{\cong}{\longrightarrow}\opnm{Hom}_N(\scr{T},f_*\scr{S}),$$
and this property characterizes $f^{-1}\scr{T}$, up to isomorphism, in the category of sheaves over $M$.


%%%%%%%%%%%%%%%%%%%%%%%%%%%%%%%%%
\subsection{Locally Free Modules}
\label{modules}

By fixing a commutative ground ring $R$, we can define a sheaf of $R$-modules as a functor $\tsf{Op}(M)^{\circ}\to \tsf{Mod}_R$ with values in the category of $R$-modules. There is a useful generalization of this definition, which involves considering a sheaf of rings instead of a fixed one.

Let $\scr{O}$ be a sheaf of (commutative) $\comp$-algebras over a a space $M$ (which will usually be a sheaf of functions). A sheaf $\scr{M}$ over $M$ is said to be an \emph{$\scr{O}$-module} if 
\begin{enumerate}
\item for each open subset $U\subset M$, $\scr{M}(U)$ is an $\scr{O}(U)$-module and
\item for each inclusion $i:V\subset U$ of open subsets, the restriction $i^*:\scr{M}(U)\to \scr{M}(V)$ is $\scr{O}(U)$-linear; that is, $i^*(x+y)=i^*(x)+i^*(y)$ and $i^*(ax)=a|_Vi^*(x)$ for $x,y\in \scr{M}(U)$ and $a\in \scr{O}(U)$, where $a|_V$ is the image of $a\in \scr{O}(U)$ by the restriction map $\scr{O}(U)\to \scr{O}(V)$.
\end{enumerate}
The $\scr{O}$-module $\scr{M}$ is said to be \emph{locally-free} if there exists an open cover $\mathfrak{U}$ of $M$ such that the restriction $\scr{M}|_U$ is isomorphic to $\scr{O}^n|_U$ for some integer $n\geqslant 1$, which is called the \emph{rank} of $\scr{M}$. Though many of the result in following paragraphs are valid for general $\scr{O}$-modules, in the sequel we shall work with locally free modules of finite rank. For further details, the reader is referred to \cite{tennison:_sheaf}. 

\begin{notation}
The notation $\scr{O}_M$ is usually adopted for sheaves of maps over some space $M$ (topological space, smooth manifold, scheme). The restriction $\scr{O}_M|_U$ of $\scr{O}_M$ to $U$ will be denoted by $\scr{O}_U$. If the base manifold is clear, then we will denote $\scr{O}_M$ just by $\scr{O}$.
\end{notation}

\begin{defi}
Let $\scr{R}$ and $\scr{A}$ be sheaves of rings over s space $M$. The sheaf $\scr{A}$ is called an \emph{$\scr{R}$-algebra} if a homomorphism of sheaves of rings $\varphi :\scr{R}(U)\to \scr{A}(U)$ exists such that for each inclusion $V\subset U$, the square
$$
\xymatrix{
\scr{R}(U) \ar[r] \ar[d]_{\varphi_U} & \scr{R}(V) \ar[d]^{\varphi_V} \\
\scr{A}(U) \ar[r] & \scr{A}(V).}
$$
commutes. If $\scr{R}$ is a sheaf of commutative rings, then the morphism $\scr{R}\to \scr{A}$ should be central, in the sense that for each $U\in \tsf{Op}(M)$, the image of $\scr{R}(U)$ is contained in the center of $\scr{A}(U)$.
\end{defi}

\begin{obs}
Given a sheaf of rings $\scr{R}$, the \emph{center} of $\scr{R}$ is defined by the assignment $U\mapsto Z(\scr{R}(U))$, where $Z(R)$ denotes the center of the ring $R$. This correspondence does not define a sheaf in general: let $\sigma \in Z(\scr{R}(U))$; then $\sigma \tau=\tau \sigma$ for each section $\tau \in \scr{R}(U)$. Applying the restriction map $Z(\scr{R}(U))\to Z(\scr{R}(V))$ we can only deduce that $\sigma |_V$ commutes with all the sections in the image of the restriction $\scr{R}(U)\to \scr{R}(V)$; but if this map is not surjective, then there is no way to assure that $\sigma |_V$ will commute with \emph{all} the sections in $\scr{R}(V)$. 
\end{obs}

A homomorphism of sheaves of rings $\phi :\scr{R}\to \scr{Q}$ is called \emph{central} if $\phi_U:\scr{R}(U)\to \scr{Q}(U)$ is a central ring homomorphism.

\begin{proposition}\label{central_stalks}
A ring homomorphism $\phi :\scr{R}\to \scr{Q}$ is central if and only if $\phi_x:\scr{R}_x\to \scr{Q}_x$ is central for each $x$.
\end{proposition}
\begin{proof}
Fix a point $x$ and let $[U,\sigma ]\in \scr{R}_x$ be such that $[U,\phi_U(\sigma )]$ is in the center of $\scr{Q}_x$. Then, if $[V,\tau ]\in \scr{Q}_x$ is an arbitrary point, we must have that $[U\cap V,\phi_U(\sigma )\tau ]$ should be equal to $[U\cap V,\tau \phi_U(\sigma )]$ over $U\cap V$. But, by naturality of morphisms of sheaves, the restriction of $\phi_U(\sigma )$ to $U\cap V$ is equal to $\phi_{U\cap V}(\sigma |_{U\cap V})$, which belongs to the center of $\scr{Q}(U\cap V)$. This proves the ``only if'' part.

To prove the other implication, assume that $\phi_x(\scr{R}_x)$ is in the center of $\scr{Q}_x$ for each $x$. Let $U\subset M$ be an open subset, $\sigma \in \scr{R}(U)$, $\tau \in \scr{Q}(U)$ and let $x\in U$. Then $\phi_x[U,\sigma ]\in Z(\scr{Q}_x)$; in particular, there should exist an open neighborhood $V_x\subset U$ of $x$ such that $\phi_{V_x}(\sigma )\tau = \tau \phi_{V_x}(\sigma )$ over $V_x$. That is, the sections $\phi_U(\sigma )\tau$ and $\tau \phi_U(\sigma )$ coincide on $V_x$ for each $x\in U$. As $\scr{Q}$ is a sheaf and $U=\bigcup_{x\in U}V_x$, the result follows.
\end{proof}

Assume that $\scr{M}$ is a locally free-sheaf of $\scr{O}$-modules over $M$ of, say, rank $n$. If $x\in M$, then there exists a neighbourhood $U\ni x$ such that $\scr{M}|_U\cong \scr{O}^n_U$. In particular, each stalk $\scr{M}_x$ is isomorphic to $\scr{O}_x^n$. 

Given two $\scr{O}$-modules $\scr{M}$ and $\scr{N}$, a morphism $\eta :\scr{M}\to \scr{N}$ is a sheaf homomorphism which is also $\scr{O}$-linear; that is, for each $U\in \tsf{Op}(M)$, $\eta_U:\scr{M}(U)\to \scr{N}(U)$ is an $\scr{O}(U)$-linear homomorphism (compatible with restrictions). The set of such morphisms will be denoted by $\operatorname{Hom}_\scr{O}(\scr{M},\scr{N})$. This defines the category $\tsf{Mod}_{\scr{O}_M}$ of $\scr{O}_M$-modules.

On the other hand, as all this structures are compatible with restrictions, we can define the sheaf $\underline{\operatorname{Hom}}_\scr{O}(\scr{M},\scr{N})$ by the assignment
$$U\longmapsto \operatorname{Hom}_{\scr{O}_U}(\scr{M}|_U,\scr{N}|_U)$$
(compare with the construction of the bundle of homomorphisms in \ref{bundles_operations}).

Free-modules have many desirable properties; indeed, many devices used for modules over fields (i.e. vector spaces) are available for free $R$-modules when $R$ is a commutative ring. These facts of course translates to the sheaves $\scr{O}^n$ and also to locally-free sheaves (at a local level). For example, it is well known that every vector space has a basis (i.e. a system of linearly independent generators). If $N$ is a free $R$-module, then $N\cong R^n$ for some $n$. In $R^n$, consider the set $B=\{e_1,\dots ,e_n\}$, where $e_i$ is the vector which $i$-th coordinate is equal to $1$ (or some other unit of $R$) and all the others are zero. Then $B$ is a basis of $R^n$ and, if $f:N\cong R^n$ is an isomorphism, then $\{f^{-1}(e_1),\dots ,f^{-1}(e_n)\}$ is a basis of $N$. This statements are also valid in $\scr{O}^n$ by taking constant maps $e_i(x)=u_i$ for each $x$, where $u_i\neq 0$ is a unit.

Denote by $\operatorname{M}_{k\times n}(\scr{O})$ the sheaf which to each open subset $U$ assigns the $\scr{O}(U)$-module $\operatorname{M}_{k\times n}(\scr{O}(U))$ of $k\times n$ matrices with coefficients in $\scr{O}(U)$. Then,
\begin{equation}\label{iso_matr_hom}
\underline{\operatorname{Hom}}_{\scr{O}}(\scr{O}^n,\scr{O}^k)\cong \operatorname{M}_{k\times n}(\scr{O}),
\end{equation}
which can be deduced from a standard linear algebra argument. If $n=k$, we will denote $\operatorname{M}_{n \times n}(\scr{O})$ by $\operatorname{M}_n(\scr{O})$. From equation \eqref{iso_matr_hom} we can easily prove the following

\begin{lemma}
If $\scr{M}$ and $\scr{N}$ are locally-free of rank $n$ and $k$ respectively, then $\underline{\operatorname{Hom}}_{\scr{O}}(\scr{M},\scr{N})$ is also locally-free, of rank equal to $nk$.
\end{lemma}

As usual, given an $\scr{O}$-module $\scr{M}$, we define its dual module $\scr{M}^*$ by
$$\scr{M}^*=\underline{\operatorname{Hom}}_\scr{O}(\scr{M},\scr{O}).$$

\begin{lemma}
If $\mathscr{M}$ is a locally-free $\mathscr{O}_M$-module, then also is $\mathscr{M}^*$.
\end{lemma}
\begin{proof}
Let $x\in M$ and $U\ni x$ such that $\mathscr{M}|_U\cong \mathscr{O}_U^n$. Let $\{e_1,\dots ,e_n\}$ be a basis for $\mathscr{M}|_U$. Then the map
$$\phi \longmapsto (\phi (e_1),\dots ,\phi (e_n))$$
defines an isomorphism $\mathscr{M}^*|_U\cong \mathscr{O}_U^n$.
\end{proof}

As one would expect, if $\{e_1,\dots ,e_n\}$ is a local basis for the locally-free $\scr{O}$-module $\scr{M}$, then the set $\{e^1,\dots ,e^n\}$, where $e^i:\scr{M}(U)\to \scr{O}(U)$ is defined by
$$e^i(e_j)=\delta_{ij}$$
is the local basis of $\scr{M}$ dual to $\{e_1,\dots ,e_n\}$.

\begin{lemma}
If $\mathscr{M}$ is a locally-free $\mathscr{O}$-module, then we have a canonical isomorphism
$$\mathscr{M}^{**}\cong \mathscr{M}.$$
\end{lemma}
\begin{proof}
Let $\eta:\scr{M}\rightarrow \scr{M}^{**}$ be the map given by
$$\eta (e)=e^{**},$$
where $e^{**}:\scr{M}^*\rightarrow \scr{O}$ is given by
$$e^{**}(\phi )=\phi (e).$$
Fix a point $x\in M$; we then only need to show that the stalk map
$$\eta_x:\scr{M}_x\longrightarrow \scr{M}_x^{**}$$
is an isomorphism of $\scr{O}_x$-modules.

Suppose first that $\eta_x(e_x)=0$. Then
$$\phi_x(e_x)=0$$
for each $\phi_x\in \scr{M}_x^*$. As $\scr{M}^*_x$ is also free, by taking a basis this easily implies that necessarily $e_x=0$.

Let now $\varepsilon \in \scr{M}^{**}_x$. If $\{e_{1,x},\dots ,e_{n,x}\}$ is a basis for $\scr{M}_x$, let $\{\phi_{1,x},\dots ,\phi_{n,x}\}$ be its dual basis. Assume
$$\varepsilon (\phi_{i,x})=f_{i,x}.$$
Then, defining $u_x=\sum_if_{i,x}e_{i,x}$, we have that
$$\varepsilon (\phi_{i,x})=f_{i,x}=\phi_{i,x}(u_x)$$
for each $i$, and thus $\varepsilon =u_x^{**}$.
\end{proof}

The direct sum of $\scr{M}\oplus \scr{N}$ of two locally free $\scr{O}$-modules $\scr{M},\scr{N}$ over $M$ is again a locally free $\scr{O}$-module, and its rank is the sum of the ranks of each summand. 

Given two $\scr{O}$-modules $\scr{M}$ and $\scr{N}$, the tensor product $\scr{M}\otimes_{\scr{O}}\scr{N}$ (or just $\scr{M}\otimes \scr{N}$ if the sheaf $\scr{O}$ is clear) is the sheaf associated to the presheaf given by
$$U\longmapsto \scr{M}(U)\otimes_{\scr{O}(U)}\scr{N}(U).$$
If $\scr{M}$ and $\scr{N}$ are locally free of ranks $n$ and $k$ respectively, then $\scr{M}\otimes \scr{N}$ is also locally free, of rank $nk$.
As colimits commute with tensor products, we have that
$$(\scr{M}\otimes_{\scr{O}} \scr{N})_x\cong \scr{M}_x\otimes_{\scr{O}_x}\scr{N}_x.$$

The following result comprises some important properties of tensor products, and its proof may be found in \cite{kn:gortz_wed}. We try to omit the reference to the sheaf $\scr{O}$ as it is usually clear form the context.

\begin{proposition}\label{tensor_hom}
Let $\scr{M}$ and $\scr{N}$ and $\scr{P}$ be locally free $\scr{O}$-modules over a space $M$.
\begin{enumerate}
\item There exists a linear adjunction
$$\underline{\opnm{Hom}}(\scr{M}\otimes \scr{P},\scr{N})\cong \underline{\opnm{Hom}}(\scr{M},\underline{\opnm{Hom}}(\scr{P},\scr{N}).$$ 
\item If $\scr{M}$ or $\scr{N}$ is of finite rank, then we have a canonical isomorphism
$$\underline{\opnm{Hom}}(\scr{M},\scr{P})\otimes \scr{N}\cong \underline{\opnm{Hom}}(\scr{M},\scr{P}\otimes \scr{N}).$$
\end{enumerate}
\end{proposition}

The following corollary is a useful consequence of the previous result.

\begin{cor}\label{tensor_hom_dual}
For locally free $\scr{O}$-modules (of finite rank, as usual), we have isomorphisms
\begin{enumerate}[a.]
\item $\underline{\opnm{Hom}}(\scr{M},\scr{N})\cong \scr{M}^*\otimes \scr{N}$ and
\item $(\scr{M}\otimes \scr{N})^*\cong \scr{M}^*\otimes \scr{N}^*$.
\end{enumerate}
\end{cor}
\begin{proof}
The first item follows readily from item 2 of the previous result, taking $\scr{P}=\scr{O}$. To prove \emph{b}, we use item \emph{a} and also item 1 from the previous proposition:
$$
\begin{aligned}
(\scr{M}\otimes \scr{N})^* &\cong \underline{\opnm{Hom}}(\scr{M}\otimes \scr{N},\scr{O}) \\
													 &\cong \underline{\opnm{Hom}}(\scr{M},\underline{\opnm{Hom}}(\scr{N},\scr{O})) \\
													 &\cong \underline{\opnm{Hom}}(\scr{M},\scr{N}^*) \\
													 &\cong \scr{M}^*\otimes \scr{N}^*. \\
\end{aligned}
$$
\end{proof}

\begin{ej}
Let $\eta :\scr{M}\to \scr{N}$ be a homomorphism of locally free $\scr{O}_M$-modules. Then $\opnm{Ker}\eta$ and $\opnm{Im}\eta$ need not be locally free modules (cf. theorem \ref{sheaf_bundle} and the last paragraph of section \ref{vector_bundles}).
\end{ej}

We shall end this section with the construction of fibres. An important particular class of modules is the one consisting of modules over algebras $\scr{O}$ for which $\scr{O}_x$ is a local ring for each $x$; that is, it contains only one maximal ideal, which we denote by $\mathfrak{m}_x$.

\begin{defi}
The sequence of projections
\begin{equation}\label{fiber_sheaf}
\scr{O}(U)\longrightarrow \scr{O}_x\longrightarrow \scr{O}_x/\mathfrak{m}_x,
\end{equation}
is called the \emph{evaluation map}. If $f$ is a section of $\scr{O}$ over $U$, then its image will be denoted by $f(x)$.
\end{defi}

\begin{obs}\label{stalks_evaluation}
Let $A$ be a commutative, local $\comp$-algebra with maximal ideal $\mathfrak{m}$. Then we have a direct sum decomposition $A=\langle 1 \rangle \oplus \mathfrak{m}$ of the vector space $A$, where $\langle 1 \rangle$ is the vector subspace generated by the unit. If $[x]$ denotes the class of $x$ (mod. $\mathfrak{m}$), then the correspondence $z\mapsto [z1]$ defines a canonical isomorphism $\comp \to A/\mathfrak{m}$. The inverse of this map is defined in the following way: if $a\in A$, then we can write it as $a=z1+x$ where $x\in \mathfrak{m}$. The assignment $a\mapsto z$ defines an algebra homomorphism $A\to \comp$ with kernel equal to $\mathfrak{m}$.

Thus, if $\scr{O}$ is a sheaf of $\comp$-algebras with local stalks, the evaluation map can be regarded as a map with values in $\comp$ (the same applies to $\re$-algebras); in fact, the family of vector spaces $\bigsqcup_{x\in M}\scr{O}_{M,x}/\mathfrak{m}_x$ is a trivial bundle: the map
$$M\times \comp \longrightarrow \bigsqcup_{x\in M}\scr{O}_{M,x}/\mathfrak{m}_x$$
given by $(x,z)\mapsto (x,[z1])$ is an isomorphism by the previous discussion.
\end{obs}

Component-wise operations provides an evaluation map
$$\scr{O}^n(U)\longrightarrow \scr{O}^n_x\longrightarrow \scr{O}^n_x/\mathfrak{m}_x^{\oplus n}$$
given by $(f_1,\dots ,f_n)\mapsto (f_1(x),\dots ,f_n(x))$.

\begin{ej}
Let $\scr{O}$ be any sheaf of functions (e.g. continuous, smooth, holomorphic, etc). In this case, $\mathfrak{m}_x=\{f_x\in \scr{O}_x\; | \; f(x)=0\}$. Let $\opnm{ev}_x :\scr{O}_x\to \comp$ be the map $\opnm{ev}_x (f_x)=f(x)$. This map has kernel equal to $\mathfrak{m}_x$ and is the inverse of the isomorphism defined in remark \ref{stalks_evaluation}. Thus, the image of $f\in \scr{O}(U)$ by the projections \eqref{fiber_sheaf} is precisely $f(x)$.
\end{ej}

The following easy lemma lets us generalize this facts to any locally-free sheaf.

\begin{lemma}
Let $\alpha :\scr{O}^n\to \scr{O}^n$ be an $\scr{O}$-linear isomorphism. Then, for each $x\in M$,
$$\alpha_x (\mathfrak{m}^{\oplus n}_x)=\mathfrak{m}^{\oplus n}_x,$$
where $\alpha_x:\scr{O}^n_x\to \scr{O}^n_x$ is the induced stalk map.
\end{lemma}
\begin{proof}
Fix a point $x\in M$. We then have
$$\alpha_x (f_1,\dots ,f_n)=\Bigl (\sum_i\lambda_{1i}f_i,\dots ,\sum_i\lambda_{ni}f_i\Bigr ),$$
where $(\lambda_{ij})$ is an invertible $n\times n$-matrix with coefficients in $\scr{O}_x$. Now, if $(f_1,\dots ,f_n)\in \mathfrak{m}^{\oplus n}_x$, then
$$\Bigl (\sum_i\lambda_{ki}f_i\Bigr )(x)=\sum_i\lambda_{ki}(x)f_i(x)=0$$
for each $k$. Thus, $\alpha_x(\mathfrak{m}^{\oplus n}_x)\subset \mathfrak{m}^{\oplus n}_x$, and the result follows.
\end{proof}

\begin{cor}
Let $\phi ,\psi:\scr{M}|_U\cong \scr{O}^n$ be two local trivilizations for $\scr{M}$. Then, for each $x\in U$,
$$\phi_x^{-1}(\mathfrak{m}^{\oplus n}_x)=\psi_x^{-1}(\mathfrak{m}^{\oplus n}_x).$$
\end{cor}

Denoting again by $\mathfrak{m}^{\oplus n}_x$ the preimage of $\mathfrak{m}^{\oplus n}_x$ by any trivialization, we can thus define an evaluation map
$$\scr{M}(U)\longrightarrow \scr{M}_x\longrightarrow \scr{M}_x/\mathfrak{m}_x^{\oplus n},$$
which we denote by $\sigma \mapsto \sigma (x)$.

These facts suggest the following
\begin{defi}
The quotient $\scr{M}_x/\mathfrak{m}_x^{\oplus n}$ is called the \emph{fibre of $\scr{M}$ over $x\in M$} and will be denoted by $F_x(\scr{M})$.
\end{defi}

In particular, note that the fibre $F_x(\scr{M})$ is a vector space over the field $\scr{O}_x/\mathfrak{m}_x\cong \comp$. If $\eta :\scr{M}\to \scr{N}$ is a linear homomorphism, then we have an induced map $\overline{\eta}_x:F_x(\scr{M})\to F_x(\scr{N})$ which makes the following diagram
$$
\xymatrix{\scr{M}(U) \ar[r]^{\eta_U} \ar[d] & \scr{N}(U) \ar[d] \\
\scr{M}_x \ar[r]^{\eta_x} \ar[d] & \scr{N}_x \ar[d] \\
F_x(\scr{M}) \ar[r]^{\overline{\eta}_x} & F_x(\scr{N})}
$$
commutative, where the vertical maps are canonical projections.

Let us now recall a basic result (\cite{lang:_algebra}, Ch. {\sc xvi} $\S2$, proposition 2.7.):

\begin{proposition}\label{fibre}
Let $R$ be a commutative ring with 1, $\mathfrak{a}\subset R$ an ideal and $N$ an $R$-module. Then, there exists an isomorphism
\begin{equation}\label{fiber_iso}
R/\mathfrak{a}\otimes_R N\stackrel{\cong}{\longrightarrow}N/\mathfrak{a}N.
\end{equation}
\end{proposition}

Putting $R=\scr{O}_x$, $\mathfrak{a}=\mathfrak{m}_x$ and $N=\scr{M}_x$ we have
$$F_x(\scr{M}) \cong \scr{M}_x\otimes_{\scr{O}_x}\comp .$$

Note that if $\eta :\scr{M}\to \scr{N}$ is a morphism, we also have an induced $\comp$-linear mapping $\widetilde{\eta}_x:F_x(\scr{M})\to  F_x(\scr{N})$, defined in the obvious way.



%%%%%%%%%%%%%%%%%%%%%%%%%%%%%%%%%%%%
\subsubsection{Idempotent Morphisms}

Let $\eta:\scr{M}\to \scr{M}$ be an endomorphism of the $\scr{O}_M$-module $\scr{M}$, and assume that $\eta^2=\eta$. As for vector spaces, in the category of presheaves the following isomorphism
\begin{equation}\label{decomp_idempotent}
\scr{M}\cong \opnm{Ker}\eta \oplus I_{\eta},
\end{equation}
holds, where $I_{\eta}$ is the presheaf $U\mapsto \opnm{Im}\eta_U$, and its proof is completely analogous to the case for vector spaces. If $1_{\scr{M}}$ denotes the identity map of $\scr{M}$, the morphism $1_{\scr{M}}-\eta$ is also idempotent, and $I_{1_{\scr{M}}-\eta}=\opnm{Ker}\eta$. This proves that for an idempotent linear map $\eta$, the presheaf $I_{\eta}$ is in fact equal to the image sheaf $\opnm{Im}\eta$.

Furthermore, the decomposition \eqref{decomp_idempotent} makes sense in the category of locally free modules; i.e. the kernel $\opnm{Ker}\eta$ (and thus also the image $\opnm{Im}\eta$) is also a locally free $\scr{O}_M$-module.





%%%%%%%%%%%%%%%%%%%%%%%%%%%%%
\subsection{Ringed Spaces}

A ringed space (over a ring $R$) is a topological space $M$ together with a sheaf of $R$-algebras over $M$. The idea is that this sheaf encodes all the geometric features of $M$, as it contains all admissible maps $U\to R$, for $U\subset M$ open. Moreover, the definition of ringed space allows a meaningful definition of tangent spaces in situations in which the usual definitions do not make sense.

\begin{defi}
Let $R$ be a ring. A \emph{ringed space} is a pair $(M,\scr{O}_M)$, where $M$ is a topological space and $\scr{O}_M$ is a sheaf of $R$-algebras, called the \emph{structure sheaf}. The space $(M,\scr{O}_M)$ is called a \emph{locally ringed space} if in addition to be a ringed space, each stalk $\scr{O}_{M,x}$ is a local ring.
\end{defi}

We shall usually write $M$ instead of $(M,\scr{O}_M)$, as the structure sheaf will be always clear from the context.

Locally ringed spaces are also called \emph{geometric spaces}, as all the usually encountered geometric structures lead to a structure sheaf with local stalks.

For instance, assume that $M$ is a topological manifold with a smooth structure (as usual, given by an atlas). Let $R=\re$ and $\scr{O}_{M}=C^\infty$ be the sheaf of real-valued smooth maps $U\mapsto C^\infty (U)$. This sheaf tells us precisely which maps on (open subsets of) $M$ are differentiable; and, in particular, we can recover the differentiable structure given by the atlas. So, it is completely equivalent to define the smooth structure by means of this sheaf. Another examples include analytic and complex manifolds, squemes and many others.\footnote{To be more accurate, schemes are constructed by gluing together pieces of ringed spaces.} All these examples are cases of locally ringed spaces.

\begin{defi}
A \emph{morphism} $(f,\overline{f}):(M,\scr{O}_M)\to (N,\scr{O}_N)$ \emph{of ringed spaces} consists of
\begin{enumerate}
\item A continuous map $f:M\to N$ and
\item a morphism $\overline{f}:\scr{O}_N\to f_*\scr{O}_M$ of sheaves or $R$-algebras over $N$.
\end{enumerate}
\end{defi}

An isomorphism can be described in the following way: the map $F$ is an isomorphism if and only if $f$ is a homeomorphism and $\overline{f}$ is an isomorphism of sheaves of $R$-algebras.

A morphism of locally ringed spaces is a morphism of ringed spaces such that the stalk map $\overline{f}_x:\scr{O}_{N,f(x)}\to \scr{O}_{M,x}$ is a local map of rings; i.e. $\overline{f}_x(\mathfrak{m}_{f(x)})\subset \mathfrak{m}_x$ for each $x\in M$.

The definition of ringed space, though extremely general, lets us construct tangent spaces in the following way: assume that $\scr{O}_M$ is a sheaf of $\comp$-algebras, and take some $x\in M$. Consider the ideal $\mathfrak{m}_x^2\subset \mathfrak{m}_x$. We then have the following result (for proofs the reader is adviced to consult \cite{kn:warner}).

\begin{lemma}
The quotient $\mathfrak{m}_x/\mathfrak{m}_x^2$ is a vector space of dimension $n=\dim M$.
\end{lemma}

We then define
$$T_xM:=\left (\mathfrak{m}_x/\mathfrak{m}_x^2\right )^*.$$

\begin{obs}
Unless otherwise stated, from now on we will only consider ringed spaces $(M,\scr{O}_M)$ over $R=\comp$, where $M$ is connected and:
\begin{enumerate}
\item $M$ is a smooth manifold and $\scr{O}_M$ is the sheaf of complex-valued smooth maps or
\item $M$ is a complex manifold and $\scr{O}_M$ is the sheaf of holomorphic maps.
\end{enumerate}
In particular, the stalks of the structure sheaves of these ringed spaces are local rings. The words ``map'', ``correspondence'', etc between structures involving these ringed spaces will of course be smooth or holomorphic, according to the case considered. When the base space $M$ is clear, we will use the notation $\scr{O}_x$ instead of $\scr{O}_{M,x}$. Moreover, the restriction $\scr{O}_M|_U$ to an open subset $U\subset M$ shall be denoted $\scr{O}_U$ and $\scr{O}_M(U)$ by $\scr{O}(U)$.
\end{obs}

Ringed spaces provide the adequate setting for the constructions of the direct and inverse image modules; to describe them, let $f:(M,\scr{O}_M)\to (N,\scr{O}_N)$ be a morphism of ringed spaces. We then have $f:M\to N$ and $\overline{f}:\scr{O}_N\to f_*\scr{O}_M$, which induces a structure of $\scr{O}_N$-module on $f_*\scr{M}$, which is called the \emph{direct image module}. Moreover, $f_*$ defines a functor from the category of $\scr{O}_M$-modules to the category of $\scr{O}_N$-modules
$$f_*:\tsf{Mod}_{\scr{O}_M}\longrightarrow \tsf{Mod}_{\scr{O}_N}.$$
Consider now the adjunction \ref{adjunction}; having the map $\overline{f}$ is equivalent to having a morphism $f^{-1}\scr{O}_N\to \scr{O}_M$, which is also a morphism of sheaves of $\comp$-algebras. This map makes $\scr{O}_M$ an $f^{-1}\scr{O}_N$-module. If $\scr{N}$ is an $\scr{O}_N$-module, the \emph{inverse image module} $f^*\scr{N}$ is the $\scr{O}_M$-module defined by
$$f^*\scr{N}=\scr{O}_M\otimes_{f^{-1}\scr{O}_N}f^{-1}\scr{N}.$$
As for the direct image, the inverse image defines a functor
$$f^*:\tsf{Mod}_{\scr{O}_N}\longrightarrow \tsf{Mod}_{\scr{O}_M}.$$
Moreover, the adjunction \ref{adjunction} holds for $f_*$ and $f^*$.


%%%%%%%%%%%%%%%%%%%%%%%%%%%%%%%%%%%%%%%%%%
\subsubsection{Sections of Vector Bundles}

\begin{defi}
Given a vector bundle $E$ over $M$, a \emph{section} of $E$ is a map $X:M\to E$ such that $X(x)\in E_x$ for each $x\in M$.
\end{defi}

Sections defined on open subsets $U\subset M$ (respectively on the whole space $M$) are usually called \emph{local} (respectively \emph{global}) sections. By the linear structure of the fibres, we can add sections and multiply them with maps $U\to \comp$ to obtain new ones. We then have that the set of sections over $U\subset M$ of $E$, which we denote by $\Gamma_E(U)$, is a module over the algebra $\scr{O}(U)$. Global sections will be denoted by $\Gamma (E)$ instead of $\Gamma_E(M)$.

\begin{theorem}\label{loc_free_sections}
The assignment $U\mapsto \Gamma_E(U)$ is a locally-free sheaf of $\scr{O}_M$-modules.
\end{theorem}
\begin{proof}
Operations are defined in the usual way: given sections $X$ and $Y$ over the same open subset $U$ and a map $\lambda :U\to \comp$, then the sections $X+Y$ and $\lambda X$ are given by the assignments $x\mapsto X(x)+Y(x)$ and $x\mapsto \lambda (x)X(x)$ respectively. All remaining verifications are standard computations.

Let now $U$ be an open subset of $M$ and $h:E|_U\to U\times \comp^n$ a local trivialization. Let $X\in \Gamma_E(U)$ be a local section and consider the following chain of maps
$$U\stackrel{X}{\longrightarrow}E|_U\stackrel{h}{\longrightarrow}U\times \comp^n\stackrel{\pi_2}{\longrightarrow}\comp^n,$$
where $\pr_2$ is the projection of the second coordinate. Then, the correspondence $X\mapsto \pi_2hX$ provides the desired isomorphism $\Gamma_E|_U\cong \scr{O}_U^n$.
\end{proof}

Conversely, we have the following

\begin{theorem}\label{sheaf_bundle}
If $\scr{M}$ be a locally-free $\scr{O}_M$-module of rank $n$, there exists a unique (up to isomorphism) vector bundle $E$ over $M$ of rank $n$ such that $\Gamma_E\cong \scr{M}$.
\end{theorem}
\begin{proof}
The idea is to construct a cocycle from the local triviality of $\scr{M}$. So let $\mathfrak{U}=\{U_i\}$ be an open cover of $M$ such that $\phi_i:\scr{M}|_{U_i}\cong \scr{O}_{U_i}^n$ is an $\scr{O}_{U_i}$-linear isomorphism of modules for each index $i$. Over $U_{ij}$ we then have a composite map
$$\scr{O}(U_{ij})^n\stackrel{\phi_j^{-1}}{\longrightarrow}\scr{M}(U_{ij})\stackrel{\phi_i}{\longrightarrow}\scr{O}(U_{ij})^n$$
which is a linear isomorphism. Thus, $\phi_i\phi_j^{-1}$ can be regarded as an invertible matrix in $\operatorname{M}_n(\scr{O}(U_{ij}))$. Putting
$$g_{ij}:=\phi_i\phi_j^{-1}$$
we obtain a family $\{g_{ij}\}$ which is a cocycle. Let $\{f_{ij}\}$ be another cocycle obtained from different isomorphisms $\psi_i:\scr{O}(U_{ij})^n\to \scr{O}(U_{ij})^n$, and consider the maps
$$g_i:=\psi_i\phi_i^{-1}:\scr{O}(U_i)^n\stackrel{\cong}{\longrightarrow}\scr{O}(U_i)^n.$$
Then we have that
$$
\begin{aligned}
g_ig_{ij}g_j^{-1} &= (\psi_i\phi_i^{-1})(\phi_i\phi_j^{-1})(\phi_j\psi_j^{-1}) \\
                  &= \psi_i\psi_j^{-1} = f_{ij}, \\
\end{aligned}
$$
and thus, by \ref{cocycles_iso}, the bundles defined by $\{g_{ij}\}$ and $\{f_{ij}\}$ are isomorphic. Let us denote by $E$ the bundle constructed from $\{g_{ij}\}$ and the cover $\mathfrak{U}$.

It only remains to check that $\Gamma_E\cong \scr{M}$. Consider the sheaf homomorphism $\eta :\scr{M}\to \Gamma_E$ defined in the following way: given a section $\sigma \in \scr{M}(U)$, we define $\eta (\sigma ):U\to E$ by the following rule
$$\eta (\sigma )(x)=[i,x,\sigma_i (x)],$$
where $x\in U\cap U_i$ and $\sigma_i(x)$ is the image of $\sigma$ through the following chain of maps (to ease the notation, we use the symbol $\phi_i$ also for the induced map $\phi_{i,x}$ on stalks):
$$\scr{M}(U)\longrightarrow \scr{M}_x\stackrel{\phi_i}{\longrightarrow}\scr{O}^n_x\longrightarrow \scr{O}^n_x/\mathfrak{m}^{\oplus n}_x\stackrel{\cong}{\longrightarrow}\comp^n.$$
Note that we need to pass through $\scr{O}_x^n$ as the isomorphisms $\scr{M}_x/\mathfrak{m}^{\oplus n}_x\cong \comp^n$ depend on the trivialization. We will first check that this map is well-defined.

Pick a point $x\in U_j$; then, we must verify that $[i,x,\sigma_i(x)]=[j,x,\sigma_j(x)]$, where $\sigma_j(x)$ is defined in the same fashion as $\sigma_i(x)$ but using $\phi_j$ instead of $\phi_i$. Assume that
$$
\begin{aligned}
\phi_i(\sigma_x) &:= (f^1_x,\dots ,f^n_x) \\
\phi_j(\sigma_x) &:= (g^1_x,\dots ,g^n_x). \\
\end{aligned}
$$
Then, $(f^k_x)=\phi_i\phi_j^{-1}(g^k_x)$. The rest now follows from the definition of the equivalence relation defined in the proof of theorem \ref{construct_bundles}.

By \ref{stalk_iso}, $\eta$ is an isomorphism if and only if $\eta_x:\scr{M}_x\to \Gamma_{E,x}$ is an isomorphism of $\scr{O}_x$-modules for each $x\in M$. Linearity is clear by definition of $\eta$. Assume now that $\eta_x (\sigma_x)=0$; this implies that the equality $\eta (\sigma )=0$ holds in a neighborhood of $x$, i.e. $[i,y,\sigma_i(y)]=0$ for $y$ sufficiently close (or equal) to $x$. The fibre $E_x$ is $\{[i,x,z]\; | \; z\in \comp^n\}$, and thus we have $\sigma_i(y)=0$ for each $y$. As $\phi_i$ is an isomorphism, this implies that $\sigma_y=0$; in particular, $\sigma_x=0$.

On the other hand, $\Gamma_E$ is locally-free (of rank $n$) by \ref{loc_free_sections}, and $\Gamma_{E,x}\cong \scr{O}_x^n$. Then, the map $\eta_x$ is necessarily an isomorphism.\footnote{If $R$ is a ring and $f:R^n\to R^n$ is an injective $R$-linear map, then it is also surjective.} This finishes the proof.
\end{proof}

Combining \ref{loc_free_sections} and \ref{sheaf_bundle} we can conclude that the functorial assignment
$$E\mapsto \Gamma_E$$
defines an equivalence between the category of finite-rank vector bundles over $M$ and the category of locally free $\scr{O}_M$-modules.

For compact manifolds and global sections, the previous result is precisely the Serre-Swan theorem (Serre proved this result for affine varieties and Swan for compact manifolds); it states that every module over the ring $C^\infty (M)$ of smooth functions on $M$ can be regarded as the (finitely generated and projective) module of sections $\Gamma (E)$ of some vector bundle $E$. This result was generalized in \cite{good:_cancellation} to include paracompact manifolds and later on to any base manifold in \cite{vas:_vbproj}, with the imposed condition that the bundles are of \emph{finite type}.\footnote{A vector bundle over a manifold $M$ is said to be of \emph{finite type} if
\begin{enumerate}[(a)]
\item There exists a finite set $\{f_1,\dots ,f_k\}$ of nonnegative maps $f_i:M\to \re$ with $\sum_if_i=1$ and
\item if $U_i:=\{x\; | \; f_i(x)\neq 0\}$, $E|_{U_i}$ is trivial.
\end{enumerate}}

The previous results tell us that every bundle can be recovered (uniquely, up to isomorphism) from its sheaf of sections, and conversely. We will now translate into the languaje of sections some important facts about bundles.

First, assume that $E$ is a vector bundle over $M$ of rank $n$ isomorphic to the trivial bundle $M\times \comp^n$. Let $\phi :E\to M\times \comp^n$ be an isomorphism. If $X:U\subset M\to E$ is a (local) section defined on an open subset $U$, then $\phi X$ is a section of the trivial vector bundle. Thus, for $x\in U$, $(\phi X)(x)=\phi (X(x))$ has the form $(x,\phi_X(x))$, where $\phi_X$ is a map $U\to \comp^n$. From this fact it can be deduced that a vector bundle of rank $n$ is trivializable if and only its sheaf of sections is free of rank $n$
$$\Gamma_E\cong \scr{O}^n_M.$$
Let now $E$ be a vector bundle over $M$ of rank $n$ and assume that $h:E|_U\to U\times \comp^n$ is a local trivialization. Define sections $X_i:U\to E$ ($i=1,\dots ,n$) by
$$X_i(x)=h^{-1}(x,e_i),$$
where $e_i$ is the vector which $i$-th component is equal to one and all the others to zero. Let $h_x$ be the restriction of $h$ to the fibre $E_x$; then $h_x$ is a linear isomorphism $E_x\to \comp^n$. As $h_x(X_i(x))=e_i$, then the set of sections $\{X_1,\dots ,X_n\}$ is \emph{linearly independent}; that is, for each $x\in U$, $\{X_1(x),\dots ,X_n(x)\}$ is linearly independent in $E_x$. And conversely, given a set $\{X_1,\dots ,X_n\}$ of linearly independent sections  over $U$, let $X\in E_x$ be an arbitrary vector. We can then write it as a unique linear combination $X=\sum_{i=1}^n\alpha_iX_i(x)$ and thus the map $h:E|_U\to U\times \comp^n$ given by
$$h(X):=(\pi (X),(\alpha_1,\dots ,\alpha_n))$$
is a local trivialization, where $\pi :E\to M$ is the bundle projection. We thus have the following result, which expresses the (local) triviality of a bundle by means of its sections.

\begin{proposition}
A rank-$n$ vector bundle $E$ is trivializable over some open subset $U\subset M$ if and only if there exists a set $\{X_1,\dots ,X_n\}$ of linearly independent sections over $U$.
\end{proposition}



%%%%%%%%%%%%%%%%%%%%%%%%%%%%%%%%%%%%%%%%%%%%%%%%%%%%%
%%%%%%%%%%%%%%%%%%%%%%%%%%%%%%%%%%%%%%%%%%%%%%%%%%%%%
\section{Azumaya Algebras and Twisted Vector Bundles}

In this section we will introduce some basic material regarding Azumaya algebras, as well as an introduction to twisted vector bundles. The former are strongly related to the latter, and this relationship will also appear later in chapter \ref{local_description}. The treatment of twisted bundles is mainly based at \cite{karoubi:twisted_vector}.


%%%%%%%%%%%%%%%%%%%%%%%%%%%%%
\subsection{Azumaya Algebras}
\label{subsec_azumaya}

If $\field$ is a field (which we assume to have characteristic equal
to zero), a\emph{central simple algebra} over $\field$ is a simple
(associative) algebra with center equal to $\field$. Replacing
$\field$ with a commutative local ring $R$ leads to the notion of
\emph{Azumaya algebra}; that is, an associative $R$-algebra $A$ is an
Azumaya algebra if there exists some $k\in \natu$ such
that $A\cong R^k$ as $R$-modules (i.e. it is free of finite rank) and
also the algebra homomorphism $\varphi :A\otimes_RA^{\circ }\to
\opnm{End}_R(A)\cong \opnm{M}_k(A)$ given by
$$\varphi (x\otimes y)(z)=xyz$$
is an isomorphism, where $A^{\circ}$ is the algebra with underlying set $A$ and operation given by $x\cdot y=yx$ (the right hand side is multiplication in $A$).\footnote{The algebra $A\otimes_R A^{\circ}$ is called the \emph{enveloping algebra of $A$}.} Auslander and Goldman \cite{auslander_goldman} generalized this definition to include any commutative (not necessarily local) base ring.

Behind these central simple and Azumaya algebras lies the notion of Brauer group (of the base ring), which Grothendieck \cite{grothendieck68:_le_group_de_brauer_i} generalized to define the Brauer group of a topological space $M$, by introducing the notion of Azumaya algebra over $M$.


\begin{defi}
A vector bundle $E$ over $M$ is called an \emph{Azumaya bundle} if
\begin{enumerate}
\item For each $x\in M$, the fibre $E_x$ is a $\comp$-algebra and
\item there exists a trivializing open cover $\mathfrak{U}$ of $A$ and an integer $k\geqslant 1$ such that the trivialization
$$E|_U\cong U\times \operatorname{M}_k(\comp )$$
is an isomorphism of bundles of $\comp$-algebras over $U$, for each $U\in \mathfrak{U}$. 
\end{enumerate}
\end{defi}

The definition of Azumaya bundles can also be done in terms of sheaves of sections. This was the original approach of Grothendieck.

\begin{defi}
An \emph{Azumaya algebra} over $(M,\scr{O}_M)$ is a sheaf of $\scr{O}_M$-algebras locally isomorphic to the sheaf $\operatorname{M}_k(\scr{O}_M)$.
\end{defi}

\begin{obs}\label{algebra_fibres}
By proposition 2.1 (b) of \cite{milne80:_etale_cohom} (see also section 1 of \cite{grothendieck68:_le_group_de_brauer_i}), an Azumaya algebra over $(M,\scr{O}_M)$  is a locally free sheaf of algebras such that its fibres are isomorphic to $\operatorname{M}_k(\comp)$.
\end{obs}

If $E$ is an Azumaya bundle over $M$, then its sheaf of sections $\Gamma_E$ inherits the algebra structure: if $X,Y$ are sections of $E$, then $XY$ is the section given by
$$XY(x)=X(x)Y(x)\in A_x.$$
Thus, $\Gamma_E$ is a sheaf of $\scr{O}_M$-algebras. By theorem \ref{loc_free_sections}, we have that $\Gamma_E$ is in fact locally isomorphic to the sheaf $\operatorname{M}_k(\scr{O}_M)$. The converse also holds by \ref{sheaf_bundle}.

If $\mathfrak{U}=\{U_{i}\}$ trivializes the Azumaya bundle $E$, a cocycle for $E$ over this open cover is given by maps $g_{ij}:U_{ij}\to \operatorname{Aut}(\operatorname{M}_k(\comp ))$ with values in the group of algebra automorphisms $\opnm{M}_k(\comp )\to \opnm{M}_k(\comp )$. The following theorem will be extremely useful for the discussion (for more details the reader may consult \cite{kn:lorenz_2}).

\begin{theorem}[Skolem-Noether Theorem]
Let $A$ be a central simple algebra over the field $\field$. If $\varphi :A\to A$ is an algebra isomorphism, then there exists an invertible element $x\in A$ such that $\varphi (y)=xyx^{-1}$.
\end{theorem}

As $\opnm{M}_k(\comp )$ is a central simple algebra, any automorphism $\varphi :\opnm{M}_k(\comp )\to \opnm{M}_k(\comp )$ is of the form $\varphi (B)=ABA^{-1}$ for some invertible matrix $A$. Moreover, the matrix $\lambda A$ defines the same automorphism for each $\lambda \in \comp^{\times }$. Thus
$$\operatorname{Aut}(\operatorname{M}_k(\comp ))\cong \operatorname{GL}_k(\comp )  /\comp^{\times}=:\operatorname{PGL}_k(\comp ),$$
and thus the structure group of any Azumaya algebra can be taken to the projective general linear group $\opnm{PGL}_k(\comp )$.


%%%%%%%%%%%%%%%%%%%%%%%%%%%%%%%%%%%
\subsection{Twisted Vector Bundles}

As vector bundles model cocycles in topological K-theory, twisted vector bundles represent a geometric model for twisted K-theory. The main interest for these type of bundles arose in string theory. In physics one usually needs to consider a space-time manifold $M$ together with a \emph{B-field}; these fields are precisely what is needed to define a twisting for the K-theory of $M$, and thus leads naturally to consideration of twisted cocycles. Another reason of interest in twisted K-theory is given by the Freed-Hopkins-Teleman theorem: the Verlinde ring of projective representations of the loop group of a compact Lie group $G$ can be represented as the twisted (equivariant) K-group of $G$. For more on this, the reader may consult \cite{atiyah-segal:twisted_k}.

The following is mainly based on Karoubi's article \cite{karoubi:twisted_vector}.

\begin{defi}
A \emph{twisted vector bundle} $\mathbb{E}$ over $M$ is a tuple
$$\mathbb{E}=(\mathfrak{U},U_i\times V,g_{ij},\lambda_{ijk})$$
consisting of the following data:
\begin{enumerate}
\item An open cover $\mathfrak{U}=\{U_i\}$ of $M$.
\item A (trivial) vector bundle $U_i\times V$ over each $U_i\in \mathfrak{U}$, where $V$ is a finite dimensional complex vector space (which shall usually be taken to be complex $n$-space).
\item Two families of maps $g_{ij}:U_{ij}\to \operatorname{GL}(V)$ and $\lambda_{ijk}\in \scr{O}(U_{ijk})$ such that $\lambda :=(\lambda_{ijk})$ is a $\check{\text{C}}$ech 2-cocycle, each map $\lambda_{ijk}$ takes values in $\comp^\times$ and
$$g_{ii}=1 \quad ,\quad g_{ji}=g_{ij}^{-1} \quad , \quad g_{ij}g_{jk}=\lambda_{ijk}g_{ijk}$$
over $U_{ijk}$ (Recall that $(\lambda_{ijk})$ is a $\check{\text{C}}$ech 2-cocycle if $\lambda_{jkl}\lambda_{ikl}^{-1}\lambda_{ijl}\lambda_{ijk}^{-1}=1$).
\end{enumerate}
\end{defi}

\begin{obs}
The cocycle $\lambda =(\lambda_{ijk})$ is in fact a \emph{completely normalized cocycle}; that is: $\lambda =1$ if two of the 3 indices $i,j,k$ are equal and, if $\sigma$ is a permutation of the indices $i,j,k$, then $\lambda_{\sigma (i)\sigma (j) \sigma (k)}=\lambda_{ijk}^{\opnm{sg}\sigma}$, where $\opnm{sg}\sigma$ is the sign of the permutation $\sigma$. Moreover, any $\check{\text{C}}$ech cocycle is equivalent to a completely normalized one. See  \cite{karoubi:twisted_vector} and the reference therein.
\end{obs}

If we want to emphasize the twisting $\lambda =(\lambda_{ijk})$, such a vector bundle will be also called a \emph{$\lambda$-twisted vector bundle}.

Let $\mathbb{E}=(\mathfrak{U},U_i\times V,g_{ij},\lambda_{ijk})$ and $\mathbb{F}=(\mathfrak{V},V_r\times V,f_{rs},\mu_{rst})$ be two twisted bundles. The question now is in what cases these two objects can be regarded as equal.

\begin{defi}
The twisted bundles $\mathbb{E}$ and $\mathbb{F}$ are \emph{equal} if there exists a refinement $\mathfrak{W}$ of $\mathfrak{U}$ and $\mathfrak{V}$ such that the cocycles of $\mathbb{E}$ and $\mathbb{F}$ coincide over elements of $\mathfrak{W}$.
\end{defi}

\begin{obs}
From now on, we will assume that the base space $M$ admits good covers (as, for instance, any manifold does) and that $\mathfrak{U}$ is indeed one of those covers.
\end{obs}

The proof of the following result is outlined in \cite{karoubi:twisted_vector}.

\begin{proposition}\label{tvb_torsion}
If $\mathbb{E}=(\mathfrak{U},U_i\times V,g_{ij},\lambda_{ijk})$ is a twisted vector bundle, then $\lambda$ is contained in the torsion subgroup of $\opnm{H}^3(M;\ent )$.
\end{proposition}

As for ordinary vector bundles, we can construct new twisted bundles from given ones. Consider then two twisted bundles $\mathbb{E}=(\mathfrak{U},U_i\times V,g_{ij},\lambda_{ijk})$ and $\mathbb{F}=(\mathfrak{U},U_i\times W,f_{ij},\mu_{ijk})$.

\begin{enumerate}
\item If $f:N\to M$ is a map, the pullback twisted bundle
\begin{equation}\label{tvb_pullback}
f^*\mathbb{E}=(\mathfrak{U}',U'_i\times V,g'_{ij},\lambda'_{ijk})
\end{equation}
is a $\lambda'$-twisted vector bundle with $\mathfrak{U}'=\{U'_i\}$, $U'_i=f^{-1}(U_i)$, $g'_{ij}=g_{ij}f$ and $\lambda'_{ijk}=\lambda_{ijk}f$.

\item Assume that $\lambda_{ijk}=\mu_{ijk}$ for each admissible $i,j$ and $k$. If $h_{ijk}=\begin{pmatrix} g_{ij} & 0 \\ 0 & f_{ij} \end{pmatrix}$, then $h_{ij}h_{jk}=\lambda_{ijk}h_{ik}$ and thus the direct sum $\mathbb{E}\oplus \mathbb{F}$ can be defined as the twisted bundle
$$\mathbb{E}\oplus \mathbb{F}=(\mathfrak{U},U_i\times V,h_{ij},\lambda_{ijk}).$$

\item The dual twisted bundle $\mathbb{E}^*$ is the twisted vector bundle given by
$$\mathbb{E}^*=(\mathfrak{U},U_i\times V^*,g^*_{ij},\lambda_{ijk}^{-1}),$$
where $g^*_{ij}:U_{ij}\to \operatorname{GL}(V^*)$ is given by $g^*_{ij}(x)(u)=u(g_{ij}(x))$.

\item The tensor product $\mathbb{E}\otimes \mathbb{F}$ is the twisted bundle
$$\mathbb{E}\otimes \mathbb{F}=(\mathfrak{U},U_i\times (V\otimes W),g_{ij}\otimes f_{ij},\lambda_{ijk}\mu_{ijk})$$
with cocycles $g_{ij}\otimes f_{ij}:U_{ij}\to \opnm{GL}(V\otimes W)$.

\item Of particular interest is the twisted vector bundle $\operatorname{Hom}(\mathbb{E},\mathbb{F})$, which is defined by
$$\operatorname{Hom}(\mathbb{E},\mathbb{F})=(\mathfrak{U},U_i\times \operatorname{Hom}_\comp (V,W),h_{ij},\lambda_{ijk}^{-1}\mu_{ijk}),$$
where $h_{ij}:U_{ij}\to \operatorname{GL}(\operatorname{Hom}_{\comp }(V,W))$ is given by $h_{ij}(x)(u)=f_{ij}(x)ug_{ij}(x)^{-1}$. If $\mathbb{F}$ is also a $\lambda$-twisted bundle (i.e. $\mu =\lambda$), then the data defining $\operatorname{Hom}(\mathbb{E},\mathbb{F})$ in fact defines an ordinary vector bundle (there is no twisting!), which is denoted by $\operatorname{HOM}(\mathbb{E},\mathbb{F})$. If $\mathbb{E}=\mathbb{F}$, then $\operatorname{HOM}(\mathbb{E},\mathbb{F})$ will be denoted $\operatorname{END}(\mathbb{E})$.
\end{enumerate}

\begin{obs}
Note that all the twistings for these new twisted bundles are also completely normalized 2-cocycles. 
\end{obs}

\begin{defi}
Let $\mathbb{E}=(\mathfrak{U},U_i\times V,g_{ij},\lambda_{ijk})$ and $\mathbb{F}=(\mathfrak{U},U_i\times W,f_{ij},\mu_{ijk})$ be twisted vector bundles over $M$. A \emph{morphism} $\phi :\mathbb{E}\to \mathbb{F}$ is a family of bundle morphisms
$$\phi_i:U_i\times V\longrightarrow U_i\times W$$
such that the following square
\begin{equation}\label{diag_tvbm}
\xymatrix{
U_{ij}\times V \ar[r]^{\phi_j} \ar[d]_{1\times g_{ij}} & U_{ij}\times W \ar[d]^{1\times f_{ij}} \\
U_{ij}\times V \ar[r]_{\phi_i} & U_{ij}\times W}
\end{equation}
commutes.
\end{defi}

Composition of two morphisms $\phi :\mathbb{E}\to \mathbb{F}$ and $\psi :\mathbb{F}\to \mathbb {G}$ is defined by composing the families $\{\phi_i\}$ and $\{\psi_i\}$. We will denote by $\tsf{TVB}(M)$ the category of twisted vector bundles over $M$. If $\lambda$ is a (fixed) twisting, we will adopt the notation $\tsf{TVB}_{\lambda}(M)$ for the category of $\lambda$-twisted vector bundles over $M$.

As usual, we will say that $\phi :\mathbb{E}\to \mathbb{F}$ is an \emph{isomorphism} if there exists another morphism $\psi :\mathbb{F}\to \mathbb{E}$ such that $\phi \psi$ and $\psi \phi$ are the respective identities; for a twisted bundle $\mathbb{E}$, its identity map is given by the family of identities $\text{id}:U_i\times V\to U_i\times V$. We denote $\psi$ by $\phi^{-1}$.

%\begin{lemma}
%The set of arrows $\operatorname{Hom}_{\tsf{TVB}(M)}(\mathbb{E},\mathbb{F})$ defines a twisted vector bundle isomorphic to $\operatorname{Hom}(\mathbb{E},\mathbb{F})$.
%\end{lemma}
%\begin{proof}
%For any morphism $\phi :\mathbb{E}\to \mathbb{F}$, we have the commutativity condition \eqref{diag_tvbm}
%$$f_{ij}\phi_j=\phi_ig_{ij},$$
%over $U_{ij}$, which implies that $\phi_i=f_{ij}\phi_jg_{ij}^{-1}$, which is precisely the expression for cocycles of the twisted bundle $\operatorname{Hom}(\mathbb{E},\mathbb{F})$. Let us denote this cocycles again by $h_{ij}$. Then we have
%$$
%\begin{aligned}
%h_{ij}(x)h_{jk}(x)(u) &= h_{ij}(x)\bigl (f_{jk}(x)ug_{jk}(x)^{-1}\bigr ) \\
%                      &= f_{ij}(x)f_{jk}(x)u\bigl (g_{ij}(x)g_{jk}(x)\bigr )^{-1} \\
%                      &= \lambda_{ijk}(x)^{-1}\mu_{ijk}(x)f_{ik}(x)ug_{ik}(x)^{-1} \\
%                      &= \lambda_{ijk}(x)^{-1}\mu_{ijk}(x)h_{ik}(x)(u).\\
%\end{aligned}
%$$
%\end{proof}

An inmediate consequence of the definition of morphism is the following

\begin{lemma}\label{isomorphic}
Two twisted bundles $\mathbb{E}=(\mathfrak{U},U_i\times V,g_{ij},\lambda_{ijk})$ and $\mathbb{F}=(\mathfrak{U},U_i\times W,f_{ij},\mu_{ijk})$. are isomorphic if and only if there exists a family of maps $\{u_i:U_i\to \operatorname{Iso}(V,W)\}$ such that
$$f_{ij}=u_ig_{ij}u_j^{-1}.$$
\end{lemma}
\begin{proof}
Assume first that $ \phi :\mathbb{E}\to \mathbb{F}$ is an isomorphism. Then, by definition of composition, it is clear that all the maps $\phi_i$ are isomorphisms. Then, take tha maps $u_i$ to be $u_i(x)=\phi_{i,x}:\{x\}\times V\to \{x\}\times W$.

Suppose now that we have a familiy of maps $\{u_i\}$. Define $\phi :\mathbb{E}\to \mathbb{F}$ to be the family consisting of the maps $\phi_i:U_i\times V\to U_i\times W$ given by
$$\phi_i(x,v)=u_i(x)(v).$$
Then, $\phi$ is a bundle isomorphism.
\end{proof}

As a corollary, we can deduce for twisted bundles the familiar isomorphism
$$\operatorname{Hom}(\mathbb{E},\mathbb{F})\cong \mathbb{E}^*\otimes \mathbb{F}.$$

\begin{lemma}\label{iso_twistings}
If $\mathbb{E}$ and $\mathbb{F}$ are isomorphic, then $\lambda =\mu$.
\end{lemma}
\begin{proof}
Let $\phi$ be an isomorphism; from \eqref{diag_tvbm} we can deduce the following commutative diagram
\begin{equation}\label{diag_tvb2}
\xymatrix{
U_{ijk}\times V \ar[r]^{\phi_k} \ar[d]_{1\times g_{jk}} & U_{ijk}\times W \ar[d]^{1\times f_{jk}} \\
U_{ijk}\times V \ar[r]^{\phi_j} \ar[d]_{1\times g_{ij}} & U_{ijk}\times W \ar[d]^{1\times f_{ij}} \\
U_{ijk}\times V \ar[r]^{\phi_i} & U_{ijk}\times W. }
\end{equation}
By definition, we have that the vertical compositions are equal to $1\times \lambda_{ijk}g_{ik}$ and $1\times \mu_{ijk}f_{ik}$, and thus
\begin{equation}\label{cocycles_isom}
\lambda_{ijk}(\phi_ig_{ik})=\mu_{ijk}(f_{ik}\phi_k).
\end{equation}
On the other hand, by lemma \ref{isomorphic}, we have that $f_{ij}=\phi_ig_{ij}\phi_j^{-1}$. Replacing this last relation in the right hand side of equation \eqref{cocycles_isom} yields
$$
\begin{aligned}
\mu_{ijk}(f_{ik}\phi_k) &= \mu_{ijk}((\phi_ig_{ik}\phi_k^{-1})\phi_k) \\
                        &= \mu_{ijk}\phi_ig_{ik}; \\
\end{aligned}
$$
comparison of this last equation with the left hand side of \eqref{cocycles_isom} finishes the proof.
\end{proof}

Operations on twisted bundles enjoy much of the properties of ordinary vector bundles. The proof of this fact, stated in the next result, can be obtained from a direct computation.

\begin{proposition}\label{ass_comm}
The operations $\oplus$ and $\otimes$ are associative, distributive and commutative, in the sense that we have natural isomorphisms
$$
\begin{aligned}
(\mathbb{E}\otimes \mathbb{F})\otimes \mathbb{G} &\cong \mathbb{E}\otimes (\mathbb{F}\otimes \mathbb{G}), \\
(\mathbb{E}\oplus \mathbb{F})\oplus \mathbb{G} &\cong \mathbb{E}\oplus (\mathbb{F}\oplus \mathbb{G}), \\
(\mathbb{E}\oplus \mathbb{F})\otimes \mathbb{G} &\cong (\mathbb{E}\otimes \mathbb{G})\oplus (\mathbb{F}\otimes \mathbb{G}),\\
\mathbb{E}\otimes \mathbb{F} &\cong \mathbb{F}\otimes \mathbb{E}, \\
\mathbb{E}\oplus \mathbb{F} &\cong \mathbb{F}\oplus \mathbb{E}. \\
\end{aligned}
$$
\end{proposition}

Further properties are given in the following

\begin{lemma}\label{prop_tvb}
Let $\mathbb{E}$ and $\mathbb{F}$ be twisted bundles. Then
\begin{enumerate}
\item If $\mathbb{E} \otimes \mathbb{F}\cong \mathbb{E}$, then $\mathbb{F}$ is an ordinary line bundle.
\item If $\mathbb{E}$ has twisting $\lambda$ and $\mathbb{F}$ has twisting $\lambda^{-1}$, then $\mathbb{E}\otimes \mathbb{F}$ is an ordinary vector bundle. In particular, $\mathbb{E}^*\otimes \mathbb{F}$ is also a vector bundle if $\mathbb{E}$ and $\mathbb{F}$ have the same twisting. Moreover, $\mathbb{L}\otimes \mathbb{L}^*$ is isomorphic to a trivial line bundle if $\mathbb{L}$ a twisted line bundle.
\item If $\mathbb{E}$ is defined over the trivial open cover $\mathfrak{U}=\{M\}$, then $\mathbb{E}$ is a trivial vector bundle, and conversely.
\end{enumerate}
\end{lemma}
\begin{proof}
To prove (1), let us assume that $\phi :\mathbb{E}\otimes \mathbb{F}\cong \mathbb{E}$ is an isomorphism, with $\mathbb{E}=(\mathfrak{U},U_i\times V,\{g_{ij}\},\{\lambda_{ijk}\})$ and $\mathbb{F}=(\mathfrak{U},U_i\times W,\{f_{ij}\},\{\mu_{ijk}\})$. Lemma \ref{iso_twistings} together with the definition of tensor product yields $\lambda_{ijk}\mu_{ijk}=\lambda_{ijk}$, and this obviously implies that $\mu_{ijk}=1$; in other words, $\mathbb{F}=L$ is an ordinary line bundle.

For (2), we note that, if $\{\lambda_{ijk}\}$ is the twisting for $\mathbb{E}$, then the twisting for $\mathbb{F}$ is $\{\lambda_{ijk}^{-1}\}$. Thus, by definition of tensor product of twisted bundles, the twisted bundle $\mathbb{E}\otimes \mathbb{F}$ has twisting given by $\{\lambda_{ijk}^{-1}\lambda_{ijk}=1\}$, and so it is an ordinary vector bundle. The assertion about $\mathbb{L}\otimes \mathbb{L}^*$ readily follows from the previous observation and the definition of tensor product.

The proof of (3) can be obtained immediately from the definition.
\end{proof}


%%%%%%%%%%%%%%%%%%%%%%%%%%%%%%%%%%%%%%%%%%%%%%%%%%%%%%%%%%%
\subsubsection{Relations With Bundles and Azumaya Algebras}

We have an obvious functor
$$\tsf{Vect}(M)\longrightarrow \tsf{TVB}(M)$$
which is fully-faithful. For a fixed twisting, we also have the following

\begin{proposition}\label{equiv_tvb_vect}
There exists an equivalence of categories $\tsf{TBV}_{\lambda}(M)\to \tsf{Vect}(M)$.
\end{proposition}
\begin{proof}
Let $\mathbb{L}$ be a fixed $\lambda$-twisted line bundle and consider the functors
$$
\xymatrix{
\tsf{TVB}_{\lambda}(M)\ar@<1ex>[r]^{\mathbb{L}^*} & \tsf{Vect}(M)\ar@<1ex>[l]^{\mathbb{L}}}
$$
given by $\mathbb{L}^*(\mathbb{E})=\mathbb{L}^*\otimes \mathbb{E}$ and $\mathbb{L}(E)=\mathbb{L}\otimes E$ (in the right hand side of this last equation, we are regarding $E$ as a twisted bundle with no twisting). By \ref{ass_comm} and \ref{prop_tvb}, we have isomorphisms
$$
\begin{aligned}
\mathbb{L}\mathbb{L}^*(\mathbb{E}) &= \mathbb{L}\otimes \mathbb{L}^*\otimes \mathbb{E}\cong \mathbb{E}, \\
\mathbb{L}^*\mathbb{L}(E) &= \mathbb{L}^*\otimes \mathbb{L}\otimes E \cong E.\\
\end{aligned}
$$
The verification of naturality of these isomorphisms is straightforward.
\end{proof}

Let now $A$ be an Azumaya algebra over $M$, locally isomorphic to $\opnm{M}_n(\comp )$. The projection
\begin{equation}\label{ppal}
\operatorname{GL}_n(\comp )\to \operatorname{PGL}_n(\comp )\cong \operatorname{Aut}(\operatorname{M}_n(\comp ))
\end{equation}
is a locally trivial principal $\comp^{\times}$-bundle; thus, on a suitable cover of $\operatorname{PGL}_n(\comp )$, this bundle is trivial, i.e. it has local sections. By shrinking the open subsets $U_i$ if necessary, we can assume that the cocycle maps for $A$, which now can be represented as maps $g_{ij}:U_{ij}\to \operatorname{PGL}_n(\comp )$, have their images contained in trivializing open subsets. Hence, composition with local sections of the bundle \eqref{ppal} provides maps
$$f_{ij}:U_{ij}\to \operatorname{GL}_n(\comp ).$$
This family of maps can be chosen so as to satisfy the equations $f_{ji}=f_{ij}^{-1}$ and $f_{ii}=1$. Moreover, we have the following

\begin{lemma}
There exists a family of maps $\lambda =\{\lambda_{ijk}\}$, with $\lambda_{ijk}:U_{ijk}\to \comp^{\times}$, such that
\begin{enumerate}
\item $f_{ij}f_{jk}=\lambda_{ijk}f_{ik}$ and
\item $\lambda$ is a completely normalized $\Check{\text{C}}$ech 2-cocycle.
\end{enumerate}
\end{lemma}

These data let us construct a twisted bundle $\mathbb{E}$ defining
$$\mathbb{E}=(\mathfrak{U},U_i\times \comp^n,\{f_{ij}\},\{\lambda_{ijk}\}).$$
Now, the twisted bundle $\operatorname{End}(\mathbb{E})$ is in fact a vector bundle $\operatorname{END}(\mathbb{E})$ with cocycle maps given by
$$h_{ij}(x)(u)=f_{ij}(x)uf_{ij}(x)^{-1},$$
i.e. $h_{ij}$ takes values in $\operatorname{PGL}_n(\comp )$ and there is no twisting. We can thus state the following relation between Azumaya algebras and twisted bundles.

\begin{theorem}[\cite{karoubi:twisted_vector}, Theorem 3.2]
\label{tvb_azumaya}
Assume $A$ is an Azumaya algebra over $M$. Then, there exists a twisted bundle $\mathbb{E}$ such that
$$A\cong \operatorname{END}(\mathbb{E}).$$
\end{theorem}

\begin{obs}
It is worth noting that as the liftings are not unique, the twisted bundle of the previous result is also not unique as well.
\end{obs}

Let now $\phi :\mathbb{E}\to \mathbb{F}$ be an isomorphism. Such a map let us define a map
$$\overline{\phi}:\operatorname{END}(\mathbb{E})\longrightarrow \operatorname{END}(\mathbb{F})$$
given by the family $\{\overline{\phi}_i:U_i\times \operatorname{End}_{\comp}(V)\to U_i\times \operatorname{End}_{\comp}(W)\}$, where
$$\overline{\phi}_i(A)=\phi_iA\phi_i^{-1}.$$
As $\operatorname{END}(\mathbb{E})$ and $\operatorname{END}(\mathbb{F})$ are ordinary bundles, we may ask whether the map $\overline{\phi}$ defines also a morphism of vector bundles.

\begin{proposition}\label{induced_morphism}
$\overline{\phi}$ is a (multiplicative) vector bundle morphism.
\end{proposition}
\begin{proof}
Let us first recall the construction of vector bundles from given cocycles given in the proof of \ref{construct_bundles}. For $\operatorname{END}(\mathbb{E})$, we have
$$\operatorname{END}(\mathbb{E})=\bigsqcup_iU_i\times \operatorname{End}_{\comp}(V)/\sim ,$$
where $(i,(x,f))\sim (j,(x',f'))$ if and only if $x=x'\in U_{ij}$ and $f'=h_{ij}(x)^{-1}(f)=g_{ij}(x)^{-1}fg_{ij}(x)$, where $g_{ij}$ are the cocycles for $\mathbb{E}$ and $h_{ij}$ the ones for $\operatorname{END}(\mathbb{E})$. Let us denote by $[i,x,f]$ the equivalence class of the pair $(i,(x,f))$. Local trivializations are then given by the assignment
$$[i,x,f]\longmapsto (x,f),$$
the map $\overline{\phi}=\{\overline{\phi}_i\}$ can then be described over $U_i$ by the equation
$$\overline{\phi}_i[i,x,f]=[i,x,\phi_i f\phi_i^{-1}].$$
Muliplicativity is clear. We need to show now that these maps coincide on the intersections $U_{ij}$.

Assume then that $x\in U_{ij}$. The element $[i,x,f]$ is represented over $U_j$ be the element $[j,x,g_{ij}(x)fg_{ij}(x)^{-1}]$. So we must verify that the equality
$$
\overline{\phi}_i[i,x,f]=\overline{\phi}_j[j,x,g_{ij}(x)fg_{ij}(x)^{-1}]
$$
holds. This equation is equivalent to
$$
[i,x,\phi_if\phi_i^{-1}]=[j,x,\phi_jg_{ij}(x)^{-1}fg_{ij}(x)\phi_j^{-1}].
$$
On the other hand, by definition of the equivalence relation, we have that this equality holds if and only if
\begin{equation}\label{eq_int}
\phi_jg_{ij}(x)^{-1}fg_{ij}(x)\phi_j^{-1} = f_{ij}(x)^{-1}\phi_if\phi_i^{-1}f_{ij}(x),
\end{equation}
where $f_{ij}$ are the cocycles for $\mathbb{F}$. As $\phi$ is a morphism, we have that $f_{ij}\phi_j=\phi_ig_{ij}$ or, equivalently, $\phi_j=f_{ij}^{-1}\phi_ig_{ij}$. Replacing this expression in the left hand side of \eqref{eq_int} finishes the proof.
\end{proof}

Let $\widehat{\tsf{TVB}}(M)$ denote the grupoid of twisted vector bundles over $M$ (that is, the only arrows we consider are the isomorphisms). We define a covariant functor
\begin{equation}\label{functor_tvb_az}
\widehat{\tsf{TVB}}(M)\longrightarrow \tsf{Az}(M)
\end{equation}
with values in the category of Azumaya algebras over $M$ in the following way: on objects, $\mathbb{E}\mapsto \operatorname{END}(\mathbb{E})$. Let now $\phi :\mathbb{E}\to \mathbb{F}$ be an isomorphism between twisted bundles, given by a family
$$\phi_i:U_i\times V\longrightarrow U_i\times W.$$
This family induces maps $\overline{\phi}_i:U_i\times \operatorname{End}(V)\to U_i\times \operatorname{End}(W)$ given by
$$\overline{\phi}_i(A)=\phi_iA\phi_i^{-1}.$$
By proposition \ref{induced_morphism}, $\phi$ induces a morphism of algebra bundles
$$\overline{\phi}:\operatorname{END}(\mathbb{E})\longrightarrow \operatorname{END}(\mathbb{F}).$$
Thus, we define $\phi \mapsto \overline{\phi}$.

Theorem \ref{tvb_azumaya} implies that this functor is essentially surjective.\footnote{Recall that a functor $F:{\bf X}\to {\bf Y}$ is \emph{essentially surjective} if given any object $Y\in {\bf Y}$ there exists an object $X\in {\bf X}$ such that $F(X)\cong Y$.} Consider now the map
$$
\operatorname{Hom}_{\widehat{\tsf{TVB}}(M)}(\mathbb{E},\mathbb{F})\longrightarrow \operatorname{Hom}_{\tsf{Az}(M)}(\operatorname{END}(\mathbb{E}),\operatorname{END}(\mathbb{F}))$$
$$\phi \longmapsto \overline{\phi}.$$
If $\phi ,\psi :\mathbb{E}\to \mathbb{F}$ are two isomorphisms such that $\overline{\phi}=\overline{\psi}$, then we have that for each $i$ and each endomorphism $A:V\to V$ the equality
$$\psi_i^{-1}\phi_iA=A\psi_i^{-1}\phi_i$$
must hold. This implies the existence of a family of maps $\{\lambda_i:U_i\to \comp^{\times}\}$ such that
$$\phi_i=\lambda_i\psi_i.$$
Thus, the map $\phi\mapsto \overline{\phi}$ is injective only after identifying $\phi$ and $\lambda \phi$.

\begin{lemma}
The family $\lambda \phi=\{\lambda_i\phi_i\}$ is a morphism if and only if $\lambda=\{\lambda_i\}$ is a 0-cocycle.
\end{lemma}
\begin{proof}
The family $\lambda \phi=\{\lambda_i\phi_i\}$ is a morphism if and only if
\begin{equation}\label{lambda_morphism}
\lambda_jf_{ij}\phi_j=\lambda_i\phi_ig_{ij}.
\end{equation}
As $\phi =\{\phi_i\}$ is a morphism, we have that $f_{ij}\phi_j=\phi_ig_{ij}$ and then equation \eqref{lambda_morphism} holds if and only if $\lambda_i=\lambda_j$ on $U_{ij}$.
\end{proof}

If $\tsf{TVB}^0(M)$ denotes the category whose objects are twisted vector bundles over $M$ and morphisms are classes of morphisms in $\widehat{\tsf{TVB}}(M)$ subject to the identification $\phi \sim \lambda \phi$ for $\lambda :M\to \comp^{\times}$, then the essentially surjective functor
$$\tsf{TVB}^0(M)\longrightarrow \tsf{Az}(M)$$
is also faithful.

Restricting to the category $\widehat{\tsf{Az}}(M)$ of Azumaya algebras with morphisms the isomorphisms, the functor
$$\tsf{TVB}^0(M)\longrightarrow \widehat{\tsf{Az}}(M)$$
is also full and then an equivalence of categories.


%%%%%%%%%%%%%%%%%%%%%%%%%%%%%%%%%%%%%%%%
\subsubsection{The Twisted Picard Group}

For the following discussion it will be useful to recall the definition of the Picard group of a manifold $M$; consider the set of isomorphism classes of (ordinary) line bundles over $M$. If $L,K$ are line bundles, then $[L]\cdot [K]:=[L\otimes K]$ provides the set of isomorphism classes of line bundles with a structure of abelian group. This group is called the \emph{Picard group of $M$} and is denoted by $\opnm{Pic}(M)$.

Analogously, twisted line bundles also enjoy some remarkable properties, like line bundles do. Given a twisted bundle $\mathbb{E}$, we shall denote by $[\mathbb{E}]$ its isomorphism class. Let us restrict ourselves to considering isomorphism classes of twisted line bundles over a manifold $M$. We define a product in the following way:
\begin{equation}\label{op_pic}
[\mathbb{L}]\cdot [\mathbb{K}]:=[\mathbb{L}\otimes \mathbb{K}],
\end{equation}
extending the one for line bundles.

\begin{theorem}
The set of isomorphism classes of twisted line bundles together with the operation \eqref{op_pic} is a $\opnm{Tor}\opnm{H}^3(M;\ent )$-graded abelian group which contains $\operatorname{Pic}(M)$ as a subgroup.
\end{theorem}
\begin{proof}
Associativity and commutativity of the operation follow from the ones of the tensor product, as stated in \ref{ass_comm}.

Let $\mathbb{L}$ be a twisted line bundle; if $\epsilon^1$ denotes the trivial line bundle over $M$, then $\mathbb{L}\otimes \epsilon^1\cong \mathbb{L}$; to see this, consider the family of maps
$$\phi_i:U_i\times (\comp \otimes \comp )\longrightarrow U_i\times \comp$$
given by $\phi_i(x,z\otimes w)=(x,zw)$. These maps define a morphism of twisted bundles
$$\phi :\mathbb{L}\otimes \epsilon^1\longrightarrow \mathbb{L},$$
with inverse given by the family $\phi_i^{-1}(x,z)=(x,z\otimes 1)$. Hence, $[\epsilon^1]=1$, the unit of the group.

Let now $[\mathbb{L}]$ be an arbitrary class. Then, $\mathbb{L} \otimes \mathbb{L}^*$ is an ordinary line bundle; denoting this bundle by $L$, we have that
$$[\mathbb{L}]^{-1}=[\mathbb{L}^*\otimes L^*].$$

From lemma \ref{iso_twistings}, we can assure that all twisted bundles in a given class have the same cocycle as twisting. Given now two twisted bundles $\mathbb{E}$ and $\mathbb{F}$ with twistings $\lambda$ and $\mu$ respectively, the twisted bundle $\mathbb{E}\otimes \mathbb{F}$ has twisting $\lambda \mu$; hence, invoking proposition \ref{tvb_torsion} proves the assertion about the grading.

The inclusion of $\opnm{Pic}(M)$ as a subgroup is clear from the previous discussion.
\end{proof}

Assume now that $\opnm{TVB}(M)$ and $\opnm{Vect}(M)$ are sets consisting of twisted bundles (with arbitrary twisting) over $M$ and vector bundles over $M$, respectively, and consider the equivalence relations $\mathbb{E}\sim \mathbb{E}\otimes \mathbb{L}$ and $E\sim E\otimes L$, where $\mathbb{L}$ is a twisted line bundle and $L$ is a line bundle. In the following result, $[\mathbb{E}]$ will denote the class of $\mathbb{E}$ according to the relation $\mathbb{E}\sim \mathbb{L}\otimes \mathbb{E}$; the same notation will be used for ordinary vector bundles.

\begin{theorem}\label{bij_tensor}
There exists a non-canonical biyection
$$\Psi :\opnm{TVB}(M)/_{\mathbb{E}\sim \mathbb{L}\otimes \mathbb{E}}\stackrel{\cong}{\longrightarrow}\opnm{Vect}(M)/_{E\sim L\otimes E}.$$
\end{theorem}
\begin{proof}
For each twisting $\lambda$, let us fix a twisted line bundle $\mathbb{L}_{\lambda}$ with that twisting. Now consider the map
$$\Psi [\mathbb{E}]=[\mathbb{E}\otimes \mathbb{L}_{\lambda^{-1}}],$$
where $\mathbb{E}$ has twisting $\lambda$.

We check that this correspondence is well-defined: first note that the twisting of $\mathbb{E}\otimes \mathbb{L}_{\lambda^{-1}}$ is $\lambda \lambda^{-1}=1$, and hence it is an ordinary line bundle. Now suppose that $[\mathbb{E}]=[\mathbb{F}]$, where $\mathbb{E}$ has twisting $\lambda$ and $\mathbb{F}$ twisting $\mu$; this implies the existence of a twisted line bundle $\mathbb{L}$ such that $\mathbb{F}\cong \mathbb{L}\otimes \mathbb{E}$. In particular, if $\mathbb{L}$ has twisting cocycle equal to $\epsilon$, then $\mu =\epsilon \lambda$. We now have to check that $[\mathbb{E}\otimes \mathbb{L}_{\lambda^{-1}}]=[\mathbb{F}\otimes \mathbb{L}_{\mu^{-1}}]$; in other words, we should find a line bundle $L$ such that $\mathbb{E}\otimes \mathbb{L}_{\lambda^{-1}}\cong \mathbb{L}\otimes \mathbb{E}\otimes \mathbb{L}_{\mu^{-1}}\otimes L$. Take now
$$L:=\mathbb{L}_\mu \otimes \mathbb{L}^*\otimes \mathbb{L}_{\lambda^{-1}};$$
then $L$ is an ordinary line bundle, as the twisting of the product of the right hand side is precisely $\mu \epsilon^{-1}\lambda^{-1}=\epsilon \lambda \epsilon^{-1}\lambda^{-1}=1$. We then have
$$\mathbb{L}\otimes \mathbb{E}\otimes \mathbb{L}_{\mu^{-1}}\otimes L\cong \mathbb{L}\otimes \mathbb{E}\otimes \mathbb{L}_{\mu^{-1}}\otimes \mathbb{L}_\mu \otimes \mathbb{L}^*\otimes \mathbb{L}_{\lambda^{-1}}\cong \mathbb{E}\otimes \mathbb{L}_{\lambda^{-1}},$$
as desired.

Assume now that $\mathbb{E}$ and $\mathbb{F}$ are twisted bundles with twistings $\lambda$ and $\mu$ respectively such that there exists a line bundle $L_0$ with $\mathbb{F}\otimes \mathbb{L}_{\mu^{-1}}\cong L_0\otimes \mathbb{E}\otimes \mathbb{L}_{\lambda^{-1}}$. Multiplying by $\mathbb{L}_{\mu}$ at both sides, we obtain
$$\mathbb{F}\otimes L_1\cong L_0\otimes \mathbb{E}\otimes \mathbb{L}_{\lambda^{-1}}\otimes \mathbb{L}_{\mu},$$
where $L_1=\mathbb{L}_{\mu}\otimes \mathbb{L}_{\mu^{-1}}$. Multiplying now by the dual line bundle $L_1^*$ yields
$$\mathbb{F}\cong \mathbb{E}\otimes \mathbb{L}_{\lambda^{-1}}\otimes \mathbb{L}_{\mu}\otimes L_0\otimes L_1^*.$$
As $\mathbb{L}_{\lambda^{-1}}\otimes \mathbb{L}_{\mu}\otimes L_0\otimes L_1^*$ is a twisted line bundle (with twisting $\mu \lambda^{-1}$), then $[\mathbb{F}]=[\mathbb{E}]$ and hence $\Psi$ is injective.

Let now $E$ be an arbitrary bundle. Then $E\otimes \mathbb{L}_{\lambda}$ is a $\lambda$-twisted vector bundle and then $\Psi [E\otimes \mathbb{L}_{\lambda}]=[E\otimes \mathbb{L}_{\lambda}\otimes \mathbb{L}_{\lambda^{-1}}]=[E]$.
\end{proof}



%%%%%%%%%%%%%%%%%%%%%%%%%%%%%%%%%%%%%%%%%%%%%%%%%%%%%
%%%%%%%%%%%%%%%%%%%%%%%%%%%%%%%%%%%%%%%%%%%%%%%%%%%%%
\section{Higher Categorical and Algebraic Structures}

The theory of higher categories originally entered into geometry and topology through the unpublished influential manuscript ``Pursuing Stacks'' of Grothedieck \cite{grothendieck:_pursuin}.  He tried to formulate a theory of higher homotopy groups in algebraic geometry using generalized coverings. This theory would be a far reaching generalization of his construction of the fundamental group \cite{kn:grothendieck2} and of Galois theory. These generalized coverings were stacks, and their fibres are $n$-homotopy types modelled using $n$-categories.
Giraud invented a particular kind of stacks called gerbes and applied them to non-abelian cohomology \cite{kn:giraud}.
In recent years the theory of higher categories has been related to developments in the study of new topological invariants of manifolds, which arise mainly from quantum field theories \cite{yetter:_cla}. 
 
A particular class of stacks which we shall encounter in the following paragraphs are called 2-vector bundles. These 2-bundles where first introduced by J.L. Brylinki \cite{brylinsky:_categ_vector_bundl_yang_mill} by categorifying the notion of sheaf of sections of a vector bundle, and it is, in turn, based upon another categorification: that of vector space, due to M. Kapranov and V. Voevodsky \cite{kn:kv}. Another definition of 2-vector bundle was later given by N. Bass, B. Dundas and J. Rognes \cite{bdr:_2vb}, \cite{BDRR:2vb_ell}; this definition is based upon {\v C}ech cocycles, and generalizes the one given by Brylinski.

We will introduce another notion of 2-vector bundle which also generalizes Brylinski's definition, but differs from the one of Bass-Dundas-Rognes in higher ranks.

%%%%%%%%%%%%%%%%%%%%%%%%%%%%%%%
\subsection{Fibred Categories}

A fibred category can be thought of as the categorical analogue of a presheaf. We will give a brief exposition of the main facts. The reader interested in a deeper treatment may consult \cite{kn:grothendieck2}, \cite{Vistoli:descent} or the more concise introduction given in \cite{Moerdijk:stacks}.

\begin{defi}
Let $\Phi:{\bf F}\to {\bf B}$ be a functor (this situation is usually stated as ``${\bf F}$ \emph{is a category over} ${\bf B}$''). A morphism $f:X\to Y$ in ${\bf F}$ is said to be \emph{cartesian} if the following condition holds: given any morphism $h:Z\to Y$ in ${\bf F}$ and any morphism $\beta :\Phi (Z)\to \Phi (X)$ in the base ${\bf B}$ such that $\Phi (f)\beta =\Phi (h)$, there exists a unique map $g:Z\to X$ such that $\Phi (g)=\beta$ and $fg=h$.
\end{defi}

This definition can be depicted in the following way:
$$
\xymatrix{
   Z\ar@{|->}[dd] \ar@{-->}[rd]_{g} \ar@/^/[rrd]^{h} & & \\
   & X\ar@{|->}[dd]\ar[r]_{f} & Y\ar@{|->}[dd] \\
   \Phi (Z) \ar[rd]_{\beta}\ar@/^/[rrd] ^(.3){\Phi (h)} |(.487)\hole \\
   & \Phi (X) \ar[r]_{\Phi (f)} & \Phi (Y),}
$$
where the objects and arrows in the ``roof'' are in ${\bf F}$ and the ones on the ``floor'', in ${\bf B}$; the connection between the ``roof'' and the ``floor'' is provided by the application of the functor $\Phi$. In this context, we will say that \emph{$X$ is a pullback of $Y$ on $\Phi (X)$}, and the notation $\alpha^*Y$ is usually adopted for $X$, where $\alpha =\Phi (f)$. Applying the previous definition (diagram chasing is always a good idea in this kind of proofs) one can verify that if $X$ and $X'$ are two pullbacks of $Y$ to $\Phi (X)=\Phi (X')$, then $X$ and $X'$ are isomorphic in ${\bf F}$.

A fibred category is then a category which admits pullbacks.

\begin{defi}
We will say that $\Phi:{\bf F}\to {\bf B}$ is a \emph{fibred category} or that ${\bf F}$ is \emph{fibred over} ${\bf B}$ if given $X,Y\in {\bf F}$ and any map $\alpha :\Phi (X)\to \Phi (Y)$, there exists a cartesian arrow $f:X\to Y$ such that $\Phi (f)=\alpha$.
\end{defi}

The previous definition resembles the definition of fibre bundle. There is another characterization of fibred categories, which resemble the definition of presheaves. Before introducing this new point of view, let us give the following

\begin{defi}
If $\Phi :{\bf F}\to {\bf B}$ is a fibred category and $U\in {\bf B}$, then the \emph{fibre} over $U$ is the full subcategory ${\bf F}(U)$ of ${\bf F}$ with objects $X\in {\bf F}$ such that $\Phi (X)=U$.
\end{defi}

We will now give a concise idea of how a fibred category defines a contravariant functor $\Phi_{{\bf F}}:{\bf B}\to \tsf{Cat}$, where $\tsf{Cat}$ is the 2-category of categories.\footnote{In fact, as the pullback $(\alpha \beta)^*Y$ is not usually equal to $\beta^*\alpha^*Y$ but only canonically isomorphic to it, $\Phi_{{\bf F}}$ is usually a \emph{pseudo-functor}. But we will not detain ourselves with more definitions, as a careful treatment of these facts is lengthy.} If $U\in {\bf B}$ is an object of the base, then $\Phi_{{\bf F}}(U)$ is defined to be the fibre ${\bf F}(U)$ over $U$. Now, the image of a morphism $\alpha :V\to U$ in ${\bf B}$ should be a functor $\alpha^*:{\bf F}(U)\to {\bf F}(V)$ (the ``restriction''). The problem now is defining this functor. So first take an object $Y\in {\bf F}(U)$; a good way of obtaining an object over $V$ (the image $\alpha^*(Y)$) in a fibred category is to pull-back $Y$. But the problem now is which one of all the isomorphic pullbacks should be chosen. This procedure of choosing pullbacks defines what is called a \emph{cleavage}, which is precisely a class $\mathfrak{K}$ of cartesian maps in ${\bf F}$ such that for each map $\alpha :V\to U$ and each object $Y$ over $U$ (that is, $\Phi (Y)=U$), there exists a unique object $\alpha^*Y$ and map $f:\alpha^*Y\to Y$ in $\mathfrak{K}$ such that $\Phi (f)=\alpha$. Cleavages always exist \cite{Vistoli:descent} and, with a cleavage at hand, we can define the functor $\alpha^* :{\bf F}(U)\to {\bf F}(V)$ by $Y\mapsto \alpha^*Y$ on objects and, if $f:X\to Y$ is an arrow in ${\bf F}(U)$, then $\alpha^*f:\alpha^*X\to \alpha^*Y$ is the unique morphism defined by the diagram
$$
\xymatrix{
   \alpha^*X\ar@{-->}[rd]_{\exists !} \ar@/^/[rrd]^{h} & & \\
   & \alpha^*Y\ar[r]_{f} & Y, }
$$
where $h$ is the composite $\alpha^*X\to X\stackrel{f}{\to}Y$. The alternative definition in terms of a pseudo-functor reads as follows.

\begin{defi}\label{fib_cat_2}
A \emph{fibred category} is a functor $\Phi :{\bf B}\to \tsf{Cat}$ with the following properties:
\begin{enumerate}
\item If $W\stackrel{\beta}{\rightarrow}V\stackrel{\alpha}{\rightarrow}U$ is a pair of composable arrows in ${\bf B}$, then, denoting $\Phi (\alpha )$ by $\alpha^*$, we should have a natural isomorphism $u({\alpha \beta}):(\alpha \beta )^*\cong \beta^*\alpha^*$.
\item For three composable maps $Z\stackrel{\gamma}{\rightarrow}W\stackrel{\beta}{\rightarrow}V\stackrel{\alpha}{\rightarrow}U$, the square
$$
\xymatrix{
(\alpha \beta \gamma )^* \ar[rr]^{u(\alpha \beta ,\gamma )}\ar[d]_{u(\alpha ,\beta \gamma )} & & \gamma^* (\alpha \beta )^* \ar[d]^{\gamma^*u(\alpha ,\beta )} \\
(\beta \gamma )^*\alpha^* \ar[rr]^{u(\beta ,\gamma )\alpha^*}  & & \gamma^*\beta^*\alpha^* }
$$
should be commutative.
\end{enumerate}
\end{defi}

And conversely, given a fibred category $\Phi :{\bf B}\to \tsf{Cat}$, we can go back to the first conception of a category over ${\bf B}$. The interested reader may consult \cite{Vistoli:descent} for a nice and complete exposition of these issues.

\begin{obs}
If the fibres of a fibred category ${\bf F}$ takes values in some subcategory ${\bf X}$ of $\tsf{Cat}$, then we will say that ${\bf F}$ is a \emph{category fibred in ${\bf X}$}. For example, if each fibre ${\bf F}(U)$ is a groupoid, then ${\bf F}$ will be called a category fibred in grupoids. A category fibred in $\tsf{Sets}$ is usually called a \emph{discrete fibred category}.\footnote{A \emph{discrete} category is a category ${\bf X}$ such that, for each object $X$, the only arrow $X\to X$ is the identity. Thus, a discrete category can be regarded as a set (and conversely).} If ${\bf B}=\tsf{Op}(M)$ for some topological space $M$, then we shall also use the term ``fibred category over $M$'' for a fibred category over $\tsf{Op}(M)$. 
\end{obs}

\begin{defi}
Let ${\bf F}\stackrel{\Phi}{\rightarrow}{\bf B}\stackrel{\Psi}{\leftarrow}{\bf G}$ be fibred categories over ${\bf B}$. A functor ${\bf F}\stackrel{H}{\longrightarrow}{\bf G}$ is said to be a \emph{fibred morphism} or \emph{morphism of fibred categories} if $\Psi H=\Phi$ and $F$ sends cartesian arrows to cartesian arrows.
\end{defi}

We also discuss here the definition of morphism of fibred categories according to the alternative viewpoint. Consider then two fibred categories $\tsf{Cat}\stackrel{\Phi}{\leftarrow}{\bf B}\stackrel{\Psi}{\rightarrow}\tsf{Cat}$; a \emph{morphism} $H:\Phi \to \Psi$ between these fibred categories consists of the following data
\begin{enumerate}
\item A family of arrows $H_U:\Phi (U)\to \Psi (U)$,
\item for each morphism $\alpha :V\to U$ in ${\bf B}$, a natural isomorphism $\eta_\alpha$ between the functors $\alpha^*H_U$ and $H_V\alpha^*$ (note that we use the same symbol $\alpha^*$ to denote both functors $\Phi (U)\to \Phi (V)$ and $\Psi (U)\to \Psi (V)$). 
\end{enumerate}
These data shoud satisfy the following compatibility condition: for a chain of maps $W\stackrel{\beta}{\to}V\stackrel{\alpha}{\to}U$ in ${\bf B}$, the diagram
$$
\xymatrix{
H_W(\alpha \beta )^* \ar[rr]^{\eta_{\alpha \beta}} \ar[d]_{\eta_{\alpha}u} && (\alpha \beta )^*H_U \ar[d]^{uH_U} \\
H_W\beta^*\alpha^* \ar[r]_{\eta_{\beta}\alpha^*} & \beta^*H_V\alpha^* \ar[r]_{\beta^*\eta_\alpha} & \beta^*\alpha^*H_U }
$$
should be commutative, where the letter $u$ denotes the maps given in definition \ref{fib_cat_2}.

Thanks to the fibred structure we have the following

\begin{proposition}
Let ${\bf F}$ and ${\bf G}$ be two fibred categories over ${\bf B}$. A fibred morphism $H:{\bf F}\to {\bf G}$ is an equivalence of categories if and only if for each $U\in {\bf B}$ the restriction $H_U:{\bf F}(U)\to {\bf G}(U)$ is an equivalence.
\end{proposition}

The following definition will be important in the next section. For our purposes, and to avoid technical issues which won't be helpful in this discussion, we shall consider ${\bf B}$ as the category of open subsets of some topological space $M$.

\begin{defi}\label{fcex_1}
Let $U\in \tsf{Op}(M)$ be an object and $\Phi_{{\bf F}}$ a fibred category with base $\tsf{Op}(M)$. Given objects $X,Y\in {\bf F}(U)$, the presheaf $\underline{\opnm{Hom}}_U(X,Y)$ on $\tsf{Op}(U)$ is defined by the correspondence
$$V\longmapsto \opnm{Hom}_{{\bf F}(V)}\left (X|_V,Y|_V\right ),$$
where $X|_V$ means the image of $X\in {\bf F}(U)$ under the restriction arrow ${\bf F}(U)\to {\bf F}(V)$. This presheaf will be denoted $H_U(X,Y)$.
\end{defi}

We end this section with several examples of fibred categories. 

\begin{ej}\label{fcex_2}
If $\tsf{Top}$ is the category of topological spaces, the usual pullback construction for bundles makes the functor $\tsf{Top}\to \tsf{Cat}$ given by $M\mapsto \tsf{Vect}(M)$ a fibred category (over the category of topological spaces). If $[\tsf{Vect},\tsf{Top}]$ denotes the category of vector bundles over arbitrary topological spaces, this fibred category can be described (from the other viewpoint) by a functor $\Phi :[\tsf{Vect},\tsf{Top}]\to \tsf{Top}$ given by
$$\Phi (E\to M)=M.$$
An important particular case is obtained replacing the base category $\tsf{Top}$ with the category $\tsf{Op}(M)$ of open subsets of a fixed space $M$. In this case, the category $[\tsf{Vect},\tsf{Op}(M)]$ will be denoted by $[\tsf{Vect},M]$.
\end{ej}

\begin{ej}\label{fcex_3}
Let $\scr{C}_{\tsf{Top}}:=[\tsf{Top},\tsf{Top}]$ be the category which objects are continuous maps $M\to N$ and morphisms $(M\to N)\to(M'\to N')$ commutative diagrams
$$
\xymatrix{
M \ar[r] \ar[d] & M'\ar[d] \\
N \ar[r] & N'.}
$$
Then the functor $\Phi :\scr{C}_{\tsf{Top}}\to \tsf{Top}$ given by $\Phi (M\to N)=N$ is a fibred category, again by the pullback construction. An important particular case is obtained by fixing the base; if $N$ is this fixed base, we thus obtain the fibred category $[\tsf{Top},N]$ (with base $\tsf{Op}(N)$), which is defined by the assignment $U\mapsto \tsf{Top}(U)$, where $\tsf{Top}(U)$ is the category of spaces over $U$: its objects are maps $M\to U$ and morphisms are commutative triangles.
\end{ej}

\begin{ej}\label{fcex_4}
Any presheaf $\scr{P}:\tsf{Op}(M)\to {\bf X}$ with values in some category ${\bf X}$ is a fibred category. This statement can be extended to arbitrary presheaves; that is, contravariant functors ${\bf B}\to {\bf X}$ (here ${\bf B}$ is the base category). 
\end{ej}

\begin{ej}\label{fcex_5}
Consider the pseudofunctor $\Phi :\tsf{Op}(M)^\circ \to \tsf{Cat}$ given on objects by $\Phi (U)=\tsf{Sh}(U)$ and, if $i:V\subset U$ is an inclusion, $\Phi (i):\tsf{Sh}(U)\to \tsf{Sh}(V)$ is the restriction $\scr{S}\mapsto \scr{S}|_V$. This functor defines a fibred structure for the category of sheaves over $M$. This property extends also to the category of sheaves of groups and (locally free) modules.

More generally, consider the pseudofunctor $\Phi :\tsf{Top}^\circ \to \tsf{Cat}$ given by $\Phi (M)=\tsf{Sh}(M)$. The inverse image construction provides $F$ with a fibred structure.

The same conclusion applies replacing the category of sheaves with the category of quasicoherent sheaves of modules. In fact, Grothendiecks's motivating example was that of quasicoherent sheaves over the category of schemes.
\end{ej}

\begin{ej}\label{fcex_7}
Given fibred categories ${\bf F}\stackrel{\Phi}{\rightarrow}{\bf B}\stackrel{\Psi}{\leftarrow}{\bf G}$ over ${\bf B}$, consider the fibred product ${\bf F}\times_{{\bf B}}{\bf G}$ defined in the following way: its objects are pairs $(X,Y)\in {\bf F}\times {\bf G}$ such that $\Phi (X)=\Psi (Y)$; in other words, there exists an object $U\in {\bf B}$ such that $X\in {\bf F}(U)$ and $Y\in {\bf G}(U)$). A map $(X,Y)\to (X',Y')$ is a pair of maps $X\to X'$ in ${\bf F}(U)$ and $Y\to Y'$ in ${\bf G}(U)$ for some $U\in {\bf B}$. A straightforward computation shows that ${\bf F}\times_{{\bf B}}{\bf G}$ is also a fibred category over ${\bf B}$. Moreover, we have projection functors ${\bf F}\leftarrow {\bf F}\times_{{\bf B}}{\bf G}\rightarrow {\bf G}$ such that the diagram
$$
\xymatrix{
{\bf F}\times_{{\bf B}}{\bf G} \ar[r] \ar[d] & {\bf F} \ar[d] \\
{\bf G} \ar[r] & {\bf B}}
$$
commutes. The $n$-folded fibred product ${\bf F}\times_{\bf B}\cdots \times_{\bf B}{\bf F}$ will be denoted by ${\bf F}^n$.
\end{ej}

\begin{ej}\label{fcex_9}
Let $f:M\to N$ be a continuous map and let ${\bf F}$ be a fibred category over $M$. The \emph{pushout} $f_*{\bf F}$ of ${\bf F}$ by $f$ is defined by the assignment $(f_*{\bf F})(V)={\bf F}(f^{-1}(V))$, and it is also a fibred category over $N$. This fact can be easily proved by noting that if  $W\subset V$ is an inclusion in $\tsf{Op}(N)$, then the induced map $f^{-1}(W)\subset f^{-1}(V)$ is an inclusion in $\tsf{Op}(M)$. The rest is deduced from the fibred structure of ${\bf F}$.
\end{ej}



%%%%%%%%%%%%%%%%%%%%%%%%%%%%%%%%%%%%%%%%%%%%%%%%%%%%%%%%%%%%%%%%%%%%%%%%
\subsubsection{The Fibred Category Structure for Twisted Vector Bundles}

We need first to define morphisms over maps $N\to M$.

\begin{defi}\label{arrows}
Let $f:N\to M$ and let $\mathbb{E}$ and $\mathbb{F}$ be twisted bundles over $N$ and $M$ respectively, given by
$$
\begin{aligned}
\mathbb{E} &= (\mathfrak{U},U_i\times V,\{g_{ij}\},\{\lambda_{ijk}\}) \\
\mathbb{F} &=(\mathfrak{U}',U'_r\times W,\{g'_{rs}\},\{\lambda'_{rst}\}). \\
\end{aligned}
$$
(Shrinking the cover if necessary, we can assume that for each $r$, there exists an index $i$ such that $f(U'_r)\subset U_i$). A \emph{morphism} $\phi :\mathbb{F}\to \mathbb{E}$ over $f$ is a family of maps
$$\phi_{ri}:U'_r\times W\longrightarrow U_i\times V$$
over the restriction $f|_{U'_r}:U'_r\to U_i$ such that the diagram
\begin{equation}
\xymatrix{
U'_{rs}\times W \ar[r]^{\phi_{sj}} \ar[d]_{1\times g'_{rs}} & U_{ij}\times V \ar[d]^{1\times g_{ij}} \\
U'_{rs}\times W \ar[r]_{\phi_{ri}} & U_{ij}\times V}
\end{equation}
commutes.
\end{defi}

Consider now the pullback bundle $f^*\mathbb{E}$, where $\mathbb{E}$ is a twisted bundle over $M$ and $f:N\to M$. We then have a map
$$\phi_f:f^*\mathbb{E}\longrightarrow \mathbb{E}$$
defined by the family $f\times 1:f^{-1}(U_i)\times V\to U_i\times V$.

\begin{proposition}\label{tvb_cartesian}
The map $\phi_f$ is a cartesian arrow.
\end{proposition}
\begin{proof}
Consider the diagram of maps and spaces
$$
\xymatrix{
P \ar[dr]^{\alpha} & \\
N\ar[r]_f & M.}
$$
and let $\mathbb{F} =(\mathfrak{U}',\alpha^{-1}(U_i)\times W,\{h_{ij}\},\{\mu_{ijk}\})$ be a twisted bundle over $P$. Let now $\psi :\mathbb{F}\to \mathbb{E}$ be any map over $\alpha$, and assume it is given by a family
$$\psi_i:\alpha^{-1}(U_i)\times W\longrightarrow U_i\times V.$$
Let $\beta :P\to N$ be a map such that $f\beta =\alpha$, and consider the map $\eta :\mathbb{F}\to f^*\mathbb{E}$ defined by the family
$$\eta_i=\beta \times \psi_i:\alpha^{-1}(U_i)\times W\longrightarrow f^{-1}(U_i)\times V.$$
This map $\eta$ is a morphism of twisted bundles over $\beta$ and is the unique map such that $\phi_f\eta =\psi$ (see the diagram below).
$$
\xymatrix{
   \mathbb{F}\ar@{|->}[dd] \ar@{-->}[rd]_{\eta} \ar@/^/[rrd]^{\psi} & & \\
   & f^*\mathbb{E}\ar@{|->}[dd]\ar[r]_{\phi_f} & \mathbb{E}\ar@{|->}[dd] \\
   P \ar[rd]_{\beta}\ar@/^/[rrd] ^(.3){\alpha} |(.487)\hole \\
   & N \ar[r]_f & M.}
$$
\end{proof}

As is usual in analogous cases, if $U$ is an open subset of $M$ and $\mathbb{E}$ is a twisted bundle over $M$, then the pullback along the inclusion $U\subset M$ is called the \emph{restriction of $\mathbb{E}$ to $U$} and is denoted by $\mathbb{E}|_U$.

The previous facts imply the following

\begin{proposition}
The assignment $M\mapsto \tsf{TVB}(M)$ defines a fibred category over $\tsf{Top}$.
\end{proposition}

The same conclusion is obtained also for the categories of $\lambda$-twisted vector bundles over $\tsf{Top}$ for a fixed twisting $\lambda$ and also for twisted bundles over some fixed space $M$. We will denote by $[\tsf{TVB},\tsf{Top}]\to \tsf{Top}$ and $[\tsf{TVB}_{\lambda},\tsf{Top}]\to \tsf{Top}$ the fibred categories of twisted vector bundles and $\lambda$-twisted vector bundles over $\tsf{Top}$, respectively. The same symbols but replacing $\tsf{Top}$ with $\tsf{Op}(M)$ for a fixed space $M$ will be used to denote the fibred categories of twisted bundles and $\lambda$-twisted bundles over $M$.

We now define a fibred category $[\widehat{\tsf{TVB}},\tsf{Top}]\to \tsf{Top}$ in the following way: objects are twisted bundles over some space $M\in \tsf{Top}$ and arrows are the cartesian ones. If $M$ is any space, then the fibre over $M$, which we denote by $\widehat{\tsf{TVB}}(M)$, is a groupoid. That is, every arrow in $\widehat{\tsf{TVB}}(M)$ is an isomorphism (and we thus obtain an example of a category fibred in grupoids). This statement can be deduced from the following:

\begin{lemma}
Let ${\bf F}$ be a fibred category over ${\bf B}$.
\begin{enumerate}
\item If $\phi :X\to Y$ and $\psi :Y\to X$ are arrows in ${\bf F}$ and $\psi$ is cartesian, then $\phi$ si cartesian if and only if the composite $\psi \phi$ is cartesian.
\item Let $X,Y$ be objects contained in the same fibre. An arrow $X\to Y$ is cartesian if and only if it is an isomorphism.
\end{enumerate}
\end{lemma}

The same conclusions apply by replacing the base category $\tsf{Top}$ with $\tsf{Op}(M)$.

\begin{obs}
As the previous lemma shows, the procedure of keeping only the cartesian arrows can be done for any fibred category. That is, if ${\bf F}$ is a fibred category over ${\bf B}$, then the category $\widehat{{\bf F}}$ with the same objects as ${\bf F}$ and arrows only the cartesian ones is a category fibred in grupoids over ${\bf B}$.
\end{obs}


%%%%%%%%%%%%%%%%%%%
\subsection{Stacks}

Just as a fibred category is the categorical analogue of a presheaf, a stack can be thought of as a categorification of the notion of sheaf. The general definition requires the introduction of sites and Grothendieck topologies, but we will avoid these facts and work only with the category $\tsf{Op}(M)$ of open subsets of a topological space $M$. Recall that, as for sheaves, a fibred category with base $\tsf{Op}(M)$ will be called a fibred category \emph{over $M$}.

The main feature of sheaves that distinguish them from presheaves is that we can glue sections. For a stack, this notion should be satisfied not only by sections (which are the objects of the fibre-categories) but also by morphims.

\begin{defi}
Let $\Phi_{{\bf F}}$ be a fibred category over $M$, viewed as a pseudo-functor. We will say that $\Phi_{{\bf F}}$ (or ${\bf F}$) is a \emph{prestack} if for each $U\in \tsf{Op}(M)$ and each pair of objects $X,Y\in {\bf F}(U)$, the presheaf $H_U(X,Y)$ defined in \ref{fcex_1} is a sheaf.
\end{defi}

The statement ``$H_U(X,Y)$ is a sheaf'' means that morphisms in ${\bf F}(U)$ can be glued together. A fibred category which verifies this fact for each $U$ is called a \emph{prestack}.\footnote{The corresponding notion for sheaves is that of \emph{separated presheaf}, which we did not define.}

\begin{ej}\label{psex_1}
Let $[\tsf{Top},N]$ be the fibred category of spaces over $N$ and fix some open subset $U\subset N$ and objects $f:M\to N$ and $g:P\to N$. Let $\{U_i\}$ be an open cover of $U$ and $\phi_i\in H_U(M,P)(U_i)$; that is $\phi_i:f^{-1}(U_i)\to g^{-1}(U_i)$ and $g\phi_i =f$. Assume that, over the (non-empty) intersections $U_{ij}=U_i\cap U_j$, the maps $\phi_i$ and $\phi_j$ coincide; that is, they agree on $f^{-1}(U_{ij})=f^{-1}(U_i)\cap f^{-1}(U_j)$. Then, basic properties of continuous maps let us glue the pieces $\phi_i$ to obtain a map $\phi:f^{-1}(U)\to g^{-1}(U)$ such that $g\phi =f$, and thus $[\tsf{Top},N]$ is a prestack.
\end{ej}

Before defining stacks, we need first the notion of descent category, which objects, roughly speaking, consist of local objects in a fibred category. When reading the next definition is recommended to keep in mind the construction of vector bundles from cocycles.

\begin{defi}
Let $\Phi_{{\bf F}}$ be a fibred category over $M$, $U\subset M$ an open subset and $\mathfrak{U}=\{U_i\}$ an open cover of $U$. The \emph{category $\tsf{Desc}(\mathfrak{U},{\bf F})$ of descent data} is defined in the following way:
\begin{enumerate}
\item Objects are pairs $(X,f)$, where $X=\{X_i\}$ is a family of objects $X_i\in {\bf F}(U_i)$ and $f=\{f_{ij}\}$ is a family of isomorphisms $f_{ij}:X_j|_{U_{ij}}\cong X_i|_{U_{ij}}$ which satisfy the so-called \emph{cocycle conditions}:
$$f_{ii}=\opnm{id}_{X_i} \quad \text{and} \quad f_{ij}f_{jk}=f_{ik},$$
where the second equality is in ${\bf F}(U_{ijk})$.
\item An arrow $(X,f)\to (Y,g)$ is a family $\{\phi_i\}$ of maps $\phi_i:X_i\to Y_i$ such that $\phi_if_{ij}=g_{ij}\phi_j$.
\end{enumerate}
\end{defi}

We have a functor $D:{\bf F}(U)\to \tsf{Desc}(\mathfrak{U},{\bf F})$ defined in the following way: given $X\in {\bf F}(U)$, $D(X)=(\{X|_{U_i}\},\opnm{id})$, where $\opnm{id}$ is the family consisting of the identity maps of $X|_{U_{ij}}$. If $f:X\to Y$, $D(f)$ is the family consisting of the maps $X|_{U_i}\to Y|_{U_i}$, obtained by aplying the pullback functor to the map $f$. A straightforward computation shows that maps can be glued over any open subset $U$ if and only if the functor $D$ is fully-faithful for each open cover $\mathfrak{U}$ of $U$. Then, the property of being a prestack can be expressed in terms of $D$ by requiring that this functor should be fully-faithful for each $U$ and each open cover of $U$. On the other hand, gluing objects defined on some cover $\mathfrak{U}$ of $U$ requires $D$ to be essentially surjective; that is, for each object $(\{X_i\},\{f_i\})\in \tsf{Desc}(\mathfrak{U},{\bf F})$ there should exist an object $X\in {\bf F}(U)$ such that $D(X)\cong (\{X_i\},\{f_i\})$. After this interlude, we can then define the notion of stack.

\begin{defi}
The fibred category ${\bf F}$ over $M$ is a \emph{stack} if the functor $D:{\bf F}(U)\to \tsf{Desc}(\mathfrak{U},{\bf F})$ is an equivalence of categories for each $U\in \tsf{Op}(M)$ and each open cover $\mathfrak{U}$ of $U$.
\end{defi}

Fibred categories of examples \ref{fcex_2}, \ref{fcex_3}, \ref{fcex_5}, \ref{fcex_9} are stacks. In example \ref{fcex_7}, if ${\bf F}$ and ${\bf G}$ are stacks, then also is the fibred product ${\bf F}\times_{{\bf B}}{\bf G}$. On the other hand, the discrete fibred category defined by a presheaf $\scr{P}:\tsf{Op}(M)\to {\bf X}$ (see example \ref{fcex_4}) is a stack if and only if $\scr{P}$ is a sheaf. Twisted and ordinary vector bundles and locally free sheaves will be treated in more detail shortly, as well as the pushout.

\begin{obs}
Let us describe what an isomorphism in the descent category looks like. First observe that the composition of two morphisms $\phi :(\{X_i\},\{f_{ij}\})\to (\{Y_i\},\{g_{ij}\})$ and $\psi :(\{Y_i\},\{g_{ij}\})\to (\{Z_i\},\{h_{ij}\})$ is obtained by composing the maps $\phi_i:X_i\to Y_i$ and $\psi_i:Y_i\to Z_i$ (compatibility with cocycles can be checked by a direct computation). Assume now that $\phi=\{\phi_i\} :(X,f)\to (Y,g)$ is an isomorphism. This implies the existence of an inverse $\phi^{-1}:(Y,g)\to (X,f)$. If $\phi^{-1}$ is the family $\{\psi_i\}$, then the equalities $\phi \phi^{-1}=\opnm{id}_{(Y,g)}$ and $\phi^{-1}\phi =\opnm{id}_{(X,f)}$ imply that necessarily each $\phi_i$ is an isomorphism and $\psi_i=\phi_i^{-1}$ for each $i$.
\end{obs}


\begin{ej}\label{st_ex1}
We will now complete the discussion started in example \ref{psex_1}. To show that $[\tsf{Top},N]$ is a stack it only remains to be shown that we can glue local objects. So let $U\subset N$ be an open subset and $\mathfrak{U}=\{U_i\}$ an open cover of $U$. In this case, an object of the descent category is a pair $(\{f_i\},\{\varphi_{ij}\})$, where $f_i:M_i\to U_i$ and $\varphi_{ij}:f_j^{-1}(U_{ij})\cong f_i^{-1}(U_{ij})$ such that $\varphi_{ii}=\opnm{id}_{M_i}$ and $\varphi_{ij}\varphi_{jk}=\varphi_{ik}$. To prove that the functor $D$ is essentially surjective we need to find a map $f:M\to U$ such that
\begin{enumerate}
\item For each $i$ there exists an isomorphism $\psi_i:M_i\cong f^{-1}(U_i)$ in the category $\tsf{Top}(U_i)$; that is, it should make the following diagram
$$
\xymatrix{
M_i \ar[rr]^{\psi_i} \ar[dr]_{f_i} & & f^{-1}(U_i) \ar[dl]^f \\
& U_i &
}
$$
commutative.
\item Over $U_{ij}$ the equality $\psi_i\varphi_{ij}=\psi_j$ holds; that is, the diagram
$$
\xymatrix{
f^{-1}_j(U_{ij})\ar[rr]^{\varphi_{ij}} \ar[dr]_{\psi_j} & & f_i^{-1}(U_{ij}) \ar[dl]^{\psi_i} \\
& f^{-1}(U_{ij}) &
}
$$
commutes.
\end{enumerate}
So let $M$ be the quotient space
$$M=\bigsqcup_iM_i\Big / \sim ,$$
where the equivalence relation is given by $(i,x)\sim (j,y)$ if and only if $U_{ij}\neq \emptyset$ and $y=f_{ji}(x)$. Denoting by $[i,x]$ the equivalence class of $(i,x)$, the map $f:M\to U$ given by $f[i,x]=f_i(x)$ verifies $D(f)\cong (\{f_i\},\{\varphi_{ij}\})$, as desired.
\end{ej}

\begin{ej}\label{st_ex2}
The same argument as the one given in the previous example shows that the fibred category $[\tsf{Vect},M]$ of vector bundles over (open subsets of a space) $M$ is also a stack.
\end{ej}

Though we state it just for $\tsf{Op}(M)$, the next result holds for arbitrary base categories.

\begin{proposition}\label{equiv_stacks}
Let $H:{\bf F}\to {\bf G}$ be a morphism of fibred categories over $M$. If $H$ is an equivalence and ${\bf F}$ is a stack, then ${\bf G}$ is also a stack.
\end{proposition}

\begin{ej}\label{st_ex3}
Let $[\tsf{TVB}_{\lambda},M]\to \tsf{Op}(M)$ be the fibered category of $\lambda$-twisted vector bundles over (open subsets of) a space $M$. Let $[\tsf{TVB},M]\to [\tsf{Vect},M]$ be the functor defined in proposition \ref{equiv_tvb_vect}. This functor is a morphism of fibred categories and is an equivalence. Thus, by proposition \ref{equiv_stacks}, $[\tsf{TVB}_{\lambda},M]\to \tsf{Op}(M)$ is also a stack.
\end{ej}

\begin{ej}\label{st_ex4}
If ${\bf F}$ is a stack, then the pushout $f_*{\bf F}$ by $f:M\to N$ is also a stack. To see this, let us consider an open subset $V\subset N$ and an open cover $\mathfrak{V}=\{V_i\}$ of $V$. We then have that $f^{-1}\mathfrak{V}:=\{f^{-1}(V_i)\}$ is an open cover of $f^{-1}(V)$. Moreover, it is easy to check that we have an equivalence $\tsf{Desc}(\mathfrak{V},f_*{\bf F})\simeq \tsf{Desc}(f^{-1}\mathfrak{V},{\bf F})$ between the descent categories. On the other hand, the equivalence $\tsf{Desc}(f^{-1}\mathfrak{V},{\bf F})\simeq {\bf F}(f^{-1}(V))$ holds because ${\bf F}$ is a stack. Then,
$$\tsf{Desc}(\mathfrak{V},f_*{\bf F})\simeq \tsf{Desc}(f^{-1}\mathfrak{V},{\bf F})\simeq {\bf F}(f^{-1}(V))= (f_*{\bf F})(V),$$
proving the assertion.
\end{ej}

\begin{ej}\label{st_ex4}
By \ref{loc_free_sections} and \ref{sheaf_bundle}, we have an equivalence between the fibred categories of vector bundles over (open subsets) of $M$ and locally-free $\scr{O}_M$-modules. Thus, by \ref{equiv_stacks}, the fibred category $U\mapsto \tsf{LF}_{\scr{O}_U}$ is also a stack.
\end{ej}


%%%%%%%%%%%%%%%%%%%%%%%%%%%%%%%%%%%%%%%%%%%%%%%%%
\subsection{2-Vector Spaces and 2-Vector Bundles}
\label{sec:algebra}


We will now give an overview of the categorical analogues of vector spaces and vector bundles. There are several definitions of 2-vector space in the literature, due to Kapranov-Voevodsky \cite{kn:kv}, Baez-Crans \cite{baezcrans:2vector}, Elgueta \cite{elgueta:2vector}, etc. We will adopt the definition of 2-vector space given by Kapranov and Voevodsky; so, for us the word ``2-vector space'' will mean ``Kapranov-Voevodsky 2-vector space''.

A complete and detailed exposition of all the following definitions is a rather lenghty task. In order to concisely introduce the concepts we need, we omit some tedious (but necessary) details. The main references for our treatment of 2-vector spaces/module categories are \cite{kn:kv}, \cite{yetter:_cla}.


%%%%%%%%%%%%%%%%%%%%%%%%%%%%%%%
\subsubsection{2-Vector Spaces}

We will assume that the reader is familiar with the notion of monoidal category, which will be central in the following discussions; for its definition and properties, the reader may consult \cite{maclane:_catwm}, \cite{kelly:_enriched}.
\begin{defi}
A \emph{rig category} is a category ${\bf R}$ with two symmetric monoidal structures $({\bf R},\oplus ,{\bf 0})$ and $({\bf R},\otimes ,{\bf 1})$ together with distributivity natural isomorphisms
$$X\otimes (Y\oplus Z)\longrightarrow (X\otimes Y)\oplus (X\otimes Z)$$
$$(X\oplus Y)\otimes Z\longrightarrow (X\otimes Z)\oplus (Y\otimes Z)$$
verifying some coherence axioms which are detailed in \cite{laplaza:_coh1}, \cite{kelly:_coh2}.\footnote{In algebra, a \emph{rig} or \emph{semiring} is a ring $R$ for which not every element $x\in R$ has an additive inverse. We adopt this terminology in this categorical setting as this is usually the case, but the term \emph{ring category} is also used.}
\end{defi}

An important example, for it will be extensively used in what follows, is the category $\tsf{Vect}$ of finite dimensional vector spaces over $\comp$ (or any other field). The operations are given by direct sum (with ${\bf 0}=\{0\}$, the trivial vector space) and tensor product (with ${\bf 1}\cong \comp$). To justify our choice of terminology, note that if $V$ is a vector space of dimension $n\geqslant 1$, then $V$ cannot have an additive inverse (that is, there is no vector space $W$ such that $V\oplus W={\bf 0}$).

\begin{notation}
From now on, $\tsf{Vect}$ will denote the category of finite dimensional, complex vector spaces.
\end{notation}

\begin{defi}
Let ${\bf R}$ be a rig category. A \emph{left module category} over
${\bf R}$ is a monoidal category $({\bf M}, \oplus , {\bf 0})$ together
with an action (bifunctor) 
$$
\otimes :{\bf R}\times {\bf M}\longrightarrow {\bf M}
$$
and natural isomorphisms
$$
\begin{aligned}
  A\otimes (B\otimes X) &\longrightarrow (A\otimes B)\otimes
  X \\
  (A\oplus B)\otimes X &\longrightarrow (A\otimes X)\oplus
  (B\otimes X) \\
  A\otimes (X\oplus Y) &\longrightarrow (A\otimes X)\oplus
  (A\otimes Y) \\
\end{aligned}
$$
$$\tau_X = \tau :{\bf 1}\otimes X\longrightarrow X \qquad \rho_A=\rho
:A\otimes {\bf 0}\longrightarrow {\bf 0} \qquad \lambda_X=\lambda :{\bf
0}\otimes X\longrightarrow {\bf 0}$$ 
for any given objects $A,B\in {\bf R}$ and $X,Y\in {\bf M}$, which are
required to satisfy coherence conditions analogous to the ones for a
rig category. \emph{Right module categories} are defined
analogously.
\end{defi}

An ${\bf R}$-module functor between ${\bf R}$-modules ${\bf M}$ and ${\bf N}$ is a functor $F:{\bf M}\to {\bf N}$ such that $F(X\oplus Y)\cong F(X)\oplus F(Y)$ (natural in $X$ and $Y$) and $F(A\otimes X)\cong A\otimes F(X)$ (natural in $A$ and $X$).

Given $n\in \natu$, consider now the product category $\tsf{Vect}^n$; its objects and maps are $n$-tuples of vector spaces and maps, respectively. The module structure is provided by the operations
$$
\begin{aligned}
(V_1,\dots ,V_n)\oplus (W_1,\dots ,W_n) &= (V_1\oplus W_1,\dots ,V_n\oplus W_n), \\
V\otimes (V_1,\dots ,V_n) &= (V\otimes V_1,\dots ,V\otimes V_n).\\
\end{aligned}
$$
Any object $(V_1,\dots ,V_n)$ can be decomposed, just like vectors in euclidean $n$-space, in the following way
$$(V_1,\dots ,V_n)=(V_1\otimes \comp_1)\oplus \cdots \oplus (V_n\otimes \comp_n),$$
where $\comp_i$ is the vector which $i$-th entry is equal to $\comp$ and all others equal to the trivial vector space. Hence, any $\tsf{Vect}$-module functor can be determined on objects by its values in each $\comp_i$,
\begin{equation}\label{module_functor}
F(V_1,\dots ,V_n)\cong (V_1\otimes F(\comp_1))\oplus \cdots \oplus (V_n\otimes F(\comp_n)).
\end{equation}

We can define some more structure to this constructions by introducing maps between maps or 2-arrows. Given two ${\bf R}$-modules ${\bf M}$ and ${\bf N}$ and module functors $F,G:{\bf M}\to {\bf N}$, we define a 2-morphism $\theta :F\to G$ as a natural transformation. This provides the category of ${\bf R}$-modules with a structure of 2-category.

\begin{defi}\label{2-vs-defi}
  A $\tsf{Vect}$-module category ${\bf V}$ is called a \emph{2-vector
    space} if it is $\tsf{Vect}$-module equivalent to the
  product $\tsf{Vect}^n$ for some natural number $n$. In other words, ${\bf V}$ is a 2-vector space if there exists a natural number $n$ and a $\tsf{Vect}$-module functor ${\bf V}\to \tsf{Vect}^n$ which is also an equivalence of categories. 
\end{defi}

The proof of the following theorem can be found in \cite{kn:kv}.

\begin{theorem}\label{prop-10}
  If $F: \tsf{Vect}^n \rightarrow \tsf{Vect}^m$ is an equivalence,
  then $n =m$.
\end{theorem}

By the previous result, the number $n$ in definition~\ref{2-vs-defi} is well defined and it is
called the \emph{rank} of the 2-vector space ${\bf V}$.

The 2-vector space $\tsf{Vect}^n$ will play, in this categorical
setting, the role that complex $n$-space $\comp^n$ plays in linear algebra.
We will denote by $2\tsf{Vect}$ the (2-)category of
2-vector spaces of finite rank.

Morphisms between 2-vector spaces can be characterized in a similar way as linear maps between vector spaces. To see this, consider first an $m\times n$ matrix
$$A=\begin{pmatrix}
V_{11} & \cdots & V_{1n} \\
\vdots & \ddots & \vdots \\
V_{m1} & \cdots & V_{mn} \\
\end{pmatrix}.
$$
Then, for an object $V:=(V_1,\dots ,V_n) \in \tsf{Vect}^n$, the product
$$AV=\left (\sum_jV_{1j}\otimes V_j,\dots ,\sum_jV_{mj}\otimes V_j\right )$$
is a well defined object of the category $\tsf{Vect}^m$; given now a map $f:=(f_1,\dots ,f_n):V\to W$, where $W:=(W_1,\dots ,W_n)$, there exists an induced map $Af:AV\to AW$ given by
$$Af=\left (\sum_j\opnm{id}_{1j}\otimes f_j,\dots ,\sum_j\opnm{id}_{mj}\otimes f_j\right ),$$
where $\opnm{id}_{ij}:V_{ij}\to V_{ij}$ is the identity map. Moreover, the correspondence
$$
\begin{aligned}
V &\mapsto AV \\
f &\mapsto Af \\
\end{aligned}
$$
is a $\tsf{Vect}$-module functor $\tsf{Vect}^n\to \tsf{Vect}^m$ and hence a morphism of 2-vector spaces. Composition of such morphisms is given by usual multiplication of matrices, and two matrices $A=(V_{ij})$ and $B=(W_{ij})$ of the same size are naturally isomorphic if and only if $V_{ij}$ is isomorphic to $W_{ij}$ for each $i,j$. 

Note that equation \eqref{module_functor} readily implies that a morphism $F:\tsf{Vect}^n\to \tsf{Vect}^m$ is naturally isomorphic to the $m\times n$ matrix with columns given by $F(\comp_1),\dots ,F(\comp_n)$. For a morphism $F:{\bf V}\to {\bf W}$ between 2-vector spaces, if $u:{\bf V}\to \tsf{Vect}^n$ and $v:{\bf W}\to \tsf{Vect}^m$ are equivalences with inverses $\widetilde{u}$ and $\widetilde{v}$ respectively, then $vF\widetilde{u}$ is naturally isomorphic to a matrix $A$, and hence $F$ can be represented as $\widetilde{v}Au$ for some matrix $A$.

Let now $A=(V_{ij})$ be an $n\times n$ matrix which is an equivalence $\tsf{Vect}^n\to \tsf{Vect}^n$, and let $B=(W_{ij})$ be an inverse. As the identity morphism of $\tsf{Vect}^n$ can be represented by the ``scalar'' matrix $\comp \opnm{Id}$, we have natural isomorphisms $AB\cong \comp \opnm{Id} \cong BA$. Taking dimensions, form the matrices $d(A):=(\dim V_{ij})$ and $d(B)=(\dim W_{ij})$. Then, as the dimension matrices has natural entries, necessarily $\det d(A)=\pm 1$. But not every matrix satisfying this property is in fact an equivalence, and this is the main problem behind the short supply of equivalences $\tsf{Vect}^n\to \tsf{Vect}^n$. For example, take $n=2$ and consider the morphisms given by the matrices
$$A_k=
\begin{pmatrix}
\comp & \comp \\
\comp^{k-1} & \comp^k \\
\end{pmatrix}.
$$
Then $d(A_k)=\left (\begin{smallmatrix} 1 & 1 \\ k-1 & k \\ \end{smallmatrix} \right )$ and $\det d(A_k)=1$. But, no matter which $k\in \natu$ we choose, there is no inverse for $A_k$, and hence it is not an equivalence of 2-vector spaces. The example below explicitly shows the scarcity of equivalences for $n=2$.

\begin{ej}
Let $A=(V_{ij})$ be an autoequivalence of $\tsf{Vect}^2$ and $B=(W_{ij})$ and inverse. Let $a_{ij}:=\dim V_{ij}$, $b_{ij}:=\dim W_{ij}$ and then $d(A)=(a_{ij})$ and $d(B)=(b_{ij})$. From the natural isomorphisms $AB\cong \comp \opnm{Id}\cong BA$ we deduce that the following equations must hold
\begin{equation}\label{eq_ej_2}
a_{i1}b_{1j}+a_{i2}b_{2j}=\delta_{ij},
\end{equation}
for $i,j=1,2$. In particular, the matrix $d(B)$ is the inverse of the matrix $d(A)$; hence
$$d(B)=\varepsilon \begin{pmatrix}
a_{22} & -a_{12} \\
-a_{21} & a_{11} \\
\end{pmatrix},
$$
where $\varepsilon =\pm 1$ is the determinant of $d(A)$. If $\varepsilon =1$, then necessarily $a_{12}=a_{21}=0$; this fact together with equation \eqref{eq_ej_2} yields
$$a_{ii}b_{ii}=1$$
for $i=1,2$, and then $a_{11}=a_{22}=1$. For $\varepsilon =-1$ we obtain $a_{ii}=0$ for $i=1,2$ and $a_{12}=a_{21}=1$. Thus, the only equivalences $\tsf{Vect}^2\to \tsf{Vect}^2$ (up to isomorphism) have the form
$$\begin{pmatrix}
\comp & 0 \\
0 & \comp \\
\end{pmatrix}\quad , \quad
\begin{pmatrix}
0 & \comp \\
\comp & 0 \\
\end{pmatrix}.
$$
\end{ej}



%%%%%%%%%%%%%%%%%%%%%%%%%%%%%%%%
\subsubsection{2-Vector Bundles}

The notion of 2-vector bundle (of rank 1) was introduced by Brylinski in \cite{brylinski:_catvb} as a way of describing some cohomology classes associated to symplectic manifolds in terms of 2-vector spaces (as an alternative to gerbes). His definition resembles the definition of the sheaf of sections of a vector bundle. Another notion of 2-vector bundle was proposed by Baas, Dundas and Rognes ({\sc bdr}) in \cite{bdr:_2vb} searching for a geometric description of elliptic cohomology. Their definition, which resembles the definition of cocycles for a vector bundles, generalizes the one given by Brylinski.

For our purposes, a generalization to higher ranks of Brylinski's definition is given and in the end of chapter \ref{local_description}, a connection with {\sc bdr} 2-vector bundles is stablished.

We now briefly recall the definition of additive category. More details are given at section \ref{sheaf_boundary_conditions}.

\begin{defi}
A category ${\bf M}$ is called \emph{additive} if the following conditions hold:
\begin{enumerate}
\item given $X,Y\in {\bf M}$, $\opnm{Hom}_{\bf X}(X,Y)$ is an abelian group;
\item the composition pairing $\opnm{Hom}_{\bf X}(X,Y)\times \opnm{Hom}_{\bf X}(Y,Z)\to \opnm{Hom}_{\bf X}(X,Z)$ is bilinear;
\item There exists an object $0\in {\bf M}$ which is both initial and terminal (a \emph{zero} object) and
\item there exists a product $(X_1,X_2)\mapsto X_1\oplus X_2$.\footnote{Recall that the object $X_1\oplus X_2$ is a \emph{product} of $X_1$ and $X_2$ in a category ${\bf M}$ if there exists (projections) $\pr_i:X_1\oplus X_2\to X_i$ ($i=1,2$) such that for each object $Y$ and arrows $f_1:Y\to X_1$ and $g:Y\to X_2$ there exists a unique map $f:Y\to X_1\oplus X_2$ and $\pr_if=f_i$ for each $i$. This product is unique, up to isomorphism.}
\end{enumerate}
\end{defi}

We will say that the category ${\bf R}$ acts on the category ${\bf M}$ if there exists a functor ${\bf R}\times {\bf M}\longrightarrow {\bf M}$. If ${\bf R}\to{\bf B}\leftarrow {\bf M}$ are fibred categories or stacks over ${\bf B}$, then an action of ${\bf R}$ on ${\bf M}$ is a morphism of fibred categories ${\bf R}\times_{{\bf B}}{\bf M}\longrightarrow {\bf M}$. According to the extra structure enjoyed by ${\bf M}$, we will ask the action to preserve such structure. For instance, if the category ${\bf M}$ is additive, then we should have a natural distributivity isomorphism $A\cdot (X\oplus Y)\cong A\cdot X\oplus A\cdot Y$, plus other properties involving ${\bf 1}$ and ${\bf 0}$.

The definition of 2-vector bundle given by Brylinski in \cite{brylinski:_catvb} reads as follows.

\begin{defi}
A fibred category ${\bf M}\to \tsf{Op}(M)$ is said to be a \emph{2-vector bundle of rank $1$ over $M$} if the following conditions hold:
\begin{enumerate}
\item For each open subset $U\subset M$, the fibre ${\bf M}(U)$ is an additive category.
\item There exists an action $(E,X)\mapsto E\cdot X$ of the (fibred) category $[\tsf{Vect},M]$ on ${\bf M}$.
\item Given any $x\in M$, there exists an open neighborhood $U\ni x$ and an object $X_U\in {\bf M}(U)$ (called a \emph{local generator}) such that the functor $\tsf{Vect}(U)\to {\bf M}(U)$ given by $E\mapsto E\cdot X_U$ is an equivalence of categories, where $\cdot$ denotes the action.
\item ${\bf M}\to \tsf{Op}(M)$ is a stack.
\end{enumerate}
\end{defi}

We now extend the definition to higher ranks. Instead of the (fibred) category of vector bundles, we make use of the (fibred) category of locally-free sheaves over $M$; see example \ref{fcex_9}.

\begin{defi}\label{2bundle_n}
A fibred category ${\bf M}\to \tsf{Op}(M)$ is said to be a \emph{2-vector bundle of rank $n$ over $M$} if the following conditions hold:
\begin{enumerate}
\item For each open subset $U\subset M$, the fibre ${\bf M}(U)$ is an additive category.
\item There exists an action $(\scr{M},X)\mapsto \scr{M}\cdot X$ of the (fibred) category $\underline{\tsf{LF}}_{\scr{O}_M}$ of locally free $\scr{O}_M$-modules on ${\bf M}$ (for each $U$, $\underline{\tsf{LF}}_{\scr{O}_M}(U)$ is given by $\tsf{LF}_{\scr{O}_U}$).
\item Given any $x\in M$, there exists an open neighborhood $U\ni x$ and objects $X_1,\dots X_n$ in ${\bf M}(U)$ (called \emph{local generators}) such that the functor $\tsf{LF}^n_{\scr{O}_U}\to {\bf M}(U)$ given by
$$(\scr{M}_1,\dots ,\scr{M}_n)\longmapsto \scr{M}_1\cdot X_1\oplus \cdots \oplus \scr{M}_n\cdot X_n$$
is an equivalence of categories.
\item ${\bf M}\to \tsf{Op}(M)$ is a stack.
\end{enumerate}
\end{defi}

\begin{obs}
Note that the local equivalence of the previous definition preserves both the action and the additive structure; that is, if $\Phi$ is such an equivalence, $\scr{L}\in \tsf{LF}_{\scr{O}_U}$ and $\scr{M},\scr{N} \in \tsf{LF}^n_{\scr{O}_U}$, then
$$\Phi \left ((\scr{L}\otimes \scr{M})\oplus \scr{N}\right )\cong \left (\scr{L}\otimes \Phi (\scr{M})\right )\oplus \Phi (\scr{N}).$$
\end{obs}

\begin{ej}
Let $M=\{x\}$ be a one-point space. A 2-vector bundle of rank $n$ over $M$ is then an additive category ${\bf M}$ equivalent to the category $\tsf{LF}^n_{\scr{O}}$. As $\scr{O}(M)\cong \comp$, then ${\bf M}$ is equivalent to the n-fold product of the category of $\comp$-modules; that is, it is a 2-vector space (of rank $n$). 
\end{ej}

\begin{ej}
We have a well defined action (the tensor product) of vector bundles on twisted bundles, obtained by considering vector bundles as twisted bundles with no twisting. Proposition \ref{equiv_tvb_vect} shows that if $\mathbb{L}$ is a $\lambda$-twisted vector bundle, then the assignment $E\mapsto E\otimes \mathbb{L}$ defines an equivalence of categories. Thus, $[\tsf{TVB}_{\lambda},M]$ is a 2-vector bundle of rank 1.
\end{ej}

The following result shall be useful later.

\begin{proposition}
Let $\Phi :\tsf{LF}^n_{\scr{O}_M}\to \tsf{LF}^m_{\scr{O}_M}$ be a functor which preserves the action and the additive structure. Then there exists an $m\times n$ matrix $A:=(\scr{M}_{ij})$ of $\scr{O}_M$-modules such that $\Phi$ is naturally isomorphic to multiplication by $A$.
\end{proposition}

The proof is completely analogous to the one for 2-vector spaces. Moreover, this kind of morphisms share with 2-vector spaces the same shortage of equivalences.

Before introducing Baas-Dundas-Rognes ({\sc bdr}) 2-vector bundles, we need the following

\begin{defi}
An \emph{ordered open cover} of a topological space $M$ is a collection $\mathfrak{U}=\{U_\alpha \}_{\alpha \in A}$ of open subsets of $M$ indexed by a poset $A$ such that
\begin{enumerate}
\item $M=\bigcup_{\alpha \in A}U_\alpha$ and
\item the partial ordering on $A$ restricts to a total ordering on each finite subset $\{\alpha_1,\dots ,\alpha_k\}$ such that the intersection $U_{\alpha_1\dots \alpha_k}$ is non-empty.
\end{enumerate}
\end{defi}

In particular, note that this definition is fulfilled by manifolds, as they admit countable, locally-finite open covers (which can be turned into ordered covers with $A=\natu$); for more details on this topic, the reader is referred to \cite{lee:_ism}.

\begin{defi}\label{bdr_2bundle}
Let $A$ be a poset and $\mathfrak{U}=\{U_{\alpha }\}_{\alpha \in A}$ an ordered open cover of a topological space $M$. A \emph{Bass-Dundas-Rognes 2-vector bundle} (\emph{{\sc bdr} 2-vector bundle} for short) \emph{of rank} $n$ is an $n\times n$-matrix $E^{\alpha \beta}:=(E_{ij}^{\alpha \beta})$ of vector bundles over $U_\alpha \cap U_\beta =U_{\alpha \beta}$ (for each $\alpha < \beta$) subject to the following conditions:
\begin{enumerate}
\item $\det \left (\opnm{rk}E_{ij}^{\alpha \beta}\right )=\pm 1$.
\item For $\alpha <\beta <\gamma$ in $A$ and $U_{\alpha \beta \gamma }\neq \emptyset$, we have isomorphisms
$$\phi^{\alpha \beta \gamma}_{ik}:\bigoplus_jE_{ij}^{\alpha \beta}\otimes E_{jk}^{\beta \gamma}\stackrel{\cong}{\longrightarrow} E_{ik}^{\alpha \gamma}.$$
As for morphisms of 2-vector spaces, this condition can also be expressed in matrix form $\phi^{\alpha \beta \gamma}:E^{\alpha \beta}E^{\beta \gamma }\cong E^{\alpha \gamma }$.
\item For $\alpha <\beta <\gamma <\delta$ with $U_{\alpha \beta \gamma \delta}\neq \emptyset$, the following diagram
of bundles over $U_{\alpha \beta \gamma \delta}$ should commute
$$\xymatrix{
E^{\alpha \beta}(E^{\beta \gamma }E^{\gamma \delta }) \ar[rr] \ar[d] & & (E^{\alpha \beta }E^{\beta \gamma })E^{\gamma \delta } \ar[d] \\
E^{\alpha \beta }E^{\beta \delta } \ar[r] & E^{\alpha \delta } & E^{\alpha \gamma }E^{\gamma \delta }, \ar[l] }
$$
where the top arrow is the associativity isomorphism derived from the associativity of the tensor product of vector bundles and the other arrows are defined from the isomorphisms of the previous item.
\end{enumerate}
\end{defi}

We shall not be concerned with a general description of the relationship between the two previous definitions. In chapter \ref{local_description} we shall obtain a 2-vector bundle in the sense of definition \ref{2bundle_n} and then, using this 2-bundle, construct a {\sc bdr} 2-vector bundle.


\clearpage

{\small
%%%%%%%%%%%%%%%%%%%%%%%%%%%%%%%%%%%%%%%%%
%%%%%%%%%%%%%%%%%%%%%%%%%%%%%%%%%%%%%%%%%
\section{Resumen del Cap\'itulo \ref{bs}}

En este primer cap\'itulo se introducen los objetos que representan la columna vertebral de esta tesis, que son los fibrados vectoriales y los haces por un lado, una versi\'on categorificada de estos \'ultimos (los \emph{stacks}) y los fibrados vectoriales torcidos (\emph{twisted vector bundles}). A continuaci\'on describimos en forma breve el contenido completo de este cap\'itulo.

%%%%%%%%%%%%%%%%%%%%%%%%%%%%%%%%%
\subsection{Fibrados Vectoriales}

Se da un tratamiento conciso pero lo suficientemente abarcativo sobre los fibrados vectoriales complejos de rango finito. A grandes rasgos, un fibrado vectorial sobre una variedad suave $M$ consiste de una variedad suave $E$ junto con una proyecci\'on $\pi :E\to M$ para el cual las fibras $E_x:=\pi^{-1}(\{x\})$ son $\comp$-espacios vectoriales de dimensi\'on finita y existe un cubrimiento abierto $\mathfrak{U_i}$ para el cual se verifica la condici\'on de trivialidad local: para cada $U_i\in \mathfrak{U}$ se tiene un difeomorfismo $h_i:E|_{U_i}\stackrel{\cong}{\longrightarrow}U_i\times \comp^n$tal que $\pr_1 h_i=\pi$, donde $E|_U:=\pi^{-1}(U)$.
 Informaci\'on suficiente para describir a estos objetos se encuentra en los llamados cociclos, que forman una familia $\{g_{ij}\}$ de mapas $g_{ij}:U_{ij}\to \operatorname{GL}_n(\comp )$ que verifican
\begin{enumerate}
\item $g_{ii}=1$,
\item $g_{ji}=g_{ij}^{-1}$ y
\item $g_{ij}g_{jk}=g_{ik}$ sobre $U_{ijk}=U_i\cap U_j\cap U_k$.
\end{enumerate}
La ``informaci\'on suficiente'' a la que se hac\'ia referencia antes proviene de que a partir de un cubrimiento $\mathfrak{U}$ y una familia de cociclos definida en las intersecciones de los elementos de $\mathfrak{U}$ podemos definir un \'unico fibrado (salvo isomorfismo) que es isomorfo a un producto sobre los $U\in \mathfrak{U}$.

A continuaci\'on se definen operaciones b\'asicas entre fibrados, describiendo el \emph{pullback} por una aplicaci\'on suave, la suma directa externa, la suma directa o suma de Whitney, el fibrado dual, el producto tensorial, el fibrado de homomorfismos (y la relacion entre estos \'ultimos), los n\'ucleos e im\'agenes de morfismos de fibrados, prestando especial atenci\'on a los correspondientes a morfismos idempotentes.

%%%%%%%%%%%%%%%%%%%%%%%%%%%%%%%%%
\subsection{Haces}

El objetivo principal al introducir haces en este trabajo es mostrar (para una clase particular de estos) su \'intima relaci\'on con los fibrados.

Se define primero la noci\'on de prehaz sobre un espacio $M$ a valores en una categor\'ia ${\bf X}$ como un funtor contravariante $\scr{P}:\tsf{Op}(M)\to {\bf X}$. Las propiedades que hacen de un prehaz un haz tienen que ver con el pasaje de lo local a lo global: mas precisamente, si se tiene definida una familia de secciones $\sigma_i\in \scr{P}(U_i)$ (donde $\mathfrak{U}=\{U_i\}$ es un cubrimiento abierto de un cierto $U\subset M$) tales que $\sigma_i=\sigma_j$ en las intersecciones no vac\'ias $U_{ij}$, entonces dichas secciones se pueden ``pegar'', en el sentido que existe una \'unica secci\'on $\sigma \in \scr{P}(U)$ tal que $\sigma |_{U_i}=\sigma_i$.

Otra construcci\'on importante es la completaci\'on de un prehaz, es decir, dado un prehaz $\scr{P}$, la completaci\'on $\scr{P}^+$ es un haz con los mismos \emph{stalks} que el prehaz $\scr{P}$ (el \emph{stalk} de un prehaz sobre $x\in M$ viene dada por $\scr{P}_x=\underset{U\ni x}{\opnm{colim}}\scr{P}(U)$). Esta construcci\'on se basa principalmente en tomar las funciones $U\to \bigsqcup_{x\in U}\scr{P}_x$ que son continuas, donde $\bigsqcup$ indica uni\'on disjunta.

El siguiente paso es estudiar los morfismos de prehaces y haces. Se da un tratamiento completo, llegando a distintas caracterizaciones para morfismos inyectivos, sobreyectivos y biyectivos.

Asi como para los fibrados, para los haces tambi\'en se estudian importantes construcciones que permiten obtener haces (o prehaces en ciertos casos) de cierto(s) haz(haces) dado(s). Particular atenci\'on se le da las im\'agenes directa e inversa por una funci\'on continua y a las propiedades de adjunci\'on entre ellas.

A continuaci\'on se definen los haces localmente libres, los cuales resultar\'an estar \'intimamente relacionados a los fibrados vectoriales. Dado un haz de anillos $\scr{O}$ sobre $M$, un $\scr{O}$-m\'odulo localmente libre es un haz $\scr{M}$ sobre $M$ tal que para cada abierto $U$ de $M$, $\scr{M}(U)$ es un $\scr{O}(U)$-m\'odulo y tal que cada $x\in M$ tiene una vecindad $U\ni x$ para la cual el haz $\scr{M}$ restringido al abierto $U$ es isomorfo a $\scr{O}^n(U):=\scr{O}(U)\times \cdots \times \scr{O}(U)$. Se estudian propiedades de dichos m\'odulos y se definen el m\'odulo dual, el m\'odulo de homomorfismos y el producto tensorial, adem\'as de la suma. Tambi\'en, fundamental para establecer la relaci\'on entre m\'odulos y fibrados, se introduce y se estudia la noci\'on de fibra sobre un punto $x\in M$.

Los espacios anillados proveen el marco adecuado para definir dos construcciones de fundamental importancia, como son las im\'agenes directa e inversa en el contexto de los m\'odulos localmente libres, adem\'as de permitir dar una definici\'on general de espacio tangente sin tener que recurrir a la maquinaria del an\'alisis. Un espacio anillado es un par $(M,\scr{O})$ donde $M$ es un espacio topol\'ogico y $\scr{O}$ es un haz de anillos sobre $M$, llamado el haz de estructura. El ejemplo can\'onico a tener en mente es, por ejemplo, un espacio topol\'ogico $M$ y $\scr{O}$ es el haz de funciones continuas $\scr{O}(U)=C(U)=\{f:U\subset M \to \re \; | \; f \text{es continua}\}$. En este contexto, sea $f:(M,\scr{O}_M)\to (N,\scr{O}_N)$ una funci\'on continua entre espacios anillados y supongamos que tenemos un $\scr{O}_M$-m\'odulo localmente libre sobre $M$ y otro $\scr{N}$ sobre $N$. Entonces podemos definir la imagen inversa $f^*\scr{N}$, que resulta un $\scr{O}_M$-m\'odulo localmente libre sobre $M$ y la imagen directa $f_*\scr{M}$, que es un $\scr{O}_N$-m\'odulo localmente libre sobre $N$.

Dado un fibrado $E\to M$, el haz de secciones de $E$ es un $\scr{O}$-m\'odulo localmente libre (donde $\scr{O}$ es el haz de funciones suaves sobre $M$) $\Gamma_E$ sobre $M$ definido como sigue: para $U\subset M$, $\Gamma_E(U)$ es el conjunto de funciones continuas $\sigma :U\to E$ tales que $\sigma (x)\in E_x$. Teniendo a nuestra disposici\'on la maquinaria de los haces, se demuestra luego la equivalencia entre fibrados vectoriales y m\'odulos localmente libres.

\medskip
{\bf Teorema.}
{\it Dado un $\scr{O}$-m\'odulo localmente libre sobre $M$, existe un \'unico (salvo isomorfismo) fibrado vectorial $E\to M$ tal que los haces $\Gamma_E$ y $\scr{M}$ son isomorfos.}
\medskip

En t\'erminos functoriales, del resultado anterior se desprende que la correspondencia $E\mapsto \Gamma_E$ define una equivalencia entre la categor\'ia de fibrados vectoriales y la de $\scr{O}$-m\'odulos localmente libres.

%%%%%%%%%%%%%%%%%%%%%%%%%%%%%%%%%
\subsection{\'Algebras de Azumaya}

La definici\'on de \'algebra de Azumaya (en el contexto de los haces)
fue introducida por A. Grothendieck. Considerando fibrados, decimos
que $E\to M$ es un \'algebra de Azumaya si las fibras $E_x$ son
$\comp$-\'algebra y para cada $x\in M$ se tiene una vecindad $U\ni x$
para la cual se tiene una trivializaci\'on local
$E|_U\cong U\times \opnm{M}_k(\comp )$ que preserva las estructuras de
\'algebras. Esta clase de \'algebras est\'an \'intimamente
relacionadas con los fibrados vectoriales torcidos (\emph{twisted
  vector bundles}). Un fibrado torcido sobre $M$ es una upla
$\mathbb{E}=(\mathfrak{U},U_i\times \comp^n,g_{ij},\lambda_{ij})$,
donde $\mathfrak{U}=\{U_i\}$ es un cubrimiento abierto de $M$ y la
familia $g_{ij}$ verifica $g_{ij}g_{jk}=\lambda_{ijk}g_{ik}$, donde
$(\lambda_{ijk})$ es un 2-cociclo de $\check{\text{C}}$ech. En
particular, cuando $\lambda_{ijk}=1$, el fibrado torcido es en
realidad un fibrado usual.

Asi como se hizo con los fibrados, definimos a continuaci\'on varios
ejemplos de fibrados contruidos a partir de fibrados dados: el
pullback, la suma (para fibrados con iguales 2-cociclos), el fibrado
dual, el producto tensorial y el fibrado de homomorfismos.

Definimos tambi\'en morfismos de fibrados torcidos e introducimos la
categor\'ia $\tsf{TVB}(M)$ de fibrados torcidos sobre $M$,
caracterizando a los isomorfismos en t\'erminos de cociclos. Esto
permite mostrar que los 2-cociclos de dos fibrados isomorfos deben
coincidir.

Los siguientes p\'arrafos se encargan de estudiar las propiedades de
las operaciones definidas anteriormente, como asociatividad y
conmutatividad, entre otras, para pasar luego a describir las
relaciones entre las categor\'ias de fibrados vectoriales y las de
fibrados torcidos.

La relaci\'on entre estos fibrados y las \'algebras de Azumaya se
describe a continuaci\'on: dada un \'algebra de Azumaya $A$, existe un
fibrado torcido $\mathbb{E}$ tal que $A\cong \opnm{END}(\mathbb{E})$,
donde $\opnm{END}(\mathbb{E})$ denota el fibrado de homomorfismos
$\mathbb{E}\to \mathbb{E}$ (que es un fibrado en el sentido usual). A
continuaci\'on se demuestran varias propiedades que llevan a demostrar
la equivalencia entre la categor\'ia de fibrados torcidos cuyos
morfismos $\phi :\mathbb{E}\to \mathbb{F}$ se identifican con
$\lambda \phi$, siendo $\lambda$ un 0-cociclo, y la categor\'ia de
\'algebras de Azumaya cuyos morfismos son los isomorfismos.

A partir de las propiedades de las operaciones entre fibrados
torcidos, definimos una operaci\'on en el conjunto de clases de
isomorfismo de fibrados de l\'inea torcidos a partir del producto
tensorial y luego probamos que se obtiene un grupo
$\opnm{Tor}\opnm{H}^3(M;\ent )$-graduado que contiene al grupo de
Picard, y que llamamos el grupo de Picard torcido.

%%%%%%%%%%%%%%%%%%%%%%%%%%%%%%%%%%%%%%%%%%%%%%
\subsection{Categor\'ias Fibradas y 2-Fibrados}

En esta \'ultima secci\'on del presente cap\'itulo se introduce la noci\'on de 2-fibrado vectorial de Baas-Dundas-Rognes ({\sc bdr}), que necesitan de varias construcciones previas.

En primer lugar, la de categor\'ia fibrada, que es una categorificaci\'on de la noci\'on de prehaz: una categor\'ia fibrada sobre un espacio $M$ puede verse como una familia de categor\'ias $\{\scr{C}_U\}$ que admite pullbacks, donde $U$ recorre los abiertos de $M$. A grandes rasgos, esto significa que si $V\subset U$ es una inclusi\'on entre abiertos de $M$ y $\alpha \in \scr{C}_U$, entonces la restricci\'on $\alpha |_V\in \scr{C}_V$.\footnote{La definici\'on de categor\'ia fibrada es mucho mas general; en lugar de la categor\'ia de abiertos de un espacio $M$ se puede definir una categor\'ia fibrada en t\'erminos de un \emph{sitio de Grothendieck}. Nos restringimos al caso de los abiertos de $M$ dado que el tratamiento general resultar\'ia extenso e innecesario para este trabajo.} Se dedica considerable trabajo en dar las definiciones equivalentes de categor\'ia fibrada como sus propiedades b\'asicas y abundantes ejemplos. En particular, demostramos que la categor\'ia de fibrados torcidos goza de la propiedad de ser fibrada.

A continuaci\'on se definen los \emph{stacks}, que son la versi\'on categorificada de los haces. De una manera an\'aloga a con los prehaces y los haces, una categor\'ia fibrada resulta un \emph{stack} cuando los datos locales que coinciden en las intersecciones se pueden pegar en un objeto global. Pero a diferencia de los haces, en este caso esta exigencia se aplica no solo a los objetos sino tambi\'en a los morfismos.

Asi como para las categor\'ias fibradas, se describen numerosos ejemplos, continuando con los dados con las categor\'ias fibradas. En particular, tambi\'en probamos que la categor\'ia de fibrados torcidos cumple estas propiedades y resulta ser un \emph{stack}.

Otra estructura importante y necesaria para construir los 2-fibrados son los 2-espacios vectoriales (de rango finito) que, como en los casos anteriores, resulta un tipo de categorificaci\'on de un espacio vectorial. La versi\'on que usamos es la definida por M. Kapranov y V. Voevodsky. El ejemplo t\'ipico y m\'as importante, en el sentido que todo 2-espacio vectorial es equivalente a el, es el del producto $\tsf{Vect}^n$ de la categor\'ia de espacios vectoriales complejos de dimensi\'on finita. Un 2-vector es un objeto de esta categor\'ia, o sea una $n$-upla de espacios vectoriales complejos de dimensi\'on finita $(V_1,\cdots ,V_n)$. La suma de elementos est\'a definida (asi como lo est\'a la suma en $\comp^n$) componente a componente, por medio de la suma directa: si $(V_i)$ y $(W_i)$ son dos $n$-uplas de espacios vectoriales, entonces $(V_i)\oplus (W_i):=(V_i\oplus W_i)$. Para el producto por un ``escalar'' (que en este caso es un espacio vectorial, de ah\'i que los 2-espacios vectoriales reciban tambi\'en el nombre de $\tsf{Vect}$-m\'odulos) se tiene $V\otimes (V_i):=(V\otimes V_i)$. Luego de las definiciones b\'asicas, se analizan varias propiedades de los 2-espacios vectoriales, llegando particularmente al hecho de que las equivalencias $\tsf{Vect}^n\to \tsf{Vect}^n$ que preservan las estructuras definidas son muy escasas. Sirva como ejemplo que para el caso $n=2$, las \'unicas equivalencias (salvo isomorfismo natural) vienen dadas por
$$\begin{pmatrix}
\comp & 0 \\
0 & \comp \\
\end{pmatrix}\quad , \quad
\begin{pmatrix}
0 & \comp \\
\comp & 0 \\
\end{pmatrix}.$$
El motivo principal detr\'as de esto es la no existencia de espacios vectoriales de dimensi\'on negativa.

La primer definici\'on de 2-fibrado vectorial (de rango 1) se debe a J.L Brylinski y fue dada con el objetivo de describir ciertas clases de cohomolog\'ia de variedades simpl\'ecticas. Un 2-fibrado vectorial se define como un \emph{stack} $\{\scr{C}_U\}$ de categor\'ias aditivas para el cual
\begin{itemize}
\item se tiene una acci\'on $\tsf{Vect}(U)\times \scr{C}_U\to \scr{C}_U$ de la categor\'ia de fibrados vectoriales de rango finito para cada $U$ y
\item cada $x\in M$ tiene una vecindad $U\ni x$ para la cual existe un objeto $\alpha_U\in \scr{C}_U$ tal que la correspondencia $\tsf{Vect}(U)\to \scr{C}_U$ dada por $E\mapsto E\cdot \alpha_U$ (acci\'on) es una equivalencia.
\end{itemize}
Extendemos esta definici\'on a 2-fibrados de rango $n$ considerando la categor\'ia fibrada de $\scr{O}_M$-m\'odulos localmente libres sobre $M$; en este caso, para cada $x$ se tiene una vecindad $U\ni x$ y objetos $\alpha_1,\dots ,\alpha_n\in \scr{C}_U$ tales que la aplicaci\'on
$$(\scr{M}_1,\dots ,\scr{M}_k)\longmapsto \scr{M}_1\cdot \alpha_1\oplus \cdots \oplus \scr{M}_k\cdot \alpha_n$$
es una equivalencia $\tsf{LF}_{\scr{O}_U}^k\to \scr{C}_U$, siendo $\tsf{LF}_{\scr{O}_U}$ la categor\'ia de $\scr{O}_U$-m\'odulos localmente libres.

El cap\'itulo finaliza con la definici\'on de 2-fibrado vectorial de {\sc bdr}. Las definiciones previas de 2-fibrado pueden considerarse como una versi\'on catego\'orica del haz de secciones de un fibrado. La correspondiente a {\sc bdr} considera cociclos en lugar de secciones: a grandes rasgos, un 2-fibrado de {\sc bdr} de rango $n$ sobre $M$ consiste de lo siguiente: un cubrimiento abierto $\mathfrak{U}=\{U_\alpha \}$ de $M$ y matrices $E^{\alpha \beta}:=(E^{\alpha \beta}_{ij})_{i,j=1,\dots ,n}$ de fibrados definidos sobre $U_{ij}$ tales que
\begin{itemize}
\item $\det \left (\opnm{rk}E^{\alpha \beta}_{ij}\right )=\pm 1$ y
\item se tiene un isomorfismo $E^{\alpha \beta}E^{\beta \gamma}\cong E^{\alpha \gamma}$,
\end{itemize}
donde el producto de las matrices se hace de la manera usual, reemplazando la suma de entradas por la suma directa y el producto por el producto tensorial.

}
%%% Local Variables:
%%% mode: latex
%%% TeX-master: "master"
%%% End:




% Chapter 2
\chapter{Frobenius Structures and Field Theories}
\label{fsfts}

\vspace{250pt}

The aim of this chapter is to introduce the notion of open-closed topological quantum field theory as well as a characterization for them due to G. Moore and G. Segal. These field theories generalize closed topological field theories by considering also open strings. We shall first introduce closed topological field theories and the study their relationship with Frobenius algebras, which provide an algebraic characterization of these field theories. In view of this, we also recall some basic notions about algebras and then provide a concise description of Frobenius algebras over commutative rings. Following this, we provide some basic introduction to closed topological field theories and then describe, after defining them in detail, the characterization of open-closed theories.

We end this chapter introducing the notions of bundles of algebras (in particular, bundles of Frobenius algebras) and manifolds with multiplication and proving some basic results about them. As we will see, the information needed to define a closed topological field theory is encoded in a Frobenius algebra, and then manifolds for which their tangent bundle is a bundle of Frobenius algebras arises naturally when considering moduli spaces of such theories.






%%%%%%%%%%%%%%%%%%%%%%%%%%%%%%%%%%%%%%%%%%%%%%%%%%%%%%%%%%%%%%%%%%%%
%%%%%%%%%%%%%%%%%%%%%%%%%%%%%%%%%%%%%%%%%%%%%%%%%%%%%%%%%%%%%%%%%%%%
\section{Frobenius Algebras and Topological Quantum Field Theories}

%%%%%%%%%%%%%%%%%%%%%%%%%%%%%%%%%%%%%%%%%%%%%%%%%%%%%%%
\subsection{Quantum Field Theories}
\label{sec_frobalgtfts}

Several reasons led mathematicians into a search for a precise formulation of a field theory in mathematical terms. The first such definition is due to G. Segal \cite{segal:_cft}, who axiomatized Conformal Field Theories. Then, inspired by this earlier work, Atiyah made a similar contribution for Topological Theories \cite{atiyah:_tqft}. We shall first introduce the general definition and then focus on the 2-dimensional case, in which Frobenius algebras have a pre-eminent role. We shall give only rough ideas, referring the reader to the appropriate literature for details.

We first introduce a category which is essential for the definition of a Topological Field Theory ({\sc tft}). A thoroughly description of this category can be found in Kock's book \cite{kock:_frobenius}. Given a positive integer $D$, let $\tsf{Cob}(D)$ be the category whose objects are smooth, closed, oriented, $(D-1)$-dimensional manifolds; given two such manifolds $\Sigma_1,\Sigma_2$, a morphism $W:\Sigma_1\to \Sigma_2$ is given by an oriented cobordism (that is, the arrow $W$ is in fact a $D$-dimensional smooth, oriented manifold $W$ such that $\partial W=\Sigma_1 \sqcup \Sigma_2^-$; here, the minus superscript refers to the opposite orientation). There is another layer of structure, provided by maps between cobordisms; given two cobordisms $W,W':\Sigma_1\to \Sigma_2$, a morphism $f:W\to W'$ is a smooth map such that $f|_{\Sigma_i}$ is the identity for $i=1,2$.

An important feature of the category $\tsf{Cob}(D)$ is that it comes equipped with a product, given by the dijoint union. The identity map $\Sigma \to \Sigma$ is given by the cylinder $W=\Sigma \times I$.

\begin{defi}[\cite{atiyah:_tqft}, \cite{kock:_frobenius}]
Let $R$ be the ring $\re$ or $\comp$.\footnote{Atiyah also considers the case $R=\ent$.} A \emph{Topological Quantum Field Theory} ({\sc tqft} or just {\sc tft} for short) in dimension $D$ over the ground ring $R$ is given by a functor
$$Z:\tsf{Cob}(D)\longrightarrow \tsf{Vect}_R$$
from the cobordism category $\tsf{Cob}(D)$ to the category of (finite-dimensional) $R$-vector spaces and linear maps, subject to the following conditions:
\begin{enumerate}
\item If $W\cong W':\Sigma_1\to \Sigma_2$ are isomorphic cobordisms, then $Z(W)=Z(W')$ (``diffeomorphism'' here means ``orientation-preserving diffeomorphism'');
%\item 
\item $Z$ is multiplicative; that is, $Z(\Sigma_1\sqcup \Sigma_2)=Z(\Sigma_1)\otimes_RZ(\Sigma_2)$. This also applies to cobordisms: if $W$ is the disjoint union of $W'$ and $W''$, then $Z(W)=Z(W')\otimes Z(W'')$ and
\item $Z(\emptyset )=R$.
\end{enumerate}
\end{defi}

\begin{obs}
As $Z$ is a functor, note that the image of the cylinder $\Sigma \to \Sigma$ is the identity map $\text{id}:Z(\Sigma )\to Z(\Sigma )$.
\end{obs}

\begin{obs}
The restriction to finite-dimensional vector spaces does not exclude other cases, as one can show that the vector space $Z(\Sigma )$ is finite-dimensional, no matter which manifold $\Sigma \in \tsf{Cob}(D)$ we choose; see \cite{kock:_frobenius}, proposition 1.2.28.
\end{obs}

From now on we will consider 2-dimensional {\sc tft}s; that is, we will work with $D=2$.

The relationship between 2-dimensional {\sc tqft}s and Frobenius algebras has been well-known for experts, but a detailed proof of this interaction was not published until 1997 in Abrams' thesis \cite{abrams:_frobenius}. We shall now recall some basic general notions about algebras and later introduce Frobenius algebras, to end up with a description of the interaction between {\sc tft}s and this type of algebras.

%%%%%%%%%%%%%%%%%%%%%%%%%%%%%%%%%%%%%%%%%
\subsection{Frobenius Algebras}\label{sfa}

In the following paragraphs we shall be involved in giving a concise description of Frobenius algebras over commutative rings and over $\comp$ in particular. All algebras are assumed to be associative and artinian (for algebras over $\comp$, we also assume that $A$ is a finite-dimensional $\comp$-vector space). Recall that an \emph{artinian ring} is a ring which satisfies the descending chain condition (dcc). In this kind of rings, every prime ideal is also maximal; details of these facts can be found in \cite{kn:at-mc}.

%%%%%%%%%%%%%%%%%%%%%%%%%%%%%%%%%%
\subsubsection{Algebras Over $\comp$}

We shall begin with a discussion of some general properties of associative, finite dimensional $\comp$-algebras. We first consider the commutative case and then provide a brief discussion for noncommutative algebras.

Let $A$ be an $n$-dimensional complex vector space. Given any linear operator $f:A\to A$, we have a decomposition of $A$ into generalized eigenvector subspaces
$$A=\bigoplus_{i=1}^k\opnm{Ker} (f-\lambda_i)^{n_i},$$
where $\lambda_1,\dots ,\lambda_k$ are the eigenvalues of $f$. These subspaces $V_i:=\opnm{Ker} (f-\lambda_i)^{n_i}$ are also invariant under $f$ and, moreover, the operator $f-\lambda_i$ is nilpotent on $V_i$.

Let now $g:A\to A$ be an operator such that $gf=fg$, and consider the decomposition
$$A=\bigoplus_{i=1}^r\opnm{Ker} (g-\mu_i)^{m_i}.$$
Put $W_i:=\opnm{Ker} (g-\mu_i)^n$. We then have:
\begin{enumerate}
\item The subspaces $V_i$ are invariant under $g$: We need to check that, if $x\in V_i$, then so is $g(x)$. Assume that $x\in V_i$; then, as $g$ commutes with $f$, $g$ also commutes with $f-\lambda_i$ and thus with $(f-\lambda_i)^{n_i}$. We then have $(f-\lambda_i)^{n_i}(g(x))=g\left ((f-\lambda_i)^{n_i}(x)\right )=0$.
\item For each $i$, there exists an index $j$ such that $V_i=W_j$ (the proof of this fact relies on the spectral theorem, and we omit it).
\end{enumerate}

We conclude that no matter which operator commuting with $f$ we choose, the decomposition
\begin{equation}\label{gen_eigenspace}
A=\bigoplus_{i=1}^kV_i,
\end{equation}
is the same, up to order.

Assume now that our vector space $A$ is also an associative and commutative algebra with unit 1. In this case we have, for each $x\in A$, a multiplication operator $L_x:A\to A$, $L_x(y)=xy$. As $A$ is commutative, these operators commute with each other, and so the previous considerations apply. The algebra structure now lets us derive some other consequences.

For $x\in A$, let us denote by $\lambda_{x,i}$ the $i$-th eigenvalue corresponding to $L_x$. According to the decomposition \eqref{gen_eigenspace} we can define a correspondence $\Lambda_i:A\to \comp$ which assigns to $x\in A$ the eigenvalue $\lambda_{x,i}$. We let $A^*$ be the dual space.

\begin{lemma}
$\Lambda_i\in A^*$ and is a morphism of algebras for each $i=1,\dots ,k$.
\end{lemma}
\begin{proof}
Linearity is deduced from the equality $L_{\lambda x+\mu y}=\lambda L_x+\mu L_y$. Let now $z$ be an eigenvector for $L_{xy}$ with eigenvalue $\lambda_{xy}$, and assume that $z\in V_i$. Then, $z$ is also an eigenvector for $L_x$ and $L_y$, say corresponding to $\lambda_x$ and $\lambda_y$ respectively, and we can write
$$\lambda_{xy}z=L_{xy}(z)=L_x(L_y(z))=\lambda_x\lambda_y z;$$
hence, $\Lambda_i (xy)=\Lambda_i (x)\Lambda_i (y)$. As $L_1$ is the identity map, then we also have $\Lambda_i (1)=1$.
\end{proof}

By the direct sum decomposition \eqref{gen_eigenspace}, for each $i$ there exists a unique $e_i\in V_i$ such that
\begin{equation}\label{unit_decomp}
1=e_1+\cdots +e_k.
\end{equation}
We can thus write $e_i=e_1e_i+\cdots +e_i^2+\cdots +e_ke_i$. As the subspaces $V_j$ are invariant under every translation, we have that $L_{e_i}(e_j)=e_ie_j\in V_j$. As $V_i\cap V_j=\{0\}$, we have the following orthogonality relations
$$e_ie_j=\delta_{ij}e_i.$$
For $i=j$, the previous identity implies that each $e_i$ is idempotent.

\begin{proposition}
We have that $V_i=e_iA$ for each $i$; in particular, $V_i$ is an algebra with unit $e_i$.
\end{proposition}
\begin{proof}
It is clear that $e_iA\subset V_i$. Let now $x\in V_i$. By \eqref{unit_decomp}, we can then write
$$x=e_1x+\cdots +e_kx.$$
As $e_j\in V_j$, then $L_x(e_j)\in V_j$ (by the previous argument), and thus $e_jx=0$ for $j\neq i$. Then, $x=e_ix\in e_iA$.
\end{proof}

By the previous facts, the eigenvalues of $L_{e_i}$ are 0 (with multiplicity $n-1$) and $1$, and eigenvectors corresponding to the eigenvalue 1 are objects of $e_iA$. In other words, $\Lambda_i(e_j)=\delta_{ij}$.

If $\mathfrak{a}\subset A$ is an ideal, then it can be decomposed as a sum $\mathfrak{a}=\bigoplus_i\mathfrak{a}_i$, where $\mathfrak{a}_i$ is the ideal $\mathfrak{a}\cap e_iA$. In particular, the maximal ideals $\mathfrak{m}$ of $A$ are of the form
$$\mathfrak{m}=e_1A\oplus \cdots \oplus e_{i-1}A\oplus \mathfrak{m}_i\oplus e_{i+1}A\oplus \cdots \oplus e_kA,$$
where $\mathfrak{m}_i$ is a maximal ideal of $e_iA$.

\begin{proposition}
\begin{enumerate}
\item For each $i$, $V_i=e_iA$ is a local algebra with maximal ideal given by $\mathfrak{m}_i:=e_iA\cap \opnm{Ker}\Lambda_i$.
\item The algebra $A$ has exactly $n$ maximal ideals, given by $\opnm{Ker}\Lambda_i=\mathfrak{m}_i\oplus \bigoplus_{j\neq i}e_jA$, $i=1,\dots ,n$.
\end{enumerate}
\end{proposition}
\begin{proof}

\end{proof}

Note that, if $\varphi :A\to \comp$ is a morphism of algebras, then by the previous result there exists an index $i$ such that $\varphi =\Lambda_i$.

An important particular case is obtained when all the endomorphisms $L_x$ are diagonalizable. In that case we say that the algebra is \emph{semisimple}.

\begin{theorem}\label{sem_comm_alg}
The following assertions are equivalent:
\begin{enumerate}
\item The algebra $A$ is semisimple (i.e. all the maps $L_x$ are diagonalizable).
\item There exists a decomposition $A=\bigoplus_{i=1}^ne_iA$ where $e_iA\cong \comp$ are one-dimensional subspaces.
\item There exists an element $x_0\in A$ such that $L_{x_0}$ has $n$ distinct eigenvalues.
\end{enumerate}
\end{theorem}
\begin{proof}
$(1)\Rightarrow (2)$: The subspaces $e_iA$ are in this case eigenspaces (spanned by $e_i$) associated to eigevalues of the operators $L_x$, which are all diagonalizable.

$(2)\Rightarrow (3)$: Let $x_0:=\lambda_1 e_1+\dots +\lambda_ne_n$, where $\lambda_i\neq \lambda_j$. The rest follows from the equality $e_ie_j=\delta_{ij}e_i$.

$(3)\Rightarrow (1)$: As all the eigenvalues of $L_{x_0}$ are distinct, it is diagonalizable, and we have a decomposition $A=\bigoplus_{i=1}^n \opnm{Ker} (L_{x_0}-\lambda_i)$. But all these kernels are invariant under every operator $L_x$, and the result now follows.
\end{proof}

The second item above shows that $A$ is a sum of simple rings (rings without non-trivial (two-sided) ideals). This is a particular case of the Artin-Wedderburn theorem \ref{aw}. Another characterization of semisimple algebras can be given in terms of nilpotent elements.

\begin{proposition}
Let $A$ be an associative and commutative algebra of dimension $n$. Then $A$ is semisimple if and only if $A$ has no nilpotent elements.
\end{proposition}
\begin{proof}
Assume $x\in A$ is nilpotent; i.e. $x^k=0$ for some positive integer $k$ and suppose that $x=\sum_i\alpha_ie_i$. As $e_ie_j=\delta_{ij}e_j$, we have that
$$0=x^k=\alpha_1^ke_1+\cdots +\alpha_n^ke_n,$$
which implies that $\alpha_i^k=0$ and thus $\alpha_i=0$ for all $i$.

Assume now that $A$ has no nilpotent elements, and let $x_0\in A$. There exists a decomposition $A=\bigoplus_ie_iA$, where $e_iA=\opnm{Ker} (L_{x_0}-\lambda_i)^{n_i}$. We will check now that every element in $V_i$ is an eigenvector. So assume that $x\in V_i$. Then $(x_0-\lambda_i)^{n_i}x=0$, and thus $(x_0-\lambda_i)^{n_i}x^{n_i}=0$. As $A$ has no nilpotents, then $(x_0-\lambda_i)x=0$, which implies that $L_{x_0}(x)=\lambda_ix$, as desired.

\end{proof}

\begin{obs}
Semisimplicity is also defined in terms of the Jacobson radical: $A$ is semisimple if and only if its Jacobson radical (the intersection of all maximal ideals of $A$) is trivial. In an artinian ring (like our $A$ for example), the Jacobson radical is equal to the nilradical of $A$, i.e. the set (ideal) consisting of all nilpotent elements. Thus, if $A$ has no nilpotent elements, then $A$ is semisimple. For more details, see \cite{lam:_ncrings}, \cite{lang:_algebra} and \cite{zar_samuel:_commalg}.
\end{obs}

In particular, note that the maximal ideals of $A$ are given by $\mathfrak{m}_i=\bigoplus_{j\neq i}\comp e_j$ for $i=1,\dots ,n$.

\begin{obs}
In this case the set $\{e_1,\dots ,e_n\}$ is a basis for $A$. Moreover, as the only eigenvalues for $L_{e_i}$ are 0 (with multiplicity equal to $n-1$) and 1 (corresponding to $e_iA$), the set $\{\Lambda_1,\dots ,\Lambda_n\}$ is the basis of $A^*$ dual to $\{e_1,\dots ,e_n\}$.
\end{obs}

Let now $R$ be a commutative ring with unit, and let $A$ be an $R$-algebra, which is not necessarily commutative. In the previous paragraphs, for $R=\comp$ and $A$ commutative, we have obtained a decomposition $A=\bigoplus_ie_iA$ of $A$ into a sum of one-dimensional subspaces. This is a particular case of a celebrated result, which holds for any semisimple $R$-algebras. Before its statement, lets us discuss the notion of semisimplicity for an arbitrary ring.

\begin{defi}
An $R$-algebra $A$ is called \emph{left semisimple} if all left $A$-modules are semisimple, i.e. they are direct sums of submodules which have no non-trivial submodules.
\end{defi}

The notion of \emph{right semisimplicity} is defined analogously, and turns out to be completely equivalent to the notion of left-semisimplicity (see \cite{lam:_ncrings} for details), and thus we can talk about semisimple $R$-algebras just as in the commutative case.

The following is the key result of this section. 

\begin{theorem}[Artin-Wedderburn]\label{aw}
If $A$ is a semisimple $R$-algebra, then
$$A\cong \operatorname{M}_{d_1}(D_1)\oplus \cdots \oplus \operatorname{M}_{d_k}(D_k),$$
where $D_i,\dots ,D_k$ are division rings.
\end{theorem}

The case for a commutative algebra $A$ over $R=\comp$ is stated and proved in theorem \ref{sem_comm_alg} above.


%%%%%%%%%%%%%%%%%%%%%%%%%%%%%%%%%%%%%%%%%%
\subsubsection{Complex Frobenius Algebras}

Frobenius algebras are algebras $A$ with a fixed isomorphism $A\cong A^*$. This kind of algebras were first considered by Frobenius when studying algebras $A$ such that its first and second regular representations $\rho_1:A\rightarrow \operatorname{End}_\comp (A)$ and $\rho_2:A\rightarrow \operatorname{End}_\comp (A^* )$ are isomorphic. These representations are given by the assignments $\rho_1(x)(y)=xy$ and $\rho_2(x)(\varphi )(y)=\varphi (xy)$, where $x,y\in A$ and $\varphi \in A^*$. An isomorphism between these representations is a linear bijection $f:A\to A^*$ such that $\rho_2(x)f=f\rho_1(x)$ for each $x\in A$. The existence of such an isomorphism $f$ is equivalent to the existence of a linear form $\theta: A\to \comp$ such that $f(x)(y)=\theta (xy)$.

\begin{defi}\label{def_frob_alg}
Let $A$ be a finite dimensional, associative $\comp$-algebra with unit. Assume that there exists a linear form $\theta :A\to \comp$ on $A$ such that the bilinear form $(x,y)\mapsto \theta (xy)$ is non-degenerate. Then the pair $(A,\theta )$ is called a \emph{Frobenius algebra}. The Frobenius structure is called \emph{symmetric} if $\theta (xy)=\theta (yx)$ for all $x,y\in A$.
\end{defi}

The previous definition implies that every commutative Frobenius algebra is symmetric.

\begin{obs}
From now on, we will only consider \emph{symmetric Frobenius algebras}.
\end{obs}

Instead of using a linear form $\theta$, we can equivalently define a Frobenius algebra as an algebra $A$ together with a bilinear form $g:A\otimes A\rightarrow \comp $ such that $g$ is non-degenerate and \emph{multiplication invariant}
$$g(xy,z)=g(x,yz)$$
for all $x,y,z \in A$. In fact, given $(A,\theta )$, we can define such a bilinear form by setting
$$g(x,y):=\theta (xy).$$
And conversely, given $g$, we have the linear form $\theta (x) :=g(x,1)$. The symmetric structure is reflected in $g$ by the equation $g(x,y)=g(y,x)$.

\begin{obs}
Note that multiplication invariance is necessary for having a well defined link between $g$ and $\theta$.
\end{obs}

Another (equivalent) way of defining a Frobenius structure is by means of a trilinear form $c:A^{\otimes 3}\rightarrow A$ such that, as for $g$, is non-degenerate and multiplication invariant. In this case, having $\theta$, we define $c$ as $c(x,y,z):=\theta (xyz)$ and conversely, $\theta (x)=c(x,1,1)$.

%%%%%%%%%%%%%%%%%%%%%%%%%%%%%%%%%%%%%%%%%%%%%%%%%%%%%%%%%%%
\subsubsection{Commutative Frobenius Algebras over $\comp$}

The following is a list of equivalent ways of defining a Frobenius structure on a commutative $\comp$-algebra $A$.

\begin{proposition}\label{equiv_conditions_frob}
For an associative, commmutative $\comp$-algebra $A$ with unit and equipped with a linear form $\theta :A\to \comp$, the following conditions are equivalent:
\begin{enumerate}
\item $(A,\theta )$ is a Frobenius algebra.
\item The subspace $\opnm{Ker} \theta$ contains no non-trivial ideals.
\item There is a symmetric, non-degenerate bilinear form $g:A\otimes A\rightarrow \comp$ defined on $A$, which is multiplication invariant.
\item There is a symmetric, non-degenerate and multiplication invariant 3-tensor $c:A^{\otimes 3}\rightarrow \comp$.
\item There is a canonical isomorphism $A\cong A^*$.
\end{enumerate}
\end{proposition}
\begin{proof}
$(1)\Rightarrow (2)$: Assume that $\mathfrak{a}\subset \opnm{Ker} \theta$ is an ideal; if $x\in \mathfrak{a}$, then $\theta (xy)=0$ for each $y\in A$, and thus $x=0$.

$(2)\Rightarrow (3)$: Given $\theta$, define $g(x,y)=\theta (xy)$.

$(3)\Rightarrow (4)$: Given $g$, define $c(x,y,z)=g(xy,z)$.

$(4)\Rightarrow (5)$: Given $c$, we have a non-degenerate form $\theta :A\to \comp$ given by $\theta (x)=c(x,1,1)$. This form provides an isomorphism $\overline{\theta}:A\cong A^*$ given by $\overline{\theta}(x)(y)=\theta (xy)$.

$(5)\Rightarrow (1)$: Let $\Phi :A\to A^*$ be an isomorphism. Define $\theta :A\to \comp$ by $\theta =\Phi (1)$.
\end{proof}

\begin{obs}
If we denote by $S^iA^*$ the space of symmetric tensors, then the symmetry condition for $g$ and $c$ is expressed as $g\in S^2A^*$ and $c\in S^3A^*$. Recall that, given $\varphi ,\psi \in A^*$, then the symmetric product $\varphi \psi$ is given by
$$\varphi \psi =\frac{1}{2}(\varphi \otimes \psi + \psi \otimes \varphi ).$$
In particular, for $\varphi=\psi$, we have that $\varphi \psi =\varphi \otimes \psi$ (recall that $\varphi \otimes \psi :A\otimes A\to \comp$ is given by $(x,y)\mapsto \varphi (x)\psi (y)$).
\end{obs}

Assume now that $A=\bigoplus_i\comp e_i$ is a semisimple Frobenius $\comp$-algebra with linear form $\theta :A\to \comp$. Then note that, for each $i$, we have $\theta (e_i)\neq 0$; indeed, as $e_ie_j=\delta_{ij}e_i$, if $\theta (e_i)=0$ then $\overline{\theta}(e_i)=0$, contradicting the fact that $(x,y)\mapsto \theta (xy)$ is non-degenerate.

The proof of the following result is a straightforward computation.

\begin{lemma}
Let $A=\bigoplus_i \comp e_i$ be a semisimple Frobenius algebra with linear form $\theta$. Then, if $\{e^1,\dots ,e^n\}\subset A^*$ is the dual basis for $\{e_1,\dots ,e_n\}$, then
$$\theta =\sum_i\lambda_ie^i \quad , \quad g=\sum_i\lambda_i(e^i)^2 \quad {\rm and} \quad c=\sum_i\lambda_i(e^i)^3,$$
where $\lambda_i =\theta (e_i)$.
\end{lemma}
\begin{proof}
Compute the right hand side of the previous equalities, using the duality between $\{e_i\}$ and $\{e^i\}$ and the definition of symmetric product.
\end{proof}


%%%%%%%%%%%%%%%%%%%%%%%%%%%%%%%%%%%%%%%
\subsubsection{The Noncommutative Case}

Definition \ref{def_frob_alg} and its consequences can be applied to an associative, unital but not neccesarily commutative $\comp$-algebra with some changes. Note that if $A$ is a noncommutative algebra, the symmetry of $\theta$ could no longer be available (this property is always present in the commutative case). Instead of giving a detailed description of the noncommutative case, we focus on the main relevant results.

The analogue of proposition \ref{equiv_conditions_frob} is the following

\begin{proposition}
For an associative $\comp$-algebra $A$ with unit and linear form $\theta :A\to \comp$, the following conditions are equivalent:
\begin{enumerate}
\item $(A,\theta )$ is a Frobenius algebra.
\item The subspace $\opnm{Ker} \theta$ contains non non-trivial left or right ideals.
\item There is a non-degenerate bilinear form $g:A\otimes A\rightarrow \comp$ defined on $A$, which is multiplication invariant.
\item There is a canonical isomorphism of left $A$-modules $A\cong A^*$.
\item There is a canonical isomorphism of right $A$-modules $A\cong A^*$.
\end{enumerate}
\end{proposition}

\begin{obs}
As in this case $A$ may be noncommutative, there are some subtleties to take care about; for example, for a bilinear map $g:A\otimes A\to \comp$ there is a notion of nondegeneracy in the first coordinate and another one in the second. But, at the end, any one of these ``left'' and ``right'' notions lead to the same concepts. In the next sections, we deal with some of these concepts in the case of an arbitrary (commutative) coefficient ring. For algebras over fields these issues are exposed with detail in \cite{kock:_frobenius}.
\end{obs}

%%%%%%%%%%%%%%%%%%%%%%%%%%%%%%%%%%%%%%%%%%%%%%%%%%%%%%%%%
\subsubsection{Frobenius Algebras Over Commutative Rings}

The definition of Frobenius algebra generalizes to include algebras over arbitrary commutative rings. In what follows, $R$ denotes a commutative ring with unit. Recall also that the algebra structure is provided by a ring homomorphism $\iota :R\to A$, and this map makes $A$ both a left and right $R$-module defining $ax$ as $\iota (a)x$ and $xa$ as $x\iota (a)$, respectively. We will consider the case for $A$ not necessarily commutative from the beginning, as the commutative case may be deduced easily from this general case.

\begin{defi}
Let $A$ be a non-necessarily commutative $R$-algebra which verifies the following properties:
\begin{enumerate}
\item $A$ is projective and finitely generated as an $R$-module.
\item There exists an isomorphism of left $R$-modules $\Theta :A\to A^*$.
\end{enumerate}
Then the pair $(A,\Theta )$ is called a \emph{Frobenius algebra (over $R$)}.
\end{defi}

Note first that, unlike the case for $R=\comp$, we stated the definition in terms of the isomorphism between $A$ and its dual. This is just for simplicity; for if we have an $R$-linear map $\theta :A\to R$, then the condition of $(x,y)\mapsto \theta (xy)$ being non-degenerate only assures that the induced map $\overline{\theta}:A\to A^*$ is injective, and so we have to add another condition for surjectivity.

By considering $\theta:=\Theta (1):A\to R$ we obtain a linear map such that $\overline{\theta}=\Theta$. As we stated in the previous paragraph, we could have started with $\theta$, asking the following two conditions:
\begin{enumerate}[a.]
\item $\theta$ is non-degenerate (which assures the injectivity of $\Theta$); in other words, if $\Theta (x)(y)=0$ for each $x\in A$, then $y=0$.
\item the induced map $\overline{\theta}$ is surjective (we have to explicitly ask for this condition): that is, given a linear form $\varphi :A\to R$, then there exists a point $x\in A$ such that $\overline{\theta}(x)(y)=\varphi (y)$ for all $y\in A$.
\end{enumerate}

Having an isomorphism $\Theta$ of left $R$-modules induces a right $R$-module isomorphism $\Theta':A\to A^*$,
$$\Theta'(x)(y)=\Theta (y)(x)$$
and conversely. Thus, the definition of Frobenius algebra can be stated replacing the left isomorphism $\Theta$ with $\Theta'$. In case of using $\Theta'$, the condition of non-degeneracy is the same as the one given above, replacing $\Theta$ with $\Theta '$. But in terms of $\Theta$, we have that $\theta'$ is non-degenerate if and only if the condition $\Theta (x)(y)=0$ for each $y$ implies that $x=0$. Likewise, the condition for surjectivity states that given a linear form $\varphi :A\to R$, then there exists a point $x\in A$ such that $\Theta '(x)(y)=\Theta (y)(x)=\varphi (x)$ for all $y\in A$.

The Frobenius algebra $(A,\Theta)$ is said to be \emph{symmetric} if $\Theta (x)(y)=\Theta '(x)(y)$. Recall that all Frobenius algebras that we will encounter are symmetric.

\begin{obs}
The condition on $A$ to be a projective $R$-module is a useful generalization \cite{eilen_naka:_dim_II}. However, in the cases that we consider, the coefficient ring is always a local ring, and thus the notions of projective module and free-module are the same.
\end{obs}



%%%%%%%%%%%%%%%%%%%%%%%%%%%%%%%%%%%%%%%%%%%%%%%%%%%%%%%%%%%%%%%%%%%%%%%%
\subsubsection{Another Characterization for Semisimple Frobenius Algebras}

There is a more geometric approach to commutative, semisimple Frobenius $\comp$-algebras. This characterization is used in \cite{moore_segal1}; we first recall some basic definitions.

Let $X=\text{Spec}\, A$ be the prime spectrum of the algebra $A$, i.e. the set of prime ideals $\mathfrak{p}\subset A$ ($A$ itself is \emph{not} considered). If $\mathfrak{a}$ is any ideal in $A$, let $V(\mathfrak{a})$ be the set of prime ideals in $A$ which contain $\mathfrak{a}$. Define a topology on $X$ by declaring the sets of the form $V(\mathfrak{a})$ to be closed. This is the Zariski topology, and induces on $X=\text{Spec}\, A$ a structure of a quasi-compact topological space.\footnote{A space is said to be \emph{quasi-compact} if it is compact but not Hausdorff.}

For any prime ideal $\mathfrak{p}\subset A$, we can consider the localization $A_{\mathfrak{p}}$ of $A$ at $\mathfrak{p}$, which is a local ring with maximal ideal $\{x/s \; : \; x\in \mathfrak{p} \text{ and } s\in A\setminus \mathfrak{p}\}$. Given $x\in A$, let $V(x)$ denote the closed subset defined by the ideal generated by $x$. Let us also denote by $A_x$ the ring $A$ localized at $x$ (i.e. by considering the subset $\{x^n \; : \; n\geqslant 0\}$). Now, the subsets $U_x:=V(x)^c$ are easily seen to be members of a basis for the Zariski topology; we then define
$$\scr{O}(U_x):=A_x.$$
This assignment (which can be extended to every open subset of $X$) is a sheaf of rings on $X$ and it is called the structure sheaf. In particular, we have that the stalk $\scr{O}_\mathfrak{p}$ is isomorphic to the localization $A_\mathfrak{p}$. Detailed constructions can be found in \cite{mumford:_red}.

\begin{lemma}
If $(A,\theta )$ is a semisimple Frobenius algebra over $\comp$, then $X$ is a finite topological space, with cardinal equal to the dimension of $A$.
\end{lemma}
\begin{proof}
Every ideal of $A\cong \bigoplus_{i=1}^n\comp e_i$ is isomorphic to an ideal of the form $\bigoplus_i\mathfrak{a}_i$, where $\mathfrak{a}_i$ is an ideal of $\comp e_i$. As each summand $ \comp e_i$ is isomorphic to $\comp$ we have that $X=\{\mathfrak{m}_1,\dots ,\mathfrak{m}_n\}$, where $\mathfrak{m}_i=\bigoplus_{j\neq i}\comp e_j$.
\end{proof}

Using the notation on preceeding paragraphs, we have that
$$\scr{O}(X)=\scr{O}(U_1)=A_1\cong A,$$
by the obvious isomorphism $\frac{x}{1}\mapsto x$. Now, defining a linear form $\theta_X :\scr{O}(X)\rightarrow \comp$ by the assignment $\frac{x}{1}\mapsto \theta (x)$, we obtain a Frobenius algebra structure on $\scr{O}(X)$ and thus, by definition, an isomorphism of Frobenius algebras $(\scr{O}(X),\theta_X)\cong (A,\theta )$ (a morphism of Frobenius algebras is an algebra homomorphism which preserve the linear forms; see section \ref{subsec_morphisms} for the appropriate definitions).

Identifying $X=\text{Spec}\, A$ with the set of orthogonal idempotents $\{e_1,\dots ,e_n\}$ such that $\sum_ie_i=1$, let $\comp^X$ denote the set of maps $X\rightarrow \comp$. Let $\chi_i$ denote the characteristic function for the set $\{e_i\}$, i.e. $\chi_i(e_i)=1$ and $\chi_i(x)=0$ otherwise. Given $x\in A$, we can write it as a linear combination $x=\sum_i\lambda_ie_i$ over $\comp$. Then, it is easy to see that the assignment
$$x\longmapsto \sum_i\lambda_i\chi_i$$
defines an isomorphism between the algebras $A$ and $\comp^X$. The linear form which defines the Frobenius structure on $\comp^X$ is $\theta_X (\chi_i)=\theta (e_i)$, and can be regarded as a measure on $X$.

\begin{obs}
The conclusion in the previous paragraph has a converse statement; assume that $X$ is a finite measure space with objects $e_1,\dots ,e_n$ and measure $\mu$. Denoting, as before, by $\chi_i$ the characteristic function of the set $\{e_i\}$, then any measurable function $f:X\to \comp$ can be represented as $f=\sum_i\lambda_i\chi_i$, where $\lambda_i=f(e_i)$. Let $A$ be the space of measurable functions and define $\theta :A\to \comp$ by
$$\theta (f)=\sum_i\lambda_i\mu\left (\{e_i\}\right ).$$
Then, the pair $(A,\theta )$ is a Frobenius algebra.
\end{obs}


%%%%%%%%%%%%%%%%%%%%%%%%%%%%%%
\subsubsection{The Euler Element}

Let $(A,\theta )$ be a non-necessarily commutative and symmetric Frobenius algebra over $\comp$ and let $g:A\otimes A\rightarrow \comp$ the induced non-degenerate bilinear form. We then have a $g$-orthogonal basis $B=\{e_1,\dots ,e_n\}$ of the $\comp$-vector space $A$ which diagonalizes $g$; more precisely,
$$g(e_i,e_j)=\theta (e_ie_j)=0$$
when $i\neq j$. Let $\{e^1,\dots ,e^n\}$ be the dual basis for $B$. We define the element $\chi_B \in A$ by the following formula
$$\chi_B :=\sum_i e_i\overline{\theta }^{-1}(e^i).$$
Suppose now that $\overline{\theta}^{-1}(e^i)=\sum_j\lambda^{(i)}_je_j\in A$. Then
\begin{equation}\label{coef_euler}
\delta_{ik}=e^i(e_k)=\overline{\theta}(\overline{\theta}^{-1}(e^i))(e_k)=\sum_j\lambda^{(i)}_j\theta (e_je_k)=\lambda_k^{(i)}\theta (e_k^2),
\end{equation}
and thus $\overline{\theta}^{-1}(e^i)=\frac{e_i}{\theta (e_i^2)}$, which gives the following expression
$$\chi_B=\sum_i\frac{e^2_i}{\theta (e_i^2)}.$$

We will now get rid of the subindex $B$.

\begin{pyd}
The definition of $\chi_B$ does not depend on the choice of basis $B$. Its common value will be denoted by $\chi$ an called the \emph{Euler element} or the \emph{distinguished element} of $A$.
\end{pyd}
\begin{proof}
A more general statement is proved in proposition \ref{cardy_well_def}.
\end{proof}

\begin{obs}
In fact, it is not necessary to invoque an orthogonal basis for the definition of $\chi$. This kind of basis was taken into account just to simplify the computations.
\end{obs}

This Euler element can be used to recover the traces of the multiplication endomorphisms.

\begin{proposition}\label{traza_euler}\label{euler_trace}
If $x\in A$, then $\theta (x\chi )={\rm tr}(L_x)$. In particular, $\theta (\chi )=\dim_\comp (A)$.
\end{proposition}
\begin{proof}
By linearity, it suffices to prove the result for the operators $L_{e_i}$. So let $B=\{e_1,\dots ,e_n\}$ and suppose that
$$L_{e_i}(e_j)=e_ie_j=\sum_i\gamma_{ij}^ke_k.$$
Then, $\text{tr}(L_{e_i})=\sum_j\gamma_{ij}^j$. On the other hand, we compute
\begin{equation}\label{euler_traza}
e_i\chi =\sum_je_ie_j\overline{\theta}^{-1}(e^j)=\sum_{j,k}\gamma_{ij}^ke_k\overline{\theta}^{-1}(e^j)=\sum_{j,k}\frac{\gamma_{ij}^k}{\theta (e_j^2)}e_ke_j.
\end{equation}
Applying $\theta$ to equation \eqref{euler_traza} we get
$$
\begin{aligned}
\theta (e_i\chi ) &= \sum_{j,k}\gamma_{ij}^k\frac{\theta (e_ke_j)}{\theta (e_j^2)} 
                         = \sum_j\gamma_{ij}^j\frac{\theta (e_j^2)}{\theta (e_j^2)} \\
                         &= \sum_j \gamma_{ij}^j 
                         = \text{tr}(L_{e_i}), \\
\end{aligned}
$$
as desired.
\end{proof}

\begin{defi}
The \emph{trace form} for $A$ is the symmetric bilinear form $\text{tr}:A\otimes A\rightarrow \comp$ defined by the equation
$$\text{tr}(x\otimes y)=\opnm{tr}(L_{xy})=\opnm{tr}(L_xL_y).$$
\end{defi}

The following result proves that semisimplicity is strongly related to the Euler element.

\begin{proposition}
The trace form is non-degenerate if and only if the Euler element is invertible.
\end{proposition}
\begin{proof}
Assume first that the trace form is non-degenerate. The Euler element $\chi$ is invertible if and only if the linear map $L_\chi$ is invertible. Suppose that $x\in \opnm{Ker} L_\chi$, i.e. $\chi x=0$ and let $y\in A$ be any vector.  Then, by the previous proposition and the symmetry of $\theta$,
$$\opnm{tr}(x\otimes y)=\opnm{tr}(L_{xy})=\theta (xy\chi )=\theta (\chi xy)=0$$
for each $y\in A$. As $\text{tr}$ is non-degenerate, $x=0$ and then $L_\chi$ is an isomorphism.

Suppose now that $\chi$ is a unit in $A$ and that $\text{tr}(L_xL_y)=0$ for all $x\in A$. Then we have
$$0=\text{tr}(L_xL_y)=\theta (xy\chi).$$
As $(x,y)\mapsto \theta (xy)$ is non-degenerate, we must have $y\chi=0$, and the result now follows.
\end{proof}

Before proving a corollary, we state a theorem of Dieudonn�.

\begin{theorem}[\cite{schafer:_naalgebras}, Theorem 2.6]\label{dieudonne}
Assume $A$ is a finite dimensional algebra over a field $\mathbbm{F}$ (of arbitrary characteristic) satisfying:
\begin{enumerate}
\item the trace form $\operatorname{tr}:A\otimes A\to \mathbbm{F}$ is non-degenerate and
\item $\mathfrak{a}^2\neq 0$ for every ideal $\mathfrak{a}\neq 0$ in $A$.
\end{enumerate}
Then $A$ is semisimple.
\end{theorem}

\begin{cor}\label{euler_inv}
If the Euler element $\chi \in A$ is invertible then $A$ is semisimple.
\end{cor}
\begin{proof}
If the Euler element $\chi$ is invertible, then the trace form is non-degenerate. Let now $\mathfrak{a}$ be a non-zero ideal in $A$ and suppose that $xy=0$ for each $x,y\in \mathfrak{a}$ (elements of $\mathfrak{a}^2$ are defined as finite sums $\sum_ix_iy_i$ with $x_i,y_i\in \mathfrak{a}$). Take now a basis $B=\{x_1,\dots ,x_r,x_{r+1},\dots ,x_k\}$ of $A$ such that
\begin{enumerate}
\item $B$ is orthogonal; i.e. $\theta (x_ix_j)=0$ if $i\neq j$ and
\item $\{x_1,\dots ,x_r\}$ is a basis of $\mathfrak{a}$.
\end{enumerate}
Let $\{x^1,\dots ,x^k\}$ be the basis dual to $B$ and consider now the equation \eqref{coef_euler}
$$\delta_{ij}=x^i(x_j)=\lambda^{(i)}_j\theta (x_j^2),$$
where the coefficients $\lambda^{(i)}_j$ are defined by $\overline{\theta}^{-1}(x^i)=\sum_j\lambda^{(i)}_jx_j$. If we take $1\leqslant i=j\leqslant r$ then, as $x_j\in \mathfrak{a}$, $x_j^2=0$ and the previous equation makes no sense. This contradiction shows that such an ideal $\mathfrak{a}\neq 0$ cannot exist. The corollary now follows from theorem \ref{dieudonne}.
\end{proof}

\begin{obs}
In fact, semisimplicity of the algebra $A$ is \emph{equivalent} to the invertibility of $\chi$. See \cite{abrams:_frobenius}, Theorem 2.3.3.
\end{obs}



%%%%%%%%%%%%%%%%%%%%%%%
\subsubsection{Morphisms}
\label{subsec_morphisms}

Given Frobenius $\comp$-algebras $(A,\theta )$ and $(B,\tau )$, a \emph{morphism} $\varphi :(A,\theta )\rightarrow (B,\tau )$ is an algebra homomorphism $\varphi :A\rightarrow B$ such that $\tau \varphi =\theta$. By an algebra homomorphism we mean a $\comp$-linear map which is multiplicative and preserves the unit.

\begin{lemma}
Any morphism $\varphi :(A,\theta )\rightarrow (B,\tau )$ between Frobenius algebras is injective.
\end{lemma}
\begin{proof}
Assume $\varphi (x)=0$ and let $y\in A$. Then $\theta (xy)=\tau (\varphi (xy))=\tau (\varphi (x)\varphi (y))=0$; thus, as $\theta$ is non-degenerate, $x=0$.
\end{proof}

In particular, any morphism $(A,\theta )\rightarrow (A,\theta )$ is an isomorphism (i.e. $\varphi^{-1}$ is also a morphism of Frobenius algebras).

\begin{obs}
The fact that Frobenius algebras are also coalgebras alters the landscape a little bit more. Given a Frobenius $\comp$-algebra $(A,\theta )$, there exists a unique coassociative comultiplication on $A$ for which $\theta$ is the counit and certain relation (called the Frobenius relation) holds. If we bring this coalgebra structure to the stage, then we can define a morphism of Frobenius algebras as a morphism of algebras which preserves the linear form and also the coalgebra structure. With this definition, the category of Frobenius algebras is in fact a grupoid; i.e. every morphism between Frobenius algebras is an isomorphism. For a detailed treatment, we refer the reader to \cite{kock:_frobenius}.
\end{obs}

We denote by $\operatorname{Hom}_{\comp {\rm -alg}}((A,\theta ),(B,\tau ))$ the set of algebra homomorphisms $A\rightarrow B$ which preserve the linear forms.

\begin{lemma}
Let $(A,\theta )$ be an $n$-dimensional, commutative, semisimple Frobenius algebra. Then, if $\Sigma_n$ denotes the group of permutation of $n$ letters, we have a group isomorphism
$$\operatorname{Hom}_{\comp \text{-}{\rm alg}}((A,\theta ),(A,\theta ))\cong \Sigma_n.$$
\end{lemma}
\begin{proof}
Every homomorphism $(A,\theta )\rightarrow (A,\theta )$ is completely defined by its values on the idempotents $e_1,\dots ,e_n$ which define the decomposition $A=\bigoplus_ie_iA$. In particular, any permutation $\sigma$ defines a morphism (isomorphism in fact)
$$\varphi (e_i)=e_{\sigma (i)}$$
of the Frobenius algebra $(A,\theta )$.

Let now $\varphi :(A,\theta )\rightarrow (A,\theta )$ be an isomorphism of the semisimple Frobenius algebra $A$; then the images $\varphi (e_i)$ are again central, orthogonal idempotents. Now assume that
$$\varphi (e_i)=\sum a_je_j.$$
As $e_ie_j=\delta_{ij}e_i$, we have that the complex coefficients $a_i$ are equal to $0$ or $1$. Thus, $\varphi (e_i)=\sum_{j\in J}e_j$, where $J=\{j\; | \; a_j=1\}$. Considering the inverse map, we have
$$e_i=\varphi^{-1}\Bigl (\sum_{j\in J}e_j\Bigr )=\sum_{j\in J}\varphi^{-1}(e_j).$$
Unless $\# J=1$, the previous decomposition for $e_i$ is impossible, as the following argument shows: assume that the idempotent $e_i$ can be decomposed as a sum $a+b$ of two orthogonal elements; let $a=\sum_k\lambda_ke_k$ and $b=\sum_k\mu_ke_k$; then $0=ab=\sum_k\lambda_k\mu_ke_k$, and hence $\lambda_k\mu_k=0$. This fact implies the existence of subsets $I_a,I_b\subset \{1,\dots ,n\}$ such that $I_a\cup I_b=\{1,\dots ,n\}$, $I_a\cap I_b=\emptyset$ and $a=\sum_{k\in I_a}\lambda_ke_k$, $b=\sum_{k\in I_b}\mu_ke_k$; now
$$e_i=a+b=\sum_{k\in I_a}\lambda_ke_k+\sum_{k\in I_b}\lambda_ke_k;$$
as $\{e_1,\dots ,e_n\}$ is a basis, then $a=0$ (if $i\in I_b$) or $b=0$ (if $i\in I_a$). The lemma is proved.
\end{proof}


%%%%%%%%%%%%%%%%%%%%%%%%%%%%%%%%%%%%
\subsubsection{The Structure Equations}
\label{structure_equations}

We will now provide a more analytic approach to the Frobenius algebra structure on a finite dimensional vector space. Instead of specifying a product and other relations in terms of maps, we will introduce these notions by means of coordinates on a fixed basis.

Let us first fix some notation and terminology. Let $A$ be a finite dimensional complex vector space with a nondegenerate bilinear form $g:A\otimes A\to \comp$ defined on $A$ and fix a basis $B=\{e_1,\dots ,e_n\}$ of $A$. Let $g_{ij}:=g(e_i,e_j)$ be the coefficients of the matrix of $g$ in terms of the basis $B$. As $g$ is nondegenerate, we have an isomorphism $\widetilde{g}:A\to A^*$. If $x\in A$, then the linear form $\widetilde{g}(x)$ is said to be obtained from $x$ by \emph{lowering an index}; considering the inverse map $\widetilde{g}^{-1}:A^*\to A$, the vector $\widetilde{g}^{-1}(\varphi )$ is said to be obtained from the linear form $\varphi$ by \emph{raising an index} (these lowering and raising refers to the coefficients in terms of the basis $B^*$ and $B$ respectively). Moreover, if $B^*$ denotes the basis of $A^*$ dual to $B$, the matrix of the linear map $\widetilde{g}$ with respect to the basis $B$ and $B^*$ is equal to $(g_{ij})$, and thus the matrix of $\widetilde{g}^{-1}$ with respect to $B^*$ and $B$ is $(g_{ij})^{-1}$.

Assume now that $A$ is a vector space as in the previous paragraph and assume that we have a trilinear map $c:A\otimes A\otimes A\to \comp$ such that the bilinear form $g:A\otimes A\to \comp$ given by $g(x,y)=c(x,y,e_1)$ is nondegenerate. Before moving on, let us make a couple of remarks: 
\begin{itemize}
\item The first one is that the vector $e_1$ will play the role of the unit of the algebra $A$;
\item the second is that we cannot start from the bilinear form $g$, as we need the mapping $c$ and the construction of $c$ from $g$ involves the multiplication on $A$, which we are trying to define.
\end{itemize}
Let $c_{ijk}:=c(e_i,e_j,e_k)$ and $g_{ij}:=c(e_i,e_j,e_1)=g(e_i,e_j)$. Thus, in the basis $B$, the coordinate expressions of $c$ and $g$ are given by
$$g=\sum_{i,j}g_{ij}\; e_i\otimes e_j \quad ,\quad c=\sum_{i,j,k}c_{ijk}\; e_i\otimes e_j\otimes e_k.$$
Fixing $i,j$ we obtain a linear form $c_{ij}:A\to \comp$ given by $c_{ij}(e_k)=c(e_i,e_j,e_k)$. In the basis $B^*$, this map is written as $c_{ij}=\sum_kc_{ijk}e^k$. Now we compute
$$
\begin{aligned}
\widetilde{g}^{-1}(c_{ij}) &= \sum_kc_{ijk}\widetilde{g}^{-1}(e^k) \\
                  				 &= \sum_kc_{ijk}\left (\sum_rg^{kr}e_r\right ) \\
                  				 &= \sum_r\left (\sum_kg^{kr}c_{ijk}\right )e_r, \\
\end{aligned}
$$
where $(g^{ij})$ denotes the inverse of the matrix $(g_{ij})$. Let $c_{ij}^k:=\sum_rg^{kr}c_{ijk}$. We now construct a product structure on $A$ by defining
$$e_ie_j=\sum_kc_{ij}^ke_k.$$
The unitarity relations for $A$ are deduced from the previous equation by computing $e_1e_i=e_i$ for each $i$, which shields the identities
$$c_{1i}^k=\delta_{ik}.$$
Particularly important in Quantum Field Theory are the equations expressing the associativity of the product, which are given by
$$\sum_rc_{ij}^rc_{rk}^s=\sum_rc_{jk}^rc_{ir}^s,$$
for $i,j,k,s=1,\dots ,n$. In particular, the vector space $A$ becomes a Frobenius algebra, symmetric if and only if the bilinear form $g$ is symmetric (recall that the linear map $\theta :A\to \comp$ can be defined from $c$ by $\theta (e_i)=c(e_i,e_1,e_1)$).





%%%%%%%%%%%%%%%%%%%%%
\subsubsection{Examples}
Frobenius algebras are rather ubiquitous: there are important examples of them not only in algebra, but also in geometry and physics. We will list some of them below, an, of course, encounter more in subsequent chapters. 

\paragraph{Matrix Algebras}
Let $(A,\theta )$ be a finite-dimensional Frobenius algebra and consider the matrix algebra $M_n(A)$; the composite map
$$M_n(A)\stackrel{\text{tr}}{\longrightarrow }A\stackrel{\theta }{\longrightarrow }\comp$$
is easily seen to be a non-degenerate form on $M_n(A)$. Thus, $M_n(A)$ is again a Frobenius algebra. Moreover, as $\text{tr}(ab)=\text{tr}(ba)$, this Frobenius structure is symmetric.

Assume now that $A$ is a finite dimensional simple $\comp$-algebra; then $A$ is (isomorphic to) $M_n(\comp )$ for some $n\in \natu$ (in fact, there exists a division algebra $D$ over $\comp$ such that $A\cong M_n(D )$; but $\comp$ is algebraically closed, and so $D\cong \comp$). Then, the trace map provides $A$ with a structure of a symmetric Frobenius algebra.

More generally, if $A$ is semisimple, then $A\cong \bigoplus_iM_{d_i}(\comp )$ and 
$$\theta :=\sum_i\text{tr}_i$$
is a Frobenius form, where $\text{tr}_i$ is the trace map on $M_{d_i}(\comp )$.



\paragraph{Group Algebras}
If $G$ is a finite group (not necessarily abelian), then, by Maschke's theorem (see \cite{lang:_algebra}, Ch. {\sc xviii}, theorem 1.2), the group-algebra $\comp G$ is semisimple and thus can be given a structure of Frobenius algebra. Without relying on the Artin-Wedderburn isomorphism, we can define a non-degenerate linear form $\theta :\comp G\rightarrow \comp$ directly by setting
$$\theta \Bigl ( \sum_{g\in G}\lambda_gg \Bigr ):=\lambda_1,$$
where $1$ is the identity in $G$. In fact, this definition for $\theta$ shows that we can indeed define a Frobenius structure on $\field G$, where $\field$ is any field.



\paragraph{Characters}
Let $G$ be a finite group of order $n$. A \emph{class function} on $G$ is a map $\chi :G\rightarrow \comp$ such that $\chi (ghg^{-1})=\chi (h)$ for all $g,h\in G$. Let us denote by $R(G)$ the $\comp$-algebra of class functions on $G$ (the ``$R$'' comes from ``representation'', and $R(G)$ is usually called the \emph{representation ring} of $G$; see below). We can define an inner product on $R(G)$ by the formula
$$\langle \chi ,\xi \rangle =\frac{1}{n}\sum_{g\in G}\chi (g)\overline{\xi (g)}.$$
Now, a (\emph{linear}) \emph{representation} of the group $G$ is a group homomorphism $\rho :G\rightarrow \operatorname{Hom}_\comp (V,V)$, where $V$ is a (finite-dimensional) complex vector space. Such a representation induces a class function $\chi_\rho :G\rightarrow \comp$ given by taking the trace of each endomorphism $\rho (g):V\rightarrow V$. It is a well-known result that characters of irreducible representations\footnote{A representation $\rho :G\rightarrow \operatorname{Hom}_\comp (V,V)$ is called \emph{irreducible} if there exist no non-trivial invariant subspaces of $V$ (i.e. subspaces $W\neq 0,V$ such that $\rho (g)(W)\subset W$; see \cite{serre:_representations}, theorem 1 (Maschke's theorem in the context of representation theory).} of  a group $G$ forms an orthonormal basis (with respect to the previously defined inner product) for $R(G)$. Then, in particular, this inner product is non-degenerate and provides $R(G)$ with a structure of a Frobenius algebra.




\paragraph{Cohomology Rings}
Let $M$ be a compact, orientable, $n$-dimensional smooth manifold. For each $i=0,\dots ,n$ we can consider its $i$th-de Rham cohomology group $H^i(M)$, which is a real vector space. The wedge product of differential forms endows
$$H^*(M)=\bigoplus_{i=0}^nH^i(M),$$
with a structure of a (graded) ring. As $M$ is compact and orientable, we have a volume form (which is a nowhere vanishing $n$-form), and so we can integrate differential forms over $M$. By Stokes theorem, integration is still well-defined when working with closed forms modulo exact forms; thus, we have a linear form
$$\int_M :H^*(M)\rightarrow \re .$$
Now, Poincar\'e duality states that, for such a manifold, the pairing given by
$$H^i(M)\otimes H^{n-i}(M)\longrightarrow \re$$
$$\omega \otimes \tau \longmapsto \int_M\omega \wedge \tau$$
is non-degenerate. This induces a non-degenerate pairing
$$H^*(M)\otimes H^*(M)\longrightarrow \re$$
which endows $H^*(M)$ with the structure of a Frobenius algebra. Note that, as every $k$-form is zero for $k>n$, this algebra cannot be semisimple, as it has nilpotent elements.

\paragraph{The Verlinde Algebra}
The theory of Riemann surfaces has proved to be extremely useful tool not only in mathematics, but also in theoretical physics, particularly in the study of Conformal Field Theories (CFTs for short). In \cite{verlinde:2dcft} , E. Verlinde studies a certain type of CFT, called rational, by considering the \emph{fusion rules} of the primary fields of the theory. These fusion rules are in fact the structure constants of an algebra, which turns out to have a Frobenius structure.

Let $\EuScript{M}_{0,3}$ denote the moduli space of Riemann surfaces of genus $g=0$ and with $n=3$ punctures. If $G$ is a correlation function (certain map defined on the moduli space $\EuScript{M}_{0,3}$), there are some vector bundles $V_{0,3}$ over $\EuScript{M}_{0,3}$ associated with the map $G$. Let $V_{0,ijk}$ be the components of the bundle $V_{0,3}$ corresponding to the sphere with fields $\phi_i,\phi_j,\phi_k$ at the three punctures and set
$$N_{ijk}=\text{rank}\; V_{0,ijk}.$$
By considering certain conjugation matrices to raise the index $k$, the fusion rule for the operators $\phi_i$ and $\phi_j$ is expressed by
$$\phi_i \cdot \phi_j=\sum_kN_{ij}^k\phi_k.$$
These coefficients $N_{ij}^k$ are in fact the structure constants for the multiplication of the fusion rule (Verlinde) algebra (cf. section \ref{structure_equations}). A further analysis shields an associativity equation involving the constants $N_{ij}^k$, as well as commutativity. Moreover, the matrices $N_i$ given by $(N_i)_{jk}=N_{ij}^k$ are mutually commuting and symmetric; thus, they can be diagonalized simultaneously. These structure provides the fusion rule algebra with the structure of a commutative, semisimple, Frobenius structure.

\paragraph{More From Physics}

Other examples of Frobenius algebras in physics besides the one described in the previous entry are given by quantum cohomology of manifolds \cite{abrams:_frobenius} and the chiral ring of certain Landau-Ginzburg theories \cite{dvv:top_strings}.




%%%%%%%%%%%%%%%%%%%%%%%%%%%%%%%%%%%%%%%%
\subsection{The Correspondence Between {\sc tft}s and Frobenius Algebras}

Let $R=\comp$ and $Z:\tsf{Cob}(2)\to \tsf{Vect}$ be a 2-dimensional {\sc tqft}, where $\tsf{Vect}$ is the category of (finite-dimensional) complex vector spaces. In this case, objects of $\tsf{Cob}(2)$ can be taken to be disjoint unions of circles and the empty set. In fact, the standard circle $S^1$ can be regarded as a generator with respect to the product $\sqcup$, as every object of $\tsf{Cob}(2)$ is diffeomorphic to a disjoint union of circles. Let $A$ be the image of the generator $S^1$,
$$Z(S^1)=A.$$
We will make a brief description of how $A$ becomes a Frobenius algebra from properties of the funtor $Z$. Pictures are also included to help in clarifying ideas. For more details about the meaning of the following figures, see section \ref{remarks}.

Multiplication of the algebra is given by the image of the <<pair of pants>> cobordism between $S^1\sqcup S^1$ and $S^1$; in other words, the arrow $S^1\sqcup S^1 \to S^1$ is mapped by $Z$ to an arrow $A\otimes A\to A$, which is the multiplication of the algebra $A$. The unit is given by the image of the cobordism between the empty set $\emptyset$ and $S^1$, while the Frobenius form $\theta$ is obtained by applying $Z$ to the cobordism $S^1\to \emptyset$. See figure \ref{frobenius_closed} for a pictorial description.

\begin{figure}[!ht]
\begin{center}
\includegraphics[width=10cm]{frobenius_closed} \\
\end{center}
\vspace{-10pt}
\caption{Frobenius algebra structure for $A=Z(S^1)$. Morphisms on top are in $\tsf{Cob}(2)$ and the ones at the bottom are the linear maps ontained in $\tsf{Vect}$ after applying the functor $Z$: (A) Unit of the algebra $A$; (B) <<Pair of pants>> cobordism which provides the multiplication in the algebra $A$; (C) Linear form making $A$ a Frobenius algebra.}
\label{frobenius_closed}
\end{figure}

Further properties of the algebra $A$ come from cobordism equivalences; examples of these properties are associativity and commutativity. A brief description is given in figure \ref{frobenius_closed_2}.

\begin{figure}[!ht]
\begin{center}
\includegraphics[width=10cm]{frobenius_closed_2} \\
\end{center}
\vspace{-10pt}
\caption{Properties of the Frobenius algebra $A$ deduced from cobordism equivalences: (A) Commutativity; (B) This property is expressing the fact that the image of $1\in \comp$ by the map $\comp \to A$ is precisely the unit of the algebra $A$. It is worth noting that the cilinder in the right hand side corresponds to the identity map $A\to A$; (C) Associativity.}
\label{frobenius_closed_2}
\end{figure}

It only remains to provide a meaning to the phrase ``morphism of {\sc tqft}s''; so let $Z_1,Z_2$ be two 2-dimensional {\sc tqft}s. A \emph{morphism} $\Phi :Z_1\to Z_2$ is a natural transformation which preserves the multiplicative structure; i.e. it is a family of linear maps
$$\Phi=\{\Phi_n:A_1^{\otimes n}\to A_2^{\otimes n}\; | \; n\geqslant 0\},\footnote{Recall that $S^1$ is a generator for the category $\tsf{Cob}(2)$.}$$
where $A_1:=Z_1(S^1)$, $A_2=Z_2(S^1)$, $A^{\otimes 0}=\comp$, $\Phi_0=\text{id}:\comp \to \comp$, $\Phi_1:A_1\to A_2$, $\Phi_n=\Phi_1^{\otimes n}$, and such that the diagram
$$
\xymatrix{
A_1^{\otimes n} \ar[r]^{\Phi_n} \ar[d]_{Z_1(W)} & A_2^{\otimes n} \ar[d]^{Z_2(W)} \\
A_1^{\otimes k} \ar[r]^{\Phi_k} & A_2^{\otimes k} }
$$
commutes for each cobordism $W:(S^1)^{\sqcup n}\to (S^1)^{\sqcup k}$.

\begin{obs}
The ``multiplicative structure'' in this categorical context is what is known as a \emph{monoidal structure}. In fact, as $\tsf{Cob}(D)$ and $\tsf{Vect}_R$ are monoidal categories, then any {\sc tqft} $Z$ must be a monoidal functor (i.e. a functor which preserves the multiplicative structure) and any morphism $\Phi :Z_1\to Z_2$ between {\sc tqft}s should be a monoidal natural transformation (i.e. a natural transformation which is compatible with the products). For details about monoidal categories, the reader is referred to the classical reference \cite{kn:mclane}.
\end{obs}

Let now $\tsf{TQFT}(2)$ be the category of $2$-dimensional {\sc tqft}s and $\tsf{Frob}$ the category of finite-dimensional, unital, commutative Frobenius algebras over $\comp$.

\begin{theorem}[\cite{abrams:_frobenius}, Theorem 3.3.1. See also the appendix of \cite{moore_segal1}]\label{abrams}
The functor $\tsf{TQFT}(2)\to \tsf{Frob}$ given by the assignments
$$
\begin{aligned}
Z &\longmapsto (Z(S^1),\theta ) \\
\Phi &\longmapsto \Phi_1 \\
\end{aligned}
$$
is an equivalence of categories.
\end{theorem}

\begin{obs}
Moreover, the previous equivalence also preserves the multiplicative structure.
\end{obs}




%%%%%%%%%%%%%%%%%%%%%%%%%%%%%%%%%%%%%%%%%%%%%%%%%%%%%%%%%%%%%%%
%%%%%%%%%%%%%%%%%%%%%%%%%%%%%%%%%%%%%%%%%%%%%%%%%%%%%%%%%%%%%%%
\section{Calabi-Yau Categories: Open-Closed Field Theories}
\label{sec_octfts}

Let us consider the case $D=2$ and $R=\comp$. Field theories as the ones considered in the previous section are called \emph{closed} field theories, as they describe the behaviour of closed strings (represented by manifolds diffeomorphic to $S^1$). But this representation is rather restrictive, as strings can also be regarded as spaces diffeomorphic to a closed interval (in fact, the word ``string'' first reminds us of a curve isomorphic to an interval and not to $S^1$). In the general case, these open-closed theories are obtained when one considers (compact) manifolds with boundary besides closed ones.

As for closed theories, there is a precise formulation of an open-closed theory, which was given by G. Moore and G. Segal in \cite{moore_segal1}. Unlike the ones for closed theories, the axioms for open-closed theories are quite involved, as we will soon check. This is mainly because of the interaction between open and closed strings, which translates into a significant amount of algebro-geometric relations.

The first step is to give a precise definition of the geometric category for the open-closed theory. As in the case for closed theories, in the following paragraphs we also include pictorial descriptions of the structures involved.


%%%%%%%%%%%%%%%%%%%%%%%%%%%%%%%%%%%
\subsection{Algebro-Geometric Data}
\label{data}

An open-closed {\sc tqft} of dimension 2 (over $\comp$) consists of the following objects:

\begin{enumerate}
\item A category $\scr{B}$, called the \emph{category of labels}, \emph{boundary conditions} or \emph{branes}. Its objects will be denoted by letters $a,b,c,\dots$. Morphisms are defined in the following way: given labels $a,b$, an arrow $a\to b$ is a 1-dimensional, oriented, smooth manifold with boundary. This is to be interpreted as a closed, oriented, 1-dimensional interval such that its endpoints (connected components of the boundary) are labeled by the objects $a$ and $b$.

$$\begin{xy}
(0,0)*+{a}; (10,0)*+{b.} **\dir{-} ?(.6)* %
\dir{>}
\end{xy}$$

We then require first that the set of arrows from $a$ to $b$, denoted $O_{ab}$, is a finite-dimensional $\comp$-vector space, and the composition law $O_{ab}\otimes O_{bc}\to O_{ac}$ should be an associative, bilinear product (if $\sigma :a\to b$ and $\tau :b\to c$ are arrows in $\scr{B}$, we will denote the image of $\sigma\otimes \tau$ by $\tau \sigma$). More structure enjoyed by $\scr{B}$ will be discussed soon.

\item A cobordism category $\tsf{Cob}_{\scr{B}}(2)$, defined in the following way: its objects are disjoint unions of the empty set and compact, oriented, 1-dimensional manifolds such that their boundary is either empty or their boundary components are labelled by objects of $\scr{B}$ (i.e. its objects are disjoint unions of the empty set, manifolds diffeomorphic to the oriented circle $S^1$ (empty boundary) or the closed oriented interval $[0,1]$; the connected components of the boundary (the two extreme points) are labelled using boundary conditions; that is, objects of the category $\scr{B}$). Given objects $\Sigma_1,\Sigma_2$, an arrow $W:\Sigma_1\to \Sigma_2$ is a 2-dimensional manifold $W$ such that $\partial W=\Sigma_1\cup \Sigma_2\cup W'$, where $W'$, which is called the \emph{constrained boundary}, is a cobordism from $\partial \Sigma_1$ to $\partial \Sigma_2$ (we will come again later to this). In particular, the strip corresponding to the cobordism between the interval with endpoints $a$ and $b$ with itself should correspond to the identity map of the vector space $O_{ab}$ (the given description is suggesting, as in the closed case, the existence of a functor between the cobordism category $\tsf{Cob}_{\scr{B}}(2)$ and the category of vector spaces; see section \ref{remarks} for more on this topic). Check figure \ref{frobenius_openclosed_1} for pictorial details.

\item Each vector space $O_{aa}$ comes equipped with a nondegenerate linear form $\theta_a:O_{aa}\to \comp$ (that is, the bilinear map $O_{aa}\otimes O_{aa}\to \comp$ given by $\sigma \otimes \tau \mapsto\theta_a(\tau \sigma)$ is nondegenerate).

\item Generalizing the previous item, each composite map
\begin{equation}\label{ngpair}
O_{ab}\otimes O_{ba}\longrightarrow O_{aa}\stackrel{\theta_a}{\longrightarrow} \comp
\end{equation}
is a perfect pairing and
\begin{equation}\label{symmetry}
\theta_{a}(\sigma \tau )=\theta_b(\tau \sigma ).
\end{equation}
See figure \ref{frobenius_openclosed_perfectpairings}.

\item For each label $a\in \scr{B}$, there exist transition maps $\iota_a:A\to O_{aa}$ and $\iota^a:O_{aa}\to A$. These maps should verify the following additional properties:
	\begin{enumerate}
	\item $\iota_a$ is a unit-preserving algebra homomorphism and $\iota^a$ is $\comp$-linear.
	
	\item $\iota_a$ is central; i.e. the equality
	$$\iota_a(x)\sigma =\sigma \iota_b (x)$$
	holds for each $x\in A$ and $\sigma \in O_{ab}$.
	
	\item There exists and adjoint relation between $\iota_a$ and $\iota^a$ given by
	$$\theta (\iota^a(\sigma )x)=\theta_a(\sigma \iota_a (x))$$
	for any $\sigma \in O_{aa}$.
	
	\item The \emph{Cardy condition}: we need a little work before defining this property. First of all, it should be noted that the vector space $O_{ba}$ is canonically isomorphic to $O_{ab}^*$ by means of a nondegenerate pairing like \eqref{ngpair}. Let $\overline{\theta}_{ab}:O_{ba}\to O_{ab}^*$ be the induced isomorphism,
	$$\overline{\theta}_{ab}(\tau )(\sigma )=\theta_a(\sigma \tau).$$
	Let now $\{\sigma_i\}$ be a basis for $O_{ab}$ and let $\{\sigma^i\}$ be its dual basis. Define a linear map $\pi_b^a:O_{aa}\to O_{bb}$ by the equation
	$$\pi_b^a(\tau )=\sum_i\sigma_i\tau \overline{\theta}_{ab}^{-1}(\sigma^i).$$
	Then $\pi_b^a,\iota_b$ and $\iota^a$ should verify the so-called \emph{Cardy condition}
	$$\pi_b^a=\iota_b\iota^a.$$
		\end{enumerate}
\end{enumerate}

For the interpretation of the following pictures it is necessary to have in mind the figures corresponding to the closed sector \ref{frobenius_closed} and \ref{frobenius_closed_2}.

\begin{figure}[!ht]
\begin{center}
\includegraphics[width=10cm]{frobenius_openclosed_1} \\
\end{center}
\vspace{-10pt}
\caption{Basic components for the open sector of an open-closed {\sc tft}; figures represent objects and arrows (cobordisms) between intervals and unions of them; enpoints are labelled using objects of the category $\scr{B}$. Below these figures, the algebraic data encoded by these geometric structures is displayed (see section \ref{remarks} for the functorial framework): {\bf (A)} The basic object for the open sector, a labeled interval, which is also viewed as an arrow between labels $a$ and $b$. {\bf (B)} The pairing corresponding to the <<pair of pants>> cobordism. {\bf (C)} Frobenius form for the algebra $O_{aa}$. {\bf (D)} Unit for the algebra $O_{aa}$; {\bf (E)} The cilinder corresponds to the identity map.}
\label{frobenius_openclosed_1}
\end{figure}

Note that by restricting to closed manifolds, we obtain a Frobenius algebra $(A,\theta )$, corresponding to the closed sector.

\begin{figure}[!ht]
\begin{center}
\includegraphics[width=10cm]{frobenius_openclosed_perfectpairings} \\
\end{center}
\vspace{-10pt}
\caption{Perfect pairings. As for the closed sector, recall that the cylinder on the right corresponds to the identity map $\opnm{id}:O_{ab}\to O_{ab}$.}
\label{frobenius_openclosed_perfectpairings}
\end{figure}

\begin{figure}[!ht]
\begin{center}
\includegraphics[width=10cm]{frobenius_openclosed_propertiestransitions} \\
\end{center}
\vspace{-10pt}
\caption{Properties of the transition homomorphism $\iota_a$: {\bf (A)} The map $\iota_a$ is multiplicative; the figure on the left represents the map $A\otimes A\to O_{aa}\otimes O_{aa}\to O_{aa}$ given by $x\otimes y\mapsto \iota_{a}(x)\iota_a(y)$ and the figure on the right represents the composition map $A\otimes A\to A\to O_{aa}$ given by $x\otimes y\mapsto \iota_a(xy)$. {\bf (B)} This relation expresses the fact that $\iota_a$ is unit-preserving; on the left we have the composition $\comp \to A\to O_{aa}$ of the unit map with $\iota_a$ and on the right, the unit for the algebra $O_{aa}$. {\bf (C)} This last image corresponds to the centrality condition; that is, to the fact that the image of the homomorphism $\iota_a$ lies within the centre of the algebra $O_{aa}$. On the left, we have the composite $A\otimes O_{ba}\to O_{bb}\otimes O_{ba}\to O_{ba}$ given by $x\otimes \sigma \mapsto \sigma \iota_a(x)$; the image on the right corresponds to the map $O_{ba}\otimes A\to O_{ba}\otimes O_{aa}\to O_{ba}$ given by $\sigma \otimes x\mapsto \iota_a(x)\sigma$.}
\label{frobenius_openclosed_propertiestransitions}
\end{figure}

\begin{figure}[!ht]
\begin{center}
\includegraphics[width=10cm]{frobenius_openclosed_adjointrelation} \\
\end{center}
\vspace{-10pt}
\caption{The adjoint relation. The figure on the left corresponds to $\theta (\iota^a(\sigma )x)$ and the one on the right to the term $\theta_a(\sigma \iota_a (x))$. Take a look again at figure \ref{frobenius_openclosed_1}.}
\label{frobenius_openclosed_adjointrelation}
\end{figure}

\begin{figure}[!ht]
\begin{center}
\includegraphics[width=10cm]{frobenius_openclosed_cardy} \\
\end{center}
\vspace{-10pt}
\caption{The Cardy condition. The first diagram on the left, the <<double twist>>, represents the linear map $\pi^a_b$. The one on the right is the composite $\iota_b\iota^a$.}
\label{frobenius_openclosed_cardy}
\end{figure}



%%%%%%%%%%%%%%%%%%%%%%%%%%%%%%%%%%%%%%%%%%%%
\subsection{Some Remarks on the Definitions}
\label{remarks}

Before turning to the characterization of the category of branes, let us discuss some important issues.


\subsubsection{Generalities on $\scr{B}$}

Given a label $a\in \scr{B}(U)$, the existence of a linear form $\theta_a:O_{aa}\to \comp$ makes $O_{aa}$ into a non-necessarily commutative Frobenius algebra, also symmetric by equation \eqref{symmetry}. Regarding the map $\pi^b_a$, note that its definition was given after fixing a basis of the vector space $O_{ba}$. The independence of the chosen basis is proved (in a more general setting) in proposition \ref{cardy_well_def}.


\subsubsection{String Interactions and Cobordisms}

To be accurate, the definition of closed and open-closed field theories are based on the figures that we included later to clarify the algebraic data, and not conversely. These pictures describe the evolution of closed and open strings and their interactions in time, and are called \emph{world-sheets} or also \emph{spacetimes}. There are four kinds of 2-dimensional {\sc tft}s, according to the properties of these world-sheets:
\begin{itemize}
\item Closed oriented (repectively unoriented) theories: They only consider closed oriented (respectively unoriented) strings (i.e. 1-dimensional manifolds diffeomorphic to the circle). These are the objects of the category $\tsf{Cob}(2)$ defined before.
\item Open-closed oriented (respectively unoriented) theories: Besides closed strings, we also take into consideration open, oriented (respectively unoriented) strings (i.e. 1-dimensional manifolds diffeomorphic to a closed interval).
\end{itemize}
The world-sheets corresponding to open and/or closed strings are depicted in section \ref{data}, after the description of the objects  involved in an open-closed theory. The shape of these world-sheets is a consequence of the interactions between strings and, in a functorial interpretation, they are regarded as arrows between disjoint unions of 1-manifolds. The allowable interactions, which are taken from \cite{pol:_string1}, are shown in figure \ref{interactions}, and these include splittings or joinings (for both types of strings), open $\leftrightarrow$ closed transitions, etc. The algebraic conditions imposed to the structure maps are derived from homotopy equivalences between the different world-sheets, which turn into equalities in the target algebraic category (the category of complex vector spaces in this case).

\begin{figure}[!ht]
\begin{center}
\includegraphics[width=10cm]{strings} \\
\end{center}
\vspace{-10pt}
\caption{String interactions (O stands for ``open string'' and C for ``closed string''): {\bf (A)} C+O$\leftrightarrow$O. {\bf (B)} O+O$\leftrightarrow$O+O. {\bf (C)} O$\leftrightarrow$C. {\bf (D)} O+O$\leftrightarrow$O. {\bf (E)} C+C$\leftrightarrow$C.}
\label{interactions}
\end{figure}


Before giving the functorial definition, let us first describe morphisms in more detail. Let $\Sigma_1$ and $\Sigma_2$ be disjoint unions of 1-dimensional, oriented manifolds. A morphism $\Sigma_1\to \Sigma_2$ will be a 2-dimensional manifold $W$ such that $\partial W=\Sigma_1 \cup \Sigma_2 \cup W'$, where $W'$ is a cobordism from $\partial \Sigma_1$ to $\partial \Sigma_2$.\footnote{As the boundaries of $\Sigma_1$ and $\Sigma_2$ consist of a finite number of points, then this cobordism can be regarded as an arrow in the category $\tsf{Cob}(1)$.} This cobordism is called the \emph{constrained boundary}; see figure \ref{constrained_boundary}.

\begin{figure}
\begin{center}
\includegraphics[width=8cm]{constrained_boundary} \\
\end{center}
\vspace{-10pt}
\caption{A cobordism from a disjoint union of an open and a closed string to a disjoint union of two open strings. The constrained boundary is marked with red lines.}
\label{constrained_boundary}
\end{figure}

There is another layer of structure, which is attached to the endpoints of the open strings; these are called D-branes, and several considerations lead to consider them as part of an additive category, which we have denoted by $\scr{B}$. These branes are boundary conditions for the boundary of the string; in other words, they impose restrictions to the behaviour of the strings in spacetime. Recall that, given objects $a,b\in \scr{B}$, arrows between them (i.e. open strings with labelled endpoints, which are all diffeomorphic) are represented by a vector space $O_{ab}$. That is, we are distinguishing all the topologically-equivalent open strings by means of the behaviour of its endpoints.

Now, we define the cobordism category $\tsf{Cob}_{\scr{B}}(2)$: its objects are 1-dimensional manifolds diffeomorphic to disjoint unions of circles (closed strings) and closed intervals (open strings) with endpoints labelled with objects of $\scr{B}$; arrows between these manifolds are the previously described cobordisms (with constrained boundaries considered for open strings).

\begin{figure}[!ht]
\begin{center}
\includegraphics[width=7cm]{transitions} \\
\end{center}
\vspace{-10pt}
\caption{The transition maps $\iota_a$ and $\iota^a$ are given by the decay of a closed string into an open one (i.e. a cobordism from the circle $S^1$ to the interval $\begin{xy}
(0,0)*+{a}; (10,0)*+{a} **\dir{-} ?(.6)* %
\dir{>}
\end{xy}$) and viceversa, respectively.}
\label{transitions}
\end{figure}

Let us now sketch a functorial definition for an open-closed theory: it is a functor
$$Z:\tsf{Cob}_{\scr{B}}(2)\longrightarrow \tsf{Vect}$$
from the cobordism category to the category of finite-dimensional complex vector spaces, such that:

\begin{itemize}

\item $Z$ sends disjoint unions to tensor products (i.e. it is a monoidal functor);

\item Diffeomorphic cobordisms have equal images through $Z$;\footnote{As was considered before for closed theories, the word ``diffeomorphism'' here means ``orientation-preserving diffeomorphism''.}

\item the image of an open string $\begin{xy}
(0,0)*+{a}; (10,0)*+{b} **\dir{-} ?(.6)* %
\dir{>}
\end{xy}$, is the vector space $O_{ab}$.

\end{itemize}

Moreover, $Z$ is subject to the following conditions:

\begin{enumerate}

\item The restriction of $Z$ to the subcategory $\tsf{Cob}(2)$ is a closed theory.

\item For each label $a\in \scr{B}$, there exists a linear form $\theta_a:O_{aa}\to \comp$ which makes $O_{aa}$ a Frobenius algebra (the product is given by the pair of pants for open strings).

\item The composition of $\theta_a$ and the image of the pair of pants cobordism $O_{ab}\otimes O_{ba}\to O_{aa}$ (see figure \ref{frobenius_openclosed_1}B) is a perfect pairing. In particular, $Z$ is involutory; that is, the image of a 1-manifold (circle or interval) with the opposite orientation is canonically isomorphic to the corresponding dual vector space. For example, $Z(\begin{xy}
(0,0)*+{b}; (10,0)*+{a} **\dir{-} ?(.6)* %
\dir{>}
\end{xy})=O_{ba}\cong O_{ab}^*$.

\item The diagram
$$\xymatrix{
O_{ab}\otimes O_{ba} \ar[dd]_{\text{twist}} \ar[r] & O_{aa} \ar[dr]^{\theta_a} & \\
 & & \comp \\
O_{ba}\otimes O_{ab} \ar[r] & O_{bb} \ar[ur]_{\theta_b} & }
$$
commutes.

\item The image of the closed-to-open cobordism (see figure \ref{interactions}) is a central algebra homomorphism, denoted by $\iota_a$.

\item If $\iota^a$ denotes the image of the open-to-closed transition, then $\theta (\iota^a(\sigma )x)=\theta_a (\sigma \iota_a(x))$, where $\sigma$ is an element of the vector space $O_{aa}$ and $\theta $ is the linear form of the Frobenius algebra $A=Z(S^1)$ corresponding to the closed sector (the image of the circle).

\item The Cardy condition holds.

\end{enumerate}


Consistency of the previous algebraic structures is proved in a sewing theorem in the appendix of \cite{moore_segal1}, using techniques of Morse theory. There are several interpretations of branes in physics. For a nice, basic and brief exposition of different interpretations of branes in string theory, the reader is referred to \cite{moore:_whatisbrane}.


%%%%%%%%%%%%%%%%%%%%%%%%%%%%%%%%%%%%%%%%%%%%%%%%%%%%%%%
\subsection{Boundary Conditions in the Semisimple Case}
\label{subsec_boundary_semisimple}

In this section we will discuss some results of G. Moore and G. Segal \cite{moore_segal1} regarding the structure of the algebras $O_{ab}$ corresponding to the open sector. We will only consider the case for which the Frobenius algebra $A$ of the closed sector is semisimple.

Let $A$ be an associative, commutative, semisimple Frobenius algebra over $\comp$, and supppose $\dim_\comp A=n$. We then have a system of orthogonal idempotents $e_1,\dots ,e_n$ which determine the simple components; i.e.
$$A\cong \bigoplus_i\comp e_i,$$
and each summand $\comp e_i$ is isomorphic to $\comp$.

\begin{theorem}[\cite{moore_segal1}, Theorem 2]\label{theorem_2}
For each object $a\in \scr{B}$, the algebra $O_{aa}$ is semisimple.
\end{theorem}
\begin{proof}
Let $\sigma_i:=\iota_a(e_i)$; then, $\{\sigma_1,\dots ,\sigma_n\}$ is a set of central, orthogonal idempotents in $O_{aa}$; as $\iota_a(1)=1$ and $1=\sum_ie_i$,
$$1=\sum_i\sigma_i$$
and thus $O_{aa}$ can be decomposed as a sum $\bigoplus \sigma_iO_{aa}$. We will show that each summand is a simple algebra.

Let $O_i$ be the ideal $\sigma_iO_{aa}$; then, as $\sigma_i$ is central, $O_i$ is an algebra over $\comp e_i\cong \comp$, and so we can restrict our attention to each summand.

By definition of $\pi_a^a$ and centrality, we have that the restriction of $\pi^a_a$ to $O_i$ takes values in $O_i$. Assume now that $\iota^a(\sigma_ix)=\sum_k\alpha_ke_k$; applying $\iota_a$ we obtain $\iota_a\iota^a (\sigma_ix)=\sum_k\alpha_k\sigma_k$. On the other hand, we have that $\pi^a_a(\sigma_ix)=\sigma_iy$ for some $y\in O_{aa}$. By the Cardy condition, we then have that $\sigma_iy=\sum_k\alpha_k\sigma_k$. Multiplying by $\sigma_i$ and by $\sigma_{j}$ for $j\neq i$, we obtain that $\alpha_k=\delta_{ik}$. This implies that $\iota^a(\sigma_i x)=\alpha_ie_i$ or, in other words, that the restriction of $\iota^a$ to $O_i$ takes values in $\comp e_i$. We can then conclude that there exists a complex number $\alpha$ such that
$$\iota^a(\sigma_i)=\alpha e_i.$$
By the Cardy condition, we have
$$\alpha \sigma_i=\iota_a (\iota^a (\sigma_i))=\pi_a^a (\sigma_i)=\chi_{O_i},$$
where $\chi_{O_i}$ is the Euler element of the algebra $O_i$ (the last equality holds as $\sigma_i$ is the unit of the algebra $O_i$). Applying $\theta_a$ to this last equality, we get
$$\alpha \theta_a(\sigma_i)=\theta_a (\chi_{O_i})=\dim_\comp O_i$$
by \ref{euler_trace}. So if $\sigma_i\neq 0$ then $\dim_{\comp}O_i>0$, $\alpha \neq 0$ and hence the Euler element $\chi_{O_i}$ is invertible. By \ref{euler_inv}, the algebra $O_i$ is then semisimple and can be represented as a sum
$$O_i=\bigoplus_jO_{ij}$$
of simple algebras. By definition, the map $\pi_a^a$ sends each summand $O_{ij}$ to itself. We will rely again on the Cardy condition to show that the algebra $O_i$ is in fact simple. Assume that $\tau_j$ is the unit of the simple algebra $O_{ij}$, and then $O_i=\sum_j\tau_jO_i$ (that is, $O_{ij}=\tau_jO_i$); then $\iota^a(\tau_j)=\alpha' e_i$, and applying $\iota_a$ we obtain that $\iota_a(\iota^a(\tau_j))=\alpha \alpha' \sigma_i$. By the Cardy condition, it is valid to write the identity
$$\alpha \alpha'\sigma_i=\lambda \tau_j$$
for some complex number $\lambda$. But, as $\sigma_i=\sum_j\tau_j$, for the previous equality to make sense it is necessary that $\tau_k=0$ for $k\neq j$; in other words, $O_i=O_{ij}$ and thus it is simple. This finishes the proof.
\end{proof}

\begin{obs}\label{dimensions}
By the previous result, we have that $O_{aa}$ can be regarded as a sum $\bigoplus_i M(a,i)$ of matrix algebras $M(a,i):=\opnm{M}_{d_{(a,i)}}(\comp )$. In other words, we can find complex vector spaces $V_{a,i}$ such that
\begin{equation}\label{iso_theorem2_ms}
O_{aa}\cong \bigoplus_{i=1}^n\opnm{End}(V_{a,i}),
\end{equation}
where $\dim V_{a,i}=d(a,i)$. Moreover, the matrix algebra $\opnm{M}(a,i)=\opnm{End}(V_{a,i})$ corresponds under the isomorphism \eqref{iso_theorem2_ms} with the subalgebra $\iota_a(e_i)O_{aa}$. Elements of $O_{aa}$ will be denoted by a tuple $\sigma =(\sigma_1,\dots ,\sigma_n)$, where $\sigma_i\in M(a,i)$. If $\varepsilon_i\in O_{aa}$ denotes the tuple consisting of the identity matrix $1_{a,i}\in M(a,i)$ in the $i$-th coordinate and all others equals to zero, then $\iota_a(e_i)=\varepsilon_i$ or is equal to zero.
\end{obs}

We can give an explicit characterization for the morphisms $\theta_a$, $\iota^a$ and $\pi^a_b$. For $\sigma =(\sigma_1,\dots ,\sigma_n )\in O_{aa}$, the equality $\theta_a(\sigma \tau)=\theta_a (\tau \sigma)$ implies that
$$\theta_a(\sigma )=\sum_i\lambda_i \opnm{tr}(\sigma_i)$$
for some constants $\lambda_i \in \comp$.

We will now find an expression for the isomorphism $\overline{\theta}_a^{-1}$ (recall that $\overline{\theta}_a:O_{aa}\to O_{aa}^*$ is given by $\overline{\theta}_a(\sigma )(\tau )=\theta_a(\tau \sigma )$). For simplicity, in this computation we will work with one summand $M(a,i)$, considering
$$\overline{\theta}_a:M(a,i)\longrightarrow M(a,i)^*.$$
Let us denote by $\{\varepsilon_{jk}\}$ the canonical basis for $M(a,i)$ (the only non-zero entry of the matrix $\varepsilon_{jk}$ is the one corresponding to the $j$-th row and the $k$-th column), and let $\{\varepsilon^{jk}\}$ be the corresponding dual basis. Fix now $j,k$ and assume that $\overline{\theta}_a^{-1}(\varepsilon^{jk})=\sum_{r,s}\alpha_{rs}\varepsilon_{rs}$. Applying $\overline{\theta}_a$ and then evaluating at $\varepsilon_{lt}$ we obtain $\alpha_{rs}=\frac{\delta^{jl}_{kt}}{\lambda_i }$ and thus
$$\overline{\theta}_a^{-1}(\varepsilon^{jk})=\frac{\varepsilon_{kj}}{\lambda_i}.$$

Recall that the adjoint relation for $\iota_a$ and $\iota^a$ is given by
$$\theta (\iota^a (\sigma )x)=\theta_a(\sigma \iota_a(x)),$$
where $\sigma \in O_{aa}$ is arbitrary. Take $\sigma =(\sigma_1,\dots ,\sigma_n)\in O_{aa}$, $x=e_i$ and assume that $\iota^a(\sigma )=\sum_j\beta_je_j$. By the adjoint relation we then have $\theta (\iota^a(\sigma )e_i)=\beta_i \theta (e_i)=\theta_a (\sigma \varepsilon_i)=\theta_a (\sigma_i\varepsilon_i)=\lambda_i\opnm{tr}(\sigma_i)$ and thus
$$\iota^a(\sigma )=\sum_i\frac{\lambda_i\opnm{tr}(\sigma_i)}{\theta (e_i)}e_i.$$

We can now use the Cardy condition to derive an expression for the map $\pi^a_b$. Let $\sigma:=(\sigma_1,\dots ,\sigma_n)\in O_{aa}$; then, as $\pi^a_b=\iota_b\iota^a$, we have that
$$
\pi^a_b(\sigma ) = \iota_b \left (\sum_i\frac{\lambda_i\opnm{tr}(\sigma_i)}{\theta (e_i)}e_i\right ) = \sum_i\frac{\lambda_i\opnm{tr}(\sigma_i)}{\theta (e_i)}\iota_b(e_i)
$$

Fix now a label $a$ and consider $\pi^a_a:O_{aa}\to O_{aa}$. As $\pi_a^a$ preserves summands (see the proof of \ref{theorem_2}), we can restrict our attention to the restriction $\pi^a_a:M(a,i)\to M(a.i)$. Let $\varepsilon_{jk}\}$ be the canonical basis of $M(a,i)$ and $\{\varepsilon^{jk}\}$ its dual. We then have
$$
\begin{aligned}
\chi_{M(a,i)}=\pi_a^a (1_{a,i}) &= \sum_{j,k}\varepsilon_{jk}\overline{\theta}_a^{-1}(\varepsilon^{jk}) \\
																&= \frac{1}{\lambda_i}\sum_{j,k}\varepsilon_{jk}\varepsilon_{kj} \\
																&= \frac{1}{\lambda_i}\sum_k\left (\sum_j\varepsilon_{jj}\right ) \\
																&= \frac{d_{a,i}}{\lambda_i}1_{a,i}.\\
\end{aligned}
$$
On the other hand,
$$\iota_a(\iota^a(1_{a,i}))=\frac{\lambda_i\opnm{tr}(1_{a,i})}{\theta (e_i)}\iota_a(e_i)=\frac{\lambda_i d_{a,i}}{\theta (e_i)}1_{a,i}.$$
By the Cardy condition, $\pi^a_a(1_{a,i})=\iota_a(\iota^a(1_{a,i}))$ and thus $\frac{d_{a,i}}{\lambda_i}=\frac{\lambda_i d_{a,i}}{\theta (e_i)}$ which yields the equality
$$\lambda_i^2=\theta (e_i).$$
Fixing a square root $\lambda_i=\sqrt{\theta (e_i)}$ for each $i$, we arrive at the following expressions
$$
\begin{aligned}
\theta_a (\sigma ) &= \sum_i\sqrt{\theta (e_i)} \opnm{tr}(\sigma_i), \\
\iota^a(\sigma )   &= \sum_i\frac{\opnm{tr}(\sigma_i)}{\sqrt{\theta (e_i)}}e_i, \\
\pi_b^a(\sigma )   &= \sum_i\frac{\opnm{tr}(\sigma_i)}{\sqrt{\theta (e_i)}}\iota_b(e_i), \\
\end{aligned}
$$
where in the last equality, the trace $\opnm{tr}$ is the one corresponding to $O_{aa}$.

A characterization like the one provided in theorem \ref{theorem_2} holds for the spaces $O_{ab}$.

\begin{lemma}[\cite{moore_segal1}]\label{ms_theorem2_bis}
If $C$ is semisimple, then for each pair $a,b\in \scr{B}$ we have an isomorphism
\begin{equation}\label{semisimple_2bis}
O_{ab}\cong \bigoplus_{i=1}^n\operatorname{Hom}_{\comp }(V_{a,i},V_{b,i}),
\end{equation}
for some finite-dimensional complex vector spaces $V_{a,i},V_{b,i}$.
\end{lemma}

Note that the vector spaces in the right hand side of equation \eqref{semisimple_2bis} are the ones appearing in the decompositions of $O_{aa}$ and $O_{bb}$; see remark \ref{dimensions}.

\begin{proof}
By the centrality condition, we have that
$$O_{ab,i}:=\iota_a(e_i)O_{ab}=O_{ab}\iota_b(e_i),$$
and $O_{ab,i}$ is then a $(O_{a,i},O_{b,i})$-bimodule, where $O_{a,i}:=\iota_a(e_i)O_{aa}$. By the previous result, there exists vector spaces $V_{a,i}$ and $V_{b,i}$ such that $O_{a,i}\cong \operatorname{End}_{\comp}(V_{a,i})$ and $O_{b,i}\cong \operatorname{End}_{\comp}(V_{b,i})$

Things to check:
\begin{itemize}
\item The unique irreducible representation of $\opnm{End}(V)$ is $V$.
\item The unique $(\opnm{End}(V),\opnm{End}(W))$-bimodule is $V^*\otimes W$.
\end{itemize}

Hence, a nonnegative integer $n_{ab}$ exists verifying
$$O_{ab,i}\cong (V_{a,i}^*\otimes V_{b,i})^{n_{ab}}.$$
Let $\{v_\alpha\}$ and $\{w_\beta\}$ be basis for $V_{a,i}$ and $V_{b,i}$ respectively. Then $\left \{v_{\alpha ,k}^*\otimes w_{\beta ,k}\right \}$ ($k=1,\dots ,n$) is a basis for $O_{ab,i}$, where $\{v^*_\alpha \}$ is the basis of $V_{a,i}^*$ dual to $\{v_\alpha \}$ (the index $k$ indicates the corresponding summand $V_{a,i}^*\otimes V_{b,i}$). We can now invoke the Cardy condition. If $\sigma \in O_{aa}$, then by definition of $\pi^a_b$ we have that
$$\pi^a_b(\sigma )=n_{ab}\sum_i\opnm{tr}_{V_{i,a}}(\sigma )\iota_a(e_i).$$
Comparison with the expression for $\iota_b\iota^a(\sigma )$ yields $n_{ab}=1$.
\end{proof}

\begin{obs}\label{dimensions_2}
Note that the vector spaces $V_{a,i}$ can be taken as the ones appearing on remark \ref{dimensions}.
\end{obs}



%%%%%%%%%%%%%%%%%%%%%%%%%%%%%%%%%%%%%%%%%%%%%%%%%%%%%%%%
\subsection{The Maximal Category of Boundary Conditions}
\label{max_cat_moore_segal}

This section will be devoted to the description of a particular class of categories of boundary conditions. We will just write a brief overview of the main definitions and results. For details, the interested reader may consult the original article \cite{moore_segal1}. In chapter \ref{local_description}, all the statements are proved in a more general setting.

For the following definition to make sense we need to consider small categories.

\begin{defi}
We will say that a category of branes $\scr{B}$ is \emph{maximal} if, given another category of branes $\scr{B}$, there exists an injective map $\opnm{sk}\scr{B}'\to \opnm{sk}\scr{B}$, where $\opnm{sk}$ stands for ``skeleton''.
\end{defi}

The following theorem is crucial for the description of the category $\scr{B}$.

\begin{theorem}
Any maximal category of boundary conditions $\scr{B}$ enjoys the following properties:
\begin{itemize}
\item $\scr{B}$ is additive.
\item There exists a functorial action $\tsf{Vect}\times \scr{B}\to \scr{B}$ of the category of finite dimensional complex vector spaces and
\item $\scr{B}$ is pseudo-abelian (for the definition of pseudo-abelian category, see \ref{linear_cats}).
\item There exists a label $a_0$ such that $\iota_{a_0}:A\to O_{a_0a_0}$ is an isomorphism.
\end{itemize}
\end{theorem}

Let us give a brief discussion of the ideas behind this theorem (a complete treatment is given in \ref{maximal_cardy_fibrations}). Basically, we can enlarge any category of boundary conditions by defining an additive structure and/or a functorial action of the category of vector spaces and/or kernels of idempotent maps. In other words, given labels $a,b$ and a complex vector space $V$, we can build up a new category of boundary conditions in which the labels $a\oplus b$ and $V\otimes a$ are meaningful (that is, for these new labels we can define all the transition homomorphisms and verify that the centrality condition, adjoint relation and Cardy condition hold). A similiar consideration holds regarding the pseudo-abelian structure: we have idempotent elements $p\in O_{aa}$; then, we can consider both the kernel $\opnm{Ker}p$ and the cokernel $\opnm{Coker}p$ and verify that all the axioms are still satisfied after adding these objects to the collection of branes.

\begin{proposition}
For each $i=1,\dots ,n$ there exists an object $a_i\in \scr{B}$ such that $O_{a_ia_i}\cong \comp$ as $\comp$-algebras.
\end{proposition}

This proposition is equivalent, thanks to \ref{ms_theorem2_bis}, to the existence of a boundary condition $a_0$ such that $\iota_{a_0}:C\cong O_{a_0a_0}$ (see chapter 4 for more details). It basically states that we have one-dimensional vector spaces among the open algebras.

The following result classifies maximal categories of labels in the semisimple case.

\begin{theorem}[\cite{moore_segal1}, Theorem 3]\label{theorem_3}
If the Frobenius algebra $A$ corresponding to the closed sector is semisimple, then the category of branes $\scr{B}$ is equivalent to the category $\tsf{Vect}(X)$ of vector bundles over the space $X=\{e_1,\dots ,e_n\}$ consisting of the orthogonal idempotents in $A$ such that $\sum_ie_i=1$.
\end{theorem}

Let now $E\to X$ be a vector bundle over $X$; then, if $E_i$ denotes the fiber over $e_i\in X$, the assignment
$$E\longmapsto (E_1,\dots ,E_n)$$
defines an equivalence (in fact, an isomorphism) between the category $\tsf{Vect}(X)$ and the $n$-fold product $\tsf{Vect}^n$. Hence, $\scr{B}$ is a 2-vector space of rank $n$.




%%%%%%%%%%%%%%%%%%%%%%%%%%%%%%%%%%%%%%%%%%%%%
%%%%%%%%%%%%%%%%%%%%%%%%%%%%%%%%%%%%%%%%%%%%%
\section{Bundles of Algebras and F-manifolds}

Vector bundles with an algebra structure on the fibers will be the main characters in most part of this work, so we will first focus on generalities about this kind of bundles.

\begin{obs}
We will work with ringed spaces $(M,\scr{O}_M)$. As a matter of notation, we will often write only $M$ instead of $(M,\scr{O}_M)$ and also $\scr{O}$ for the structure sheaf, when no possibility of confusion about the base manifold can occur. On the other hand, these ringed spaces will always be smooth ($C^\infty $) manifolds or complex manifolds, with the usual structure sheaves.
\end{obs}

Let $M$ be a ringed space with structure sheaf $\scr{O}_M$. A \emph{bundle of algebras} over $M$ is a (complex or holomorphic) vector bundle $E\rightarrow M$ together with a bundle map
$$\mu :E\otimes E\longrightarrow E$$
(equivalently, with an $\scr{O}_M$-linear morphism $\Gamma (E)\otimes \Gamma (E)\rightarrow \Gamma (E)$) such that, for each $x\in M$, the restriction $\mu_x$ of $\mu$ to $E_x\otimes E_x$ is a multiplication which induces an associative $\comp$-algebra structure on $E_x$. Moreover, we require the existence of a global section $1:M\rightarrow \Gamma (E)$ such that $1(x)=1_x$ is the unit of the algebra $E_x$.

These algebra bundles are also called \emph{bundles with multiplication}. If $X,Y$ are sections of $E$, we will denote their product by $XY$. When $E=TM$ for some space $M$, then $M$ is called a \emph{manifold with multiplication} (\emph{on the tangent sheaf}).

The next examples show some important examples of algebra bundles in the literature.

\begin{ej}\label{azumaya_example}
An \emph{Azumaya bundle} or \emph{Azumaya algebra over $M$} is a vector bundle $E$ over $M$ such that the fibers $E_x$ are isomorphic to a matrix algebra $\text{M}_n(\comp )$; see section \ref{subsec_azumaya}. Equivalently, a sheaf of algebras $\EuScript{A}$ over $M$ is called an \emph{Azumaya algebra} over $M$ if it is locally isomorphic to the sheaf $\text{M}_n(\scr{O}_M)$ (this is the same as saying that $\EuScript{A}$ is locally free as a sheaf of $\scr{O}_M$-modules and the reduced fibre $\EuScript{A}_x\otimes_{\scr{O}_{M,x}}k_x$ is isomorphic to $\text{M}_n(\comp )$ for each $x\in M$, where $k_x$ is the field $\scr{O}_{M,x}/\{f \, | \, f(x)=0\}$). By defining a certain equivalence relation on these isomorphism classes we obtain the Brauer group $\text{Br}(M)$ of $M$. By a theorem of Serre, for certain spaces $M$ (e.g. compact ones), this Brauer group is isomorphic to the torsion subgroup of the third cohomology group $H^3(M;\ent )$; see \cite{grothendieck68:_le_group_de_brauer_i}. 
\end{ej}

\begin{ej}
Algebra bundles were considered by Dixmier and Douady in \cite{dd:_champs} to give a geometric description of the third cohomology group of a topological space: if $H$ is a separable Hilbert space, $\text{U}(H)$ its unitary group and $\mathbbm{P}\text{U}(H)$ the corresponding projective group, then there exists a bijection between the group of isomorphism classes of principal $\mathbbm{P}\text{U}(H)$-bundles and the third cohomology group $H^3(M;\ent )$. As the group $\mathbbm{P}\text{U}(H)$ can be identified with the group of automorphisms $\scr{K}\rightarrow \scr{K}$ of the $C^*$-algebra of compact operators on $H$, we then obtain that the group $H^3(M,\ent )$ is in bijective correspondence with isomorphism classes of (locally trivial) bundles over $M$ with fiber $\scr{K}$. As $\scr{K}^{\otimes 2}\cong \scr{K}$, the set of isomorphism classes of algebra bundles with fiber $\scr{K}$ is a group under the tensor product, which is called the infinite Brauer group; the previous bijection then turns out to be a group isomorphism. See also \cite{brylinski:_loop_spaces} and \cite{parker:_brgroup}.
\end{ej}

%%%%%%%%%%%%%%%%%%%%%%%%%%%%%%%%%%%%%%%%%%%%%
\subsection{The Spectral Cover of a Manifold}

We shall now focus on the definition of the spectral cover of a bundle of algebras. We consider the particular case that is useful to us and refer the reader to the appropriate literature for further details.

Assume that $E$ is a bundle of algebras over $M$ with the property that for each $x\in M$, the fibre $E_x$ is a commutative, semisimple $\comp$-algebra. That is, $E_x$ has a decomposition $E_x=\bigoplus_ie_i(x)E_x$, where $\{e_i(x)\}$ is a basis of orthogonal, simple idempotents for $E_x$. Consider now the subset $S_E\subset E^*$ consisting of algebra homomorphisms; that is, over each $x\in M$, $S_E$ contains all linear functionals $\varphi_x:E_x\to \comp$ such that $\varphi_x$ is multiplicative and $\varphi_x(1)=1$. We give to $S_E$ the subspace topology.

\begin{proposition}\label{alg_bundle_idemp_sec}
Let $x_0\in M$ be a point such that $E_{x_0}$ is semisimple. Then, there exists an open neighborhood $U\ni x_0$ such that $E_x$ is semisimple for each $x\in U$. Moreover, there exist unique, up to reordering, local sections $e_1,\dots ,e_n:U\to E$ such that $e_ie_j=\delta_{ij}e_i$ and $E=\bigoplus_ie_iE$ over $U$.
\end{proposition}

Such an open subset will be said to be \emph{semisimple}.

\begin{proof}
Assume that $E_{x_0}$ is semisimple, with decomposition $E_{x_0}=\bigoplus_ie_i(x_0)E_{x_0}$. We then have an isomorphism of algebras $E_{x_0}\to \comp^n$, where the algebra structure on the right is the trivial one. This isomorphism is given by $e_i(x_0)\mapsto e_i$, where $e_i$ is the $i$-th vector of the canonical basis. Let $X_0$ (which we can identify with a tuple $z_0\in \comp^n$) be a vector such that the left translation $L_{X_0}$ has $n$ distinct eigenvalues $\lambda_{1,0},\dots ,\lambda_{n,0}$ (and thus $z_0=(\lambda_{1,0},\dots ,\lambda_{n,0}))$. We can then find an open subset $U\ni x_0$ and maps $\lambda_1,\dots ,\lambda_n:U\to \comp$ such that
\begin{enumerate}
\item $\lambda_i(x_0)=\lambda_{i,0}$ for each $i$ and
\item $\lambda_i(x)\neq \lambda_j(x)$ for each $x\in U$ and distinct $i,j$.
\end{enumerate}
We now define a (local) section $X:U\to \comp^n$ by
$$X(x)=(\lambda_1(x),\dots ,\lambda_n(x)).$$
Then, for each $x\in U$, the map $L_{X(x)}\in E_x$ has $n$ distinct eigenvalues, and thus the algebra $E_x$ is semisimple.

The idempotent sections $e_i$ are defined in this trivialization chart by the equation
$$e_i(x)=e_i,$$
and uniqueness follows from uniqueness of the decomposition \eqref{gen_eigenspace}.
\end{proof}

The previous result produces the following

\begin{cor}
The set $S_E$ together with the canonical projection $\pi :S_E\to M$ is a $\dim M$-sheeted covering space.
\end{cor}
\begin{proof}
Pick a point $x\in M$ and let $U\ni x$ be a semisimple neighborhood, with local idempotent sections $e_1,\dots ,e_n:U\to E$, where $n=\dim M$. If $\varphi_x:E_x\to \comp$ is an algebra homomorphism, then its kernel is a maximal ideal. Hence, there exists an index $i$ such that
$$\opnm{Ker}\varphi_x =\bigoplus _{j\neq i}e_j(x)E_x.$$
In other words, we have $\varphi_x(e_j(x))=\delta_{ij}$, and we can then identify $S_E$ with a subset of $E$ itself, namely by the correspondence $\varphi_x\mapsto e_i(x)$. In particular, this shows also that $\pi^{-1}(U)$ is precisely the disjoint union of $n$ copies of $U$, each sheet corresponding to the image of $U$ by each idempotent section.
\end{proof}

\begin{defi}
When $E=TM$, the covering $\pi :S_{TM}\to M$ is called the \emph{spectral cover of $M$}. We will denote it just by $S$ instead of $S_{TM}$.
\end{defi}

\begin{obs}
The \emph{caustic} $K\subset M$ consists precisely of points $x\in M$ for which $E_x$ is not semisimple. The caustic is either empty or an hypersurface in $M$ (\cite{hertling:_fman}, proposition 2.6). We will deal with bundles for which $K=\emptyset$. In this case, the spectral cover is an (unramified) $n$-sheeted covering space; ramifications appear over points $x\in K$. For more details, see \cite{hertling:_fman}.
\end{obs}

These constructions are part of a more general framework, namely that of the \emph{analytic spectrum}, introduced by C. Houzel \cite{houzel_gal2} to study finite morphism of analytic spaces. He defines the analytic spectrum for algebras of finite presentation over an analytic space, which include finite algebras (those algebras which are coherent modules): let $\Gamma$ be a finite presentation $\scr{O}_M$-algebra and $f:N\to M$ a space over $M$ (in particular, if $E$ is a vector bundle, then its sheaf of sections is coherent and thus of finite presentation). Define a contravariant functor $S_\Gamma$ from spaces over $M$ to the category of sets by
$$S_\Gamma (N,f)=\operatorname{Hom}_{\scr{O}_N\text{-}{\rm alg}}(f^*\Gamma ,\scr{O}_N)$$
(the pair $(N,f)$ is short for $f:N\to M$).\footnote{Note that if $\Gamma$ is an $\scr{O}_M$-algebra, then so is $f^*\Gamma$.} This functor is then representable, and we have a bijection between $S_\Gamma (N,f)$ and holomorphic maps $N\to \operatorname{Specan} \Gamma$, where $\operatorname{Specan}\Gamma$ is the analytic spectrum. Even with these nice algebras, the space $\operatorname{Specan}\Gamma$ may have singularities. For detailed descriptions we refer the reader to \cite{houzel_gal2}; check also \cite{fischer:_cng}. The case in which we are interested deals with a bundle of algebras $E$ such that $E_x$ is semisimple for each $x$ (see below). If $M=N$ and $f:M\to M$, then the construction of the analytic spectrum provides a bijection between the subspace of the dual bundle $(f^*E)^*$ consisting of morphisms of algebras and maps $M\to \operatorname{Specan}\Gamma_E$.\footnote{Note that there is an isomorphism between $\Gamma_{E^*}$ and $\Gamma_{E}^*=\opnm{Hom}_{\scr{O}_M}(\Gamma_E,\scr{O}_M)$ induced by the pairing between $\Gamma_{E^*}$ and $\Gamma_E$.} For $f=\text{id}_M$, this is just expressing that every morphism of algebras $\varphi :E\to \comp$ is determined by a map $M\to \operatorname{Specan}\Gamma_E$ (for each $x$ this is just choosing the kernel of the restriction $\varphi_x:E_x\to \comp$).

\begin{proposition}\label{isom_1}
For a bundle of algebras $E$ over $M$ there exists an isomorphism of $\mathscr{O}_M$-algebras
\begin{equation}\label{iso_2}
\pi_*\scr{O}_{S_E}\cong \Gamma_E,
\end{equation}
\end{proposition}
\begin{proof}
consider the sequence of maps
$$\Gamma_E\longrightarrow p_*\mathscr{O}_{E^*}\longrightarrow \pi_*\mathscr{O}_{S_E},$$
$$X\longmapsto \widetilde{X}\longmapsto \widetilde{X}|_S$$
where $p:E^*\rightarrow M$ is the canonical projection (we are considering $S_E$ as a subspace of $E^*$; then $\pi$ is just the restriction of $p$ to $S_E$), and $\widetilde{X}:p^{-1}(U)=E^*|_U\rightarrow \comp$ is the map given by
$$\widetilde{X}(x,\varphi )=\varphi (X(x)).$$
The composite map
\begin{equation}\label{iso}
\Gamma_E\longrightarrow \pi_*\scr{O}_{S_E}
\end{equation}
is then easily seen to be an isomorphism of $\mathscr{O}_M$-algebras (recall that $(x,\varphi )\in S_E$ if and only if $\varphi$ is an algebra homomorphism).

The inverse can be described easily: Given a map $\widetilde{f}:\pi^{-1}(U)\rightarrow \comp$, let $X_{\widetilde{f}}\in \Gamma_E(U)$ be the local section defined as follows: pick an $x\in U$ an assume that $U$ is semisimple (if it is not, we can choose a smaller open neighborhood around $x$); let $\{e_i\}$ be a local frame of idempotent sections for $E|_U$. Then
$$X_{\widetilde{f}}(x)=\sum_i\widetilde{f}(x,\varphi_i)e_i(x),$$
where $\varphi_i:E_x\rightarrow \comp$ is the algebra homomorphism which verifies $\varphi_i(e_i(x))\neq 0$ (in fact, $\varphi_i (e_i(x))=1$ as $\varphi_i(1)=1$). The assignment $\widetilde{f}\mapsto X_{\widetilde{f}}$ is then the inverse of \eqref{iso}.
\end{proof}

Combining the previous result with propositions \ref{direct_covering} and \ref{fibre}, for a point $x_0\in M$ we obtain isomorphisms
$$
\Gamma_{E,x_0} \cong \bigoplus_{y\in \pi^{-1}(x_0)}\scr{O}_{S_E,y}$$
$$
E_{x_0}        \cong \Gamma_{E,x_0}\otimes_{\scr{O}_{x_0}}\comp\cong \bigoplus_{y\in \pi^{-1}(x_0)}\scr{O}_{S_E,y} \otimes_{\scr{O}_{x_0}}\comp .$$

Moreover, each summand $\scr{O}_{S,y}\otimes_{\scr{O}_{x_0}}\comp$ is invariant under ths action of any multiplication operator, and thus it is the space of generalized eigenvectors.

We can now prove the following result, which is in fact Housel's definition of the spectral cover.

\begin{proposition}
Let $E\to M$ be a bundle of associative and commutative algebras. Then
\begin{enumerate}
\item The analytic spectrum $S_E$ represents the functor (which we denote with the same symbol) $S_E (N,f)=\operatorname{Hom}_{\scr{O}_N-\text{alg}}(f^*E,\comp)$ from spaces over $M$ to the category of sets (here $\comp$ means the trivial line bundle $N\times \comp$).
\item If $E_x$ is semisimple for each $x$, then $\pi :S_E\to M$ is a covering space.
\end{enumerate}
\end{proposition}
\begin{proof}
Let us first fix some notation: for $y\in N$, the orthogonal complement (with respect to the product of the algebra $E_{f(y)}$) of the simple component spanned by $u(y)$ is the hyperplane spanned by all the other simple idempotents; we will denote this complement by $\langle u(y) \rangle ^{\perp}$. We define a biyection
$$\Phi :C^\infty (N,S_E)\longrightarrow \operatorname{Hom}_{\scr{O}_N-\text{alg}}(f^*E,\comp )$$
by the following rule: for $u:N\to S_E$, let $\Phi (u):f^*E\to \comp$ be the unique map which verifies
\begin{enumerate}[(a)]
\item $\Phi (u)_y:E_{f(y)}\to \comp$ is a unit-preserving morphism of algebras for each $y\in N$ and
\item $\opnm{Ker} (\Phi (u)_y)=\langle u(y)\rangle ^{\perp}$.
\end{enumerate}
Assume that $\Phi (u)=\Phi (v)$; then, for each $y\in N$, $\langle u(y) \rangle ^{\perp }=\opnm{Ker} \Phi (u)_y=\opnm{Ker} \Phi (v)_y=\langle v(y)\rangle ^{\perp }$, and then necessarily $u(y)=v(y)$. To check surjectivity, let $\varphi :f^*E\to N\times \comp$ be a morphism of algebra bundles. Define $u:N\to S_E$ by the assignment $u(y)=e_{\varphi}(y)$, where $e_{\varphi}(y)$ is the unique simple idempotent which verifies $\varphi_y (e_{\varphi}(y))=1$, where $\varphi_y:E_{f(y)}\to \comp$ is the restriction of $\varphi$ to the fibre $E_{f(y)}$. To check smoothness, consider the following commutative diagram
$$
\xymatrix{
N\ar[r]^u \ar[dr]_f & S_E \ar[d]^{\pi } \\
                    & M.}
$$
Then, smoothness of $u$ follows from smoothness of $\pi$, $f$ and the next item.

For the second assertion, let $x\in M$ and $U\ni x$ a semisimple neighborhood, with local frame $\{e_1,\dots ,e_n\}$. Then $\pi^{-1}(U)=\bigsqcup_i\widetilde{U}_i$, where $\widetilde{U}_i\cong U$ is the image of the section $e_i:U\to E|_U$.
\end{proof}

%We will now focus on the analytic spectrum of a bundle of algebras. This spectrum was introduced by Houzel in \cite{houzel_gal2} to study finite morphisms of analytic spaces; it is a generalization of the usual notion of spectrum of a ring. From now on, it will be assumed that the multiplication on the bundles in consideration is also commutative.
%
%We work on the category of analytic (ringed) spaces over a fixed analytic space $(M,\scr{O}_M)$; this is the general setting, but we will in fact restrict our attention only to complex manifolds (i.e. non-singular analytic spaces) and, in particular, vector bundles over $M$.\footnote{Recall that, given ringed spaces $(X,\scr{O}_X)$ and $(Y,\scr{O}_Y)$, a morphism $(X,\scr{O}_X)\rightarrow (Y,\scr{O}_Y)$ is a holomorphic map $f:X\rightarrow Y$ together with a sheaf morphism $\scr{O}_Y\rightarrow f_*\scr{O}_X$ (or, equivalently, a sheaf map $f^*\scr{O}_Y\rightarrow \scr{O}_X$, by the adjunction between the inverse and direct image sheaves).} Regarding analytic spaces, the interested reader may consult \cite{fischer:_cng}. 
%
%\begin{defi}
%A sheaf $\EuScript{A}$ of $\scr{O}_M$-algebras over $M$ is said to be \emph{finite} if and only if it is a coherent sheaf of $\scr{O}_M$-modules.
%\end{defi}
%
%We will not discuss the definition of coherent sheaf here, which is treated in many books of basic algebraic geometry, as for example \cite{hartshorne:_alg_geom}. The fact that really interest us is that, given a bundle of algebras $E$, its sheaf of sections $\Gamma_E$, being a locally-free $\scr{O}_M$-module, is in particular coherent and so it is a finite $\scr{O}_M$-algebra over $M$.
%
%Let $E$ be a vector bundle over $M$ and $f:N\rightarrow M$ a space over $M$. Define a contravariant functor $S_E$ from spaces over $M$ to the category of sets, given by $$S_E(N)=\operatorname{Hom}_{\scr{O}_N \text{-alg}}(f^*\Gamma_E,\scr{O}_N).$$
%Equivalently, we can consider the functor (which we denote by the same symbol) $S_E(N)=\operatorname{Hom}_{\scr{O}_M \text{-alg}}(\Gamma_E,f^*\scr{O}_N)$, by means of the adjunction between $f^*$ and $f_*$.
%This functor turns out to be representable (\cite{houzel_gal2}, Proposition 1).
%
%\begin{defi}\label{anspec}
%The \emph{analytic spectrum} of $\Gamma_E$ (or just $E$) is the analytic space $\text{Specan}\, E$ over $M$ which represents the functor $S_E$.
%\end{defi}
%
%\begin{obs}
%Even in this restricted setting, the analytic spectrum may have singularities.
%\end{obs}
%
%\begin{obs}
%The analytic spectrum is defined for more general sheaves of algebras, those of finite presentation, by considering the functor $N\mapsto S_\EuScript{A}(N)$, for an $\scr{O}_M$-algebra $\EuScript{A}$ of finite presentation and complex spaces $N,M$. For a detailed treatment, we refer again to \cite{houzel_gal2}.
%\end{obs}
%
%From now on, we will focus only on the case of a vector bundle $E$ over $M$ and the finite algebra $\Gamma_E$.
%
%We will denote the analytic spectrum of a bundle $E$ by $(\text{Specan}\, E,\pi )$ where $\pi :\text{Specan}\, E\rightarrow M$ is the projection, or just by $\text{Specan}\, E$, as in definition \ref{anspec}. We thus obtain a pair $(\text{Specan}\, E,\eta )$, with
%$$\eta :\text{Hol}_M(-,\text{Specan}\, E)\stackrel{\cong}{\longrightarrow} S_E$$
%a natural isomorphism, where $\text{Hol}_M(N,L)$ denotes the space of holomorphic maps $N\rightarrow L$ over $M$. By Yoneda's lemma, the natural isomorphism $\eta$ can be identified with the morphism of $\scr{O}_M$-algebras
%$$\eta_\Lambda(\text{id}_\Lambda):\Gamma_E\longrightarrow \pi_*\scr{O}_\Lambda,$$
%where $\Lambda =\text{Specan}\, E$ (to ease the notation). This map is an isomorphism (\cite{houzel_gal2}, Proposition 8).
%
%The following result gives a description of the fibers of the analytic spectrum and justifies its name.
%
%\begin{proposition}
%Let $\pi :\Lambda \rightarrow M$ be the analytic spectrum of $E$. Then:
%\begin{enumerate}
%\item each fiber $\Lambda_x=\pi^{-1}(x)$ has finite cardinality (in fact, $\pi :\Lambda\rightarrow M$ is a finite morphism, as it is also closed);
%\item $\Lambda_x$ is in bijective correspondence with maximal ideals in $E_x$ and
%\item the projection $\pi :\Lambda \rightarrow M$ is finite and flat of degree equal to the dimension of $M$.
%\end{enumerate}
%\end{proposition}
%
%For the case we are considering, there is a nice description of the analytic spectrum, as a subset of the dual bundle $E^*$.
%
%Given $E$ with multiplication, consider the symmetric bundle $SE$. As for vector spaces, we can view $\Gamma_E$ inside $\Gamma_{SE}$. Now, sections of $SE$ can be identified with polynomial maps on $E^*$ in the following way: assume $\{X_1,\dots ,X_m\}$ is a local frame for $E$ over $U$. Then every local section $Y$ of $\Gamma_{SE}$ is of the form $Y=\sum \lambda_{i_1,\dots ,i_m}X_1^{i_1}\otimes \cdots \otimes X_{m}^{i_m}$ for some maps $\lambda_{i_1\dots i_m}\in \scr{O}(U)$. If $(x,\xi )$ is a point in $E^*$, then
%$$Y(x,\xi )=\sum \lambda_{i_1,\dots ,i_m}(x)\xi (X_1(x))^{i_1} \dots \xi (X_{m}(x))^{i_m}.$$
%Assume now that we restrict attention only to points $(x,\xi )$ such that $\xi :E_x\rightarrow \comp$ is also an algebra homomorphism (with the notation of preceeding paragraphs, $(x,\xi ) =(x,\Lambda_i )$ for some $i=1,\dots ,n$). Suppose that near $x$ we have $\bigl (\sum_i\lambda_iX_i\bigr )(y,\Lambda_j)=0$ for each $j$; then
%$$\Lambda_j\Bigl (\sum_i\lambda_i(y)X_i(y)\Bigr )=0,$$
%and, being $\{X_1,\dots ,X_m\}$ a local frame, necessarily $\lambda_i=0$ in a neighbourhood of $x$ for each $i=1,\dots ,m$. Thus, the map
%\begin{equation}\label{canonical_map}
%\Gamma_E\longrightarrow \pi_*\scr{O}_{\Lambda }
%\end{equation}
%is injective, where $\Lambda =\{(x,\xi )\in E^* \, | \, \xi :E_x\rightarrow \comp \text{ is an algebra homomorphism}\}$. On the other hand, let $f:\pi^{-1}(V)\rightarrow \comp$ be a map defined in a neighbourhood $V$ of $x\in M$. We then have that the linear system
%$$\lambda_1 \Lambda_j(X_1)+\cdots +\lambda_m \Lambda_j(X_m)=f(\Lambda_j)\quad (j=1,\dots ,n)$$
%has infinitely many solutions as $n\leqslant m$.
%
%
%\begin{proposition}
%The map \eqref{canonical_map} is an isomorphism of $\scr{O}_M$-algebras.
%\end{proposition}
%
%\begin{proposition}
%The set $\Lambda =\bigsqcup_{x\in \text{supp}\, \Gamma_E}\operatorname{Hom}_{\comp \text{-alg}}(E_x,\comp )\subset E^*$ is an analytic spectrum for $E$.
%\end{proposition}
%\begin{proof}
%We have to construct an isomorphism
%$$\eta_N:\text{Hol}_M(N,\Lambda )\longrightarrow S_E(N)$$
%natural in $N$. 
%\end{proof}
%
%
%
%
%If $E_x=\bigoplus_{i=1}^{n_x}e_iE_x$ is the eigenspace decomposition for the fiber $E_x$, recall from section \ref{sfa} that the set of maximal ideals of $E_x$ is in bijective correspondence with the set $\{\Lambda_1,\dots ,\Lambda_{n_x}\}=\operatorname{Hom}_{\comp \text{-alg}}(E_x,\comp )\subset E^*_x$. Then, the support of the sheaf $\Gamma_E$
%$$\text{supp}\, \Gamma_E=\{x\in M \, | \, E_x\neq 0\}$$
%is the image of the structural morphism $\pi :\Lambda\rightarrow M$. Thus, we have that  
%$$\Lambda=\bigsqcup_{x\in \text{supp}\, \Gamma_E}\operatorname{Hom}_{\comp \text{-alg}}(E_x,\comp )\subset E^*.$$
%
%There is a more concrete description, without relying in the previous proposition: recall that for any finite-dimensional $\comp$-algebra $V$, the symmetric algebra $SV$ can be identified with polynomial maps $V^*\rightarrow \comp$; more precisely, if $\{x_1,\dots ,x_n\}$ is a basis for $V$, then
%$$SV\cong \comp [x_1,\dots ,x_n];$$
%if $p$ is a monomial $x_{i_1}^{r_1}\otimes \dots \otimes x_{i_k}^{r_k}$, then $p(\varphi )=\varphi (x_{i_1})^{r_1}\dots \varphi (x_{i_k})^{r_k}$.\footnote{It is often more natural to define the symmetric algebra as $SV^*$; the definition we use here is chosen mainly because $V$ is an algebra, and we need to keep track on its multiplication.} Consider now the regular representation $L:V\rightarrow \operatorname{Hom}_\comp (V,V)$; this morphism can be extended to a morphism of algebras
%$$L:SV\longrightarrow \operatorname{Hom}_\comp (V,V)$$
%in an obvious way. Let $\text{ev}_1:\operatorname{Hom}_\comp (V,V)\rightarrow V$ be the evaluation map $\text{ev}_1(f)=f(1)$. Then the kernel of the composite map
%$$\text{ev}_1L:S(V)\longrightarrow V$$
%(which is the (surjective) morphism that maps the multiplication in $SV$ to the multiplication in $V$) is an ideal $\mathfrak{a}$ in $SV$. Assume that the first element of the basis of $V$ is $x_1=1$, the unit of the algebra $V$. Then $\mathfrak{a}$ is the ideal generated by the polynomials
%\begin{equation}\label{generators}
%x_1-1 \quad , \quad x_i\otimes x_j-\sum_k\lambda_{ij}^kx_k \quad (i,j=1,\dots ,n),
%\end{equation}
%where $x_ix_j=\sum_k\lambda_{ij}^kx_k$ in $V$.
%
%The previous constructions can be applied in the context of vector bundles: if $E$ is a vector bundle with multiplication, then we get an algebra bundle $SE$; then, the $\scr{O}_M$-sheaf of sections $\Gamma_{SE}$ is isomorphic to the $\scr{O}_M$-sheaf of functions in $\tau_*\scr{O}_{E^*}$ which are polynomials on the fibers (here $\tau :E^*\rightarrow M$ is the bundle projection). The kernel of the morphism $\text{ev}_1L:\Gamma_{SE}\rightarrow \Gamma_E$ defines an ideal sheaf in $\tau_*\scr{O}_{E^*}$, which is defined locally by the polynomials \eqref{generators}. Pick now an arbitrary open subset $W\subset E^*$ and let $\tau (W)= U\subset M$. Then we can construct an ideal
%$$\mathfrak{A}(W)\subset \scr{O}_{E^*}(W)$$
%by generating it with the restrictions $p|_W$, where $p\in \opnm{Ker} \{(\text{ev}_1L):\Gamma_{SE}(U)\rightarrow \Gamma_E(U)\}$.
%
%\begin{proposition}
%For the analytic spectrum $L$ of $E$, we have the following conclusions:
%\begin{enumerate}
%\item The support of the quotient sheaf $\scr{O}_{E^*}/\mathfrak{A}$ is the analytic spectrum $L=\text{Specan}\, E$ and the map
%$$\Gamma_E\longrightarrow \tau_*\scr{O}_L$$
%is an isomorphism of $\scr{O}_M$-algebras.
%\item The previous isomorphism induces isomorphisms $e_iE_x\cong \scr{O}_{L,\Lambda_i}$ over each $x\in M$.
%\end{enumerate}
%\end{proposition}
%\begin{proof}
%
%\end{proof}
%
%
%\begin{cor}\label{splitting}
%For each $x\in M$ there exists an open neighbourhood $U\ni x$ such that the restriction $E|_U$ splits as a sum of multiplication invariant subbundles $e_iE$.
%\end{cor}
%
%\begin{obs}\label{caustic}
%Given a point $x\in M$ and a neighbourhood $U$ as in corollary \ref{splitting}, the eigenspace decomposition over a point $y\in U$ may have more summands than the one induced by the splitting around $x$. Let $P(x)$ be the vector given by
%$$P(x)=(\dim_\comp e_{1,x}E_x,\dots ,\dim_\comp e_{n,x}E_x).$$
%Now, if $M$ is connected, there exists a unique vector $\beta$ such that the subset $\{x\in M\, | \, P(x)=\beta \}$ is open in $M$. The complement of this subset is called the \emph{caustic} and is an hypersurface or empty. Note that $E_x$ is semisimple if and only if $P(x)=(1,\dots ,1)$. In this case, if $U\ni x$ is as in corollary \ref{splitting}, then for every $y\in U$, the induced decomposition over $y$ is the same as its eigenspace decomposition. This is treated in detail in \cite{hertling:_fman}.
%\end{obs}
%
%We are interested in the case where each $e_iE$ has rank 1; i.e. when the multiplication is semisimple. We have that the subset $\{x\in M \, | \, E_x \text{ is semisimple}\}$ is open and we have locally defined (and uniquely determined) sections
%$$e_i:U\longrightarrow E|_U$$
%such that $e_ie_j=\delta_{ij}e_i$; i.e. $\{e_1,\dots ,e_n\}$ is a local basis of orthogonal, idempotent sections. Singularities of $\Lambda$ arises in the complement of semisimple points, and so in a neighbourhood of a semisimple point, $L$ is actually a manifold.
%
%Assume now that each point $x\in M$ is semisimple. Then, in a neighbourhood of each point we have a splitting of the bundle $E$ as a sum of line bundles $e_iE$
%$$E=\bigoplus_{i=1}^ne_iE;$$
%each subbundle $e_iE$ is called an \emph{eigenline bundle} for $L$: if $X$ is any local section of $E$, then the multiplication operator $L_X$ can be diagonalized and each fiber $(e_iE)_x$ is an eigenspace for the restriction of $L$ to $E_x$.

%\begin{obs}
%The analytic spectrum $\pi :\Lambda \rightarrow M$ is branched over the caustic, i.e. over the points where $E_x$ is not semisimple (see remark \ref{caustic}). So, on the open subset of semisimple points, $\pi$ is unramified and thus \'etale. In fact, semisimplicity is equivalent to being \'etale (see \cite{hertling:_fman}, Theorem 3.2).
%\end{obs}

In the following sections we shall encounter bundles of algebras with an additional layer of structure, namely a nondegenerate, symmetric linear form $\theta :E\to \comp$ (recall that in the context of vector bundles, $\comp$ denotes the trivial vector bundle $M\times \comp$). In this case, $\theta$ defines an isomorphism $\overline{\theta}:E\cong E^*$ defined in the usual way. Moreover, if $X,Y$ are sections of $E$, then the equation
$$g(X,Y):=\theta (XY)$$
defines a metric on $E$. Frobenius manifolds provide examples of bundles with this property.


%%%%%%%%%%%%%%%%%%%%%%%%%%%%%%%%%%%%%%%%%%
\subsection{F-Manifolds}

We now take $E=TM$, the tangent bundle to an $n$-dimensional connected manifold $M$, and suppose that we have an associative and commutative multiplication on $TM$, with a global vector field $1:M\rightarrow TM$. We will also assume that this multiplication is semisimple at each point of $M$. In this case, the analytic spectrum of $TM$ will be called the spectral cover.

\begin{defi}
A manifold $M$ such that $T_xM$ is semisimple for each $x\in M$ is called \emph{massive}.\footnote{This terminology comes from \emph{massive perturbations} in a conformal field theory.}
\end{defi}

We then have a local decomposition
\begin{equation}\label{decomp}
TM|_U=\bigoplus_{i=1}^ne_iTM
\end{equation}
of $TM$ into line bundles and the set $\{e_1,\dots ,e_n\}$ is a basis of orthogonal idempotent sections of $TM$ over $U$, with $\sum_ie_i=1$.

Given this idempotent local fields, we would like to to know if they come from a system of local coordinates. This is equivalent to the commutativity condition
$$[e_i,e_j]=0$$
for all $i,j=1,\dots ,n$ and for each $U$ with a decomposition \eqref{decomp}.

\begin{defi}
An \emph{F-manifold} is a manifold with multiplication $M$ such that the following product rule
\begin{equation}\label{f_manifold}
\EuScript{L}_{XY}(\mu )=X\EuScript{L}_Y(\mu )+Y\EuScript{L}_X(\mu )
\end{equation}
holds for all local vector fields $X,Y$ on $M$ ($\mu$ is the multiplication tensor and $\EuScript{L}$ the Lie derivative).
\end{defi}

As $\mu$ is a $(2,1)$-tensor, so is $\EuScript{L}_X(\mu )$ and it can be computed as
\begin{equation}\label{lie_derivative}
\EuScript{L}_X(\mu )(Y,Z)=[X,YZ]-[X,Y]Z-[X,Z]Y.
\end{equation}

An inmediate consequence of this definition is the following

\begin{lemma}
$\EuScript{L}_{e_i}(\mu )=0$ for each $i=1,\dots ,n$.
\end{lemma}
\begin{proof}
An easy computation using \eqref{f_manifold} and the equality $e_i^2=e_i$ shows that $\EuScript{L}_{e_i}(\mu )=2e_i\EuScript{L}_{e_i}(\mu )$. Multiplying by $e_i$, we then have that $e_i\EuScript{L}_{e_i}(\mu )=0$, and the result follows.
\end{proof}

\begin{proposition}\label{canonical_coord}
Let $M$ be an F-manifold. For each $x\in M$, there exists a neighbourhood $U\ni x$ with local coordinates $(x_1,\dots ,x_n)$ such that
$$e_i=\partial_{x_i}.$$
\end{proposition}
\begin{proof}
Pick a semisimple neighbourhood $U\ni x$ and let $TM|_U=\bigoplus_{i=1}^ne_iTM$. We must show that $[e_i,e_j]=0$ for each $i,j=1,\dots ,n$. By the previous lemma and equation \eqref{lie_derivative}
\begin{equation}\label{eigenvector}
0=\EuScript{L}_{e_i}(\mu )(e_j,e_j)=[e_i,e_j]-2e_j[e_i,e_j],
\end{equation}
which implies that $[e_i,e_j]$ is an eigenvector with (constant) eigenvalue equal to $\frac{1}{2}$ for the multiplication operator $L_{e_j}$; i.e. $[e_i,e_j]\in e_jTM$. Applying $L_{e_j}$ to equation \eqref{eigenvector} shields $0=e_j[e_i,e_j]=L_{e_j}([e_i,e_j])$, as desired.
\end{proof}

\begin{defi}
A coordinate chart as the one obtained in proposition \ref{canonical_coord} is called a \emph{canonical coordinates chart}.
\end{defi}

Note that this canonical coordinates are uniquely determined, up to reordering; in such an open subset we then have a chart $(x_1,\dots ,x_n)$ such that $\{\partial_{x_1},\dots ,\partial_{x_n}\}$ is a basis of orthogonal idempotents and each line bundle $\partial_{x_i}TM$ over $U$ is a simple summand of $TM|_U$. Massive manifolds can then be classified as the only F-manifolds which admit canonical coordinates.

\begin{obs}
The approach adopted here is the one in \cite{hertling:_fman}, and shows that this canonical coordinates, as defined by Dubrovin for Frobenius manifolds in \cite{dubrovin:_2dtft} (cf. also \cite{hitchin:_frob_manifolds}), are available for more general manifolds with multiplication, i.e. F-manifolds. These F-manifolds where first considered by Y. Manin, motivated by K. Saito's work, to avoid the metric as part of the structure.
\end{obs}

We now define a particular class of vector fields, which have an important role when dealing with Frobenius manifolds.

\begin{defi}
Let $M$ be an F-manifold. An \emph{Euler vector field of weight} $d\in \comp$ is a global vector field $\chi \in \EuScript{T}(M)$ such that
$$\EuScript{L}_\chi (\mu )(X,Y)=dXY$$
for all vector fields $X,Y$.
\end{defi}

Of particular importance are Euler fields of weight $d=1$ (if no weight is mentioned, we will assume that it has weight equal to 1), and not every F-manifold has such vector fields; see \cite{hertling:_fman}, section 3.2. From equation \eqref{f_manifold} follows easily that the unit field $1$ is an Euler field of weight $d=0$.

\begin{ej}
The canonical (and most important, in the sense that every F-manifold of dimension $n$ is locally equivalent to it) example of an F-manifold is complex $n$-space $\comp^n$; let $(z_1,\dots ,z_n)$ denote the usual coordinate chart and  let $e_i:=\partial_{z_i}$; define the multiplication by the formula
$$e_ie_j:=\delta_{ij}e_i.$$
Then
\begin{enumerate}
\item the multiplication is semisimple and satisfies equation \eqref{f_manifold};
\item $\sum_ie_i$ is the unit field and
\item every massive F-manifold is locally like this manifold.
\end{enumerate}
\end{ej}


\clearpage

{\small
%%%%%%%%%%%%%%%%%%%%%%%%%%%%%%%%%%%%%%%%%%%%
%%%%%%%%%%%%%%%%%%%%%%%%%%%%%%%%%%%%%%%%%%%%
\section{Resumen del Cap\'itulo \ref{fsfts}}

El objetivo central de este cap\'itulo es el de introducir las teor\'ias cu\'anticas de campo abiertas-cerradas como asi tambi\'en la clasificaci\'on de estas dada por G. Moore y G. Segal en el caso semisimple. Para esto se necesita primero introducir las teor\'ias cerradas, las cuales est\'an \'intimamente ligadas a las \'algebras de Frobenius, a las cuales tambi\'en les dedicamos una concisa introducci\'on. Finalizamos con los fibrados de \'algebras, los cuales, junto con las teor\'ias abiertas-cerradas, juegan un papel fundamental en lo que resta de este trabajo.

%%%%%%%%%%%%%%%%%%%%%%%%%%%%%%%%%%%%%%%%%%%%%%
\subsection{Teor\'ias Topol\'ogicas de Campos}

Comenzemos con una definici\'on previa. Dado un entero positivo $D$, definimos la categor\'ia de cobordismos $\tsf{Cob}(D)$ como la categor\'ia cuyos objetos son variedades suaves, orientadas y cerradas de dimensi\'on $D-1$; dadas dos tales variedades $\Sigma_1,\Sigma_2$, unm morfismo $\Sigma_1\to \Sigma_2$ es un cobordismo orientado (es decir, el morfismo es una variedad suave y orientada $W$ de dimensi\'on $D$ tal que $\partial W=\Sigma_1\sqcup \Sigma_2^-$, donde el super\'indice $^-$ indica orientaci\'on opuesta). Una Teor\'ia Cu\'antica de Campos (Topol\'ogica) (abreviado {\sc tft} por sus siglas en ingl\'es) de dimensi\'on $D$ sobre un anillo conmutativo $R$ (que en nuestro caso consideramos igual a $\re$ \'o $\comp$) consiste de un funtor $Z:\tsf{Cob}(D)\to \tsf{Vect}_R$ de la categor\'ia de cobordimos en la categor\'ia de $R$-espacios vectoriales de dimensi\'on finita que verifica:
\begin{itemize}
\item Si $W\cong W'$ son cobordismos difeomorfos, entonces $Z(W)=Z(W')$.
\item $Z$ es multiplicativo, en el sentido que $Z(\Sigma_1\sqcup \Sigma_2)=Z(\Sigma_1)\otimes Z(\Sigma_2)$.
\item $Z(\emptyset )=R$.
\end{itemize}
A partir de ahora, consideramos $D=2$. Estas teor\'ias de campo mantienen una estrecha relaci\'on con las \'algebras de Frobenius, tema que se discute a continuaci\'on.

\subsubsection{{\small \'Algebras de Frobenius}}

Estas \'algebras fueron consideradas originalmente por Frobenius, quien estudiaba \'algebras $A$ cuyas primer y segunda representaciones regulares eran isomorfas. Esto es equivalente a la existencia de una forma lineal $\theta :A\to \comp$ tal que la forma bilineal dada por $(x,y)\mapsto \theta (xy)$ es no-degenerada. En particular (equivalentemente) tenemos que $A\cong A^*$. Particularmente importantes para nosotros son las \'algebras conmutativas y semisimples, y en ellas nos enfocamos en lo que sigue. Recordemos que una $\comp$-\'algebra es semisimple si es suma de subm\'odulos simples (es decir, que no tienen subm\'odulos no triviales). En particular si $\dim_\comp A=n$, se demuestra la existencia de idempotentes simples $e_1,\dots ,e_n$ (que forman una base) tales que $A=\bigoplus_{i=1}^ne_iA$ (en particular, cada sumando $e_iA$ es un \'algebra simple con neutro igual a $e_i$) y $\sum_{i=1}^ne_i=1$.
Existe una caracterizaci\'on de las \'algebras de Frobenius semisimples dada por G. Moore y G. Segal, que describimos brevemente a continuaci\'on.

Llamemos $X$ al espectro de ideales primos $\opnm{Spec}A$ del \'algebra $A$. Entonces se puede mostrar que $X$ es un espacio topol\'ogico finito, cuyo cardinal es igual a la dimensi\'on de $A$. Consideramos entonces el \'algebra $\comp ^X$ de funciones $X\to \comp$. Si $\chi_i$ denota la funci\'on caracter\'istica del conjunto $\{e_i\}$, entonces la correspondencia $x\mapsto \sum_i\lambda_i\chi_i$ define un isomorfismo entre las \'algebras $A$ y $\comp ^X$, donde $x=\sum_i\lambda_ie_i$.

A continuaci\'on se define un elemento importante asociado a un \'algebra $A$, que llamamos el \emph{elemento de Euler}. Dada una base $\{e_i\}$ de $A$, sea $\{e^i\}$ su dual. Se define el elemento de Euler $\chi \in A$ por la f\'ormula
$$\chi =\sum_ie_i\overline{\theta}^{-1}(e^i),$$
donde $\overline{\theta}:A\to A^*$ es el isomorfismo inducido por $\theta$ (la definici\'on no depende de la base elegida). Es notable destacar que la existencia de un inverso para $\chi$ en $A$ es equivalente a que la traza $\opnm{tr}:A\otimes A \to \comp$, $\opnm{tr}(x\otimes y)=\opnm{tr}(L_{xy})$ sea no degenerada (dado $x\in A$, $L_x:A\to A$ es el operador de multiplicaci\'on). Esto provee, v\'ia un teorema de Dieudonn\'e, una manera de deducir si cierta \'algebra $A$ es semisimple: $\chi \in A$ es inversible si y solo si $A$ es semisimple.
A continuaci\'on se definen los homomorfismos de \'algebras de Frobenius y se da una descripci\'on del grupo de endomorfismos de un \'algebra semisimple y conmutativa.
Completamos la introducci\'on a las \'algebras de Frobenius dando una descripci\'on de las ecuaciones de estructura de un \'algebra, que expresan el producto, la asociatividad, la conmutatividad y la existencia de un elemento neutro en base a coordenadas en una base fija. Se complementa con una descripci\'on de varios ejemplos en el \'algebra, la geometr\'ia y la f\'isica en donde aparecen \'algebras de Frobenius.

\subsubsection{{\small La Correspondencia Entre \'Algebras de Frobenius y {\sc tft}s}}

En esta secci\'on se decribe la relaci\'on entre las teor\'ias de campo y las \'algebras de Frobenius, conocida por los especialistas desde hace tiempo y demostrada finalmente por L. Abrams en su tesis, y de la cual incluimos un breve resumen.

Dada una {\sc tft} de dimensi\'on 2, representada por un functor $Z:\tsf{Cob}(2)\to \tsf{Vect}_\comp$, llamemos $A$ al espacio $Z(S^1)$, donde $S^1$ indica el c\'irculo unitario. Considerando entonces los cobordismos $\emptyset \to S^1$, $S^1\sqcup S^1\to S^1$ (<<pantalones>>) y $S^1\to \emptyset$, al aplicar $Z$ obtenemos respectivamente la unidad de $A$, la multiplicaci\'on y la forma lineal $\theta$. Distintas propiedades topol\'ogicas se traducen al aplicar $Z$ en propiedades algebraicas del \'algebra $A$, que resulta ser un \'algebra de Frobenius. Mas a\'un, la correspondencia es tambi\'en v\'alida en el otro sentido; y de esto se puede deducir una equivalencia entre la categor\'ia de teor\'ias topol\'ogicas de campos $\tsf{TQFT}(2)$ de dimensi\'on 2, y la categor\'ia de $\comp$-\'algebras de Frobenius con unidad, conmutativas, de dimensi\'on finita.


\subsection{Teor\'ias Abiertas-Cerradas}

Las cuerdas cerradas no describen todas las opciones originalmente consideradas por los f\'isicos. El caso general, adem\'as de las cuerdas cerradas, inlcuye tambi\'en a las cuerdas abiertas. Asi como para las teor\'ias cerradas, se tiene tambi\'en una formulaci\'on precisa de las teor\'ias que admiten tambi\'en cuerdas abiertas, dada por G. Moore y G. Segal \cite{moore_segal1}. Pasamos a continuaci\'on a discutir las nuevas estructuras introducidas para construir una teor\'ia que admita tambi\'en las cuerdas abiertas.

La diferencia principal con las teor\'ias cerradas es la introducci\'on de una categor\'ia de condiciones de borde, la \emph{categor\'ia de branas}, que notamos por $\scr{B}$. Los objetos de $\scr{B}$ consisten de ``etiquetas'' asignadas a los extremos de los intervalos que representan a las cuerdas abiertas, que notamos por $a,b,c,\dots $; un morfismo $a\to b$ en esta categor\'ia es precisamente una variedad suave, orientada, con borde de dimensi\'on 1. Notando por $O_{ab}$ el conjunto de mapas $a\to b$, requerimos entonces que $O_{ab}$ sea un $\comp$-espacio vectorial tal que la ley de composici\'on $O_{ab}\otimes O_{bc}\to O_{ac}$ sea asociativa y bilineal.

La existencia de las nuevas cuerdas abiertas hace que tambi\'en debamos cambiar la categor\'ia $\tsf{Cob}(2)$ por una nueva, que notamos $\tsf{Cob}_\scr{B}(2)$, construida a partir de la primera adjuntando a los intervalos con extremos descriptos por objetos de $\scr{B}$. Los morfismos en esta nueva categor\'ia son tambi\'en cobordismos $W:\Sigma_1\to \Sigma_2$ entre uniones disjuntas de c\'irculos e intervalos de tal forma que $\partial W=\Sigma_1\cup \Sigma_2\cup W'$, donde $W'$ es un cobordismo $\partial \Sigma_1\to \partial \Sigma_2$.

Asi como las teor\'ias cerradas, este tipo de teor\'ias tiene tambi\'en una descripci\'on funtorial, que viene dada por un funtor
$$Z:\tsf{Cob}_\scr{B}(2)\longrightarrow \tsf{Vect},$$
cuya restricci\'on a la categor\'ia $\tsf{Cob}(2)$ es una teor\'ia cerrada. La imagen de un intervalo con extremos $a,b\in \scr{B}$ se nota $O_{ab}$. A continuaci\'on damos una descripci\'on de las estructuras algebraicas subyacentes.

Dada una brana $a\in \scr{B}$, los espacios vectoriales $O_{aa}$ debe tambi\'en estar munidos de una forma lineal $\theta_{a}:O_{aa}\to \comp$ de tal forma que la forma bilineal $O_{aa}\otimes O_{aa}\to O_{aa}\stackrel{\theta}{\to}\comp$ sea no degenerada; en particular, $(O_{aa},\theta_a )$ es un \'algebra de Frobenius, no necesariamente conmutativa). Para otro objeto $b\in \scr{B}$, tenemos tambi\'en el espacio vectorial $O_{ab}$, relacionado con $O_{aa}$ via la composici\'on
$$O_{ab}\otimes O_{ba}\longrightarrow O_{aa}\stackrel{\theta_a}{\longrightarrow}\comp,$$
que debe ser una forma no degenerada. En particular resulta $O_{ba}\cong O_{ab}^*$.

La interacci\'on entre cuerdas abiertas y cerradas se describe de la siguiente manera: una cuerda cerrada puede evolucionar a una abierta con el mismo extremo, digamos $a\in \scr{B}$, y viceversa. Estas evoluciones resultan ser cobordismos, es decir, morfismos en la categor\'ia $\tsf{Cob}_\scr{B}(2)$. La imagen de estos cobordismos se notan $\iota_a :A\to O_{aa}$ (cerrada a abierta) e $\iota^a:O_{aa}\to A$ (abierta a cerrada). Propiedades de estas interacciones fuerzan a exigir que $\iota_a$ sea un homomorfismo central de $\comp$-\'algebras y que $\iota^a$ sea $\comp$-lineal.

Otras propiedades de estos morfismos los relacionan con las formas lineales $\theta$ y $\theta_a$, que proveen las estructuras de \'algebras de Frobenius a $A$ y $O_{aa}$ respectivamente. Mas precisamente, se debe verificar la relaci\'on de adjunci\'on $\theta (\iota^a(\sigma )x)=\theta_a(\sigma \iota_a (x))$, donde $x\in A$ y $\sigma \in O_{aa}$.

Una \'ultima condici\'on, llamada la \emph{condici\'on de Cardy}, debe verificarse; la describimos a continuaci\'on. Consideremos una base  $\{\sigma_i\}$ de $O_{ab}$ y sea $\{\sigma ^i\}$ su dual. Definimos un mapa lineal $\pi^a_b:O_{aa}\to O_{bb}$ por la ecuaci\'on
$$\pi_b^a(\tau )=\sum_i\sigma_i\tau \overline{\theta}_{ab}^{-1}(\sigma^i).$$
Entonces, $\pi_b^a,\iota_b$ e $\iota^a$ deben verificar
$$\pi_b^a=\iota_b\iota^a.$$


\subsubsection{{\small Caracterizaci\'on de una Categor\'ia de Branas Maximal}}

Para lo que sigue, se considera que el \'algebra del sector cerrado $A$ es semisimple. Por medio de la condici\'on de Cardy podemos deducir los siguientes datos fundamentales:
\begin{itemize}
\item Las \'algebras $O_{aa}$ son semisimples (en otras palabras, son isomorfas a sumas de \'algebras de matrices)
\item En general, para $a,b\in \scr{B}$ no necesariamente iguales, tenemos que $O_{ab}$ es isomorfo a un espacio vectorial de la forma $\bigoplus_i\operatorname{Hom}_\comp (V_{a,i},V_{b,i})$.
\end{itemize}

Una categor\'ia de branas $\scr{B}$ es \emph{maximal}si y solo si dada cualquier otra tal categor\'ia $\scr{B}'$, se tiene un mapa inyectivo $\opnm{sk}\scr{B}'\to \opnm{sk}\scr{B}$. En particular, las siguientes propiedades se verifican para una categor\'ia maximal
\begin{itemize}
\item $\scr{B}$ es aditiva.
\item Se tiene definida una acci\'on $V\otimes a$ de los espacio vectoriales complejos de dimensi\'on finita sobre $a\in \scr{B}$.
\item $\scr{B}$ es pseudo-abeliana.
\item Existe una brana $a_0$ para la cual $\iota_{a_0}:A\to O_{a_0a_0}$ es un isomorfismo; equivalentemente, para cada \'indice $i$ se tiene una brana $a_i\in \scr{B}$ tal que $O_{a_ia_i}\cong \comp$ como $\comp$-\'algebras.
\end{itemize}

Esto da lugar a la siguiente caracterizaci\'on dada por G. Moore y G. Segal.

\medskip
{\bf Teorema.}
{\it Si el \'algebra de Frobenius $A$ correspondiente al sector cerrado de una teor\'ia abierta-cerrada es semisimple, entonces la categor\'ia de branas $\scr{B}$ (maximal) es equivalente a la categor\'ia $\tsf{Vect}(X)$ de fibrados vectoriales sobre el espacio finito $X=\{e_1,\dots ,e_n\}$ formado por los idempotentes ortogonales del \'algebra $A$ tales que $\sum_ie_i=1$.}


%%%%%%%%%%%%%%%%%%%%%%%%%%%%%%%%%%%%%%%%%%%%%%%%%%
\subsection{Fibrados de \'Algebras y F-variedades}

Sea $M$ una variedad y $\scr{O}_M$ un haz de funciones sobre $M$. Un fibrado de \'algebras sobre $M$ es un fibrado complejo (suave u holomorfo) $E\to M$ junto con un morfismo de fibrados $\mu :E\otimes E\to E$ (multiplicaci\'on) tal que para cada $x\in M$, la restricci\'on $\mu_x$ de $\mu$ a $E_x\otimes E_x$ induce en $E_x$ una estructura de $\comp$-\'algebra asociativa con unidad $1_x$. Se pide adem\'as que exista una secci\'on, que notamos $1:M\to E$, tal que $1(x)=1_x$ para cada $x\in M$. Notemos que esta definici\'on no implica la existencia de trivialidad local, en el siguiente sentido: dado $x\in M$, sabemos que existe una vecindad $U\ni x$ tal que $E|_U$ es isomorfo a $U\times \comp^n$; pero la definici\'on de fibrado de \'algebras no implica que esta trivializaci\'on local preserve la estructura de \'algebra. Ver la siguiente secci\'on.

Diremos que $M$ es una \emph{variedad con multiplicaci\'on} si $TM$ es un fibrado de \'algebras.


\subsubsection{{\small El Recubrimiento Espectral}}

Sea $E$ un fibrado de \'algebras sobre $M$. La siguiente proposici\'on es fundamental en la siguiente discusi\'on.

\medskip
{\bf Proposici\'on.}
{\it Sea $x_0\in M$ tal que $E_{x_0}$ es semisimple. Entonces existe una vecindad $U\ni x_0$ tal que $E_x$ es semisimple para cada $x\in U$. Mas a\'un, existe una bse local de secciones $e_1,\dots ,e_n:U\to E$ tal que $e_ie_j=\delta_{ij}e_i$ y $E=\bigoplus_ie_iE$ sobre $U$.}
\medskip

Tenemos adem\'as que, en el contexto del resultado anterior, el conjunto de puntos $x\in M$ tales que $E_x$ no es semisimple puede ser una hipersuperficie (ver la discusi\'on del p\'arrafo anterior a la presente secci\'on).

En lo que sigue vamos a considerar fibrados tales que $E_x$ es semisimple. Notemos con $S_E$ al conjunto de homomorfismos de \'algebras $E_x\to \comp$ ($x\in M$).

\medskip
{\bf Proposici\'on y Definici\'on.}
{\it La proyecci\'on can\'onica $\pi :S_E\to M$ es un recubrimiento de $n$ hojas. Cuando $M$ es una variedad con multiplicaci\'on y $E=TM$, llamamos a $S_E$ el \emph{recubrimiento espectral} de $M$.}










 


}


%%% Local Variables:
%%% mode: latex
%%% TeX-master: "master"
%%% End:


% Chapter 3
\chapter{Cardy Fibrations}
\label{cfib}

\vspace{250pt}

The first part of this chapter is devoted to the introduction of some
basic notions from category theory. Additive and pseudo-abelian
categories are needed in the next chapter to study maximal Cardy
fibrations; we bundled all the definitions in this chapter just for
convenience.

%%%%%%%%%%%%%%%%%%%%%%%%%%%%%%%
\section{Calabi-Yau Categories}
\label{sheaf_boundary_conditions}

Let $R$ be a commutative ring with unit. A category ${\bf X}$ is said
to be \emph{enriched over the category of $R$-modules} if for
arbitrary objects $a,b\in {\bf X}$, $\opnm{Hom}_{\bf X}(a,b)$ is an
$R$-module and the composition map is $R$-bilinear. In particular, if
$R=\ent$, we say that ${\bf X}$ is enriched over the category of
abelian groups.

Recall also that an object $a\in {\bf X}$ is said to be \emph{initial}
(respectively \emph{terminal}) if for each $b\in {\bf X}$,
there exists a unique arrow $a\to b$ (respectively $b\to a$).

\begin{defi}\label{linear_cats}
  Let ${\bf X}$ be a category and $R$ a commutative, unital ring. Then
  ${\bf X}$ is called
  \begin{enumerate}

  \item an \emph{$R$-linear category} if it is enriched
    over the category of $R$-modules;

  \item an \emph{additive category} if it is
    $\ent$-linear, has an initial object $0$ and for each pair of
    objects $a,b\in {\bf X}$ there exists a sum $a\oplus b\in {\bf
      X}$;

  \item a \emph{pseudo abelian category} if it is additive
    and given any object $a\in {\bf X}$, for each idempotent $\sigma
    :a\rightarrow a$ (i.e. $\sigma^2=\sigma$) there exists an object
    $\opnm{Ker} \sigma \in {\bf X}$, called the \emph{kernel of}
    $\sigma$, such that the canonical arrow
    \begin{equation}\label{canonical_arrow}
      \opnm{Ker} \sigma \oplus \opnm{Ker} (1_a -\sigma )\longrightarrow a
    \end{equation}
    is an isomorphism;

  \item a \emph{Calabi-Yau category (over $R$)} ({\sc cy} category for
    short) if it is $R$-linear, the objects $\operatorname{Hom}_{{\bf X}}(a,a)$ 
		are finitely generated, projective $R$-modules and, for each object $a\in {\bf X}$, 
		there exists a linear form
$$\theta_a:\operatorname{Hom}_{{\bf X}}(a,a)\longrightarrow R$$
such that the composite
\begin{equation}\label{pair}
  \operatorname{Hom}_{{\bf X}}(a,b)\otimes_R
  \operatorname{Hom}_{{\bf X}}(b,a)\longrightarrow 
  \operatorname{Hom}_{{\bf X}}(a,a)\stackrel{\theta_a}{\longrightarrow }R
\end{equation}
is a perfect pairing (the first arrow is the composition map $\sigma
\otimes \tau \mapsto \tau \sigma$) and, given arbitrary arrows $\sigma
:a\to b$ and $\tau :b\to a$, the equality
$$\theta_a(\tau \sigma )=\theta_b(\sigma \tau )$$
holds.
\end{enumerate}
\end{defi}

Let us add some more comments on the previous definitions. For
details, the reader is referred to 
\cite{maclane:_catwm, kn:grothendieck_sga2, freyd:abelian, costello07:_tcft_cy}.

{\sc Additive Categories.} Recall that in a category ${\bf X}$, a
\emph{zero object} $0\in {\bf X}$ is an object which is both initial
and terminal. The sum operation $\oplus$ is usually called a
\emph{biproduct} and, given objects $a_1,a_2\in {\bf X}$, there exist
projection $\pr_k:a_1\oplus a_2\to a_k$ and inclusion morphisms
$i_k:a_k\to a_1\oplus a_2$ ($k=1,2$) enjoying the following
properties:
\begin{itemize}
\item $a_1\oplus a_2$ (together with the projections $\pr_1$ and
  $\pr_2$) is a product.
\item $a_1\oplus a_2$ (together with the inclusions $i_1$ and $i_2$)
  is also a coproduct.
\item $\pr_ki_k=1_{a_k}$ ($k=1,2$).
\item $\pr_li_k=0$ for $l\neq k$, where $0$ is the zero object of the
  abelian group $\opnm{Hom}_{\bf X}(a_k,a_l)$.
\end{itemize}
Schematically, the biproduct structure for $a_1\oplus a_2$ is given by
the following diagrams:
\begin{equation}\label{biproduct}
  \xymatrix{ & b \ar[dr] \ar[dl] \ar[d] &  & & b & \\
    a_1 & a_1\oplus a_2 \ar[l]^-{\pr_1} \ar[r]_-{\pr_2} & a_2 & a_1 \ar[r]_-{i_1} \ar[ur] & a_1\oplus a_2 \ar[u] & a_2 ,\ar[l]^-{i_2} \ar[lu]}
\end{equation}
where the diagonal maps are given and the vertical arrows are uniquely
determined by the other morphisms (the diagram on the left corresponds
to the product structure and the one on the right to the
coproduct). In this setting, a morphism $\sigma :a_1\oplus a_2\to
b_1\oplus b_2$ can be represented as a matrix
$$\sigma =\left (\begin{smallmatrix} \sigma_{11} & \sigma_{21} \\ \sigma_{12} & \sigma_{22} \end{smallmatrix} \right ),$$
where $\sigma_{ij}:a_i\to b_j$.

Some examples: in the category of sets, there is no zero object (the
empty set is the initial object while any singleton is terminal); the
product is the cartesian product, and the coproduct is the disjoint
union; in the category of vector spaces over a field, the direct sum
is both a product and a coproduct; moreover, this category is
additive, the zero object being the trivial vector space. In the
category of groups, the zero object is the trivial group, but there is
no biproduct: the product is the direct product, and the coproduct is
the free product.

{\sc Pseudo-Abelian Categories.} Let $\sigma :a\to b$ be a morphism in
an additive category ${\bf X}$; the \emph{kernel} $K(\sigma )$
\emph{of} $\sigma$ is a pair $(\opnm{Ker}\sigma ,k)$, where
$\opnm{Ker}\sigma$ is an object of ${\bf X}$ and $k=k_{\sigma
}:\opnm{Ker}\sigma \to a$ an arrow such that $\sigma k=0\in
\opnm{Hom}_{\bf X}(\opnm{Ker}\sigma ,b)$. Moreover, $\opnm{Ker}\sigma$
is the ``biggest'' object with this property, in the sense that if
$k':K'\to a$ is another arrow such that $\sigma k'=0$, then there
exists a unique morphism $i:K'\to \opnm{Ker}\sigma$ such that
$k'=ki$. In a pseudo-abelian category ${\bf X}$, every idempotent
$\sigma :a\to a$ has a kernel; as $1_a-\sigma$ is also idempotent,
then $\opnm{Ker}(1_a-\sigma)$ is also defined; the canonical arrow
\eqref{canonical_arrow} is the unique map $\opnm{Ker}\sigma \oplus
\opnm{Ker}(1_a-\sigma )\to a$ which makes the diagram
$$
\xymatrix{
  & a & \\
  \opnm{Ker}\sigma \ar[ur]^{k_{\sigma}} \ar[r]_-{i_1} &
  \opnm{Ker}\sigma \oplus \opnm{Ker}(1_a-\sigma ) \ar[u] &
  \opnm{Ker}(1_a-\sigma ), \ar[ul]_{k_{1_a -\sigma }} \ar[l]^-{i_2} }
$$
commutative.

An important example of a pseudo-abelian category is the category
$\tsf{Vect}(M)$ of vector bundles over a manifold $M$; see section
\ref{bundles_operations}).

Another term used to describe this situation is to say that the
idempotent $\sigma$ \emph{splits}. In fact, the definition of
pseudo-abelian category given here restricts to additive categories,
but the notion of idempotent splitting can be given in an arbitrary
category. Moreover, given any category ${\bf X}$ in which idempotents
do not split, a new category $\widehat{\bf X}$, called the
\emph{idempotent completion}, \emph{Karoubi envelope} or \emph{Cauchy
  completion of} ${\bf X}$ can be constructed in a way such that
\begin{itemize}
\item the category ${\bf X}$ embeds naturally in $\widehat{\bf X}$ and
\item every idempotent in $\widehat{\bf X}$ splits.
\end{itemize}
We sketch the construction of the category $\widetilde{{\bf X}}$: Its
objects are pairs $(a,\sigma )$, where $\sigma :a\to a$ is an
idempotent map. A morphism $(a,\sigma )\to (b,\tau )$ is an arrow
$f:a\to b$ in ${\bf X}$ such that $f\sigma =f=\tau f$, and composition
is the same as the one in ${\bf X}$; the identity arrow of an object
$(a,\sigma )$ is $\sigma$. The embedding ${\bf X}\to \widehat{\bf X}$
is given by the assignment $a\mapsto (a,1_a)$. In the
additive-category setting, the objects $(a,\sigma )$ and
$(a,1_a-\sigma )$ should be interpreted as $\opnm{Ker}\sigma $ and
$\opnm{Ker}(1_a-\sigma )$ (they are in fact kernels in the additive
category $\widehat{{\bf X}}$), and the isomorphism $(a,1_a)\to
(a,\sigma )\oplus (a,1_a-\sigma )$ is given by the matrix $\left
  (\begin{smallmatrix} \sigma & 1_a-\sigma \\ \end{smallmatrix} \right
)$, with inverse $\left (\begin{smallmatrix} \sigma \\ 1_a-\sigma
    \\ \end{smallmatrix} \right )$. If $\sigma :a\to a$ is an
idempotent map ${\bf X}$, then we can view it in $\widehat{\bf X}$ as
an arrow
$$\sigma :(a,\sigma )\oplus (a,1_a-\sigma )\longrightarrow (a,\sigma )\oplus (a,1_a-\sigma ),$$
and hence as a matrix $\left (\begin{smallmatrix} \sigma_{11} &
    \sigma_{21} \\ \sigma_{12} & \sigma_{22} \end{smallmatrix} \right
)$. As the composite maps $\sigma (1_a-\sigma )$ and $(1_a-\sigma
)\sigma$ are both equal to 0, then $\sigma = \left
  (\begin{smallmatrix} 0 & 0 \\ 0 & 1_a \end{smallmatrix} \right )$.

For details, the reader is referred to
\cite{borceux:_cauchy_completion, natan:_envelope} and
references therein.

{\sc Calabi-Yau Categories.} The notion of Calabi-Yau category comes
from physics. In fact, in one of the aforementioned references,
K. Costello shows that $A_\infty$ Calabi-Yau categories classify
open-closed topological conformal field theories. In a {\sc cy} category,
for each object $a\in {\bf X}$, the existence of a trace $\theta_a$
implies that the hom-set $\operatorname{End}_{{\bf
    X}}(a)=\operatorname{Hom}_{{\bf X}}(a,a)$ is a Frobenius
$R$-algebra. Equivalently, as the pairing \eqref{pair} is
non-degenerate, we have that the $R$-module $\operatorname{Hom}_{{\bf
    X}}(b,a)$ is canonically isomorphic to the dual module
$\operatorname{Hom}_{{\bf X}}(a,b)^*$. A {\sc cy} category
\emph{in the sense of Moore and Segal} is a {\sc cy} category which
satisfies the conditions listed in section \ref{data}; in other words,
it is a {\sc cy} category which models an open-closed topological field
theory.

We can generalize these notions to fibred categories over a manifold
$M$.

\begin{defi}
  Let $\scr{R}$ be a sheaf of commutative rings with unit. A presheaf
  of categories $\scr{B}$ over $M$ is said to be
  \emph{$\scr{R}$-linear} iff for every open subset $U\subset M$, the
  category $\scr{B}(U)$ is $\scr{R}(U)$-linear and all the structures
  are compatible with pullbacks. Presheaves of additive,
  pseudo-abelian and of $\scr{R}$-linear Calabi-Yau categories are
  defined analogously.
\end{defi}

Note that if $\scr{B}$ is an $\scr{R}$-linear Calabi-Yau category over
$M$, then for each open subset $U\subset M$ and each object $a\in
\scr{B}(U)$, we have that $\operatorname{Hom}_{\scr{B}(U)}(a,a)$ is a
Frobenius $\scr{R}(U)$-algebra. As for $\scr{R}$-modules, this
statement can be generalized by saying that the presheaf
$\underline{\opnm{Hom}}_U(a,a)$ is a Frobenius
$\scr{R}|_U$-algebra.

\begin{obs}
  We use the term \emph{presheaf of categories} as a synonym for
  \emph{fibred category}.
\end{obs}


%%%%%%%%%%%%%%%%%%%%%%%%%%%%%%%%%%%%%%%%
\section{Calabi-Yau and Cardy Fibrations}

In \cite{moore_segal1}, Moore and Segal define a model for an
open-closed topological field theory of dimension 2. An account of
these results was given in section \ref{sec_octfts} of chapter
\ref{fsfts}. Theorem \ref{theorem_3} provides an algebraic
characterization of a maximal category of boundary conditions, which turns
out to be (non-canonically) equivalent to the category of finite-rank, complex vector bundles over the spectrum of the Frobenius algebra $A$.

Moore and Segal's construction can be regarded as a theory over a one-point space, say $\{x\}$. By replacing
\begin{itemize}
\item $\{x\}$ by an $F$-manifold $M$,\footnote{In fact, broadly speaking, we shall need only to consider manifolds $M$ such that $T_xM$ is a Frobenius algebra and such that for each $x$, a frame of idempotent, orthogonal sections exists.}
\item the closed algebra $C$ by the tangent bundle $TM$ (i.e. over each point $x\in M$, the closed algebra is the fibre $T_xM$) and
\item the spectrum of the algebra $A$ by the spectral cover of $M$
\end{itemize}
we shall obtain not just a category but a sheaf of categories which has relations which 2-vector bundles, as Segal conjectured. 


%%%%%%%%%%%%%%%%%%%%%%%%%%%%%%%%%%
\subsection{Calabi-Yau Fibrations}

From now on, we shall work with a ringed space $(M,\scr{O}_M)$ with the following properties:

\begin{itemize}
\item $TM$ is a bundle of algebras, i.e. $M$ is a manifold with multiplication and $\scr{O}_M$ is the usual structure sheaf (i.e. the sheaf of smooth functions in case $M$ is a smooth manifold; in particular, note that $\scr{O}_{M,x}$ is a local ring for each $x\in M$).
\item There exists a linear form $\theta :\Gamma(TM)\to \scr{O}_M$ making each fibre $T_xM$ a commutative Frobenius $\comp$-algebra.
\item $M$ is massive; i.e. each tangent space $T_xM$ is semisimple. In particular, for each $x\in M$ there exists a neighborhood $U\ni x$ and a frame of sections $\{e_1,\dots ,e_n\}$ defined over $U$ such that $e_ie_j=\delta_{ij}e_i$ and $\sum_i e_i=1$. In this case, we shall also say that $U$ is semisimple.
\end{itemize}

For simplicity, we shall refer to such a space as a \emph{semisimple manifold with multiplication} or just \emph{massive/semisimple manifold}. The reader should be aware that this name hides all the properties listed before.

Let $M$ denote a semisimple manifold with multiplication, with
structure sheaf $\scr{O}=\scr{O}_M$ and let $\scr{B}$ be an
$\scr{O}$-linear {\sc cy} category over $M$. For objects $a,b\in
\scr{B}(U)$, let us denote by $\Gamma_{ab}$ the presheaf
$\underline{\operatorname{Hom}}_U(a,b)$ over $U$ given by
\begin{equation}\label{sheaf_maps}
  V\longmapsto \operatorname{Hom}_{\scr{B}(V)}(a|_V,b|_V).
\end{equation}
By definition of {\sc cy} category, we have that $\Gamma_{aa}$ is a
Frobenius $\scr{O}_U$-algebra for each $a\in
\scr{B}(U)$. We shall denote the linear form corresponding to $\Gamma_{aa}$
by $\theta_a$.


\begin{notation}
  Recall that if the base manifold is clear, we shall supress the
  subscript of the structure sheaf when taking local sections;
  e.g. instead of using the notation $\scr{O}_M(U)$ for $U\subset M$,
  we will only write $\scr{O}(U)$; and the restriction $\scr{O}_M|_U$
  shall be denoted $\scr{O}_U$. The same considerations are applied to
  the tangent sheaf $\scr{T}_M$ of a manifold $M$.
\end{notation}

We now turn to the relevant definitions.

\begin{defi}\label{cy_fibration}
  A \emph{Calabi-Yau ({\sc cy}) fibration over a semisimple manifold $M$} is
  a pair $(\scr{B},\mathfrak{U})$ (the open cover shall be omitted
  form the notation), where $\scr{B}$ is a {\sc cy} category over $M$ and
  $\mathfrak{U}=\{U_\alpha \}$ is an open cover of $M$, subject to the
  following conditions:
  \begin{enumerate}
  \item Each $U_\alpha \in \mathfrak{U}$ is semisimple.
  \item $\scr{B}$ is a stack.\footnote{In particular, the presheaf
      \eqref{sheaf_maps} is a sheaf.}
  \item Given any $U_\alpha\in \mathfrak{U}$ and objects $a,b\in
    \scr{B}(U_\alpha)$, the sheaf $\Gamma_{ab}$ is a locally-free
    $\scr{O}_{U_\alpha}$-module of finite rank. Objects of $\scr{B}(U)$ are called \emph{labels},
    \emph{boundary conditions} or \emph{D-branes} over $U$.
	%\item For each $a\in \scr{B}(U)$, the locally free module $\Gamma_{aa}$ is a symmetric Frobenius 
	%algebra with linear form $\theta_a$.
  \item For each $U_\alpha \in \mathfrak{U}$ and each object $a\in
    \scr{B}(U_\alpha )$, we have transition (sheaf) homomorphisms
    \begin{displaymath}
      \iota_a:\scr{T}_{U_\alpha}\longrightarrow \Gamma_{aa} 
      \quad , \quad \iota^a:\Gamma_{aa}\longrightarrow \scr{T}_{U_\alpha}.     
    \end{displaymath}
The previous data is subject to the following conditions:

\begin{enumerate}
\item $\iota_a$ is a morphism of $\scr{O}_{U_\alpha}$-algebras
  (preserves multiplication and unit) and $\iota^a$ is an
  $\scr{O}_{U_\alpha}$-linear map.\footnote{In particular, $\iota_a$
    provides $\Gamma_{aa}$ with a $\scr{T}_{U_\alpha}$-algebra
    structure.}
\item $\iota_a$ is central: given $X\in \scr{T}(V)$ and $\sigma \in
  \Gamma_{ab}(V)$, we have
  \begin{equation}\label{centrality}
    \sigma \iota_a(X)=\iota_b(X)\sigma
  \end{equation}
  in $\Gamma_{ab}(V)$, for each $V\subset U_\alpha$.
\item There is an adjoint relation between $\iota_a$ and $\iota^a$
  given by
  \begin{equation}\label{adjoint}
    \theta (\iota^a(\sigma )X)=\theta_a(\sigma \iota_a(X)),
  \end{equation}
  for each $X\in \scr{T}_{U_\alpha}$ and $\sigma \in
  \Gamma_{aa}$.\footnote{Recall that, given a sheaf $\scr{S}$ over
    some space $M$, the notation $x\in \scr{S}$ means that $x\in
    \scr{S}(U)$ for some arbitrary open subset $U\subset M$.}
\end{enumerate}
\end{enumerate}
\end{defi}

\begin{obs}
  For some technical considerations (see definition \ref{maximal}), we
  will assume that our {\sc cy} fibrations $\scr{B}$ verify that for each
  open subset $U\subset M$, the skeleton $\opnm{sk}\scr{B}(U)$ of the
  category $\scr{B}(U)$ is a set.
\end{obs}


%%%%%%%%%%%%%%%%%%%%%%%%%%%%%
\subsection{Cardy Fibrations}

For $U_\alpha \in \mathfrak{U}$ open and $a,b\in \scr{B}(U_\alpha )$,
pick a local basis $\{\sigma_i\}$ of $\Gamma_{ab}$ and let
$\{\sigma^i\}$ be a basis of $\Gamma_{ab}^*$ dual to
$\{\sigma_i\}$. Define the map $\pi^a_b:\Gamma_{aa}\to \Gamma_{bb}$ by
$$\pi^a_b(\sigma )=\sum_i\sigma_i\sigma \sigma^i.$$
Some comments are in place: the sequence of maps
\begin{equation}\label{duality}
  \Gamma_{ba}\otimes \Gamma_{ab}\longrightarrow \Gamma_{bb}\stackrel{\theta_a}{\longrightarrow}\scr{O}_U
\end{equation}
induces a duality isomorphism
$\Gamma_{ba}\stackrel{\cong}{\longrightarrow}\Gamma_{ab}^*$. The dual
basis in the definition of $\pi_b^a$ is in fact the preimage of the
dual basis of $\{\sigma_i\}$ under this isomorphism. Another key
observation is stated in the following

\begin{proposition}\label{cardy_well_def}
  The map $\pi^a_b$ does not depend on the chosen (local) basis.
\end{proposition}
\begin{proof}
  As $\Gamma_{aa}$, $\Gamma_{bb}$ and $\Gamma_{ba}$ are locally-free,
  we can pick an open cover $\mathfrak{U}_\alpha$ of $U_\alpha$ such
  that $\Gamma_{aa}|_V\cong \scr{O}^{n_{a}}$, $\Gamma_{ba}|_V\cong
  \scr{O}^{n_{ba}}$, etc. for each $V\in \mathfrak{U}_\alpha$. Pick
  then a basis $B=\{e_1,\dots ,e_{n_{ba}}\}$ for
  $\Gamma_{ba}|_V$.\footnote{By a \emph{basis} we mean a system of
    linearly independent generators $e_1,\dots ,e_{n_{ba}}\in
    \Gamma_{ba}(V)$ such that $\{e_1|_W,\dots ,e_{n_{ba}}|_W\}$ is
    also linearly independent and generates $\Gamma_{ba}(W)$ for each
    $W\subset V$. For instance, let $u_1,\cdots ,u_{n_{ba}}\in
    \scr{O}(V)$ be units; then, if $e_i=(0,\dots ,0,1,0,\dots ,0)$,
    the sections $u_1e_1,\dots ,u_{n_{ab}}e_{n_{ba}}$ form a basis.}
  Let $B'=\{e^1,\dots ,e^{n_{ba}}\}$ be the corresponding dual basis
  for $\Gamma_{ba}^*$. Then, in terms of this basis we have $\pi_b^a
  (\sigma )=\sum_ie_i\sigma e^i$. Let $D=\{f_1,\dots ,f_{n_{ba}}\}$ be
  another basis over $V$ with dual basis $D'$. We then have
$$f_i=\sum_j\lambda_{ij}e_j \quad \text{and} \quad f^i=\sum_j\mu^{ij}e^j.$$
Replacing these linear combinations in the equality
$\delta_{ij}=f^i(f_j)$ we obtain
$$\delta_{ij}=\sum_k\mu^{ik}\lambda_{jk}.$$
If $A:=(\lambda_{ij})$ and $B:=(\mu^{ij})$ then the previous equality
implies that $AB^t=I$ or, equivalently, $A^tB=I$, which in terms of
the coefficients is expressed by
$\delta_{ij}=\sum_k\lambda_{ki}\mu^{kj}$. We now compute
$$
\begin{aligned}
  \sum_if_i\sigma f^i &= \sum_i\Bigl (\sum_j\lambda_{ij}e_j\Bigr )\sigma \Bigl (\sum_k\mu^{ik}e^k\Bigr ) \\
  &= \sum_{j,k}\Bigl (\sum_i\lambda_{ij}\mu^{ik}\Bigr )e_j\sigma e^k \\
  &=\sum_{j,k}\delta_{jk}e_j\sigma e^k \\
  &=\sum_je_j\sigma e^j, \\
\end{aligned}
$$
as desired.
\end{proof}

Then, when defining $\pi_b^a$ locally on each $V$, we have that, by
the previous computation, these expressions coincide over non-empty
overlaps, and thus can be glued together to obtain a morphism over
$U_\alpha \in \mathfrak{U}$
$$\pi_b^a :\Gamma_{aa}\longrightarrow \Gamma_{bb}.$$
This final layer of structure is included in the following

\begin{defi}\label{cardy_fib}
  A Calabi-Yau fibration $\scr{B}$ is called a \emph{Cardy fibration}
  if the following condition, called the \emph{Cardy
    condition}, holds for each open subset $U_\alpha \in
  \mathfrak{U}$: For $a,b\in \scr{B}(U_\alpha )$,
$$\pi^a_b=\iota_b\iota^a.$$
In other words, the following triangle
$$
\xymatrix{
  \Gamma_{aa} \ar[dr]_{\iota^a}\ar[rr]^{\pi^a_b} & & \Gamma_{bb} \\
  & \scr{T}_U \ar[ur]_{\iota_b} & }
$$
should commute.
\end{defi}

We shall deal with Cardy fibrations all along.

\begin{defi}
  A Cardy fibration $\scr{B}$ is said to be \emph{trivializable} if
  and only if conditions (3), (4)a-c in definition \ref{cy_fibration}
  and the Cardy condition hold also for any open subset of each
  $U_\alpha \in \mathfrak{U}$.
\end{defi}

A characterization of a certain kind of \emph{trivializable} Cardy
fibrations shall be given in the next chapter.

% Hasta aqui 2013-09-30
%%%%%%%%%%%%%%%%%%%%%%%%%%%%%%%%%%%%%%%
\subsection{Global Objects}

We shall now deduce some further structure enjoyed by globally defined
boundary conditions. These properties are needed in chapter \ref{dbtvb}.

We first note that for a proper open subset $U$ of $M$ ($U\neq
U_\alpha$ for each $\alpha$), and objects $a,b\in \scr{B}(U)$, the
sheaves $\Gamma_{ab}$ need not be locally free. But this situation is
slightly different when considering $U=M$.

Take global objects $a,b\in \scr{B}(M)$; hence, $a_\alpha
:=a|_{U_\alpha}, b_\alpha :=b|_{U_\alpha}\in \scr{B}(U_\alpha )$ and
$\Gamma_{a_\alpha b_\alpha }$ is a locally free $\scr{O}_{U_\alpha
}$-module, which in turn implies that $\Gamma_{ab}$ is a locally free
$\scr{O}$-module.

We also have transition homomorphisms
$$\iota_{a_\alpha}:\scr{T}_{U_\alpha }\longrightarrow \Gamma_{a_\alpha a_\alpha }\quad , \quad \iota^{a_\alpha}:\Gamma_{a_\alpha a_\alpha }\longrightarrow \scr{T}_{U_\alpha }.$$
Pick now an open subset $U_\beta \in \mathfrak{U}$ such that
$U_{\alpha \beta }\neq \emptyset$ and let $a_\beta
:=a|_{U_\beta}$. For $U_{\alpha \beta }$, as $\Gamma_{a_\alpha
  a_\alpha }(U_{\alpha \beta })=\Gamma_{a_\beta a_\beta }(U_{\alpha
  \beta })=\Gamma_{aa}(U_{\alpha \beta})$, we have maps
$$
\iota_{a_\alpha ,U_{\alpha \beta}},\iota_{a_\beta ,U_{\alpha \beta}} :
\scr{T}(U_{\alpha \beta })\longrightarrow \Gamma_{aa}(U_{\alpha \beta
}),
$$
which we also shall denote by $\iota_{a_\alpha }$ and $\iota_{a_\beta
}$ for notation's sake.

Let now $X\in \scr{T}(U_{\alpha \beta })$ and let $\sigma \in
\Gamma_{aa}$. The centrality condition \eqref{centrality} implies that
over $U_{\alpha \beta }$ the equality
$$\sigma |_{U_{\alpha \beta}} \iota_{a_\alpha} (X) = \iota_{a_\beta} (X)\sigma |_{U_{\alpha \beta }}$$
holds. Taking $\sigma =1_a$ we conclude that the morphisms
$\iota_{a_\alpha }|_{U_{\alpha \beta }}$ and $\iota_{a_\beta
}|_{U_{\alpha \beta }}$ are equal, and hence can be glued into a
global algebra homomorphism
$$\iota_a:\scr{T}_M\longrightarrow \Gamma_{aa}.$$

An analogous conclusion can be derived for the other transition map;
for this we use tha adjoint relation \eqref{adjoint}. First note that
the restrictions of the linear forms $\theta_{a_\alpha}$ and
$\theta_{a_\beta }$ to $U_{\alpha \beta}$ are the same, as they are
both equal to the restriction $\theta_a|_{U_{\alpha \beta}}$. Then,
using this fact together with the adjoint relation over $U_{\alpha
  \beta }$ we obtain
$$\theta (\iota^{a_\alpha }(\sigma )X)=\theta_{a_\alpha}(\sigma \iota_{a_\alpha}(X))=\theta_{a_\beta}(\sigma \iota_{a_\beta}(X))=\theta (\iota^{a_\beta }(\sigma )X)$$
for each vector field $X:U_{\alpha \beta }\to TM$ and each section
$\sigma \in \Gamma_{aa}(U_{\alpha \beta})$. Hence, the equality
$$\theta \left ((\iota^{a_\alpha }(\sigma )-\iota^{a_\beta}(\sigma ))X\right )=0$$
holds for each $X$ and $\sigma$. As $\theta$ is non degenerate, we can
then conclude that the morphisms $\iota^{a_\alpha }|_{U_{\alpha \beta
  }}$ and $\iota^{a_\beta }|_{U_{\alpha \beta }}$ are equal, thus
obtaining a global map
$$\iota^a:\Gamma_{aa}\longrightarrow \scr{T}_M.$$

A similar procedure shows that the map $\pi^a_b$ exists also for
global objects $a,b\in \scr{B}(M)$. Moreover, the verification of the
centrality condition, adjoint relation and Cardy condition for these
``new'' maps can be deduced with no difficulties from the local
versions.


\clearpage

{\small
%%%%%%%%%%%%%%%%%%%%%%%%%%%%%%%%%%%%%%%%%%%%
%%%%%%%%%%%%%%%%%%%%%%%%%%%%%%%%%%%%%%%%%%%%
\section{Resumen del Cap\'itulo \ref{cfib}}

En este cap\'itulo se definen los objetos que componen el n\'ucleo de este trabajo, los cuales, a grandes rasgos, son b\'asicamente familias de teor\'ias topol\'ogicas de campos, indexadas por una variedad con multiplicaci\'on particular.


%%%%%%%%%%%%%%%%%%%%%%%%%%%%%%%%%%%%%%%%%%%
\subsection{Categor\'ias de Calabi-Yau}

Para lo que sigue ser\'a necesario introducir ciertos tipos de categor\'ias. Daso un anillo conmutativo $R$ con unidad, diremos que una categor\'ia ${\bf X}$ es
\begin{itemize}
\item \emph{$R$-lineal} si est\'a enriquecida sobre la categor\'ia de $R$-modulos;
\item \emph{aditiva} si es $\ent$-lineal , tiene un objeto inicial $0$ y para cada par de objetos $a,b\in {\bf X}$ se tiene definida una suma $a\oplus b\in {\bf X}$;
\item \emph{pseudo-abeliana} si es aditiva y para cada objeto $a\in {\bf X}$ y cada idempotente $\sigma :a\to a$ existe un objeto $\opnm{Ker}\sigma \in {\bf X}$ (el \emph{n\'ucleo} de $\sigma$) tal que la aplicaci\'on can\'onica
$$ \opnm{Ker} \sigma \oplus \opnm{Ker} (1_a -\sigma )\longrightarrow a$$
es un isomorfismo;
\item una \emph{categor\'ia de Calabi-Yau} (abreviado {\sc cy}) si es $R$-lineal, los $R$-m\'odulos $\opnm{Hom}_{\bf X}(a,a)$ son finitamente generados y proyectivos y para cada objeto $a\in {\bf X}$ se tiene una forma lineal
$$\theta_a:\opnm{Hom}_{\bf X}(a,a)\longrightarrow R$$
tal que la composici\'on
$$ \operatorname{Hom}_{{\bf X}}(a,b)\otimes_R
  \operatorname{Hom}_{{\bf X}}(b,a)\longrightarrow 
  \operatorname{Hom}_{{\bf X}}(a,a)\stackrel{\theta_a}{\longrightarrow }R$$
es una forma bilineal no degenerada.
\end{itemize}

Las categor\'ias que nos interesan se construyen a partir de las anteriores, b\'asicamente considerando categor\'ias fibradas.

%%%%%%%%%%%%%%%%%%%%%%%%%%%%%%%%%%%%%%
\subsection{Fibraciones de Calabi-Yau}

En lo que sigue, $M$ ser\'a una variedad con multiplicaci\'on con las siguiente propiedades:

\begin{itemize}
\item Se tiene una forma lineal $\theta :\Gamma(TM)=:\scr{T}_M\to \scr{O}_M$ que hace a cada espacio tangente $T_xM$ una $\comp$-\'algebra de Frobenius, siendo $\scr{O}_M$ el haz estructural usual (por ejemplo, el haz de funciones suaves en caso que $M$ sea una variedad $C^\infty$; en particular, $\scr{O}_{M,x}$ es un anillo local para cada $x\in M$).
\item $M$ es masiva; es decir, $T_xM$ es semisimple para cada $x$.
\end{itemize}
\medskip

{\bf Definici\'on.} Una \emph{fibraci\'on de Calabi-Yau} ({\sc cy}) sobre una variedad semisimple $M$ es una par $(\scr{B},\mathfrak{U})$ formado por una categor\'ia de {\sc cy} $\scr{B}$ sobre $M$ y un cubrimiento abierto $\mathfrak{U}=\{U_\alpha \}$ sujetos a las siguientes condiciones:
\begin{enumerate}
\item Cada $U_\alpha$ es un abierto semisimple; es decir, existe sobre $U$ una base de secciones idempotentes ortogonales $\{e_1,\dots ,e_n\}$ tales que $\sum_ie_i=1$.
\item $\scr{B}$ es un stack.
\item Dado $U_\alpha \in \mathfrak{U}$ y $a,b \in \scr{B}(U_\alpha )$, el haz de morfismos $a\to b$, que notamos $\Gamma_{ab}$, es un $\scr{O}_{U_\alpha }$-m\'odulo localmente libre de rango finito. Los objetos de $\scr{B}(U)$ se llamar\'an \emph{condiciones de borde o $D$-branas sobre $U$}.
\item Para cada $U_\alpha \in \mathfrak{U}$ y cada $a\in \scr{B}(U_\alpha )$ se tienen morfismos de transici\'on $\iota_a:\scr{T}_{U_\alpha }\to \Gamma_{aa}$, $\iota^a:\Gamma_{aa}\to \scr{T}_{U_\alpha}$.
\end{enumerate}
Lo anterior sujeto a las siguientes condiciones:
\begin{enumerate}[(a)]
\item $\iota_a$ es un morfismo de \'algebras e $\iota^a$ es $\scr{O}_{U_\alpha }$-lineal.
\item $\iota_a$ es central: dado un campo local $X$ sobre $V\subset U_\alpha$ y $\sigma \in \Gamma_{ab}(V)$, se tiene $\sigma \iota_a(X)=\iota_b(X)\sigma$ en $\Gamma_{ab}(V)$.
\item Se tiene una relaci\'on de adjunci\'on entre $\iota_a$ e $\iota^a$ dada por $\theta (\iota^a(\sigma )X)=\theta_a(\sigma \iota_a(X))$ para cada campo $X$ y cada $\sigma :a\to a$.\footnote{Dado un haz $\scr{S}$, digamos de conjuntos para fijar ideas, la notaci\'on $x\in \scr{S}$ indica $x\in \scr{U}$ para un abierto arbitrario $U$.}
\end{enumerate}

%%%%%%%%%%%%%%%%%%%%%%%%%%%%%%%%%
\subsection{Fibraciones de Cardy}

Dado $U_\alpha \in \mathfrak{U}$ y $a,b \in \scr{B}(U_\alpha )$, sea $\{\sigma_i\}$ una base local arbitraria de $\Gamma_{ab}$ y sea $\{\sigma^i\}$ su dual. Se define un mapa $\pi^a_b:\Gamma_{aa}\to \Gamma_{bb}$ por la ecuaci\'on
$$\pi^a_b(\sigma )=\sum_i\sigma_i\sigma \sigma^i.$$
Un comentario sobre esta definici\'on: se tiene un isomorfismo $\Gamma_{ba}\to \Gamma_{ab}^*$ inducido por la forma bilineal $\Gamma_{ba}\otimes \Gamma_{ab}\to \Gamma_{bb}\to \scr{O}_U$; la base dual a la que nos referimos est\'a en realidad formada por las preimagenes de $\sigma^i$ bajo el isomorfismo anterior. Mas a\'un, una demostraci\'on elemental muestra que el mapa $\pi^a_b$ no depende de la base elegida.

Definimos a continuaci\'on los objetos que estudiaremos en detalle en lo que resta del trabajo.
\medskip

{\bf Definici\'on.} Una fibraci\'on de {\sc cy} se dice una \emph{fibraci\'on de Cardy} si la siguiente ecuaci\'on, llamada \emph{condici\'on de Cardy}, se verifica en cada $U_\alpha \in \mathfrak{U}$: $\pi^a_b=\iota_b\iota^a$.
\medskip

{\bf Observaci\'on.} Es importante hacer notar (y lo usaremos mas adelante), que los morfismos $\iota_a,\iota^a$ y $\pi^a_b$ existen tambi\'en sobre $M$; es decir, si $a,b\in \scr{B}(M)$, podemos entonces considerar las restricciones $a|_{U_\alpha}$ y $b|_{U_\alpha}$ y tambi\'en los morfismos $\iota_{a|_{U_\alpha}}$, $\iota_{b|_{U_\alpha}}$ y $\pi^{a|_{U_\alpha}}_{b|_{U_\alpha}}$. Dadas las propiedades que verifican los morfismos locales, podemos pegar estos mapas en mapas globales $\iota_a$, $\iota^a$, $\pi^a_b$.









}








%%% Local Variables: 
%%% mode: latex
%%% TeX-master: "master"
%%% End: 


% Chapter 4
\chapter{Local Description of Cardy Fibrations}
\label{local_description}
 
\vspace{250pt}

%%%%%%%%%%%%%%%%%%%%%%%%%%%%%%%%%%%%%%%%%%%%%%%%%%%%%%%%%%
\section{Algebraic Properties of Maximal Cardy Fibrations}

This section will be devoted to describing in detail the stack of boundary conditions $\mathscr{B}$. The idea is to describe all posible branes for a given category; to accomplish this, we shall first deal with morphisms and later with the whole category.

As we are only interested in maximal fibrations, we introduce them now. Given a category ${\bf X}$, recall that $\opnm{sk}{\bf X}$ denotes its skeleton.

\begin{defi}\label{maximal}
A Cardy fibration $\scr{B}$ over a manifold $M$ is said to be \emph{maximal} if given another Cardy fibration $\scr{B}'$ over $M$, then there exists an injective map $\opnm{sk}\scr{B}'\to \opnm{sk}\scr{B}$.
\end{defi}

Our first goal now is to show that the stalks of a Cardy fibration are maximal categories in the sense of Moore and Segal. The idea is to pick a point $x\in M$ and prove that all the fibres over $x$ of the sheaves involved in this discussions define a brane category as discussed in \cite{moore_segal1}. This approach will let us generalize all the results to Cardy fibrations.

Let us fix a point $x\in M$ and an index $\alpha$ such that $U_\alpha$ is semisimple and $x\in U_\alpha$. Given arbitrary labels $a,b\in \mathscr{B}(U_\alpha )$, let us denote by $E_{ab}$ the fibre over $x$ for the sheaf $\Gamma_{ab}$ (we omit reference to the point $x$ to keep the notation as simple as possible). We need to show that the vector spaces $T_xM$ and $E_{ab}$, together with the appropriate morphisms, form a {\sc cy} category in the sense of Moore and Segal.

Let us denote by $p_{ab}$ (or just $p$ if the labels are clear) the sequence of proyections
\begin{equation}\label{proy}
\Gamma_{ab}(U_\alpha )\longrightarrow \Gamma_{ab,x}\longrightarrow E_{ab},
\end{equation}
where $\Gamma_{ab,x}$ is the stalk over $x$ of the sheaf $\Gamma_{ab}$. Let $1_a$ be the unit in $\Gamma_{aa}(U_\alpha )$; let us identify a label $a\in \scr{B}(U_\alpha )$ with $1_a$, and denote $p_{aa}(1_a)$ by $\overline{a}$. We now define the category of boundary conditions $\overline{\scr{B}}_x$; its objects are given by
$$\operatorname{Obj}\overline{\mathscr{B}}_x=\{\overline{a}=p_{aa}(1_a) \; | \; a\in \mathscr{B}(U_\alpha )\}.$$
If $\overline{a},\overline{b}\in \overline{\mathscr{B}}_x$, consider the corresponding units $1_a\in \Gamma_{aa}(U_\alpha )$ and $1_b\in \Gamma_{bb}(U_\alpha )$. Then
$$\operatorname{Hom}_{\overline{\mathscr{B}}_x}(\overline{a},\overline{b}):=E_{ab}.$$
With this definition, $\operatorname{Hom}_{\overline{\mathscr{B}}_x}(\overline{a},\overline{b})$ is a $\comp$-vector space, with dimension equal to the rank of $\Gamma_{ab}$. We shall denote this vector space by $O_{\overline{a}\overline{b}}$.

We also have the linear forms $\theta :\scr{T}_M\to \scr{O}$ and $\theta_a:\Gamma_{aa}\to \scr{O}$ which induce linear maps on the fibres
$$
\begin{aligned}
\overline{\theta}_x &: T_xM\longrightarrow \comp \\
\theta_{\overline{a}} &: O_{\overline{a}\overline{a}}\longrightarrow \comp \\
\end{aligned}
$$
 which provide $T_xM$ and $O_{\overline{a}\overline{a}}$ with a Frobenius $\comp$-algebra structure.

In the same fashion, the transition morphisms $\iota_a$ and $\iota^a$ induce maps
$$T_xM\stackrel{\iota_{\overline{a}}}{\longleftarrow}O_{\overline{a}\overline{a}}\stackrel{\iota^{\overline{a}}}{\longrightarrow}T_xM.$$

\begin{lemma}
Let $x_0,x_1\in U_\alpha$. Then the categories $\overline{\scr{B}}_{x_0}$ and $\overline{\scr{B}}_{x_1}$ are isomorphic.
\end{lemma}
\begin{proof}
Let us consider two labels $a,b\in \scr{B}(U_\alpha )$; to distinguish between the two fibres, let us go back to the previous notation: $F_x(\scr{M})$ is the fibre over $x$ of the locally free module $\scr{M}$; likewise, let us denote by $p_{aa}^0$ (for $x_0$) or $p_{aa}^1$ (for $x_1$) the projection \eqref{proy}. By connectivity assumptions, the ranks of $\Gamma_{aa}$ and $\Gamma_{ab}$ are constant and we can therefore fix isomorphisms
$$\phi_{aa}:F_{x_0}(\Gamma_{aa})\cong F_{x_1}(\Gamma_{aa})\quad \text{and} \quad \phi_{ab} :F_{x_0}(\Gamma_{ab})\cong F_{x_1}(\Gamma_{ab})$$
such that the diagrams
$$
\xymatrix{
 & F_{x_0}(\Gamma_{aa}) \ar[dd]^{\phi_{aa}} \\
\Gamma_{aa}(U_\alpha )\ar[ur] \ar[dr] & \\
 & F_{x_1}(\Gamma_{aa})}
\xymatrix{
 & F_{x_0}(\Gamma_{ab}) \ar[dd]^{\phi_{ab}} \\
\Gamma_{ab}(U_\alpha )\ar[ur] \ar[dr] & \\
 & F_{x_1}(\Gamma_{ab})}
$$
commute, where the unlabelled arrows are canonical projections. In particular, this commutativity implies that, for example, $p_{aa}^0(1_a)\in F_{x_0}(\Gamma_{aa})$ is mapped onto $p_{aa}^1(1_a)$.

We now define a functor $F:\overline{\scr{B}}_{x_0}\to \overline{\scr{B}}_{x_1}$; on objects, if $\overline{a}_0:=p_{aa}^0(1_a)$, then
$$F(\overline{a}_0)=\phi_{aa}(\overline{a}_0).$$
Let now $\sigma :\overline{a}_0\to \overline{b}_0$ be an arrow in $\overline{\scr{B}}_{x_0}$. That is, $\sigma$ is an element of $F_{x_0}(\Gamma_{ab})$. Then we define
$$F (\sigma )=\phi_{ab}(\sigma ).$$
The inverse of this functor is constructed in the same way, by considering $\phi_{aa}^{-1}$ and $\phi_{ab}^{-1}$.
\end{proof}

\begin{theorem}\label{ms_over_point}
The category $\overline{\mathscr{B}}_x$, together with the Frobenius algebra $T_xM$ and the structure maps $\overline{\theta}_x$, $\theta_{\overline{a}}$, $\iota_{\overline{a}}$ and $\iota^{\overline{a}}$ ($\overline{a}\in \overline{\scr{B}}_x$) defines a brane category in the sense of Moore and Segal.
\end{theorem}
\begin{proof}
Given objects $\overline{a}$ and $\overline{b}$, by definition $\operatorname{Hom}_{\overline{\mathscr{B}}}(\overline{a},\overline{b})=E_{ab}$ is a $\comp$-vector space. Thus, $\overline{\scr{B}}_x$ is $\comp$-linear. All remaining properties for a brane category can be proved by following the definition of the Cardy fibration $\scr{B}$.
\end{proof}

From theorem \ref{ms_over_point} we can deduce the following

\begin{theorem}\label{theorem2}
Let $a\in \scr{B}(U_\alpha )$. Then, the sheaf $\Gamma_{aa}$ is locally isomorphic to a sum $\bigoplus_i\operatorname{M}_{d(a,i)}(\mathscr{O}_{U_\alpha})$ of matrix algebras.
\end{theorem}
\begin{proof}
Fix $x_0\in U_\alpha$ and let $\{e_1,\dots ,e_n\}$ be a frame of orthogonal, idempotent sections in $\scr{T}(U_\alpha )$. Then, for the category $\overline{\scr{B}}_{x_0}$, we have Moore and Segal's Theorem 2 (\ref{theorem_2}) at our disposal. We have that $O_{\overline{a}\overline{a}}=\bigoplus_i\iota_{\overline{a}}(e_i(x_0))O_{\overline{a}\overline{a}}$; by \ref{theorem_2},
\begin{equation}\label{theorem2_ms_point}
O_{\overline{a}\overline{a}}=\operatorname{Hom}_{\overline{\scr{B}}_{x_0}}(\overline{a},\overline{a})\cong \bigoplus_{i=1}^n\text{M}_{d(x_0,\overline{a},i)}(\comp );
\end{equation}
moreover, the matrix algebra $\opnm{M}_{d(x_0,\overline{a},i)}(\comp )$ corresponds to the summand $\iota_{\overline{a}}(e_i(x_0))O_{\overline{a}\overline{a}}$. On the other hand, we have that, locally around $x_0$, the sheaf $\Gamma_{aa}$ is isomorphic to $\scr{O}^{n_a}_{U_\alpha}$ for some integer $n_a$. But the previous properties together with remark \ref{algebra_fibres} implies that the algebra isomorphism \eqref{theorem2_ms_point} extends to a neighborhood of $x_0$, as we wanted to prove.
\end{proof}

\begin{obs}\label{remark_summands}
From the previous result we can also deduce that the matrix algebra $M_{d(a,i)}(\scr{O}_V)$ corresponds (locally) to the subalgebra $\iota_a(e_i)\Gamma_{aa}$.
\end{obs}

For $a,b\in \scr{B}(U_\alpha )$, and again by the {\sc cy} structure of $\overline{\scr{B}}_x$, we have an isomorphism
$$O_{\overline{a}\overline{b}}=\operatorname{Hom}_{\overline{\scr{B}}_x}(\overline{a},\overline{b})\cong \bigoplus_{i=1}^n\operatorname{Hom}_\comp \left (\comp^{d(\overline{a},i)},\comp^{d(\overline{b},i)}\right ),$$
and thus the following result, which is proved following the same procedure of the previous theorem (note that in this case we have the idempotent morphism $L_i:\Gamma_{ab}\to \Gamma_{ab}$, $L_i(\sigma )=\iota_b(e_i)\sigma$ which, by the centrality condition \eqref{centrality}, coincides with the morphism $\Gamma_{ab}\to \Gamma_{ab}$ given by $\sigma \mapsto \sigma \iota_a(e_i)$). 

\begin{theorem}\label{theorem2bis}
In the situation of theorem \ref{theorem2}, for $a,b\in \mathscr{B}(U_\alpha )$ we have a local isomorphism between $\Gamma_{ab}$ and $\bigoplus_{i=1}^n\operatorname{Hom}_{\mathscr{O}_{U_\alpha}}\left (\mathscr{O}_{U_\alpha}^{d(a,i)},\mathscr{O}_{U_\alpha}^{d(b,i)}\right )$.
\end{theorem}

\begin{obs}\label{remark_summands_2}
Observe that the dimensions $d(a,i)$ in theorem \ref{theorem2bis} are the same as the ones in \ref{theorem2}; this is deduced form the proof of Moore and Segal's theorem 2 in \cite{moore_segal1}. And also in this case, the summand $\operatorname{Hom}_{\mathscr{O}_V}\left (\mathscr{O}_V^{d(a,i)},\mathscr{O}_V^{d(b,i)}\right )$ corresponds to the submodule $\iota_b(e_i)\Gamma_{ab}|_V=\Gamma_{ab}|_V\iota_a(e_i)$.
\end{obs}

From these last results, and following the same procedures done in section \ref{subsec_boundary_semisimple}, we can derive local expressions for the morphisms $\theta_a$, $\iota^a$ and $\pi^a_b$. Let $a,b\in \scr{B}(U_\alpha)$ and let $x\in U_\alpha$. Assume that $U\ni x$ is a neighborhood such that $\Gamma_{aa}|_U$ is isomorphic to a sum $\bigoplus_i\opnm{M}_{d(a,i)}(\scr{O}_U)$ (in that case an element $\sigma \in \Gamma_{aa}|_U$ can be represented as a tuple $(\sigma_i)$, where $\sigma_i\in \opnm{M}_{d(a,i)}(\scr{O}_U)$). If $\{e_1,\dots ,e_n\}$ is a frame of orthogonal, idempotent sections for $\scr{T}_M$ over $U_\alpha$, then we have the following expressions for $\theta_a$, $\iota^a$ and $\pi^a_b$ over $U$:
\begin{equation}\label{local_expressions}
\begin{aligned}
\theta_a (\sigma ) &= \sum_i\sqrt{\theta (e_i)} \opnm{tr}(\sigma_i), \\
\iota^a(\sigma )   &= \sum_i\frac{\opnm{tr}(\sigma_i)}{\sqrt{\theta (e_i)}}e_i, \\
\pi_b^a(\sigma )   &= \sum_i\frac{\opnm{tr}(\sigma_i)}{\sqrt{\theta (e_i)}}\iota_b(e_i). \\
\end{aligned}
\end{equation}

In \cite{moore_segal1}, Moore and Segal also prove that a maximal category of boundary conditions is equivalent to the product $\tsf{Vect}^n$, where $n$ is the dimension of the commutative algebra corresponding to the closed sector, which is assumed to be semisimple (see section \ref{max_cat_moore_segal}). We shall show in the next sections that the localization process described above can be reversed to give an analogous result for our maximal Cardy fibrations.


%%%%%%%%%%%%%%%%%%%%%%%%%%%%%%%%%%%%%%%%%%%%%%%%%%%
\subsection{Properties of Maximal Cardy Fibrations}
\label{maximal_cardy_fibrations}

In the following sections we shall study certain ways of constructing new labels from given ones. By definition of maximality, these new labels should be considered as objects of a maximal category. This constructions shall reveal more structure which any maximal category should enjoy and, in the last section of this chapter, a characterization of maximal fibrations is given, showing that these constructions are also sufficient to construct a maximal category.


%%%%%%%%%%%%%%%%%%%%%%%%%%%%%%%%%%
\subsubsection{Additive Structure}

Let $U\subset M$ be any open subset and $a,b,c\in \scr{B}(U)$; based on properties of modules, we shall define a new label $a\oplus b$; we put
$$
\begin{aligned}
\Gamma_{(a\oplus b)c}:=\Gamma_{ac}\oplus \Gamma_{bc}, \\
\Gamma_{c(a\oplus b)}:=\Gamma_{ca}\oplus \Gamma_{cb}.\\
\end{aligned}
$$
A morphism $a\oplus b\to c$ shall be represented as a row matrix $\left ( \begin{smallmatrix} \sigma & \tau \\ \end{smallmatrix} \right )$, where $\sigma :a\to c$, $\tau :b\to c$. Likewise, an arrow $c\to a\oplus b$ is a column matrix $\left ( \begin{smallmatrix} \sigma \\ \tau \\ \end{smallmatrix} \right )$, for $\sigma :c\to a$, $\tau :c\to b$. Thus, a map $a_1\oplus a_2\to b_1\oplus b_2$ can be represented as a matrix $\left (\begin{smallmatrix} \sigma_{11} & \sigma_{21} \\ \sigma_{12} & \sigma_{22} \\ \end{smallmatrix} \right )$, where $\sigma_{ij}:a_i\to b_j$. Composition of maps is then given by multiplying matrices. As a consequence, we obtain thus a structure of additive category for each $\scr{B}(U)$.

For a new object $a\oplus b$ we define $\theta_{a\oplus b}:\Gamma_{(a\oplus b)(a\oplus b)}\to \scr{O}_U$ by
\begin{equation}\label{linear_form_additive}
\theta_{a\oplus b}\left (\begin{smallmatrix} \sigma_{11} & \sigma_{21} \\ \sigma_{12} & \sigma_{22} \\ \end{smallmatrix} \right )=\theta_{a}(\sigma_{11})+\theta_b(\sigma_{22}).
\end{equation}

Regarding nondegeneracy of the linear forms we have the following

\begin{proposition}\label{nondeg_additive}
The diagram
$$
\xymatrix{
\Gamma_{(a\oplus b)c}\otimes \Gamma_{c(a\oplus b)}\ar[d]\ar[r] & \Gamma_{(a\oplus b)(a\oplus b)}\ar[r]^-{\theta_{a\oplus b}} & \scr{O}_U\ar@{=}[d] \\
\Gamma_{c(a\oplus b)}\otimes \Gamma_{(a\oplus b)c}\ar[r] & \Gamma_{cc} \ar[r]^{\theta_c} & \scr{O}_U}
$$
is commutative, and the top and botton composite bilinear maps are non-degenerate parings (the vertical arrow on the left is the twisting map).
\end{proposition}
\begin{proof}
Let $\tau \in \Gamma_{(a\oplus b)c}$ and $\sigma \in \Gamma_{c(a\oplus b)}$ be given by $\tau =\left (\begin{smallmatrix} \tau_{11} & \tau_{21} \\ \end{smallmatrix} \right )$ and $\sigma =\left (\begin{smallmatrix} \sigma_{11} \\ \sigma_{12} \\ \end{smallmatrix} \right )$. Then, the bottom row is
$$(\sigma ,\tau) \longmapsto \theta_c(\tau_{11}\sigma_{11})+\theta_c(\tau_{21}\sigma_{12}),$$
and hence the commutativity of the diagram follows from the known analogous identities for the pairings involving the labels $a,c$ and $b,c$.

Assume now that $\theta_{a\oplus b}(\sigma \tau )=0$ for each $\tau$; put
$$\sigma =\left (\begin{smallmatrix} \sigma_{11} & \sigma_{21} \\ \sigma_{12} & \sigma_{22} \\ \end{smallmatrix} \right )\quad \text{and} \quad \tau =\left (\begin{smallmatrix} \tau_{11} & \tau_{21} \\ \tau_{12} & \tau_{22} \\ \end{smallmatrix} \right ).$$
Then
$$\theta_{a\oplus b}(\sigma \tau )=\theta_a(\sigma_{11}\tau_{11}+\sigma_{21}\tau_{12})+\theta_b(\sigma_{12}\tau_{21}+\sigma_{22}\tau_{22})=0$$
no matter which maps $\tau_{ij}$ we choose. Taking, for example, $\tau =\left (\begin{smallmatrix} \tau_{11} & 0 \\ 0 & 0 \\ \end{smallmatrix} \right )$ we obtain $\sigma_{11}=0$ by nondegeneracy of the pairing $\Gamma_{ac}\otimes \Gamma_{ca}\to \scr{O}_U$. The rest of the proof can be completed in the same fashion.
\end{proof}

\begin{obs}\label{remark_trace_sum}
Note that the previous proposition readily implies that
$$\theta_{a\oplus b}\left (\begin{smallmatrix} 0 & \sigma_{21} \\ 0 & 0 \\ \end{smallmatrix}\right )=\theta_{a\oplus b}\left (\begin{smallmatrix} 0 & 0 \\ \sigma_{12} & 0 \\ \end{smallmatrix}\right )=0;$$
just consider the equality $\theta_{a\oplus b}(\tau \sigma )=\theta_{a\oplus b}(\sigma \tau )$ and multiply by the matrices $\left (\begin{smallmatrix} 1_a & 0 \\ 0 & 0 \\ \end{smallmatrix}\right )$ and $\left (\begin{smallmatrix} 0 & 0 \\ 0 & 1_b \end{smallmatrix}\right )$.
\end{obs}

For labels $a,b,c\in \scr{B}(U_\alpha )$, note that $\Gamma_{(a\oplus b)c}$ (and also $\Gamma_{c(a\oplus b)}$ by duality) is also locally free.

We now define the transition morphisms $\iota_{a\oplus b}:\scr{T}_{U_\alpha }\to \Gamma_{(a\oplus b)(a\oplus b)}$ and $\iota^{a\oplus b}:\Gamma_{(a\oplus b)(a\oplus b)}\to \scr{T}_{U_\alpha }$ by the equations
\begin{equation}\label{transition_additive}
\begin{aligned}
\iota_{(a\oplus b)}(X) &= \left (\begin{smallmatrix} \iota_a(X) & 0 \\ 0 & \iota_b(X) \\ \end{smallmatrix} \right ), \\
\iota^{(a\oplus b)}\left (\begin{smallmatrix} \sigma_{11} & \sigma_{21} \\ \sigma_{12} & \sigma_{22} \\ \end{smallmatrix} \right ) &=\iota^a(\sigma_{11})+\iota^b(\sigma_{22}).\\
\end{aligned}
\end{equation}
In particular, note that both $\iota_{a\oplus b}$ and $\iota^{a\oplus b}$ are $\scr{O}_{U_\alpha }$-linear, and $\iota_{a\oplus b}$ is an algebra homomorphism which preserves the unit.

The following result shall be useful to prove the Cardy condition.

\begin{lemma}\label{cardy_additive}
For the maps $\pi^{a\oplus b}_c$ and $\pi^a_{b\oplus c}$ the following equalities hold
$$
\begin{aligned}
\pi^{a\oplus b}_c &= \pi^a_c+\pi^b_c \\
\pi^a_{b\oplus c} &= \left (\begin{smallmatrix} \pi^a_b & 0 \\ 0 & \pi^a_c \\ \end{smallmatrix} \right ). \\
\end{aligned}
$$
\end{lemma}
\begin{proof}
First note that if $\overline{\theta}_{c(a\oplus b)}:\Gamma_{c(a\oplus b)}\cong \Gamma_{(a\oplus b)c}^*$ is the isomorphism induced by the pairing between $\Gamma_{c(a\oplus b)}$ and $\Gamma_{(a\oplus b)c}$, then
$$
\begin{aligned}
\overline{\theta}_{c(a\oplus b)} &= \left (\begin{smallmatrix} \overline{\theta}_{ca} & \overline{\theta}_{cb}\end{smallmatrix} \right ), \\
\overline{\theta}_{c(a\oplus b)}^{-1} &= \left (\begin{smallmatrix} \overline{\theta}_{ca}^{-1} \\ \overline{\theta}_{cb}^{-1} \end{smallmatrix} \right ).
\end{aligned}
$$
Take now a local basis for $\Gamma_{(a\oplus b)c}$ of the form $\{\left (\begin{smallmatrix} \tau_i & 0 \\ \end{smallmatrix} \right ), \left (\begin{smallmatrix} 0 & \eta_j \\ \end{smallmatrix} \right )\}$, where $\{\tau_i\}$ is a local basis for $\Gamma_{ac}$ and $\{\eta_j\}$ for $\Gamma_{bc}$. For $\sigma =\left (\begin{smallmatrix} \sigma_{11} & \sigma_{21} \\ \sigma_{12} & \sigma_{22} \\\end{smallmatrix} \right )\in \Gamma_{(a\oplus b)(a\oplus b)}$ we thus have
$$
\begin{aligned}
\pi^{(a\oplus b)}_c(\sigma ) &= \sum_i \left (\begin{smallmatrix} \tau_i & 0 \end{smallmatrix} \right )\left (\begin{smallmatrix} \sigma_{11} & \sigma_{21} \\ \sigma_{12} & \sigma_{22} \\ \end{smallmatrix} \right ) \left (\begin{smallmatrix} \overline{\theta}_{ca}^{-1}(\tau^i) \\ 0 \\ \end{smallmatrix} \right ) + \sum_j \left (\begin{smallmatrix} 0 & \eta_j \end{smallmatrix} \right )\left (\begin{smallmatrix} \sigma_{11} & \sigma_{21} \\ \sigma_{12} & \sigma_{22} \\ \end{smallmatrix} \right ) \left (\begin{smallmatrix} 0 \\ \overline{\theta}_{cb}^{-1}(\eta^j) \\ \end{smallmatrix} \right ) \\
														 &= \sum_i \tau_i\sigma_{11}\overline{\theta}^{-1}_{ca}(\tau^i) + \sum_j\eta_j\sigma_{22}\overline{\theta}^{-1}_{cb}(\eta^j) \\
														 &= \pi^a_c(\sigma_{11})+\pi^b_c(\sigma_{22}). \\
\end{aligned}
$$
The other equality is completely analogous; in this case we have that $\overline{\theta}_{(b\oplus c)a}:\Gamma_{(b\oplus c)a}\to \Gamma_{a(b\oplus c)}^*$ and its inverse are given by
$$
\begin{aligned}
\overline{\theta}_{(b\oplus c)a} &= \left (\begin{smallmatrix} \overline{\theta}_{ba} \\ \overline{\theta}_{ca} \\ \end{smallmatrix} \right ) \\
\overline{\theta}_{(b\oplus c)a}^{-1} &= \left (\begin{smallmatrix} \overline{\theta}_{ba}^{-1} & \overline{\theta}_{ca}^{-1} \\ \end{smallmatrix} \right ). \\
\end{aligned}
$$
If $\left \{\left (\begin{smallmatrix} \tau_i \\ 0 \\ \end{smallmatrix} \right ), \left (\begin{smallmatrix} 0 \\ \eta_j \\ \end{smallmatrix} \right )\right \}$ is a local basis for $\Gamma_{a(b\oplus c)}\cong \Gamma_{ab}\oplus \Gamma_{ac}$, where $\{\tau_i\}$ is a basis for $\Gamma_{ab}$ and $\{\eta_j\}$ for $\Gamma_{ac}$, then
$$
\begin{aligned}
\pi^a_{(b\oplus c)} &= \sum_i\left (\begin{smallmatrix} \tau_i \\ 0 \\ \end{smallmatrix} \right )\sigma \left (\begin{smallmatrix} \overline{\theta}^{-1}_{ba}(\tau^i) & 0 \\ \end{smallmatrix} \right ) + \sum_i\left (\begin{smallmatrix} 0 \\ \eta_j \\ \end{smallmatrix} \right )\sigma \left (\begin{smallmatrix} 0 & \overline{\theta}^{-1}_{ca}(\eta^j) \\ \end{smallmatrix} \right ) \\
									  &= \sum_i \left (\begin{smallmatrix} \tau_i\sigma \overline{\theta}^{-1}_{ba}(\tau^i) & 0 \\ 0 & 0 \\ \end{smallmatrix} \right ) + \sum_j \left (\begin{smallmatrix} 0 & 0 \\ 0 & \eta_j\sigma \overline{\theta}^{-1}_{ca}(\eta^j) \\ \end{smallmatrix} \right ) \\
										&= \left (\begin{smallmatrix} \pi^a_b(\sigma ) & 0 \\ 0 & \pi^ a_c(\sigma ) \\ \end{smallmatrix} \right ). \\
\end{aligned}
$$
\end{proof}



\begin{theorem}\label{maximal_additive}
Given $a,b \in \scr{B}(U_\alpha )$, the maps $\theta_{a\oplus b}$, $\iota_{(a\oplus b)}$ and $\iota^{(a\oplus b)}$ verify the centrality, adjoint and Cardy conditions.
\end{theorem}

\begin{proof}
For the centrality condition, take $\sigma :a\oplus b\to c$, which can be represented by a matrix $\left (\begin{smallmatrix} \sigma_{11} & \sigma_{21} \\ \end{smallmatrix}\right )$. Then
$$
\begin{aligned}
\sigma \iota_{a\oplus b}(X) &= \left (\begin{smallmatrix} \sigma_{11} & \sigma_{21} \\ \end{smallmatrix}\right ) \left (\begin{smallmatrix} \iota_a(X) & 0 \\ 0 & \iota_b(X) \\ \end{smallmatrix}\right )\\
									          &= \left (\begin{smallmatrix} \sigma_{11}\iota_a(X) & \sigma_{21}\iota_b(X) \\ \end{smallmatrix}\right ).
\end{aligned}
$$
The equality $\sigma \iota_{a\oplus b}(X)=\iota_c(X)\sigma$ now follows from the centrality condition for the morphisms $\iota_a,\iota_c$ and $\iota_b,\iota_c$.

We now verify the adjoint relation $\theta_{a\oplus b}\left (\sigma \iota_{a\oplus b}(X)\right )=\theta \left (\iota^{a\oplus b}(\sigma )X\right )$; so let $\sigma :a\oplus b\to a\oplus b$ be given by $(\sigma_{ij})^t$. Then the adjoint relation between $\iota_a,\iota^a$ and the one between $\iota_b\iota^b$ let us write
$$
\begin{aligned}
\theta_{a\oplus b}\left (\sigma \iota_{a\oplus b}(X)\right ) &= \theta_{a\oplus b}\left (\begin{smallmatrix} \sigma_{11}\iota_a(X) & \sigma_{21}\iota_b(X) \\ \sigma_{12}\iota_a(X) & \sigma_{22}\iota_b(X) \\ \end{smallmatrix} \right ) \\
																											       &= \theta_a\left (\sigma_{11}\iota_a(X)\right )+\theta_b\left (\sigma_{22}\iota_b(X)\right ) \\
																														 &= \theta \left (\iota^a(\sigma_{11})X\right )+\theta \left (\iota^b(\sigma_{22})X\right ) \\
																														 &= \theta \left (\left (\iota^a(\sigma_{11})+\iota^b(\sigma_{22})\right )X\right ) \\
																														 &= \theta \left (\iota^{a\oplus b}(\sigma )X\right ), \\
\end{aligned}
$$
as desired.

For the Cardy condition, we now check that $\pi^{a\oplus b}_{c\oplus d}=\iota_{c\oplus d}\iota^{a\oplus b}$. The right hand side is
$$
\begin{aligned}
\iota_{c\oplus d}\iota^{a\oplus b}\left (\begin{smallmatrix} \sigma_{11} & \sigma_{21} \\ \sigma_{12} & \sigma_{22} \\ \end{smallmatrix} \right ) &= \iota_{c\oplus d}\left (\iota^a(\sigma_{11}) + \iota^b(\sigma_{22})\right ) \\
				  &= \left (\begin{smallmatrix} \iota_c\left (\iota^a(\sigma_{11})+\iota^b(\sigma_{22})\right ) & 0 \\ 0 & \iota_d\left (\iota^a(\sigma_{11})+\iota^b(\sigma_{22})\right ) \\ \end{smallmatrix} \right ) \\
					&= \left (\begin{smallmatrix} \pi^a_c(\sigma_{11})+\pi^b_c(\sigma_{22}) & 0 \\ 0 & \pi^a_d(\sigma_{11})+\pi^b_d(\sigma_{22}) \\ \end{smallmatrix} \right ), \\
\end{aligned}
$$
where in the last equality we used the Cardy condition. The rest now follows from lemma \ref{cardy_additive}.
\end{proof}

\begin{cor}
Any maximal Cardy fibration is additive.
\end{cor}


%%%%%%%%%%%%%%%%%%%%%%%%%%%%%%%%%%%%%%%%%%%%%%%%%%%%%%%%%%%%%%%
\subsubsection{The Action of the Category of Locally Free Modules}

In this section we shall prove that another enlargement of the category $\scr{B}$ can be made, by considering a label of the form $\scr{M}\otimes a$, where $\scr{M}$ is a locally free $\scr{O}_U$-module and $a\in \scr{B}(U)$. A consequence of this construction is that every maximal fibration enjoys, besides an additive structure, an action of the (fibred) category of locally free modules, which is compatible with the additive structure.

So let the locally free $\scr{O}_U$-module $\mathscr{M}$ be given, as well as a brane $a\in \scr{B}(U)$ over $U$. The new product brane $\scr{M}\otimes a$ is defined by
\begin{equation}\label{product_brane}
\begin{aligned}
\Gamma_{(\scr{M}\otimes a)b} &= \scr{M}^*\otimes \Gamma_{ab}, \\
\Gamma_{b(\scr{M}\otimes a)} &= \scr{M} \otimes \Gamma_{ba}, \\
\end{aligned}
\end{equation}
where the tensor product is taken over $\scr{O}_U$. In particular, we also have that
$$\Gamma_{(\scr{M}\otimes a)(\scr{N}\otimes b)}=\underline{\opnm{Hom}}(\scr{M},\scr{N})\otimes \Gamma_{ab},$$
by the canonical identification between $\scr{M}^*\otimes \scr{N}$ and $\underline{\opnm{Hom}}(\scr{M},\scr{N})$ (so an object of the form $\varphi \otimes x$ shall be regarded as a homomorphism $\scr{M}\to \scr{N}$). Note that this definition let us also define a restriction $(\scr{M}\otimes a)|_V:=\scr{M}|_V\otimes a|_V$. Moreover, if we work on a semisimple subset $U_\alpha \in \mathfrak{U}$, then $\Gamma_{(\scr{M}\otimes a)b}$ and $\Gamma_{b(\scr{M}\otimes a)}$ are locally free.

The composition pairing
\begin{equation}\label{pairing_prod}
\Gamma_{(\scr{M}\otimes a)(\scr{N}\otimes b)}\otimes \Gamma_{(\scr{N}\otimes b)(\scr{P}\otimes c)}\longrightarrow \Gamma_{(\scr{M}\otimes a)(\scr{P}\otimes c)}
\end{equation}
can be also written as
$$\scr{M}^*\otimes \scr{N} \otimes \scr{N}^* \otimes \scr{P}\otimes \Gamma_{ab}\otimes \Gamma_{bc}\longrightarrow \scr{M}^*\otimes \scr{P}\otimes \Gamma_{ac};$$
hence, the map \eqref{pairing_prod} is built from two composition pairings, the one corresponding to composition of module homomorphisms, namely $\scr{M}^*\otimes \scr{N}\otimes \scr{N}^*\otimes \scr{P}\to \scr{M}^*\otimes \scr{P}$, and the one corresponding to composition of maps of branes, $\Gamma_{ab}\otimes \Gamma_{bc}\to \Gamma_{ac}$.

\begin{lemma}
We have a duality isomorphism $\Gamma_{(\scr{M}\otimes a)b}\cong \Gamma_{b\otimes (\scr{M}\otimes a)}^*$.
\end{lemma}
\begin{proof}
This follows by definition of $\Gamma_{(\scr{M}\otimes a)b}$, from the duality between $\Gamma_{ab}$ and $\Gamma_{ba}$ and from corollary \ref{tensor_hom_dual}.
\end{proof}

\begin{proposition}
The correspondence $(\scr{M},a)\mapsto \scr{M}\otimes a$ defines an action
$$\tsf{LF}_{\scr{O}_U}\times \scr{B}(U)\longrightarrow \scr{B}(U)$$
which is compatible with the additive structure.
\end{proposition}
\begin{proof}
This is mainly a consequence of properties of the tensor product for modules. As we have defined product branes in terms of their morphisms, we should check any statement involving products by considering maps: if $a,b$ are fixed branes such that the modules $\Gamma_{ac}$ and $\Gamma_{bc}$ are isomorphic for each $c$, then necessarily $a\cong b$.

We first check that $\scr{M}\otimes (\scr{N}\otimes a)\cong (\scr{M}\otimes \scr{N})\otimes a$ by studying morphisms to an arbitrary object $b$. We have
$$
\begin{aligned}
\Gamma_{(\scr{M}\otimes (\scr{N}\otimes a))b} &= \scr{M}^*\otimes \Gamma_{(\scr{N}\otimes a)b} \\
																							&\cong \scr{M}^*\otimes \left (\scr{N}^*\otimes \Gamma_{ab}\right ) \\
																							&\cong \left (\scr{M}^*\otimes \scr{N}^*\right )\otimes \Gamma_{ab} \\
																							&\cong (\scr{M}\otimes \scr{N})^*\otimes \Gamma_{ab} \\
																							&= \Gamma_{((\scr{M}\otimes \scr{N})\otimes a)b}. \\
\end{aligned}
$$
In a similar fashion we now check that $\scr{M}(a\oplus b)\cong (\scr{M}\otimes a)\oplus (\scr{M}\otimes b)$:
$$
\begin{aligned}
\Gamma_{(\scr{M}\otimes (a\oplus b))c} &= \scr{M}^*\otimes \Gamma_{(a\oplus b)c} \\
																			 &\cong (\scr{M}^*\otimes \Gamma_{ac})\oplus (\scr{M}^*\otimes \Gamma_{bc}) \\
																			 &\cong \Gamma_{(\scr{M}\otimes a)c}\oplus \Gamma_{(\scr{M}\otimes b)c} \\
																			 &= \Gamma_{((\scr{M}\otimes a)\oplus (\scr{M}\otimes b))c}.\\
\end{aligned}
$$
The isomorphisms $\scr{O}\otimes a\cong a$ and $\scr{M}\otimes 0\cong 0$ (where $0$ is the zero object of the additive category $\scr{B}(U)$) are proved in the same way.
\end{proof}

Let now $\overline{a}=\scr{M}\otimes a$. Then, $\Gamma_{\overline{a}\overline{a}}=\underline{\operatorname{End}}_{\scr{O}_U}(\scr{M})\otimes \Gamma_{aa}$, and we define the trace $\theta_{\overline{a}}:\Gamma_{\overline{a}\overline{a}}\to \scr{O}_U$ as the following composite map
$$
\xymatrix{
\underline{\operatorname{End}}_{\scr{O}_U}(\scr{M})\otimes \Gamma_{aa} \ar[rr]^{\operatorname{tr}\otimes \operatorname{id}} & & \scr{O}_U\otimes \Gamma_{aa}\cong \Gamma_{aa} \ar[r]^-{\theta_a} & \scr{O}_U;}
$$
equivalently, $\theta_{\overline{a}}(f\otimes \sigma )=\operatorname{tr}(f)\theta_a(\sigma )$.

Before proving the relevant results, let us recall some basic notions about traces. Assume that $f:\scr{M}\to \scr{M}$ is an endomorphism of the locally free $\scr{O}_M$-module $\scr{M}$. Let $U\subset M$ be an open subset such that $\scr{M}|_U\cong \scr{O}_U^n$ and $B_U=\{e_1,\dots ,e_n\}$ a local basis. In the same fashion as for vector spaces, we can define the matrix $M_B(f)$ of $f$ in $B$, and then its trace
$$\opnm{tr}\left (M_{B_U}(f)\right )\in \scr{O}(U).$$
If $B'_U$ is another basis, then the change-of-basis formula $M_{B'_U}(f)=C_{B_UB'_U}M_{B_U}(f)C_{B_UB'_U}^{-1}$ holds also in this case, and
$$\opnm{tr}\left (M_{B_U}(f)\right )=\opnm{tr}\left (M_{B'_U}(f)\right ).$$
If $V\subset M$ is an open subset where $\scr{M}|_V\cong \scr{O}_V^n$ and $U\cap V\neq \emptyset$, then the previous formula implies that $\opnm{tr}\left (M_{B_V}(f)\right )=\opnm{tr}\left (M_{B_U}(f)\right )$ over $U\cap V$. Thus, if $M$ is connected, the \emph{trace} $\opnm{tr}(f)$ is well-defined globally on $M$.

Regarding maps $\scr{M}\to \scr{M}$ as objects of the tensor product $\scr{M}^*\otimes \scr{M}$, the trace is described as follows: as in the previous paragraph, let $B_U=\{e_1,\dots ,e_n\}$ be a local basis for $\scr{M}$ and let $B_U^*=\{e^1,\dots ,e^n\}$ be its dual basis. If $\varphi \otimes u $ is a section of $\scr{M}^*\otimes \scr{M}$ over $U$, then we can write
$$\varphi \otimes u =\left (\sum_i \alpha_ie^i\right )\otimes \left (\sum_j\beta_je_j\right )=\sum_{i,j}\alpha_i\beta_j(e^i\otimes e_j).$$
The endomorphism $e^i\otimes e_j$ is defined by the relations
$$\left (e^i\otimes e_j\right )(e_k)=e^i(e_k)e_j=\delta_{ik}e_j$$ 
and thus its trace is $\opnm{tr}(e^i\otimes e_j)=\delta_{ij}$. We can then conclude that
$$\opnm{tr}(\varphi \otimes u)=\sum_i\alpha_i\beta_i.$$

\begin{proposition}
The diagram
$$
\xymatrix{
\Gamma_{\overline{a}b}\otimes \Gamma_{b\overline{a}}\ar[d]\ar[r] & \Gamma_{\overline{a}\overline{a}}\ar[r]^-{\theta_{\overline{a}}} & \scr{O}_U\ar@{=}[d] \\
\Gamma_{b\overline{a}}\otimes \Gamma_{\overline{a}b}\ar[r] & \Gamma_{bb} \ar[r]^{\theta_b} & \scr{O}_U}
$$
is commutative, and the top and botton composite bilinear maps are non-degenerate parings (the vertical arrow on the left is the twisting map).
\end{proposition}
\begin{proof}
Commutativity of the diagram follows from the definition of the maps involved and from the equality $\theta_a(\tau \sigma )=\theta_b(\sigma \tau )$. To prove the nondegeneracy we assume that
\begin{equation}\label{trace_nondeg_tensor}
\opnm{tr}(\varphi \otimes u)\theta_a(\tau \sigma )=0
\end{equation}
for each $u\otimes \tau \in \Gamma_{b(\scr{M}\otimes a)}$; we then need to prove that $\varphi \otimes \sigma =0$; we can work on the stalk over some $x\in U$, as the maps $\varphi$ and $\sigma$ are fixed. Equation \eqref{trace_nondeg_tensor} implies that $\opnm{tr}(\varphi_x\otimes u_x)\theta_{a,x}(\tau_x\sigma_x)=0$ in $\scr{O}_x$. Pick a local basis $\{e_i\}$ for $\scr{M}$ around $x$ and let $\{e^i\}$ be its dual basis. Write $\varphi =\sum_i\alpha_ie^i$. We now assume that $\alpha_i(x)\neq 0$, and hence also its germ $\alpha_{i,x}$. Define $u_x=\frac{e_i}{\alpha_i}$. Therefore, $\varphi_x\otimes u_x=e^i_x\otimes e_{i,x}$ and $\opnm{tr}(\varphi_x\otimes u_x)=1$. This implies, by nondegeneracy of the pairing $\Gamma_{ab}\otimes \Gamma_{ba}\to \scr{O}_U$, that $\sigma_x=0$ and hence $\sigma =0$. 
\end{proof}

We now work on a semisimple subset $U_\alpha \in \mathfrak{U}$; the transition map $\iota_{\overline{a}}:\scr{T}_{U_\alpha}\to \Gamma_{\overline{a}\overline{a}}$ is defined by the equation $\iota_{\overline{a}}(X) = \text{id}_{\scr{M}}\otimes \iota_a (X)$ and $\iota^{\overline{a}}:\Gamma_{\overline{a}\overline{a}}\to \scr{T}_{U_\alpha}$ by the following chain of morphisms
$$
\xymatrix{
\underline{\operatorname{End}}_{\scr{O}_{U_\alpha}}(\scr{M})\otimes \Gamma_{aa} \ar[rr]^{\operatorname{tr}\otimes \operatorname{id}} & & \scr{O}_{U_\alpha}\otimes \Gamma_{aa}\cong \Gamma_{aa} \ar[r]^-{\iota^a} & \scr{T}_{U_\alpha};}
$$
i.e. $\iota^{\overline{a}}(f\otimes \sigma )=\operatorname{tr}(f)\iota^a(\sigma )$.

Let $\scr{M}$ and $\scr{N}$ be two locally free $\scr{O}_M$-modules and assume that $U\subset M$ is an open subset over which $\scr{M}$ and $\scr{N}$ are isomorphic to $\scr{O}^n_U$ and $\scr{O}^k_U$ respectively. Then the modules $\underline{\opnm{Hom}}_{\scr{O}_M}(\scr{M},\scr{N})$ and $\underline{\opnm{Hom}}_{\scr{O}_M}(\scr{M},\scr{N})^*$ are also trivial over $U$, the first one isomorphic to $\scr{O}^{nk}_U$, whose objects can be regarded as $k\times n$ matrices with coefficients in $\scr{O}_U$. Fix a basis $\{f_{ij}\}$ for $\underline{\opnm{Hom}}_{\scr{O}_M}(\scr{M}.\scr{N})$ over $U$ and let $\{f^{ij}\}$ be its dual basis. Define the linear map $\pi_U:\underline{\opnm{End}}_{\scr{O}_U}(\scr{M}|_U)\to \underline{\opnm{End}}_{\scr{O}_U}(\scr{N}|_U)$ by the equation
$$\pi_U(f)=\sum_{i,j}f_{ij}f\overline{\theta}^{-1}(f^{ij}),$$
where $\overline{\theta}:\underline{\opnm{Hom}}_{\scr{O}_U}(\scr{N}|_U,\scr{M}|_U)\to \underline{\opnm{Hom}}_{\scr{O}_U}(\scr{M}|_U,\scr{N}|_U)^*$ is the isomorphism given by $\overline{\theta}(f)(g)=\opnm{tr}(gf)$. A straightforward adaptation to proposition \ref{cardy_well_def} shows that this morphism $\pi_U$ does not depend on the choice of basis $\{f_{ij}\}$ and then the same conclusion as for the maps $\pi^a_b$ applies here: we have a globally defined linear map
$$\pi^{\scr{M}}_{\scr{N}}:\underline{\opnm{End}}_{\scr{O}_M}(\scr{M})\longrightarrow \underline{\opnm{End}}_{\scr{O}_M}(\scr{N}).$$

\begin{lemma}
If $f:\scr{M}\to \scr{M}$ is a linear endomorphism, then
$$\pi^{\scr{M}}_{\scr{N}}(f)=\opnm{tr}(f)\opnm{id}_{\scr{N}}.$$
\end{lemma}
\begin{proof}
It only suffices to consider $\scr{M}=\scr{O}^n_M$ and $\scr{N}=\scr{O}^k_M$. Before proving the result, let us fix some notation:
\begin{itemize}
\item We will supress the subscript $M$ in $\scr{O}_M$ and denote $\pi^{\scr{M}}_{\scr{N}}$ by $\pi^n_k$.
\item The basis $\{f_{ij}\}$ of $\underline{\opnm{Hom}}(\scr{O}^n,\scr{O}^k)$ will consist of elementary matrices. And then $\{f_{ij}^t\}$ is the basis of elementary matrices for $\underline{\opnm{Hom}}(\scr{O}^k,\scr{O}^n)$.
\item $\{e_{rl}\}$ will be also the canonical basis but for $\underline{\opnm{End}}(\scr{O}^n)$ and $\{e'_{st}\}$ for $\underline{\opnm{End}}(\scr{O}^k)$.
\end{itemize}
The previous choices, which are made just for simplicity, are justified by proposition \ref{cardy_well_def}.

We have the trace map $\overline{\theta}:\underline{\opnm{Hom}}(\scr{O}^k,\scr{O}^n)\to \underline{\opnm{Hom}}(\scr{O}^n,\scr{O}^k)^*$; assume now that $\overline{\theta}^{-1}(f^{ij})=\sum_{a,b} \lambda^{(ij)}_{ab}f^t_{ab}$. Applying $\overline{\theta}$ at both sides, we have $f^{ij}=\sum_{a,b} \lambda^{(ij)}_{ab}\overline{\theta}(f^t_{ab})$; evaluating this expression in $f_{cd}$ we obtain
$$
\begin{aligned}
\delta^{ic}_{jd} =f^{ij}(f_{cd}) &= \sum_{a,b}\lambda^{(ij)}_{ab}\opnm{tr}(f_{cd}f^t_{ab}) \\
                                 &= \sum_{a,b}\lambda^{(ij)}_{db}\opnm{tr}(e'_{cb}) \\
                                 &= \lambda^{(ij)}_{dc}, \\
\end{aligned}
$$
and thus $\overline{\theta}^{-1}(f^{ij})=\lambda^{(ij)}_{ji}f^t_{ji}=f^t_{ji}$. We now compute
$$
\begin{aligned}
\pi^n_k(e_{rl}) &= \sum_{i,j}f_{ij}e_{rl}f_{ij}^t \\
                &= \sum_if_{il}f^t_{ir} \\
                &= \delta_{rl}\sum_ie'_{ii},
\end{aligned}
$$
as desired.
\end{proof}

\begin{theorem}\label{action_cy}
With the previous definitions, the action $\tsf{LF}_{\scr{O}_U}\times_U\scr{B}|_U\to \scr{B}|_U$ is compatible with all the structures in a Cardy fibration. 
\end{theorem}
\begin{proof}
We work on a semisimple subset $U_\alpha$, and we need to verify the centrality condition, the adjoint relation and the Cardy condition. Let us fix a notation for this proof: given locally free modules $\scr{M},\scr{N}$ over $U_\alpha$ and labels $a,b\in \scr{B}(U_\alpha )$ we define $\overline{a}:=\scr{M}\otimes a$ and $\overline{b}:=\scr{N}\otimes b$.

For the centrality condition we need to check that $\iota_{\overline{b}}(X)(f\otimes \sigma )=(f\otimes \sigma )\iota_{\overline{a}}(X)$ for $f:\scr{M}\to \scr{N}$ and $\sigma :a\to b$. Then
$$
\begin{aligned}
\iota_{\overline{b}}(X)(f\otimes \sigma ) &= f\otimes \iota_b(X)\sigma \\
                                          &= f\otimes \sigma \iota_a(X) \\
                                          &= (f\otimes \sigma )\iota_{\overline{a}}(X), \\
\end{aligned}
$$
where in the second step we used the centrality condition for $\iota_a$ and $\iota_b$.

For the adjoint relation, we have
$$
\begin{aligned}
\theta (\iota^{\overline{a}}(f\otimes \sigma )X) &= \theta (\operatorname{tr}(f)\iota^a(\sigma )X) \\
&= \operatorname{tr}(f)\theta (\iota^a(\sigma )X) \\
&= \operatorname{tr}(f)\theta_a(\sigma \iota_a(X)) \\
&= \theta_{\overline{a}}(f\otimes \sigma \iota_a(X))\\
&= \theta_{\overline{a}}((f\otimes \sigma )\iota_{\overline{a}}(X)),\\
\end{aligned}
$$
where in the third step we used the adjoint relation for $\iota_a$ and $\iota^a$.

For the Cardy condition, we must check that $\pi^{\overline{a}}_{\overline{b}}:\Gamma_{\overline{a}\overline{a}}\to \Gamma_{\overline{b}\overline{b}}$ verifies $\pi^{\overline{a}}_{\overline{b}}=\iota_{\overline{b}}\iota^{\overline{a}}$. The right hand side is
$$
\begin{aligned}
\iota_{\overline{b}}\iota^{\overline{a}}(f\otimes \sigma ) &= \iota_{\overline{b}}(\operatorname{tr}(f)\iota^a(\sigma )) \\
&= \text{id}_{\scr{N}}\otimes (\operatorname{tr}(f)\iota_b(\iota^a(\sigma ))) \\
&= \operatorname{tr}(f) \; \text{id}_{\scr{N}}\otimes \pi^a_b(\sigma ). \\
\end{aligned}
$$
For the left hand side, let $\{e_{ij}\}$ be a local basis for $\underline{\operatorname{Hom}}_{\scr{O}_U}(\scr{M},\scr{N})$ and let $\{e^{ij}\}$ be the local basis of $\underline{\operatorname{Hom}}_{\scr{O}_U}(\scr{N},\scr{M})\cong \underline{\operatorname{Hom}}_{\scr{O}_U}(\scr{M},\scr{N})^*$ dual to $\{e_{ij}\}$. Then, if $\{\sigma_k\}$ is a local basis for $\Gamma_{ab}$, we have that $\{e_{ij}\otimes \sigma_k\}$ is a local basis for $\Gamma_{\overline{a}\overline{b}}$ and $\{e^{ij}\otimes \sigma^k\}$ its dual. Thus
$$
\begin{aligned}
\pi^{\overline{a}}_{\overline{b}}(f\otimes \sigma ) &= \sum_{i,j,k}(e_{ij}\otimes \sigma_k)(f\otimes \sigma )(e^{ij}\otimes \sigma^k) \\
&= \Bigl (\sum_{i,j}e_{ij}fe^{ij}\Bigr )\otimes \Bigl (\sum_k\sigma_k\sigma \sigma^k \Bigr ) \\
&= \pi^{\scr{M}}_{\scr{N}}(f)\otimes \pi^a_b(\sigma ), \\
\end{aligned}
$$
and the Cardy condition then follows from the previous lemma.
\end{proof}

We thus obtain the following

\begin{cor}
Any maximal {\sc cy} category $\scr{B}$ over $M$ comes equipped with a linear action $\tsf{LF}_{\scr{O}_M}\times \scr{B}\to \scr{B}$.
\end{cor}  



%%%%%%%%%%%%%%%%%%%%%%%%%%%%%%%%%%%%%
\subsubsection{Pseudo-Abelian Structure}

We shall now show that besides the additive structure and the action of the category of locally free sheaves, any maximal Cardy fibration should be pseudo-abelian (for generalities on pseudo-abelian categories see section \ref{sheaf_boundary_conditions}). That is to say, given $a\in \scr{B}(U)$ and an arrow $\sigma_0 :a\to a$ such that $\sigma_0^2=\sigma_0$, we shall assume that there exists branes $K_0:=\opnm{Ker}\sigma_0$ and $I_0:=\opnm{Im}\sigma_0$ (which can also be taken as $\opnm{Ker}(1_a-\sigma_0 )$) such that
\begin{itemize}
\item The brane $a$ decomposes as a sum $a\cong K_0 \oplus I_0$ and
\item using matrix notation, the map $\sigma_0$ is given by $\left (\begin{smallmatrix} 0 & 0 \\ 0 & 1_a \\ \end{smallmatrix} \right )$.
\end{itemize}
 As was done for the additive structure and the action of the category of locally free modules, the enlargement of the category of branes by adding kernels should be done by defining all the structure maps for this new object $K_0$, namely $\theta_{K_0}$, $\iota_{K_0}$, $\iota^{K_0}$, along with the verification of their properties. In particular, it should be noted that this definitions should agree with the additive structure.

First note that an arrow $K_0\to K_0$ is a composite of the form
$$K_0\stackrel{i_1}{\longrightarrow}K_0\oplus I_0\stackrel{\sigma}{\longrightarrow}K_0\oplus I_0\stackrel{\pr_1}{\longrightarrow}K_0$$
for some arrow $\sigma :a\to a$, and hence $\Gamma_{K_0K_0}\subset \Gamma_{aa}$ is a submodule. In fact, we have that
$$\Gamma_{aa}=\Gamma_{K_0K_0}\oplus \Gamma_{K_0I_0}\oplus \Gamma_{I_0K_0}\oplus \Gamma_{I_0I_0}.$$
For $a\in \scr{B}(U_\alpha )$, consider the homomorphism $\rho :\Gamma_{aa}\to \Gamma_{aa}$ given by
$$\rho \left (\begin{smallmatrix} \sigma_{11} & \sigma_{21} \\ \sigma_{12} & \sigma_{22} \end{smallmatrix} \right )=\left (\begin{smallmatrix} 0 & \sigma_{21} \\ \sigma_{12} & \sigma_{22} \end{smallmatrix} \right ).$$
Then $\rho$ is clearly a projection with kernel $\Gamma_{K_0K_0}$ which is then locally-free. A similar argument can be used to prove that for any label $b\in \scr{B}(U_\alpha )$, $\Gamma_{K_0b}$ is also locally free; consider $\Gamma_{ab}=\Gamma_{K_0b}\oplus \Gamma_{I_0b}$ and the map $\eta :\Gamma_{ab}\to \Gamma_{ab}$ which projects to $\Gamma_{I_0b}$. Proposition \ref{non_deg_pseudoab} shows that also $\Gamma_{bK_0}\cong \Gamma_{K_0b}^*$ is locally free.

We now turn to the structure maps. If $a\cong K_0\oplus I_0$, the fact that
$$\theta_a\left (\begin{smallmatrix} 0 & \sigma_{21} \\ 0 & 0 \end{smallmatrix} \right )=\theta_a\left (\begin{smallmatrix} 0 & 0 \\ \sigma_{12} & 0 \end{smallmatrix} \right )=0$$
(see remark \ref{remark_trace_sum}) suggests the definition of the linear form $\theta_{K_0}:\Gamma_{K_0K_0}\to \scr{O}_U$ by
$$\theta_{K_0}(\sigma )=\theta_a\left (\begin{smallmatrix} \sigma & 0 \\ 0 & 0 \end{smallmatrix} \right ).$$

\begin{proposition}\label{non_deg_pseudoab}
The diagram
$$
\xymatrix{
\Gamma_{K_0b}\otimes \Gamma_{bK_0}\ar[d]\ar[r] & \Gamma_{K_0K_0}\ar[r]^-{\theta_{K_0}} & \scr{O}_U\ar@{=}[d] \\
\Gamma_{bK_0}\otimes \Gamma_{K_0b}\ar[r] & \Gamma_{bb} \ar[r]^{\theta_b} & \scr{O}_U}
$$
is commutative, and the top and botton composite bilinear maps are non-degenerate parings (the vertical arrow on the left is the twisting map).
\end{proposition}
\begin{proof}
Let $\sigma \in \Gamma_{K_0b}$ and $\tau \in \Gamma_{bK_0}$; as morphisms $a\to b$ and $b\to a$ respectively, these maps can be written as matrices $\left (\begin{smallmatrix} \sigma & 0 \\ \end{smallmatrix}\right )$ and  $\left (\begin{smallmatrix} \tau \\ 0 \\ \end{smallmatrix}\right )$, respectively. The top arrow is then given by the correspondence
$$\sigma \otimes \tau \longmapsto \theta_a \left (\begin{smallmatrix} \sigma \tau & 0 \\ 0 & 0 \\ \end{smallmatrix}\right )=\theta_a\left ( \left (\begin{smallmatrix} \tau \\ 0 \\ \end{smallmatrix}\right )\left (\begin{smallmatrix} \sigma & 0 \\ \end{smallmatrix}\right )\right ),$$
which is equal to $\theta_b\left ( \left (\begin{smallmatrix} \sigma & 0 \\ \end{smallmatrix}\right )\left (\begin{smallmatrix} \tau \\ 0 \\ \end{smallmatrix}\right )\right )$.

Assume now that $\theta_{K_0}(\tau \sigma )=0$ for each map $\tau :b\to K_0$; this is equivalent to the statement that $\theta_a\left ( \left (\begin{smallmatrix} \tau \\ 0 \\ \end{smallmatrix}\right )\left (\begin{smallmatrix} \sigma & 0 \\ \end{smallmatrix}\right )\right )=0$ for each $\tau$. Write a map $\tau':b\to a\cong K_0\oplus I_0$ as $\left (\begin{smallmatrix} \tau_{11} \\ \tau_{12} \\ \end{smallmatrix}\right )$. Then
$$
\begin{aligned}
\theta_a\left ( \left (\begin{smallmatrix} \tau_{11} \\ \tau_{12} \\ \end{smallmatrix}\right )\left (\begin{smallmatrix} \sigma & 0 \\ \end{smallmatrix}\right )\right ) &= \theta_a \left (\begin{smallmatrix} \tau_{11}\sigma & 0 \\ \tau_{12}\sigma & 0 \\ \end{smallmatrix}\right ) \\
                                  &= \theta_a \left (\begin{smallmatrix} \tau_{11}\sigma & 0 \\ 0 & 0 \\ \end{smallmatrix}\right )+\theta_a \left (\begin{smallmatrix} 0 & 0 \\ \tau_{12}\sigma & 0 \\ \end{smallmatrix}\right ) = \theta_a \left (\begin{smallmatrix} \tau_{11}\sigma & 0 \\ 0 & 0 \\ \end{smallmatrix}\right ). \\
\end{aligned}
$$
This implies that $\theta_a\left (\tau'\left (\begin{smallmatrix} \sigma & 0 \\ \end{smallmatrix}\right ) \right )=0$ for each map $\tau'$ and hence $\sigma=0$, as desired.
\end{proof}

As was done with $\theta_a$, we shall now relate the expression of $\iota_a$ with the decomposition $a\cong K_0\oplus I_0$. So assume that for a vector field $X$ over $U_\alpha$,
$$\iota_a(X)=\left (\begin{smallmatrix} \varphi_{11} & \varphi_{21} \\ \varphi_{12} & \varphi_{22} \\ \end{smallmatrix} \right ).$$

\begin{lemma}
We have $\varphi_{12}=\varphi_{21}=0$.
\end{lemma}
\begin{proof}
The result follows from the centrality condition $\iota_a(X)\sigma =\sigma \iota_a(X)$, taking $\sigma = \left (\begin{smallmatrix} \sigma_{11} & 0 \\ 0 & 0 \\ \end{smallmatrix} \right )$.
\end{proof}

We then define $\iota_{K_0}:\scr{T}_U\to \Gamma_{K_0K_0}$, $\iota^{K_0}:\Gamma_{K_0K_0}\to \scr{T}_U$ by
$$
\begin{aligned}
\iota_{K_0}(X) &= \varphi_{11} \\
\iota^{K_0}(\sigma ) &= \iota^a \left (\begin{smallmatrix} \sigma & 0 \\ 0 & 0 \\ \end{smallmatrix} \right ).\\
\end{aligned}
$$
The previous lemma and the additive structure motivate the definition of $\iota_{K_0}$ while the adjoint relation, and also the additive structure, motivate that of $\iota^{K_0}$.

\begin{theorem}
The maps $\theta_{K_0}$, $\iota_{K_0}$ and $\iota^{K_0}$ satisfy the centrality, adjoint and Cardy conditions.
\end{theorem}
\begin{proof}
For the centrality condition, let $\sigma: b\to K_0$ and assume, for another idempotent $\sigma_0':b\to b$, that $b\cong K_0'\oplus I_0'$, where $K_0'$ and $I_0'$ are the kernel and image of $\sigma_0'$, respectively. Assume also that
$$\iota_b(X)=\left (\begin{smallmatrix} \varphi_{11}' & 0 \\ 0 & \varphi_{22}' \end{smallmatrix} \right ),$$
and put $\sigma':=i_1\sigma :b\to a$. If $\sigma$ is represented by the matrix $\sigma =\left (\begin{smallmatrix} \sigma_{11} & \sigma_{21} \end{smallmatrix} \right )$, then $\sigma'=\left (\begin{smallmatrix} \sigma_{11} & \sigma_{21} \\ 0 & 0 \end{smallmatrix} \right )$. The centrality condition tells us that $\iota_a(X)\sigma'=\sigma'\iota_b(X)$. Expanding this equality in matrix terms we obtain
\begin{equation}\label{cent_matrix}
\left (\begin{smallmatrix} \varphi_{11}\sigma_{11} & \varphi_{11}\sigma_{21} \\ 0 & 0 \end{smallmatrix} \right ) = \left (\begin{smallmatrix} \sigma_{11}\varphi_{11}' & \sigma_{21}\varphi_{22}' \\ 0 & 0 \end{smallmatrix} \right ).
\end{equation}
 The centrality condition $\iota_{K_0}(X)\sigma =\sigma \iota_b(X)$ follows by noting that $\iota_{K_0}(X)\sigma$ is precisely the first row of the matrix in the left hand side of equation \eqref{cent_matrix} and $\sigma \iota_b(X)$ the first row of the right hand side.

We now need to check the adjoint relation $\theta_{K_0}\left( \sigma \iota_{K_0}(X)\right )=\theta \left (\iota^{K_0}(\sigma )X\right )$ foe each vector field $X$ and $\sigma :K_0\to K_0$. Assume that $\iota_a(X)=\left (\begin{smallmatrix} \varphi_{11} & 0 \\ 0 & \varphi_{22} \\ \end{smallmatrix} \right )$. Then
$$
\begin{aligned}
\theta_{K_0}\left( \sigma \iota_{K_0}(X)\right ) &= \theta_{K_0}(\sigma \varphi_{11}) \\
																								 &= \theta_a \left (\begin{smallmatrix} \sigma \varphi_{11} & 0 \\ 0 & 0 \\ \end{smallmatrix} \right ) \\
																								 &= \theta_a  \left ( \left (\begin{smallmatrix} \sigma & 0 \\ 0 & 0 \\ \end{smallmatrix} \right )\left (\begin{smallmatrix} \varphi_{11} & 0 \\ 0 & \varphi_{22} \\ \end{smallmatrix} \right ) \right )  \\
																								 &= \theta \left (\iota^a\left (\begin{smallmatrix} \sigma & 0 \\ 0 & 0 \\ \end{smallmatrix} \right )X\right ) \\
																								 &= \theta \left (\iota^{K_0}(\sigma )X\right ), \\
\end{aligned}
$$
where in the fourth line we used the adjoint relation for $\iota_a$ and $\iota^a$.

We now turn to the Cardy condition; for the equality $\pi^{K_0}_b=\iota_b\iota^{K_0}$, consider a basis $\{\left (\begin{smallmatrix} \tau_i & 0 \\ \end{smallmatrix} \right ), \left (\begin{smallmatrix} 0 & \eta_j  \\ \end{smallmatrix} \right )\}$ for $\Gamma_{ab}\cong \Gamma_{K_0b}\oplus \Gamma_{I_0b}$, where $\{\sigma_i\}$ is a basis for $\Gamma_{K_0b}$ and $\{\eta_j\}$ for $\Gamma_{I_0b}$. We have
$$
\begin{aligned}
\iota_b\iota^{K_0}(\sigma ) &= \iota_b\iota^a \left (\begin{smallmatrix} \sigma & 0 \\ 0 & 0 \\ \end{smallmatrix} \right ) = \pi^a_b \left (\begin{smallmatrix} \sigma & 0 \\ 0 & 0 \\ \end{smallmatrix} \right ) \\
														&= \sum_i\left (\begin{smallmatrix} \tau_i & 0 \\ \end{smallmatrix} \right )\left (\begin{smallmatrix} \sigma & 0 \\ 0 & 0 \\ \end{smallmatrix} \right ) \left (\begin{smallmatrix} \overline{\theta}_{bK_0}^{-1}(\tau^i) \\ 0 \\ \end{smallmatrix} \right ) + \sum_j \left (\begin{smallmatrix} 0 & \eta_j \\ \end{smallmatrix} \right )\left (\begin{smallmatrix} \sigma & 0 \\ 0 & 0 \\ \end{smallmatrix} \right ) \left (\begin{smallmatrix} 0 \\ \overline{\theta}_{bI_0}^{-1}(\eta^j) \\ \end{smallmatrix} \right ) \\
														&= \pi_b^{K_0}(\sigma ). \\
\end{aligned}
$$
Consider now the equality $\pi^b_a=\iota_a\iota^b$; taking into account the decomposition $a\cong K_0\oplus I_0$ we have
\begin{equation}\label{rhs_cardy_pa}
\iota_a\iota^b(\sigma ) = \iota_{K_0\oplus I_0}\iota^b(\sigma ) = \left (\begin{smallmatrix} \iota_{K_0}\left (\iota^b(\sigma )\right ) & 0 \\ 0 & \iota_{I_0}\left (\iota^b(\sigma )\right ) \\ \end{smallmatrix} \right ).
\end{equation}
On the other hand, by lemma \ref{cardy_additive},
\begin{equation}\label{lhs_cardy_pa}
\pi^b_a(\sigma )= \pi^b_{K_0\oplus I_0} (\sigma )=\left (\begin{smallmatrix} \pi^b_{K_0}(\sigma ) & 0 \\ 0 & \pi^b_{I_0}(\sigma ) \\ \end{smallmatrix} \right ).
\end{equation}
Comparing equations \eqref{rhs_cardy_pa} and \eqref{lhs_cardy_pa} we obtain $\pi^b_{K_0}=\iota_{K_0}\iota^b$, as desired.
\end{proof}

Hence, we obtain the following

\begin{cor}
Any maximal {\sc cy} category $\scr{B}$ over $M$ is pseudo-abelian.
\end{cor}



%%%%%%%%%%%%%%%%%%%%%%%%%%%%%%%%%%%%%
\section{Local Structure}
\label{ss_local_structure}

The following definition shall be useful.

\begin{defi}
Let $U\subset M$ be a semisimple open subset. We shall say that a label $a\in \scr{B}(U)$ is \emph{supported on an index $i_0$} if
$$\iota_{a}(e_{i_0})=1_a.$$
Equivalently, $\iota_a(e_j)=0$ for each $j\neq i_0$. 
\end{defi}

\begin{lemma}\label{support_ij}
Let $i\neq j$ be two indices, $1\leqslant i,j\leqslant n$ and let $a,b$ be labels over a semisimple open subset of $M$. If $a$ and $b$ are supported on $i$ and $j$ respectively, then $\Gamma_{ab}=0$.
\end{lemma}
\begin{proof}
Pick an arrow $\sigma \in \Gamma_{ab}$. Then
$$\sigma =\sigma 1_a=\sigma \iota_a(e_i)=\iota_b(e_i)\sigma =0,$$
as claimed.
\end{proof}

\begin{lemma}\label{existence_support}
Let $\scr{B}$ be a maximal category of branes and $U$ a semisimple open subset. For each index $i$, $1\leqslant i\leqslant n$, there exists a label $\xi_i$ supported on $i$.
\end{lemma}
\begin{proof}
Assume that this statement is false. We shall see that the maximality of $\scr{B}$ will not allow this to happen.

So we first assume that $\iota_a(e_j)=0$ for each index $j$ and each $a\in \scr{B}(U)$. We define a new category $\scr{C}$: the objects of $\scr{C}(U)$ are objects of $\scr{B}(U)$ plus one label, which we denote by $\xi_i$. We also define
\begin{itemize}
\item $\Gamma_{\xi_i\xi_i}=\scr{O}_U$.
\item $\Gamma_{\xi_ia}=\Gamma_{a\xi_i}=0$; this definition is motivated by lemma \ref{support_ij}.
\item $\theta_{\xi_i}:\Gamma_{\xi_i\xi_i}=\scr{O}_U \to \scr{O}_U$ is the identity.
\item Let $X=\sum_j\lambda_je_j$ be a local vector field. Then $\iota_{\xi_i}:\scr{T}_U\to \Gamma_{\xi_i\xi_i}$ and $\iota^{\xi_i}:\Gamma_{\xi_i\xi_i}\to \scr{T}_U$ are given by
$$\iota_{\xi_i}(X)=\lambda_i \quad \text{and} \quad \iota^{\xi_i}(\lambda )=\lambda e_i.$$
\end{itemize}
These definitions make $\scr{C}$ a Cardy fibration, contradicting the maximality of $\scr{B}$.
\end{proof}

\begin{proposition}\label{invertibles}
Let $U$ be a semisimple neighborhood. For each index $i=1,\dots ,n$, there exists a label $\xi_i\in \scr{B}(U)$ supported on $i$ such that $\Gamma_{\xi_i\xi_i}\cong \scr{O}_U$.
\end{proposition}
\begin{proof}
Let $i$ be an index, $1\leqslant i\leqslant n$. By lemma \ref{existence_support}, we can pick a label $a_i$ supported in $i$. If $\Gamma_{a_ia_i}\cong \scr{O}_U$, then $\xi_i:=a_i$ is the label we are looking for. If not, we have that $\Gamma_{a_ia_i}$ can be taken to be a matrix algebra $\opnm{M}_{d_i}(\scr{O}_U)$ (the construction of such a label is assured by maximality of the category of branes, and can be proved by following exactly the same procedure used in the proof of lemma \ref{existence_support}). Let then $\sigma \in \Gamma_{a_ia_i}$ be an idempotent matrix, which can be regarded as a morphism $\sigma :\scr{O}_U^{d_i}\to \scr{O}_U^{d_i}$. Moreover, assume that $\sigma$ is the projection
$$\sigma (\lambda_1,\dots ,\lambda_n)=(\lambda_1,\dots ,\lambda_{i-1},0,\lambda_{i+1},\dots ,\lambda_n).$$
Then, as the category of branes is pseudo-abelian, we have that $\opnm{Ker}\sigma \cong \scr{O}_U\in \scr{B}(U)$. As $\scr{O}_U$ is indecomposable, we should have $\Gamma_{\opnm{Ker}\sigma\opnm{Ker}\sigma}\cong \scr{O}_U$, and hence $\xi_i:=\opnm{Ker}\sigma$ is the object we were looking for.
\end{proof}

\begin{lemma}\label{simple_mods}
$\Gamma_{\xi_i\xi_j}=0$ for $i\neq j$.
\end{lemma}
\begin{proof}
This is an immediate consequence of lemma \ref{support_ij}.
\end{proof}

We shall need the following decomposition for $\Gamma_{ab}$.

\begin{proposition}\label{decomposition}
For labels $a,b\in \mathscr{B}(U)$, with $U$ a semisimple neighborhood, we have an isomorphism
$$\Gamma_{ab}\cong \bigoplus_i\Gamma_{a\xi_i}\otimes\Gamma_{\xi_ib}.$$
\end{proposition}
\begin{proof}
Define the map $\phi :\bigoplus_i\Gamma_{a\xi_i}\otimes\Gamma_{\xi_ib}\rightarrow\Gamma_{ab}$ by
\begin{equation}\label{iso_decomp}
\phi (\sigma_1\otimes \tau_1,\dots ,\sigma_n\otimes \tau_n)=\sum_i\tau_i\sigma_i.
\end{equation}
Using the characterization given in \ref{theorem2bis}, we have a local isomorphism
$$
\bigoplus_i\Gamma_{a\xi_i}\otimes\Gamma_{\xi_ib} \cong \bigoplus_i \left (\bigoplus_j\underline{\operatorname{Hom}}_{\scr{O}_U}\left (\mathscr{O}_U^{d(a,j)},\mathscr{O}_U^{d(\xi_i,j)}\right )\right )\otimes \left (\bigoplus_k\underline{\operatorname{Hom}}_{\mathscr{O}_U}\left (\mathscr{O}_U^{d(\xi_i,k)},\mathscr{O}_U^{d(b,k)}\right )\right ).
$$
By \ref{simple_mods}, we have that $d(\xi_i,k)=\delta_{ik}$, and thus
$$
\bigoplus_i\Gamma_{a\xi_i}\otimes\Gamma_{\xi_ib} \cong \bigoplus_i \underline{\operatorname{Hom}}_{\scr{O}_U}\left (\mathscr{O}_U^{d(a,i)},\mathscr{O}_U\right )\otimes \underline{\operatorname{Hom}}_{\scr{O}_U}\left (\mathscr{O}_U,\mathscr{O}_U^{d(b,i)}\right ).
$$
On the other hand, by \ref{theorem2}, we also have that, locally, $\Gamma_{ab}\cong \bigoplus_i\underline{\operatorname{Hom}}_{\scr{O}_U}\left (\mathscr{O}_U^{d(a,i)},\mathscr{O}_U^{d(b,i)}\right )$. Combining these facts with \eqref{iso_decomp} we conclude that the stalk maps $\phi_x$ are in fact bijections for each $x\in U$.  
\end{proof}

A useful consequence of \ref{decomposition} is the following

\begin{cor}\label{linear_comb}
For each label $b$ over $U$, we have an isomorphism
$$b\cong \bigoplus_i\Gamma_{\xi_ib}\otimes \xi_i.$$
\end{cor}
\begin{proof}
Take any label $c$. By equations \eqref{product_brane} and duality we have
$$
\begin{aligned}
\underline{\operatorname{Hom}}_U\Bigl (\bigoplus_i\Gamma_{\xi_ib}\otimes \xi_i,c\Bigr ) &\cong \bigoplus_i\Gamma_{b\xi_i}\otimes \underline{\operatorname{Hom}}_U(\xi_i,c) \\
&\cong \bigoplus_i\Gamma_{b\xi_i}\otimes\Gamma_{\xi_ic} \\
&\cong \Gamma_{bc}.\\
\end{aligned}
$$
As $c$ is arbitrary, the result follows.
\end{proof}

Note that the coefficient modules in the previous result are unique, up to isomorphism: if $b\cong \bigoplus_i\scr{M}_i\otimes \xi_i$, then
$$\Gamma_{\xi_jb}\cong \bigoplus_i\scr{M}_i\otimes\Gamma_{\xi_j\xi_i}\cong\scr{M}_j.$$

The next result addresses some uniqueness issues.

\begin{proposition}\label{uniqueness_labels}
Let $\xi_i\in \scr{B}(U)$ be as in \ref{invertibles}, where $U$ is semisimple.
\begin{enumerate}[(1)]
\item Let $\eta_i$ be a label with the same properties as $\xi_i$. Then, there exists an invertible sheaf $\scr{L}$ over $U$ such that
$$\eta_i\cong \scr{L}\otimes \xi_i.$$
The converse statement also holds.
\item If $\scr{M}$ is a locally-free module such that $\scr{M}\otimes \xi_i\cong \xi_i$, then $\scr{M}\cong \scr{O}_U$.
\end{enumerate}
\end{proposition}
\begin{proof}
For the first item, by \ref{support_ij} and \ref{linear_comb}, we have that
$$
\begin{aligned}
\eta_i &\cong \bigoplus_j\Gamma_{\xi_j\eta_i}\otimes \xi_j \\
       &\cong \Gamma_{\xi_i\eta_i}\otimes \xi_i. \\
\end{aligned}
$$
Let $\scr{M}_i:=\Gamma_{\xi_i\eta_i}$. Then,
$$
\begin{aligned}
\scr{O}_U &\cong \Gamma_{\eta_i\eta_i} \cong \Gamma_{\left (\scr{M}_i\otimes \xi_i \right )\left (\scr{M}_i\otimes \xi_i \right )} \\
          &\cong \scr{M}_i^*\otimes \scr{M}_i\otimes \Gamma_{\xi_i\xi_i} \\
					&\cong \Gamma_{\xi_i\eta_i}^*\otimes \Gamma_{\xi_i\eta_i}.\\
\end{aligned}
$$
The converse is immediate by properties of the action $\scr{L}\otimes \xi_i$.

For (2), as $\scr{M}\otimes \xi_i \cong \xi_i$, the modules $\Gamma_{\xi_i\xi_i}$ and $\Gamma_{\xi_i\left (\scr{M}\otimes \xi_i\right )}$ are isomorphic. Hence,
$$\scr{O}_U\cong \Gamma_{\xi_i\left (\scr{M}\otimes \xi_i\right )}\cong \scr{M}\otimes \Gamma_{\xi_i\xi_i}\cong \scr{M},$$
as desired.
\end{proof}

\begin{theorem}\label{equivalences}
There exists an open cover $\mathfrak{U}$ of $M$ and an equivalence of categories
\begin{equation}\label{equiv_2vb}
\mathscr{B}(U)\simeq \tsf{LF}^n_{\scr{O}_U}
\end{equation}
for each $U\in \mathfrak{U}$, where $\tsf{LF}^n_{\scr{O}_U}$ denotes the $n$-fold fibred product of $\tsf{LF}_{\scr{O}_U}$.
\end{theorem}
\begin{proof}
Let $\mathfrak{U}=\{U_\alpha \}$ be an open cover of $M$, where each $U_\alpha$ is semisimple. Define $F_\alpha:\mathscr{B}(U_\alpha )\rightarrow \tsf{LF}^n_{\scr{O}_{U_\alpha}}$ on objects by
$$F_\alpha (a)=(\Gamma_{\xi_1a},\dots ,\Gamma_{\xi_na}),$$
where the objects $\xi_i$ are the ones of proposition \ref{invertibles}, and on arrows by $F_\alpha (\sigma )=\sigma_*$; that is, if $\sigma :a\to b$, then $F_\alpha (\sigma )(\tau_1,\dots ,\tau_n)=(\sigma \tau_1,\dots ,\sigma \tau_n)$. We now define $G_\alpha:\tsf{LF}^n_{\scr{O}_{U_\alpha}}\rightarrow\scr{B}(U_\alpha )$ on objects by
$$G_\alpha (\mathscr{M}_1,\dots ,\mathscr{M}_n)=\bigoplus_i \mathscr{M}_i\otimes \xi_i$$
and on arrows by
$$G_\alpha (f_1,\dots ,f_n)=(f_1\otimes \opnm{id}_{\xi_1},\dots ,f_n\otimes \opnm{id}_{\xi_n}),$$
where $f_i:\scr{M}_i\to \scr{N}_i$.

We then have that $F_\alpha G_\alpha (\mathscr{M}_1,\dots ,\mathscr{M}_n)=(\Gamma_{\xi_1\overline{a}},\dots ,\Gamma_{\xi_n\overline{a}})$, where $\overline{a}:=\bigoplus_j\scr{M}_j\otimes \xi_j$. Now,
$$
\begin{aligned}
\Gamma_{\xi_i\overline{a}} &\cong \bigoplus_j \underline{\operatorname{Hom}}_U(\xi_i,\mathscr{M}_j\otimes \xi_j) \\
                    &\cong \bigoplus_j \mathscr{M}_j\otimes \underline{\operatorname{Hom}}_U(\xi_i,\xi_j) \\
                    &\cong \mathscr{M}_i \\
\end{aligned}
$$
by \eqref{product_brane} and \ref{simple_mods}.

The other way, we have $G_\alpha F_\alpha (a)=\bigoplus_i\Gamma_{\xi_ia}\otimes \xi_i$, which is isomorphic to $a$ by \ref{linear_comb}.
\end{proof}

In terms of the spectral cover, over each semisimple $U\subset M$ we have $\pi^{-1}(U)=\bigsqcup_{i=1}^n\widetilde{U}_i$, where each $\widetilde{U}_i$ is homeomorpic to $U$ by the projection $\pi :S\to M$, and thus we can write the $n$-fold product $\tsf{LF}^n_{\scr{O}_U}$ as the pushout $(\pi_*\tsf{LF}_{\scr{O}_S})(U)=\tsf{LF}_{\scr{O}_{\pi^{-1}(U)}}$. But $\scr{O}_{\pi^{-1}(U)}$ is the sheaf $(\pi_*\scr{O}_S)|_U$, which is in turn isomorphic to the tangent sheaf $\scr{T}_U$ by proposition \ref{isom_1}. Moreover, if $f:M\to N$ is a continuous map, then, by definition, the fibred categories $f_*\tsf{LF}_{\scr{O}_M}$ and $\tsf{LF}_{f_*\scr{O}_M}$ are equal. Thus, combining all these facts we can deduce that
$$\pi_*\tsf{LF}_{\scr{O}_S}=\tsf{LF}_{\pi_*\scr{O}_S}\simeq \tsf{LF}_{\scr{T}_M}.$$

\begin{cor}
Given a  maximal Cardy fibration $\scr{B}$ over a massive manifold $M$, there exists an open cover $\mathfrak{U}$ of $M$ such that the category $\scr{B}(U)$ is equivalent to the category $\tsf{LF}_{\scr{T}_U}$ of locally free $\scr{T}_U$-modules. 
\end{cor}

Before stating the next result, we give a preliminary definition. Given a vector bundle $E$ we can construct the exterior powers $\bigwedge^kE$ which for a point $x\in M$ have fibre $\bigwedge^kE_x$. Given now a bundle map $\phi :E\to F$, we have that $\phi^{\wedge k}:\bigwedge^kE\to \bigwedge^kF$ is given by
$$\phi^{\wedge k}(e_1\wedge \dots \wedge e_n)=\phi (e_1)\wedge \dots \wedge \phi (e_n).$$
After this brief comment about exterior powers, we can now give the definition we need (see \cite{audin:frob} and references cited therein). A \emph{Higgs pair} for a manifold $M$ is a pair $(E,\phi )$, where $E$ is a vector bundle and $\phi :TM\to \opnm{End}(E)$ is a morphism such that $\phi \wedge \phi =0$. This last condition is expressing that for each $x\in M$, the endomorphisms $\phi_x(v)\in \opnm{End}(E_x)$ (for $v\in T_xM$) commute.

In the next result, we use the notation of the proof of theorem \ref{equivalences}.

\begin{cor}
Given $a\in \scr{B}(U_\alpha )$, the transition homomorphism $\iota_a$ consists of $n$ Higgs pairs for $U_\alpha$.
\end{cor}

The meaning of <<consists of $n$ Higgs pairs>> is explained in the following proof.

\begin{proof}
From theorem \ref{equivalences}, we have an equivalence $F_\alpha :\scr{B}(U_\alpha )\to \tsf{LF}_{\scr{O}_{U_\alpha}}^n$; in particular, given a label $a\in \scr{B}(U_\alpha )$, we have a bijection
$$\opnm{Hom}_{\scr{B}(U_\alpha )}(a,a)\longrightarrow \opnm{Hom}_{\tsf{LF}_{\scr{O}_{U_\alpha}}^n}(F_\alpha (a),F_\alpha (a)),$$
which is in fact an isomorphism of algebras
$$\Gamma_{aa}\longrightarrow \bigoplus_k\opnm{End}_{\tsf{LF}_{\scr{O}_{U_\alpha}}}\left (\Gamma_{\xi_k a}\right ).$$
We can then assume that the transition homomorphism $\iota_a:\scr{T}_{U_\alpha }\to \Gamma_{aa}$ is in fact a morphism
$$\iota_a:\scr{T}_{U_\alpha }\longrightarrow \bigoplus_k\opnm{End}_{\tsf{LF}_{\scr{O}_{U_\alpha}}}\left (\Gamma_{\xi_k a}\right );$$
in other words, the map $\iota_a$ consists of $n$ morphisms
$$\iota_a^k :\scr{T}_{U_\alpha }\longrightarrow \opnm{End}_{\tsf{LF}_{\scr{O}_{U_\alpha}}}\left (\Gamma_{\xi_k a}\right ).$$

In our case, we have that the morphism $\iota_a$ is central; this condition can be also expressed by saying that the morphisms $\iota_a^k$ are central ($k=1,\dots ,n$). Hence, for each $k=1,\dots ,n$, $\left (\Gamma_{\xi_ka},\iota_a^k\right )$ is a Higgs pair for $U_\alpha$.
\end{proof}

We shall now describe the {\sc bdr} 2-vector bundle structure for the stack $\scr{B}$.

We first point out that, being $M$ paracompact, the open cover by semisimple open subsets $\mathfrak{U}=\{U_\alpha \}$ can be taken to be indexed by a poset (which we shall not include in our notation). For each index $i=1,\dots ,n$, let $\xi^\alpha_i \in \scr{B}(U_\alpha )$ be a label as in proposition \ref{invertibles}. Let $U_\beta$ be another semisimple subset such that $U_{\alpha \beta}\neq \emptyset$ and let $\{e_i^\alpha\}$ and $\{e_i^\beta\}$ be frames of simple idempotent sections over $U_\alpha$ and $U_\beta$ respectively. We then have a permutation $u=u_{\alpha \beta}:\{1,\dots ,n\}\to \{1,\dots ,n\}$ such that, over $U_{\alpha \beta}$,
$$e_i^\alpha =e_{u(i)}^\beta .$$
By proposition \ref{uniqueness_labels}, the previous equation is equivalent to the existence of invertible sheaves $\scr{L}^{\alpha \beta}_i$ such that, over $U_{\alpha \beta}$,
$$\xi_i^\alpha \cong \scr{L}^{\alpha \beta}_{u(i)}\otimes \xi_{u(i)}^\beta .$$
Write $\xi^\alpha :=(\xi_1^\alpha ,\dots ,\xi_n^\alpha )^t$. Then, we can write the previous equation in matrix form
\begin{equation}\label{mult_matrix}
\xi^\alpha \cong A^{\alpha \beta}_u \xi^\beta ,
\end{equation}
where $A^{\alpha \beta}_u$ is a matrix obtained from the diagonal matrix
$$\opnm{diag}\left (\scr{L}^{\alpha \beta}_1,\dots ,\scr{L}^{\alpha \beta}_n\right )$$
by applying the permutation $u$ to its columns. Let now $\gamma$ be such that $U_{\alpha \beta \gamma}\neq \emptyset$ and suppose that the idempotents are permuted according to $v$ over $U_{\beta \gamma}$ and $w$ over $U_{\alpha \gamma}$.

\begin{lemma}
We have an isomorphism $A^{\alpha \beta}_uA^{\beta \gamma}_v\cong A^{\alpha \gamma}_w$ (i.e. the corresponding matrix entries on each side have isomorphic bundles).
\end{lemma} 
\begin{proof}
Assume that the idempotents are permuted according to
\begin{itemize}
\item $u$ over $U_{\alpha \beta}$,
\item $v$ over $U_{\beta \gamma}$ and
\item $w$ over $U_{\alpha \gamma}$.
\end{itemize}
Then, by uniqueness, we should have $vu=w$. Now pick a vector $\xi^\gamma$. Then, the $i$-th coordinate of $A^{\alpha \beta}_uA^{\beta \gamma}_v\xi^\gamma$ is given by
$$\scr{L}_i^{\alpha \beta }\otimes \scr{L}_{u(i)}^{\beta \gamma }\otimes \xi_{v(u(i))}^\gamma ,$$
and the one corresponding to the product $A^{\alpha \gamma}_w\xi^\gamma$ is
$$\scr{L}^{\alpha \gamma}_i\otimes \xi_{w(i)}^\gamma .$$
As both objects are isomorphic to $\xi_i^\alpha$, they are both isomorphic, and hence by \ref{uniqueness_labels},
$$\scr{L}_i^{\alpha \beta }\otimes \scr{L}_{u(i)}^{\beta \gamma } \cong \scr{L}^{\alpha \gamma}_i,$$
as desired.
\end{proof}

If $A=(E_{ij})$ is an $n\times n$ matrix of vector bundles, we denote by $\opnm{rk}A\in \opnm{M}_n(\natu_0)$ the matrix which $(i,j)$ entry is $\opnm{rk}E_{ij}$. Then, by definition,
$$\opnm{det}\left (\opnm{rk} A_u^{\alpha \beta}\right )=\pm 1.$$
Moreover, associativity of the tensor product renders the following diagram
$$\xymatrix{
A^{\alpha \beta}(A^{\beta \gamma }A^{\gamma \delta }) \ar[rr] \ar[d] & & (A^{\alpha \beta }A^{\beta \gamma })A^{\gamma \delta } \ar[d] \\
A^{\alpha \beta }A^{\beta \delta } \ar[r] & A^{\alpha \delta } & A^{\alpha \gamma }A^{\gamma \delta }, \ar[l] }
$$
commutative (see definition \ref{bdr_2bundle}). We can then state the following

\begin{theorem}
Let $M$ be a massive manifold with multiplication of dimension $n$. Then, any maximal Cardy fibration $\scr{B}$ over $M$ has a canonical {\sc bdr} 2-vector bundle of rank $n$ attached to it.
\end{theorem}


%%%%%%%%%%%%%%%%%%%%%%%%%%%%%%%%%%%%%%%%%%%%%%%%%
\subsection{The Category of Locally Free Modules}

Assume that our (semisimple) base manifold $M$ has dimension $n$ and consider the fibred category $\underline{\tsf{LF}}^n_{\scr{O}_M}$ defined by the correspondence
$$\underline{\tsf{LF}}^n_{\scr{O}_M}(U)=\tsf{LF}^n_{\scr{O}_U},$$
where the right-hand side is the $n$-folded fibred product of the category of locally free $\scr{O}_U$-modules. We shall now build a Cardy fibration from this fibred category.

Let $\mathfrak{U}$ be an open cover consisting of connected, semisimple subsets. Over each $U\in \mathfrak{U}$ we then have a frame of idempotent sections $\{e_1,\dots ,e_n\}$ of the tangent sheaf $\scr{T}_U$. Given objects ($n$-tuples) $\scr{M}:=(\scr{M}_i)$ and $\scr{N}:=(\scr{N}_i)$, a morphism $\sigma :\scr{M}\to \scr{N}$ is an $n$-tuple $(\sigma_i)$ of morphisms $\sigma_i:\scr{M}_i\to \scr{N}_i$. In particular, note that, locally, the sheaf $\Gamma_{\scr{MN}}$ is isomorphic to a sum $\bigoplus_i\opnm{M}_{n_i\times m_i}(\scr{O}_U)$ of matrix algebras, where $m_i$ and $n_i$ are, respectively, the ranks of $\scr{M}_i$ and $\scr{N}_i$. When $\scr{M}=\scr{N}$, the sheaf $\Gamma_{\scr{MM}}$ shall be denoted by $\Gamma_{\scr{M}}$.

In order to endow $\underline{\tsf{LF}}^n_{\scr{O}_M}$ with a Cardy fibration structure, we need first to define the structure maps, for which we consider equations \eqref{local_expressions}.

Let us start with the transition map $\iota_\scr{M}$. Recall that for each local vector field $X$, the image $\iota_\scr{M}(X)$ should be in the center of the endomorphism sheaf, which in this case is a sheaf isomorphic to $\scr{O}^n_U$. Hence, $\iota_\scr{M}$ should be an algebra homomorphism $\iota_\scr{M}:\scr{T}_U\to \scr{O}^n_U$, where the algebra structure on $\scr{O}_U$ is the trivial one.

For an object $\scr{M}=(\scr{M}_i)$, we define $\iota_{\scr{M}}$ in the following way: given an idempotent section $e_i$, the (idempotent) endomorphism $\iota_{\scr{M}}(e_i):\scr{M}\to \scr{M}$ is the canonical projection
$$\iota_{\scr{M}}(e_i)(x_1,\dots ,x_n)=(0,\dots ,0,x_i,0,\cdots ,0).$$
Let now $\sigma =(\sigma_i)\in \Gamma_{\scr{M}}$; then for $\iota^\scr{M}$ we must have
$$\iota^\scr{M}(\sigma )= \sum_i\frac{\operatorname{tr}(\sigma_i)}{\sqrt{\theta (e_i)}}e_i,$$
which leads to the following expression for $\theta_\scr{M}$:
$$\theta_\scr{M}(\sigma )=\sum_i\sqrt{\theta (e_i)}\operatorname{tr}(\sigma_i).$$
From these definitions we can deduce also the adjoint relation \eqref{adjoint}.

For the Cardy condition, consider $\pi^\scr{M}_\scr{N}:\Gamma_{\scr{M}}\to \Gamma_{\scr{N}}$ which is given by
$$\pi_\scr{N}^\scr{M}(\sigma_1,\dots ,\sigma_n)=\sum_i\frac{\opnm{tr}(\sigma_i)}{\sqrt{\theta (e_i)}}\iota_\scr{N}(e_i);$$
that is, if $\sigma :=(\sigma_1,\dots ,\sigma_n)$,
$$\pi_\scr{N}^\scr{M}(\sigma )(x_1,\dots ,x_n)=\sum_i\frac{\opnm{tr}(\sigma_i)}{\sqrt{\theta (e_i)}}x_i.$$
On the other hand, we have $\iota_\scr{N}\iota^\scr{M}(\sigma )=\sum_i\frac{\opnm{tr}(\sigma_i)}{\sqrt{\theta (e_i)}}\iota_\scr{N}(e_i)$, and hence
$$\iota_\scr{N}\iota^\scr{M}(\sigma )(x_1,\dots ,x_n)=\sum_i\frac{\opnm{tr}(\sigma_i)}{\sqrt{\theta (e_i)}}x_i=\pi_\scr{N}^\scr{M}(\sigma )(x_1,\dots ,x_n).$$
Additivity is provided by the direct sum of modules. The action of the category of locally free modules is given by the tensor product. The pseudo-abelian structure (in fact abelian) structure of the category of locally free modules is also well-known (see section \ref{bundles_operations}). As $\underline{\tsf{LF}}_{\scr{O}_M}$ is a stack (check example \ref{st_ex4}), then so is the $n$-fold product.

The objects $\xi_i$ are given in this case by the $n$-tuples $(0,\dots ,0,\scr{O}_U,0,\dots ,0)$.

It is worth noting that the open cover $\mathfrak{U}$ of definition \ref{cy_fibration} cannot in general be taken as $\mathfrak{U}=\{M\}$; consider the object $\scr{O}_i:=(0,\dots ,0,\scr{O}_M,0,\dots ,0)$; then $\Gamma_{\scr{O}_i}\cong \scr{O}_M$, and the transition map $\iota_{\scr{O}_i}$ can be regarded as an algebra homomorphism
$$\iota_{\scr{O}_i}:\scr{T}_M\longrightarrow \scr{O}_M,$$
which is the same as having a global section $M\to S$ of the spectral cover. If this were true, then $S$ should be trivial, which in fact implies that there exists a global frame of idempotent sections, and hence trivializing the tangent bundle of $M$.



\clearpage

{\small
%%%%%%%%%%%%%%%%%%%%%%%%%%%%%%%%%%%%%%%%%%%%%%%%%%%%%%%%
%%%%%%%%%%%%%%%%%%%%%%%%%%%%%%%%%%%%%%%%%%%%%%%%%%%%%%%%
\section{Resumen del Cap\'itulo \ref{local_description}}

En este cap\'itulo hacemos una descripci\'on completa de las que llamamos categor\'ias maximales, y demostramos que la sugerencia de G. Segal respecto a que debe existir una relaci\'on entre el moduli de teor\'ias topol\'ogicas de campos y los 2-espacios vectoriales es en efecto cierta.


%%%%%%%%%%%%%%%%%%%%%%%%%%%%%%%%%%%%%%%%%%%%%%%%%%%%%%%%%%%%%%%%%%
\subsection{Propiedades Algebraicas de las Categor\'ias Maximales}

Diremos que una fibraci\'on de Cardy $\scr{B}$ sobre $M$ es \emph{maximal} si dada otra fibraci\'on $\scr{B}'$, existe una aplicaci\'on inyectiva $\opnm{sk}\scr{B}'\to \opnm{sk}\scr{B}$, donde $\opnm{sk}$ indica el esqueleto de la categor\'ia.

Fijemos ahora un punto $x\in U_\alpha \subset M$, donde $U_\alpha$ es un abierto semisimple. Dados $a,b\in \scr{B}(U_\alpha )$, notaremos con $E_{ab}$ la fibra sobre $x$ del haz $\Gamma_{ab}$ (omitimos referencia a $x$ para simplificar la notaci\'on).\footnote{Dado un $\scr{O}_M$-m\'odulo localmente libre $\scr{M}$, recordemos que el \emph{stalk} sobre $x$ viene dado por $\scr{M}_x=\underset{U\ni x}{\opnm{colim}}\scr{M}_x$. La \emph{fibra} $F_x(\scr{M})$ sobre $x$ se define entonces por
$$F_x(\scr{M})=\scr{M}_x/\mathfrak{m}_x^{\oplus n},$$
siendo $\mathfrak{m}_x$ el ideal maximal de $\scr{O}_{M,x}$.}

Llamemos ahora $p_{ab}$ a la sucesi\'on de morfismos
$$\Gamma_{ab}(U_\alpha)\longrightarrow \Gamma_{ab,x}\longrightarrow E_{ab},$$
donde $\Gamma_{ab,x}$ indica el \emph{stalk} del haz $\Gamma_{ab}$ sobre $x$. Sea $1_a$ la identidad de $\Gamma_{aa}(U_\alpha )$, e identifiquemos a una brana $a\in \scr{B}(U_\alpha )$ con la correspondiente identidad $1_a$. Notemos tambi\'en por $\overline{a}$ a la imagen $p_{aa}(1_a)\in E_{aa}$. Definimos ahora una categor\'ia $\overline{\scr{B}}_x$ de la siguiente manera: sus objetos vienen dados por $\overline{a}$ (con $a\in \scr{B}(U_\alpha)$); dados objetos $\overline{a}$ y $\overline{b}$, el conjunto de morfismos $\overline{a}\to \overline{b}$ se define como $E_{ab}$. Las formas lineales vienen inducidas por las formas $\theta :\scr{T}_M\to \scr{O}_M$ y $\theta_a :\Gamma_{aa}\to \scr{O}_M$, las cuales inducen $\overline{\theta}_x:T_xM\to \comp$ y $\overline{\theta}_a:E_{aa}\to \comp$. De la misma forma, los morfismos de transici\'on inducen morfismos de transici\'on
$$T_xM\stackrel{\iota_{\overline{a}}}{\longleftarrow}E_{aa}\stackrel{\iota^{\overline{a}}}{\longrightarrow}T_xM.$$
\medskip

{\bf Teorema.} {\it Sean $x_0,x_1\in U_\alpha$. Tenemos entonces que
\begin{enumerate}
\item Las categor\'ias $\overline{\scr{B}}_{x_0}$ y $\overline{\scr{B}}_{x_1}$ son isomorfas.
\item La categor\'ia $\overline{\scr{B}}_x$, junto con el \'algebra $T_xM$ y los mapas de estructura $\overline{\theta}_x, \overline{\theta}_a \iota_{\overline{a}}$ e $\iota^{\overline{a}}$ definen una categor\'ia de branas en el sentido de Moore y Segal.
\end{enumerate}}
\medskip

Dos fundamentales consecuencias de esta definici\'on vienen resumidas en el siguiente resultado.
\medskip

{\bf Teorema.} {\it Sean $a,b\in \scr{B}(U_\alpha)$. Entonces
\begin{enumerate}
\item El haz $\Gamma_{aa}$ es localmente isomorfo a una suma de \'algebras de matrices $\bigoplus_i\opnm{M}_{d(a,i)}(\scr{O}_{U_{\alpha}})$.
\item El haz $\Gamma_{ab}$ es localmente isomorfo a $\bigoplus_i\opnm{Hom}_{\scr{O}_{U_\alpha}}\left (\scr{O}_{U_\alpha}^{d(a,i)},\scr{O}_{U_\alpha}^{d(b,i)}\right )$.
\end{enumerate}}


\subsubsection{{\small Propiedades de las Categor\'ias Maximales}}

La propiedad de maximalidad implica la existencia de varias propiedades importantes que este tipo de categor\'ias deben tener. En esta secci\'on damos cuenta de todas ellas.
\medskip

{\sc Estructura Aditiva.} Sea $U\subset M$ un abierto y $a,b,c\in \scr{B}(U)$. Veamos entonces que tener una estructura aditiva es perfectamente compatible con las propiedades que definen una fibraci\'on de Cardy. Definimos un objeto $a\oplus b$, poniendo
$$
\begin{aligned}
\Gamma_{(a\oplus b)c} &:= \Gamma_{ac}\oplus \Gamma_{bc} \\
\Gamma_{c(a\oplus b)} &:= \Gamma_{ca}\oplus \Gamma_{cb}. \\
\end{aligned}
$$
En particular, notar que los morfismos en $\Gamma_{(a_1\oplus a_2)(b_1\oplus b_2)}$ se pueden representar como una matriz $\begin{smallmatrix} \sigma_{11} & \sigma_{21} \\ \sigma_{12} & \sigma_{22} \\ \end{smallmatrix}$, donde $\sigma_{ij}:a_i\to b_j$.

Para los morfismos: Definimos $\theta_{a\oplus b}:\Gamma_{(a\oplus b)(a\oplus b)}\to \scr{O}_U$ por
$$
\theta_{a\oplus b}\left (\begin{smallmatrix} \sigma_{11} & \sigma_{21} \\ \sigma_{12} & \sigma_{22} \\ \end{smallmatrix} \right )=\theta_{a}(\sigma_{11})+\theta_b(\sigma_{22});
$$
y para los morfismos de transici\'on,
$$
\begin{aligned}
\iota_{a\oplus b}(X) &:=\begin{smallmatrix} \iota_a(X) & 0 \\ 0 & \iota_b(X) \end{smallmatrix} \\
\iota^{a\oplus b}\left (\begin{smallmatrix} \sigma_{11} & \sigma_{21} \\ \sigma_{12} & \sigma_{22} \\ \end{smallmatrix} \right ) &:= \iota^a(\sigma_{11})+\iota^b(\sigma_{22}). \\
\end{aligned}
$$
Las aplicaciones $\pi^{a\oplus b}_c$ and $\pi^a_{b\oplus c}$ toman la forma
$$
\begin{aligned}
\pi^{a\oplus b}_c &= \pi^a_c+\pi^b_c \\
\pi^a_{b\oplus c} &= \left (\begin{smallmatrix} \pi^a_b & 0 \\ 0 & \pi^a_c \\ \end{smallmatrix} \right ). \\
\end{aligned}
$$
{\bf Teorema.} {\it Las aplicaciones definidas anteriormente verifican la condici\'on de centralidad, la adjunci\'on y la identidad de Cardy. En particular, toda fibraci\'on de Cardy maximal tiene una estructura aditiva.}
\medskip

Notar que la \'ultima conclusi\'on del teorema proviene justamente de la maximalidad, ya que en caso de no tener estructura aditiva, podemos definir la operaci\'on $\oplus$ y los morfismos de estructura como en los p\'arrafos anteriores y definir una categor\'ia mas grande, violando la maximalidad.
\medskip

{\sc Acci\'on de un M\'odulo Localmente Libre.} Asi como la aditividad, otra propiedad que cualquier categor\'ia maximal tiene es la de admitir una acci\'on de la categor\'ia de $\scr{O}_M$-m\'odulos localmente libres. Sea $\scr{M}$ un $\scr{O}_U$-m\'odulo localmente libre y $a,b\in \scr{B}(U)$. Definimos entonces un nuevo objeto $\scr{M}\otimes a$ de la siguiente manera:
$$
\begin{aligned}
\Gamma_{(\scr{M}\otimes a)b} &= \scr{M}^*\otimes \Gamma_{ab}, \\
\Gamma_{b(\scr{M}\otimes a)} &= \scr{M} \otimes \Gamma_{ba}, \\
\end{aligned}
$$
donde el producto tensorial se toma sobre $\scr{O}_U$. En particular, obs\'ervese que
$$\Gamma_{(\scr{M}\otimes a)(\scr{N}\otimes b)}=\underline{\opnm{Hom}}(\scr{M},\scr{N})\otimes \Gamma_{ab},$$
donde $\underline{\opnm{Hom}}(\scr{M},\scr{N})$ indica el haz de mosfismos $\scr{O}_U$-lineales. La demostrici\'on de la siguiente proposici\'on se basa principalmente en las propiedades del producto tensorial de m\'odulos.
\medskip

{\bf Proposici\'on} {\it La correspondencia $(\scr{M},a)\mapsto \scr{M}\otimes a$ define una acci\'on de la categor\'ia de $\scr{O}_M$-m\'odulos localmente sobre $\scr{B}$, compatible con la estructura aditiva.}
\medskip

Sea $\overline{a}:=\scr{M}\otimes a$. Definimos el morfismo $\theta_{\overline{a}}:\Gamma_{\overline{a}\overline{a}}\to \scr{O}_U$ como la composici\'on
$$
\xymatrix{
\underline{\operatorname{End}}_{\scr{O}_U}(\scr{M})\otimes \Gamma_{aa} \ar[rr]^{\operatorname{tr}\otimes \operatorname{id}} & & \scr{O}_U\otimes \Gamma_{aa}\cong \Gamma_{aa} \ar[r]^-{\theta_a} & \scr{O}_U;}
$$
esto es, $\theta_{\overline{a}}(f\otimes \sigma )=\opnm{tr}(f)\theta_a(\sigma )$. Pasando ahora a un abierto semisimple $U_\alpha$, definimos los mapas de transici\'on de la siguiente manera: $\iota_{\overline{a}}(X)=\opnm{id}_{\scr{M}}\otimes \iota_a(X)$ e $\iota^{\overline{a}}$ por la siguiente composici\'on:
$$
\xymatrix{
\underline{\operatorname{End}}_{\scr{O}_{U_\alpha}}(\scr{M})\otimes \Gamma_{aa} \ar[rr]^{\operatorname{tr}\otimes \operatorname{id}} & & \scr{O}_{U_\alpha}\otimes \Gamma_{aa}\cong \Gamma_{aa} \ar[r]^-{\iota^a} & \scr{T}_{U_\alpha};}
$$
o sea $\iota^{\overline{a}}(f\otimes \sigma )=\opnm{tr}(f)\iota^a(\sigma )$.
\medskip

{\bf Teorema.} {\it Con las definiciones anteriores, la acci\'on $\scr{M}\otimes a$ es compatible con todas las estructuras definidas en una categor\'ia maximal. Luego, toda categor\'ia maximal viene equipada con una acci\'on de la categor\'ia de m\'dulos localmente libres.}
\medskip

{\sc Estructura Pseudo-Abeliana.} Se demuestra que cualquier categor\'ia maximal debe ser adem\'as pseudo-abeliana; esto es:dado un morfismo idempotente $\sigma_0 :a\to a$, vamos a asumir que existen branas $K_0:=\opnm{Ker}\sigma_0$ e $I_0:=\opnm{Im}\sigma_0$ tales que
\begin{itemize}
\item La brana $a$ se descompone como $a\cong K_0\oplus I_0$ y
\item usando notaci\'on matricial, el mapa $\sigma_0$ viene dado por $\begin{smallmatrix} 0 & 0 \\ 0 & 1_a \\ \end{smallmatrix}$.
\end{itemize}
Notemos en primer lugar que
$$\Gamma_{aa}=\Gamma_{K_0K_0}\oplus \Gamma_{K_0I_0}\oplus \Gamma_{I_0K_0}\oplus \Gamma_{I_0I_0},$$
de donde podemos deducir que tanto $\Gamma_{K_0K_0}$ y $\Gamma_{I_0I_0}$ son localmente libres.
Definimos ahora los morfismos de estructura para los nuevos objetos $K_0$ e $I_0$: tenemos $\theta_{K_0}:\Gamma_{K_0K_0}\to \scr{O}_U$ dado por
$$\theta_{K_0}(\sigma )=\theta_a\begin{smallmatrix} \sigma & 0 \\ 0 & 0\end{smallmatrix}.$$
Para los morfimos de transici\'on tenemos
$$
\begin{aligned}
\iota_{K_0}(X) &:= \varphi_{11} \\
\iota^{K_0}(\sigma )=\iota^a\begin{smallmatrix} \sigma & 0 \\ 0 & 0 \\ \end{smallmatrix}, \\
\end{aligned}
$$
donde $\iota_a(X)=\begin{smallmatrix} \varphi_{11} & 0 \\ 0 & \varphi_{22} \\ \end{smallmatrix}$ (los coeficientes nulos se obtienen por la condici\'on de centralidad).
\medskip

{\bf Teorema.} {\it Los objetos y morfismos anteriores son compatibles con todas las estructuras que definen una fibraci\'on de Cardy maximal. En particular, cualquier tal categor\'ia debe ser pseudo-abeliana.}


%%%%%%%%%%%%%%%%%%%%%%%%%%%%%
\subsection{Estructura Local}

A continuaci\'on se introducen objetos que resultan fundamentales en la clasificaci\'on de las categor\'ias maximales. Su existencia, nuevamente, esta garantizada por la maximalidad.
\medskip

{\bf Lema.} {\it Sea $\scr{B}$ una categor\'ia de branas maximal y $U$ semisimple. Para cada \'indice $1\leqslant i \leqslant n$ existe una brana $\xi_i$ tal que $\iota_{\xi_i}(e_k)=\delta_{ik}1_a$.}

Decimos que un tal objeto \emph{tiene soporte en $i$}. A partir del lema anterior podemos enunciar una resultado importante.
\medskip

{\bf Proposici\'on.} {\it Si $U$ es un abierto semisimple, para cada \'indice $i$ existe una brana $\xi_i\in \scr{B}(U)$ soportada en $i$ y tal que $\Gamma_{\xi_i\xi_i}\cong \scr{O}_U$. Mas a\'un, si $i\neq j$, tenemos que $\Gamma_{\xi_i\xi_j}=0$.}

Las branas de la proposici\'on anterior son \'unicas en el siguiente sentido: si $\eta_i$ es una brana con las mismas propiedades que $\xi_i$, entonces existe un m\'odulo localmente libre $\scr{L}$ de rango 1 (que se llaman tambi\'en haces invertibles) tal que $\eta_i\cong \scr{L}\otimes \xi_i$.

Llegamos asi a uno de los resultados centrales.
\medskip

{\bf Teorema.} {\it Si $\scr{B}$ es una fibraci\'on maximal de Cardy sobre $M$, existe un cubrimiento abierto $\mathfrak{U}$ de $M$ y una equivalencia de categor\'ias
$$\scr{B}(U)\simeq \tsf{LF}^n_{\scr{O}_U},$$
donde $U\in \mathfrak{U}$ y $\tsf{LF}^n_{\scr{O}_U}$ el producto (fibrado) de $n$ factores de la categor\'ia de $\scr{O}_U$-m\'odulos localmente libres.}
\medskip

En t\'erminos del recubrimiento espectral $\pi :S\to M$, sobre cada abierto semisimple $U\subset M$ tenemos que $\pi^{-1}(U)=\bigsqcup_i\widetilde{U}_i$. Luego, como adem\'as $\scr{O}_{\pi^{-1}(U)}$ es isomorfo al haz tangente $\scr{T}_U$, podemos escribir el producto $\tsf{LF}^n_{\scr{O}_U}$ como $\tsf{LF}^n_{\scr{O}_U}\simeq \tsf{LS}_{\scr{T}_U}$, deduciendo entonces que
$$\scr{B}(U)\simeq \tsf{LF}_{\scr{T}_U}.$$

Antes de enunciar el siguiente corolario damos una definci\'on preliminar. Dado un fibrado vetorial $E$, podemos construir las potencias exteriores $\bigwedge^kE$. Dado un morfismo de fibrados $\phi :E\to F$, tenemos que $\phi^{\wedge k}:\bigwedge^kE\to \bigwedge^kF$ viene dado por
$$\phi^{\wedge k}(e_1\wedge \cdots \wedge e_n)=\phi (e_1)\wedge \cdots \wedge \phi (e_n).$$
Un \emph{par de Higgs} para la variedad $M$ viene dado por un par $(E,\phi )$, donde $E$ es un fibrado vectorial y $\phi :TM\to \opnm{End}(E)$ es un morfismo de fibrados tal que $\phi \wedge \phi =0$; esta \'ultima condici\'on expresa que para cada $x\in M$, los endomorfismos $\phi_x(v):E_x\to E_x$ conmutan.
\medskip

{\bf Corolario.} {\it Dado una abierto semisimple $U\subset M$ y una brana $a\in \scr{B}(U)$, el morfismo de transici\'on $\iota_a$ consiste de $n$ pares de Higgs sobre $U$.}
\medskip

Describimos a continuaci\'on la estructura de 2-fibrado vectorial de Baas-Dundas-Rognes ({\sc bdr}) de la categor\'ia de branas $\scr{B}$. Para cada abierto semisimple $U_\alpha \in \mathfrak{U}$, sean $\xi_i^\alpha$ ($i=1,\dots ,n$) branas soportadas en $i$ tales que $\Gamma_{\xi_i^\alpha \xi_i^\alpha}\cong \scr{O}_U$. Sea $U_\beta$ tal que $U_{\alpha \beta}:=U_\alpha \cap U_\beta \neq \emptyset$ y $\{e_i^\alpha \},\{e_i^\beta \}$ bases de idempotentes ortogonales sobre $U_\alpha$ y $U_\beta$ respectivamente. Sobre $U_{\alpha \beta}$ tenemos una permutaci\'on $u:=u_{\alpha \beta}:\{1,\dots ,n\}\to \{1,\dots ,n\}$ tal que $e_i^\alpha =e_{u(i)}^\beta$ sobre $U_{\alpha \beta}$. Esta identidad implica la existencia de haces invertibles $\scr{L}_i^{\alpha \beta}$ tales que
$$\xi_i^\alpha \cong \scr{L}_{u(i)}^{\alpha \beta}\otimes \xi_{u(i)}^\beta.$$
Pongamos $\xi^\alpha :=(\xi_1^\alpha ,\dots ,\xi_n^\alpha )^t$. Entonces podemos escribir la ecuaci\'on anterior en forma matricial
$$\xi^\alpha \cong A_u^{\alpha \beta}\xi^\beta,$$
donde $A_u^{\alpha \beta}$ es la matriz obtenida de
$$\opnm{diag}\left (\scr{L}_1^{\alpha \beta},\dots ,\scr{L}_n^{\alpha \beta}\right )$$
aplicando la permutaci\'on $u$ a sus columnas. Supongamos ahora que $\gamma$ es tal que $U_{\alpha \beta \gamma}\neq \emptyset$ y supongamos que los idempotentes se permutan por $v$ sobre $U_{\beta \gamma}$ y por $w$ sobre $U_{\alpha \gamma}$. Entonces vale el isomorfismo
$$A_u^{\alpha \beta}A_v^{\beta \gamma}\cong A_w^{\alpha \gamma}.$$
Podemos entonces enunciar el resultado que da respuesta positiva a la sugerencia de G. Segal.
\medskip

{\bf Teorema.} {\it Sea $M$ una variedad con multiplicaci\'on semisimple de dimensi\'on $n$. Entonces, toda fibraci\'on de Cardy maximal $\scr{B}$ sobre $M$ viene equipada con un 2-fibrado vectorial de {\sc bsr} can\'onico de rango $n$.}












}







% Chapter 5
\chapter{D-Branes and Twisted Vector Bundles}
\label{dbtvb}

\vspace{250pt}

In this chapter we will focus on obtaining a relationship between branes and twisted vector bundles. This is accomplished by first constructing a particular class of functor from the category of $\scr{O}_S$-modules to the category of modules over the tangent sheaf of $M$ and then by noting that the $\scr{O}_S$-modules that we deal with are in fact Azumaya algebras. Though the following constructions are a little bit technical, the main results are based on the existence of a global section of the pullback sheaf $\pi^{-1}\scr{T}$ over the spectral cover of $M$.


%%%%%%%%%%%%%%%%%%%%%%%%%%%
\section{Algebras over $M$}\label{algebras_over_m}

Recall that if $U$ is a semisimple subset of $M$, we have a decomposition
$$\mathscr{T}_M|_U\cong e_1\scr{T}_M|_U\oplus \cdots \oplus e_n\scr{T}_M|_U$$
of the tangent sheaf into invertible free subsheaves $e_i\scr{T}_M|_U$, and $\{e_1\dots ,e_n\}$ is the (unique, up to reordering) local frame consisting of orthogonal, simple idempotents. Then, this decomposition applies also to the stalks $\scr{T}_{M,x}$ for each $x\in M$. Now, the spectral cover of $M$ is the (lagrangian) submanifold $S\subset T^*M$ consisting of the points $(x,\varphi )$ such that $\varphi :T_xM\rightarrow \comp$ is an algebra homomorphism. The local frame $\{e_1,\dots ,e_n\}$ also verifies $\sum_ie_i=1$; as $\varphi$ is an algebra homomorphism, then $\varphi (1)=1$ and $\varphi (e_i(x))$ is idempotent in $\comp$. These facts imply that there exists a unique local section $e^{\varphi}$ such that $\varphi (e^{\varphi}(x))=1$ and $\varphi (e_j(x))=0$ if $e_j\neq e^{\varphi}$. We can thus locally identify points in $S$ with the idempotent sections $e_1,\dots ,e_n$ in $\scr{T}_M$ (note that $\varphi$ also can be viewed as a local 1-form).

\begin{notation}
Given a sheaf $\scr{S}$, besides the symbol $\scr{S}(U)$ we will also use the notation $\Gamma (U;\scr{S})$ to denote sections of $\scr{S}$ over $U$. We will also use a tilde $\; \widetilde{} \; $ when referring to open subsets or sections of sheaves over the spectral cover of $M$. If $\scr{S}$ is a sheaf and $\sigma \in \Gamma (U;\scr{S})$ is a local section, then its value at a point $x\in U$ will be denoted by $\sigma_x$ when regarding it as a section of the \'etale space $\sigma :U\longrightarrow \bigsqcup_{x\in U}\scr{S}_x$.
In addition, from now on we will supress the subscript and denote the tangent sheaf $\scr{T}_M$ just by $\scr{T}$. The subscripts are only used when restricting; that is, if $U\subset M$, we use the symbol $\scr{T}_U$ to denote the restriction $\scr{T}|_U$. For disjoints unions $\bigsqcup_iA_i$, an object $(i,x)\in A_i$ will also be denoted just by $x$ when the index is clear from the context.
\end{notation}

Let now $\mathscr{A}$ be an algebra over $M$, i.e. a sheaf of (non necessarily commutative) $\scr{O}_M$-algebras, and assume also that $\scr{A}$ is locally-free as an $\scr{O}_M$-module. Let
$$\iota :\mathscr{T}_M\longrightarrow \mathscr{A}$$
be a central morphism; this map provides $\scr{A}$ with a structure of $\mathscr{T}_M$-algebra.

\begin{lemma}
If $S$ is the spectral cover of $M$ with projection $\pi :S\rightarrow M$, the topological inverse image $\pi^{-1}\scr{T}$ is a sheaf of rings (and of $\pi^{-1}\scr{O}_M$-modules) and $\pi^{-1}\scr{A}$ is a $\pi^{-1}\scr{T}$-algebra by means of the central morphism $\pi^{-1}\iota :\pi^{-1}\mathscr{T}\longrightarrow \pi^{-1}\mathscr{A}$ which is given by
$$\pi^{-1}\iota (\sigma )_{\varphi }=\iota_{\pi (y)}(\sigma (y)).$$
\end{lemma}
\begin{proof}
Recall that, for a sheaf over $\scr{S}$ over $M$, $\pi^{-1}\scr{S}$ is the sheaf given by $\pi^{-1}\scr{S}(\widetilde{U})=\scr{S}(\pi (\widetilde{U}))$. From this definition, the statement of the lemma readily follows.
\end{proof}

In the following we shall consider the ringed space $(S,\scr{O}_S)$ and also $M$ with two different ringed structures: one given by $\scr{O}_M$ and the other by the sheaf of algebras $\scr{T}$. By proposition \ref{isom_1}, we have distinguished maps $u_1:\scr{O}_M\to \pi_*\scr{O}_S$ and $u_2:\scr{T}\to \pi_*\scr{O}_S$, which can be regarded as the inclusion $f\mapsto f1$ and the identity, respectively. This maps define two morphisms of ringed spaces $(\pi ,u_1):(S,\scr{O}_S)\to (M,\scr{O}_M)$ and $(\pi ,u_2):(S,\scr{O}_S)\to (M,\scr{T})$. By the adjunction between $\pi_*$ and $\pi^{-1}$ we have change-of-ring morphisms
\begin{equation}\label{change_of_rings}
\pi^{-1}\scr{O}_M\longrightarrow \scr{O}_S \quad \text{and} \quad \pi^{-1}\scr{T}\longrightarrow \scr{O}_S,
\end{equation}
and the inverse images
$$
\begin{aligned}
\pi^*\scr{T} &= \scr{O}_S\otimes_{\pi^{-1}\scr{O}_M}\pi^{-1}\scr{T} \\
\pi^*\scr{A}   &= \scr{O}_S\otimes_{\pi^{-1}\scr{T}}\pi^{-1}\scr{A} \\
\end{aligned}
$$
are $\scr{O}_S$-algebras. By considering the morphism
$$
\xymatrix{
\pi^*\scr{T} \ar[rr]^{1\otimes \pi^{-1}\iota} && \pi^*\scr{A} },$$
the sheaf $\pi^*\scr{A}$ turns out to be a $\pi^*\scr{T}$-algebra. The actions that provide these algebra structures will be described explicitly after introducing some other tools that we need.

\begin{lemma}
Let $\scr{A}$ be a sheaf of commutative $\scr{R}$-algebras over $S$, where $\scr{R}$ is a sheaf of commutative rings. Then $\pi_*\scr{A}$ is a sheaf of $\pi_*\scr{R}$-algebras. 
\end{lemma}
\begin{proof}
This follows immediately from properties of $\pi$ and the definition of the pushout $\pi_*$: as $\pi :S\to M$ is a covering map, we have that, for a sheaf $\widetilde{\scr{S}}$ over $S$ and $U$ an open subset of $M$,
$$\pi_*\widetilde{\scr{S}}(U)=\widetilde{\scr{S}}(\pi^{-1}(U)).$$
From this definition the lemma follows immediately.
\end{proof}

In what follows, we regard $S$ as being a submanifold of $T^*M$; i.e. points of $S$ are multiplicative linear maps $\varphi :T_xM\to \comp$, where $x=\pi (\varphi )$. We now define a global section $\sigma_0\in \Gamma (S;\pi^{-1}\scr{T})$ in the following way: we let $\sigma_0:S\to \bigsqcup_{\varphi \in S}\scr{T}_{\pi (\varphi )}$ be given by
$$\sigma_0(\varphi ):=(\varphi ,e^{\varphi}_x),$$
where $x=\pi (\varphi )$ and $e^{\varphi}_x$ is the germ at $x$ of the unique idempotent local section $e^{\varphi}:U\to TM$ which verifies $\varphi (e^{\varphi}(x))=1$. Note that $\sigma_0$ induces a section $1\otimes \sigma_0\in \Gamma (S;\pi^*\scr{T})$ and, moreover, $\sigma_0$ as well as $1\otimes \sigma_0$ are idempotent. Likewise, $\sigma_0$ also induces (global) idempotent sections on $\pi^{-1}\scr{A}$ and $\pi^*\scr{A}$ given by $\pi^{-1}\iota (\sigma_0)$ and $1\otimes \pi^{-1}\iota (\sigma_0)$, respectively. To be more explicit, we have
$$
\begin{aligned}
1\otimes \sigma_0 \in \Gamma (S;\pi^*\scr{T}) \quad , \quad 1\otimes \sigma_0 &:S\longrightarrow \bigsqcup_{\varphi \in S}\scr{O}_{S,\varphi}\otimes_{\scr{O}_{M,\pi (\varphi )}}\scr{T}_{\pi (\varphi )}, \\
\pi^{-1}\iota (\sigma_0) \in \Gamma(S;\pi^{-1}\scr{A}) \quad , \quad \pi^{-1}\iota (\sigma_0) &: S\longrightarrow \bigsqcup_{\varphi \in S}\scr{A}_{\pi (\varphi )}, \\
1\otimes \pi^{-1}\iota (\sigma_0) \in \Gamma (S;\pi^*\scr{A}) \quad , \quad 1\otimes \pi^{-1}\iota (\sigma_0) &: S\longrightarrow \bigsqcup_{\varphi \in S}\scr{O}_{S,\varphi}\otimes_{\scr{T}_{\pi (\varphi )}}\scr{A}_{\pi (\varphi )}, \\
\end{aligned}
$$
given by the following expressions:
$$
\begin{aligned}
(1\otimes \sigma_0)_\varphi &= 1\otimes e^{\varphi}_x ,\\
\pi^{-1}\iota (\sigma_0)_\varphi &= \iota_x(e^{\varphi}_x), \\
(1\otimes \pi^{-1}\iota (\sigma_0))_\varphi &= 1\otimes \iota_x(e^{\varphi}_x), \\
\end{aligned}
$$
where $x=\pi (\varphi )$.

\begin{proposition}\label{subsheaf}
Let $\mathscr{A}$ be an algebra over a space $M$ and let $e\in \mathscr{A}(M)$ be a global idempotent section. Then the assignment
$$U\longmapsto e\scr{A}(U)=\{e\sigma \ | \ \sigma\in\scr{A}(U)\}$$
is a sheaf of ideals.\footnote{Note that $e\scr{A}$ is also a ring with identity equal to $e$.}
\end{proposition}
\begin{proof}
Let $\{U_i\}$ be an open cover of an open subset $U\subset M$; for each index $i$, let $\sigma_i\in e\scr{A}(U_i)$ such that $\sigma_i=\sigma_j$ over $U_{ij}$. Then we have:
\begin{enumerate}
\item for each $i$, there exists a section $\tau_i\in \mathscr{A}(U_i)$ such that $\sigma_i=e\tau_i$ and
\item as $\mathscr{A}$ is a sheaf, there exists a unique section $\sigma \in \mathscr{A}(U)$ with $\sigma|_{U_i}=\sigma_i$ for each $i$.
\end{enumerate}
Consider now the section $e\sigma \in e\scr{A}(U)$. Then, over $U_i$ we have
$$(e\sigma )|_{U_i}=e\sigma_i=e(e\tau_i)=e\tau_i=\sigma_i,$$
and thus, by uniqueness, $\sigma =e\sigma \in \mathscr{A}(U)$.
\end{proof}

\begin{notation}
The sheaves $(1\otimes \sigma_0)\pi^*\scr{T}_M$ and $(1\otimes \pi^{-1}\iota (\sigma_0))\pi^*\scr{A}$, will be denoted by $\scr{T}^*_0$ and $\scr{A}^*_0$ respectively. The notation $\epsilon^{\varphi}_x$ will be adopted for the germ $\iota_x(e^{\varphi}_x)$.
\end{notation}

By the previous result, the sheaves $\scr{T}^*_0$ and $\scr{A}^*_0$ are $\scr{O}_S$-algebras and their stalks are given by the expressions
$$
\begin{aligned}
\scr{T}^*_{0,\varphi } &= \scr{O}_{S,\varphi }\otimes_{\scr{O}_{M,x}}e^{\varphi}_x\scr{T}_x, \\
\scr{A}^*_{0,\varphi } &= \scr{O}_{S,\varphi }\otimes_{\scr{T}_x}\epsilon^{\varphi}_x\scr{A}_x, \\
\end{aligned}
$$
where $x=\pi (\varphi )$.

\begin{notation}
From now on, we will supress the coefficient rings in the notation of the tensor product.
\end{notation}

\begin{proposition}\label{isom_2}
There exists a canonical isomorphism of $\scr{O}_S$-algebras
$$\scr{T}^*_0\stackrel{\cong}{\longrightarrow}\scr{O}_S.$$
\end{proposition}
\begin{proof}
The correspondence $\scr{O}_S\to \scr{T}^*_0$ given by
$$f\longmapsto f\otimes \sigma_0.$$
provides the desired isomorphism.
\end{proof}

Combining \ref{isom_1} and \ref{isom_2} we have the following

\begin{cor}
There exists a canonical isomorphism of $\scr{O}_M$-algebras
$$\pi_*\scr{T}^*_0\stackrel{\cong}{\longrightarrow}\scr{T}.$$
\end{cor}

As $\pi :S\to M$ is a covering map, proposition \ref{direct_covering} can be invoqued to describe the stalks of the pushout $\pi_*\scr{A}^*_0$; if $x\in M$, then
$$\left (\pi_*\scr{A}^*_0\right )_x\cong \bigoplus_{\varphi \in \pi^{-1}(x)}\scr{O}_{S,\varphi}\otimes \epsilon^{\varphi}_x\scr{A}_x.$$

Let now $U\subset M$ be an arbitrary open subset and let $\sigma \in \Gamma (U;\scr{A})$ be a section over $U$. Applying the inverse image functor $\pi^{-1}$ we obtain a section $\pi^{-1}\sigma \in \Gamma (\pi^{-1}(U);\pi^{-1}\scr{A})$ given by $(\pi^{-1}\sigma )_{\varphi}=\sigma_{\pi (\varphi )}$; that is, $\pi^{-1}\sigma$ repeats the values of $\sigma$ on the fibre. Finally, we obtain a section $\overline{\sigma}\in \Gamma (U;\pi_*\scr{A}^*_0)=\Gamma (\pi^{-1}(U);\scr{A}^*_0)$ by the formula
\begin{equation}\label{final_isomorphism}
\overline{\sigma}_x=\sum_{\varphi \in \pi^{-1}(x)}1\otimes \epsilon^{\varphi}_x\sigma_x.
\end{equation}

Before studying the assignment $\sigma \mapsto \overline{\sigma}$ in more detail, we will explicitly describe the algebra structures in pushouts and pullbacks that we have encountered. This is fairly easy to do because $\pi$ is a covering map. Recall first that $\iota$ provides the $\scr{T}$-algebra structure on the algebra $\scr{A}$ by means of the action $X\cdot \sigma =\iota (X)\sigma$, and that $\scr{O}_S$ enjoys a structure of $\pi^{-1}\scr{O}_M$ as well as $\pi^{-1}\scr{T}$-module by \eqref{change_of_rings}.

\begin{enumerate}

\item Action on $\pi^{-1}\scr{A}$: this is provided by applying the functor $\pi^{-1}$ to $\iota$, and makes $\pi^{-1}\scr{A}$ a $\pi^{-1}\scr{T}$-module. As a section in $\Gamma (\widetilde{U};\pi^{-1}\scr{T})$ (respectively in $\Gamma (\widetilde{U};\pi^{-1}\scr{A})$) can be regarded as a vector field over the projection $\pi (\widetilde{U})$ (respectively as a section in $\Gamma (\pi (\widetilde{U});\scr{A})$), then this action is the same as the one given by $\iota$.

\item Action on $\pi^*\scr{A}$: This is induced by the morphism $1\otimes \pi^{-1}\iota$. If $f,g:\widetilde{U}\to \comp$ are maps, $\widetilde{X}\in \Gamma (\widetilde{U};\pi^{-1}\scr{T})$ and $\widetilde{\sigma}\in \Gamma (\widetilde{U};\pi^{-1}\scr{A})$, then $(f\otimes \widetilde{X})\cdot (g\otimes \widetilde{\sigma})=fg\otimes \widetilde{X}\cdot \widetilde{\sigma}$ (the first term is just the product map and the action in the second is the one of the previous item). This makes $\pi^*\scr{A}$ a $\pi^*\scr{T}$-algebra.

\item Action on $\scr{A}^*_0$: This action makes $\scr{A}^*_0$ also a $\pi^*\scr{T}$-algebra, and is defined in the same way as the action of the previous item, using also the centrality of the morphism $\iota$. Moreover, this action restricts to an action of $\scr{T}^*_0\cong \scr{O}_S$, which is the same as the one inherited by the one on $\pi^*\scr{A}$.

\item Action on $\pi_*\scr{A}^*_0$: This is obtained by applying the functor $\pi_*$, and provides $\pi_*\scr{A}^*_0$ with a $\pi_*\scr{O}_S\cong \scr{T}$-algebra structure. Explicitly, let $X$ be a vector field over some open subset $U\subset M$, and assume that locally around a point $x\in U$ this vector field can be represented as $\sum_\varphi \lambda_\varphi e^{\varphi}$, and let $\sigma \in \Gamma (U;\pi_*\scr{A}^*_0)=\Gamma (\pi^{-1}(U);\scr{A}^*_0)$. If $x\in U$, then the germ $\sigma_x$ can be represented as $\sum_{\varphi \in \pi^{-1}(x)}f_{\varphi}\otimes \epsilon^{\varphi}_x\sigma_{\varphi ,x}$, where $\sigma_{\varphi}$ are sections of $\scr{A}$ over $U$. Then
$$(X\cdot \sigma)_x=\sum_{\varphi \in \pi^{-1}(x)}f_\varphi\otimes \lambda_{\varphi ,x}\epsilon^{\varphi}_x\sigma_{\varphi ,x}.$$
If $U$ is sufficiently small (so as to have a local basis of idempotents sections over it) and $\widetilde{\lambda} :\pi^{-1}(U)\to \comp$ is the map $\widetilde{\lambda}(\varphi ):=\lambda (\pi (\varphi ))$, then the right hand side of the previous equation can also be represented by $\sum_{\varphi \in \pi^{-1}(x)} f_\varphi \widetilde{\lambda}_\varphi \otimes \epsilon^{\varphi}_x\sigma_{\varphi ,x}$.

\end{enumerate}

\begin{lemma}\label{iota_decomposition}
If $U\subset M$ is a semisimple neighborhood with basis $\{e_1,\dots ,e_n\}$, there exists an isomorphism
$$\mathscr{A}|_U\cong \bigoplus_i\iota (e_i)\mathscr{A}|_U.$$
\end{lemma}
\begin{proof}
Define $\phi :\scr{A}|_U\to \bigoplus_i\iota (e_i)\scr{A}|_U$ by
$$\phi (\sigma )=\sum_i\iota (e_i)\sigma .$$
Recalling that the stalk $\Bigl (\bigoplus_i\iota (e_i)\scr{A}|_U\Bigr )_x$ is given by $\bigoplus_{\varphi}\epsilon^{\varphi}_x\scr{A}_x$, the statement of the lemma follows.
\end{proof}

\begin{obs}
Let us add a comment about (an abuse of) notation. In the next result we adopt the following representation: the local idempotents, say over some open subset $U$, shall be denoted by $e^\varphi$, where $\varphi$ is the local section of the dual bundle $T^*U$ that verifies $\varphi_x(e^\varphi (x))=1$ for each $x\in U$.
\end{obs}

\begin{theorem}\label{isom_3}
The assignment $\sigma \mapsto \overline{\sigma}$ defines an isomorphism of $\scr{T}$-algebras
$$\scr{A}\longrightarrow \pi_*\scr{A}^*_0.$$
\end{theorem}
\begin{proof}
The equalities $\overline{1}=1$ and $\overline{\sigma +\tau}=\overline{\sigma}+\overline{\tau}$ are straightforward to verify. Let us now check that $\overline{\sigma \tau}=\overline{\sigma}\; \overline{\tau}$ holds. We have
$$
\begin{aligned}
(\overline{\sigma \tau})_x &= \sum_{\varphi \in \pi^{-1}(x)}1\otimes \epsilon^{\varphi}_x\sigma_x\tau_x \\
                           &= \sum_{\varphi \in \pi^{-1}(x)}1\otimes \epsilon^{\varphi}_x\sigma_x \epsilon^{\varphi}_x \tau_x \\
                           &= \left (\sum_{\varphi \in \pi^{-1}(x)}1\otimes \epsilon^{\varphi}_x\sigma_x\right )\left (\sum_{\varphi \in \pi^{-1}(x)}1\otimes \epsilon^{\varphi}_x\tau_x\right )=\overline{\sigma}_x\overline{\tau}_x. \\
\end{aligned}
$$
Let $X$ be a vector field on $M$ with local representation $X=\sum_{\varphi \in \pi^{-1}(x)}\lambda_\varphi e^{\varphi}$. We will now check that $\overline{X\cdot \sigma}=X\cdot \overline{\sigma}$, which is almost a tautology. The left hand side is
$$
\begin{aligned}
(\overline{X\cdot \sigma})_x &= \sum_{\varphi \in \pi^{-1}(x)}1\otimes \lambda_{\varphi ,x}\epsilon^{\varphi}_x\sigma_x.\\
                             &= \sum_{\varphi \in \pi^{-1}(x)}\widetilde{\lambda}_\varphi \otimes \epsilon^{\varphi}_x\sigma_x,\\
\end{aligned}
$$
where $\widetilde{\lambda}$ is the map on $\pi^{-1}(U)$ defined by $\widetilde{\lambda}(\varphi )=\lambda (\pi (\varphi ))$. But the right hand side is precisely $(X\cdot \overline{\sigma})_x$.

We will now prove that the assignment $\sigma \mapsto \overline{\sigma}$ is a sheaf isomorphism, so we will check that at the level of stalks, the maps $\scr{A}_x\to \left (\pi_*\scr{A}^*_0\right )_x$ are bijections.

Let $\tau_x\in \left (\pi_*\scr{A}^*_0\right )_x$ be given by $\tau_x=\sum_{\varphi \in \pi^{-1}(x)}f_{\varphi}\otimes \epsilon^{\varphi}_x\sigma_{\varphi ,x}$. Assume also that $f_\varphi$ is the germ of a function, which, abusing, we denote again by $f_\varphi$, defined in a neighborhood $\widetilde{U}_\varphi$ of $\varphi$ such that $\pi |_{\widetilde{U}_\varphi}$ is a homeomorphism. If we define
$$\sigma_x=\sum_{\varphi \in \pi^{-1}(x)}(f_\varphi\pi^{-1})_x\epsilon_{\varphi ,x}\sigma_{\varphi ,x}\in \scr{A}_x,$$
then $\sigma_x\mapsto \tau_x$.

Suppose now that $\overline{\sigma}_x=\sum_{\varphi \in \pi^{-1}(x)}1\otimes \epsilon^{\varphi}_x\sigma_x=0$. As all the modules (stalks) involved are free, this equality implies immediately that $\epsilon^{\varphi}_x\sigma_x=0$ for each $\varphi \in \pi^{-1}(x)$, and thus $\sigma_x=0$. This finishes the proof.
\end{proof}

Recall now that a functor $F:{\bf X}\to {\bf Y}$ is said to be \emph{essentially surjective} if for each object $Y\in {\bf Y}$ there exists an object $X\in {\bf X}$ such that $F(X)$ is isomorphic to $Y$. For a sheaf of rings $\scr{R}$, we let $\tsf{Alg}_{\scr{R}}$ denote the category of $\scr{R}$-algebras. The previous results can then be summarized in the following

\begin{theorem}\label{esurjective}
The functor $\pi_*:\tsf{Alg}_{\scr{O}_S}\to \tsf{Alg}_{\scr{T}}$ is essentially surjective.
\end{theorem}

%\begin{lemma}\label{extension_free}
%Let $R, R'$ be commutative rings and $A$ a free $R$-module. Let $\phi :R'\to R$ be a %change-of-rings morphism. Then, $R'\otimes_RA$ is free over $R'$. Moreover, if $\{x_i\}$ is a basis for $A$, then $\{1\otimes x_i\}$ is a basis for $R'\otimes_RA$.
%\end{lemma}
%\begin{proof}
%First, note that the map $\phi$, besides changing the base ring, also gives $R'$ the structure of an $R$-module, via $a\cdot b:=\phi (a)b$. Moreover, 
%\end{proof}

Let now $\scr{A}_0$ and $\scr{A}_1$ be $\scr{O}_M$-algebras just as $\scr{A}$ in the previous paragraphs and suppose that $\scr{M}$ is an $(\scr{A}_1,\scr{A}_2)$-bimodule; that is, we have linear actions
$$\scr{A}_1\otimes \scr{M}\stackrel{\mu_0}{\longrightarrow}\scr{M}\stackrel{\mu_1}{\longleftarrow}\scr{M}\otimes \scr{A}_2,$$
which can also be represented as morphisms $\scr{A}_1\stackrel{\mu_1}{\longrightarrow}\underline{\operatorname{End}}_{\scr{O}_M}(\scr{M})\stackrel{\mu_2}{\longleftarrow}\scr{A}_2$.
Denote by $\iota_i:\scr{T}\to \scr{A}_i$ ($i=1,2$) the $\scr{T}$-algebra structure for $\scr{A}_i$. We will make two further assumptions:
\begin{enumerate}
\item The algebra structures are compatible in the sense that they verify the centrality condition $\iota_1(X)\sigma =\sigma \iota_2(X)$ for each vector field $X$ and each section $\sigma $ of $\scr{M}$.
\item $\scr{M}$ is locally-free as an $\scr{O}_M$-module.
\end{enumerate}

By means of the maps
$$\iota_1\otimes 1:\scr{T}\otimes \scr{M}\longrightarrow \scr{A}_1\otimes \scr{M}$$
$$1\otimes \iota_2:\scr{M}\otimes \scr{T}\longrightarrow \scr{M}\otimes \scr{A}_2$$
(the tensor product taken over $\scr{O}_M$), the module $\scr{M}$ inherits a structure of $(\scr{T},\scr{T})$-bimodule. But then, the centrality condition implies that both module structures are the same, and thus we can refer to $\scr{M}$ as just a $\scr{T}$-module.

The following result will be useful. The proof of a more general statement can be found in \cite{kn:kscha} (Lemma 18.3.1. and Example 17.2.7.(i)).

\begin{lemma}
Let $\scr{R}$ be a sheaf of commutative rings and $\scr{M},\scr{N}$ two $\scr{R}$-modules over $N$. If $f :M\to N$ is a continuous map, then $f^{-1}(\scr{M}\otimes_{\scr{R}}\scr{N})\cong f^{-1}\scr{M}\otimes_{f^{-1}\scr{R}}f^{-1}\scr{N}$.
\end{lemma}

The previous result implies that the $\scr{T}$-action on $\scr{M}$ lifts to an action of $\pi^{-1}\scr{T}$ on $\pi^{-1}\scr{M}$, and makes it a $\pi^{-1}\scr{T}$-module. The isomorphism $\scr{T}\to \pi_*\scr{O}_S$ together with the adjuntion between $\pi^{-1}$ and $\pi_*$ let us now define the inverse image
$$\pi^*\scr{M}=\scr{O}_S\otimes_{\pi^{-1}\scr{T}}\pi^{-1}\scr{M},$$
which is an $\scr{O}_S$-module. The action of $\pi^{-1}\scr{T}$ on $\pi^{-1}\scr{M}$ induces an action of $\pi^*\scr{T}$ on $\pi^*\scr{M}$ in the following way: consider a section of $\pi^*\scr{T}$ over some open subset $\widetilde{U}\subset S$ of the form $f\otimes \widetilde{X}$, and let $g\otimes \sigma$ be a section of $\pi^*\scr{M}$ over the same open subset. Then define
$$(f\otimes \widetilde{X})\cdot (g\otimes \sigma):=fg\otimes \widetilde{X}\sigma .$$
This action provides $\pi^*\scr{M}$ with a structure of a $\pi^*\scr{T}$-module.

\begin{obs}
The centrality condition also implies that the module structures given by $\pi^{-1}\mu_1\iota_1$ and $\pi^{-1}\mu_2\iota_2$ on $\pi^{-1}\scr{M}$ coincide.\footnote{Note that in this assertion we are considering the maps $\mu_i$ as morphisms from $\scr{A}_i$ to the sheaf of endomorphisms of $\scr{M}$.}
\end{obs}

For simplicity, fix $i=2$ (the same applies to $i=1$ \emph{mutatis mutandis}) and denote by $\mu$ and $\iota$ the maps $\mu_2$ and $\iota_2$ respectively. Consider the section $\delta :=\pi^{-1}\iota (\sigma_0)\cdot 1$. We will first state the following result, which is a generalization of \ref{subsheaf}, and its proof is completely analogous.

\begin{lemma}
Let $\scr{M}$ be a sheaf of $\scr{R}$-modules over $M$, where $\scr{R}$ is a sheaf of commutative rings. Then, if $\sigma_0 \in \Gamma (M;\scr{R})$ is an idempotent section, the correspondence
$$U\longmapsto \sigma_0 \scr{M}(U)=\{\sigma_0\tau \ | \tau \in \scr{M}(U)\}$$
is a submodule of $\scr{M}$. 
\end{lemma}

The product $1\otimes \delta$ defines a section of the inverse image $\pi^*\scr{M}$ over $S$ and thus, by the previous result, we can define the $\pi^*\scr{T}$-submodule
$$\scr{M}_0^*:=(1\otimes \delta )\pi^*\scr{M}.$$
As $\scr{M}^*_0$ is also an $\scr{O}_S$-module, the direct image $\pi_*\scr{M}_0^*$ is a $\pi_*\scr{O}_S\cong \scr{T}$-module, and its stalk is given by
$$(\pi_*\scr{M}_0^*)_x=\bigoplus_{\varphi \in \pi^{-1}(x)}\scr{O}_{S,\varphi }\otimes (\epsilon^{\varphi}_x\cdot 1)\scr{M}_x.$$

\begin{proposition}	\label{isom_4}
There exists an isomorphism of $\scr{T}$-modules
$$\pi_*\scr{M}^*_0\cong \scr{M}.$$
\end{proposition}
\begin{proof}
Given a section $\sigma \in \Gamma (U;\scr{M})$, define $\overline{\sigma}\in \Gamma (U;\pi_*\scr{M}^*_0)$ by
$\overline{\sigma}(x)=\sum_{y\in \pi^{-1}(x)}1\otimes \epsilon^{\varphi}_x\cdot \sigma_x$. The proof now follows the same patterns as the proof of \ref{isom_3}.
\end{proof}

We can now conclude with

\begin{theorem}
The direct image functor
$$\pi_*:\tsf{Mod}_{\scr{O}_S}\longrightarrow \tsf{Mod}_{\scr{T}}$$
from $\scr{O}_S$-modules to $\scr{T}$-modules is essentially surjective.
\end{theorem}

\begin{obs}
In the previous discussions, the sheaves $\scr{A},\scr{A}_1,\scr{A}_2$ plays the role of the sheaves $\Gamma_{aa}$ for $a\in \scr{B}(M)$. In the second part, the bimodule $\scr{M}$ represents $\Gamma_{ab}$ for $a,b\in \scr{B}(M)$. In what follows, we shall only be concerned with the algebras $\Gamma_{aa}$.
\end{obs}



%%%%%%%%%%%%%%%%%%%%%%%%%%%%%%%%%%%%%%%%%%%%%%%%%%%%%%%%%%%%%%%%%%%%%%
\subsection{A Correspondence Between Branes and Twisted Vector Bundles}

%%%%%%%%%%%%%%%%%%%%%%%%%%%%%%%%%%%%%%%%%%%%%%%%%%
%SE NECESITA ESTO?
%We can add some further detail to the correspondence given in theorem \ref{esurjective}. To this end, let us assume %that $\scr{R}$ is a $\scr{T}$-algebra. We know that there exists an $\scr{O}_S$-algebra $\widetilde{\scr{R}}$ which %direct image is isomorphic to $\scr{R}$. Keeping the notation of the previous section, this algebra can be explicitly %presented as $\widetilde{\scr{R}}=\scr{R}^*_0$.

%\begin{theorem}
%Let $\widetilde{\scr{S}}$ be an $\scr{O}_S$-algebra such that $\pi_*\widetilde{\scr{S}}\cong \scr{R}$ as %$\scr{T}$-algebras. Then we have an $\scr{O}_S$-algebra isomorphism $\widetilde{\scr{S}}\cong \scr{R}^*_0$.
%\end{theorem}
%\begin{proof}
%Let us first fix an isomorphism $\eta :\pi_*\widetilde{\scr{S}}\cong \scr{R}$. Define a sheaf homomorphism %$\widetilde{\eta}:\widetilde{\scr{S}}\to \scr{R}^*_0$ in the following way: let $\widetilde{U}\subset S$ be an open %subset and $\sigma:\widetilde{U}\to \bigsqcup_{\varphi \in \widetilde{U}}\widetilde{\scr{S}}_\varphi$ a local section. %Then the section $\widetilde{\eta}(\sigma)$ over $\widetilde{U}$ is defined by
%$$\widetilde{\eta}(\sigma)_\varphi =1\otimes \epsilon^{\varphi}_x\eta_x (\sigma_\varphi) \in %\scr{O}_{S,\varphi}\otimes \epsilon^{\varphi}_x\scr{R}_x,$$
%where on the right hand side we are considering $\sigma_\varphi$ as an element of $\bigoplus_{\varphi \in %\pi^{-1}(x)}\widetilde{\scr{S}}_\varphi$.
%\end{proof}
%%%%%%%%%%%%%%%%%%%%%%%%%%%%%%%%%%%%%%%%%%%%%%%%%%

Consider now a global label $a\in \scr{B}(M)$; we can then apply the machinery of the previous sections to the $\scr{T}$-algebra $\Gamma_{aa}$. Hence, by \ref{esurjective}, there exists an $\scr{O}_S$-algebra $\widetilde{\Gamma}_{aa}$ such that $\pi_* \widetilde{\Gamma}_{aa}\cong \Gamma_{aa}$.

\begin{theorem}\label{azumaya_s}
$\widetilde{\Gamma}_{aa}$ is an Azumaya algebra over $S$.
\end{theorem}
\begin{proof}
Let $x\in M$ and let $U$ be a semisimple neighborhood of $x$, with $\pi^{-1}(U)=\bigsqcup_i\widetilde{U}_i$ If $a\in \scr{B}(M)$ is a global label, then we can apply \ref{theorem2} to the restriction $a|_U$. Let $\{e_1,\dots ,e_n\}$ be a frame of simple, orthogonal idempotent sections over $U$. Suppose now that $e_i$ is the section corresponding to the sheet $\widetilde{U}_i$. By constructions in the previous section, and also theorem \ref{theorem2} and remark \ref{remark_summands}, we can write
$$
\begin{aligned}
\widetilde{\Gamma}_{aa}|_{\widetilde{U}_i} &= \iota_a(e_i)\Gamma_{aa}|_{\pi (\widetilde{U}_i)} \\
																					 &\cong \iota_a(e_i)\Gamma_{aa}|_{U} \\
																					 &\cong \opnm{M}_{d(a,i)}(\scr{O}_{U}). \\
\end{aligned}
$$
\end{proof}

Note that the dimension of the matrix algebras may vary at different sheets: if $\Gamma_{aa}$ is isomorphic over a semisimple $U$ to $\bigoplus_{i}\operatorname{M}_{d_i}(\scr{O}_M)$, then, if $\varphi \in \widetilde{U}$, $\pi (\varphi )=x\in U$ and $\widetilde{U}$ is a sufficiently small neighborhood around $\varphi$, we have that
$$\widetilde{\Gamma}_{aa}|_{\widetilde{U}}\cong \operatorname{M}_{d_i}(\scr{O}_{\widetilde{U}}).$$
If the cover $S$ is connected, then this dimension is constant. In this case, we then have a twisted vector bundle $\mathbb{E}_a$ over $S$ such that
$$\operatorname{END}(\mathbb{E}_a)\cong \widetilde{\Gamma}_{aa}.$$
From now on we shall assume that $S$ is connected.

Take now two boundary conditions $a,b\in \scr{B}(M)$ such that $\Gamma_{aa}\cong \Gamma_{bb}$. On a semisimple open subset $U_i$ we can represent both labels in the form
$$
\begin{aligned}
a|_{U_i}&=\bigoplus_k\scr{M}_k\otimes \xi_k, \\
b|_{U_i}&=\bigoplus_k\scr{N}_k\otimes \xi_k, \\
\end{aligned}
$$
where $\scr{M}_k,\scr{N}_k$ are locally free modules and $\xi_k$ are the objects of proposition \ref{invertibles}. Then, $\Gamma_{aa}|_{U_i}\cong \bigoplus_k\underline{\opnm{End}}_{\scr{O}_{U_i}}(\scr{M}_k)$ and $\Gamma_{bb}|_{U_i}\cong \bigoplus_k\underline{\opnm{End}}_{\scr{O}_{U_i}}(\scr{N}_k)$. By theorem \ref{azumaya_s} and the connectivity of $S$ we can write
\begin{equation}\label{global_labels_local_rep}
\begin{aligned}
\Gamma_{aa}|_{U_i}&\cong \underline{\opnm{End}}^{\oplus n}_{\scr{O}_{U_i}}(\scr{M}^{(i)}), \\
\Gamma_{bb}|_{U_i}&\cong \underline{\opnm{End}}^{\oplus n}_{\scr{O}_{U_i}}(\scr{N}^{(i)}). \\
\end{aligned}
\end{equation}
for some locally free modules $\scr{M}^{(i)}$ and $\scr{N}^{(i)}$ over $U_i$. As $\Gamma_{aa}$ and $\Gamma_{bb}$ are isomorphic, we can assure the existence of invertible sheaves $\scr{L}_i$ such that $\scr{N}^{(i)}\cong \scr{L}_i\otimes \scr{M}^{(i)}$. By shrinking the open subset if necessary, we can regard these invertible sheaves as free.

From equations \eqref{global_labels_local_rep} let us denote by $\widehat{\scr{M}}$ and $\widehat{\scr{N}}$ the locally free sheaves with local representation $\underline{\opnm{End}}_{\scr{O}_{U_i}}(\scr{M}^{(i)})$ and $\underline{\opnm{End}}_{\scr{O}_{U_i}}(\scr{N}^{(i)})$ respectively. Then
\begin{itemize}
\item $\widehat{\scr{M}}$ and $\widehat{\scr{N}}$ are Azumaya algebras. Hence, there exist twisted bundles $\mathbb{E}$ and $\mathbb{F}$ such that $\widehat{\scr{M}}\cong \Gamma_{\operatorname{END}(\mathbb{E})}$ and $\widehat{\scr{N}}\cong \Gamma_{\operatorname{END}(\mathbb{F})}$.
\item As $\Gamma_{aa}$ and $\Gamma_{bb}$ are isomorphic, $\widehat{\scr{M}}$ and $\widehat{\scr{N}}$ are also isomorphic. In particular, $\operatorname{END}(\mathbb{E})$ and $\operatorname{END}(\mathbb{F})$ are isomorphic.
\end{itemize}

\begin{proposition}\label{tensor_l}
Let $\mathbb{E}$ and $\mathbb{F}$ be two twisted bundles over a space $M$. Then the algebra bundles $\operatorname{END}(\mathbb{E})$ and $\operatorname{END}(\mathbb{F})$ are isomorphic if and only if there exists a twisted line bundle $\mathbb{L}$ such that $\mathbb{F}\cong \mathbb{E}\otimes \mathbb{L}$.
\end{proposition}
\begin{proof}
We make use of \ref{isomorphic}. Let $\mathbb{E}, \mathbb{F}$ be given by
$$
\begin{aligned}
\mathbb{E} &= (\mathfrak{U},U_i\times \comp^n,g_{ij},\lambda_{ijk}), \\
\mathbb{F} &= (\mathfrak{U},U_i\times \comp^n,f_{ij},\mu_{ijk}). \\
\end{aligned}
$$
For the ``if'' part, let $\mathbb{L}$ be given by $(\mathfrak{U},U_i\times \comp,\xi_{ij},\eta_{ijk})$, where $\xi_{ij}:U_{ij}\to \comp^\times$. Assume that $u_{ij}:U_{ij}\to \operatorname{GL}(\operatorname{M}_n(\comp ))$ are the cocycles for $\operatorname{END}(\mathbb{E}\otimes \mathbb{L})$; then,
$$
\begin{aligned}
u_{ij}(x)(A) &= \xi_{ij}(x)g_{ij}(x)Ag_{ij}(x)^{-1}\xi_{ij}(x)^{-1} \\
             &= g_{ij}(x)Ag_{ij}(x)^{-1}, \\
\end{aligned}
$$
which are precisely the cocycles for $\operatorname{END}(\mathbb{E})$.

For the ``only if'' part, assume that $\operatorname{END}(\mathbb{E})\cong \operatorname{END}(\mathbb{F})$ and let $\{\alpha_i:U_i\to \operatorname{GL}(\operatorname{M}_n(\comp ))\}$ be a family of maps as in \ref{isomorphic}. Then, for each $n\times n$ matrix $A$ we have
$$f_{ij}(x)Af_{ij}(x)^{-1}=(\alpha_i(x)g_{ij}(x)\alpha_j(x)^{-1})A(\alpha_i(x)g_{ij}(x)\alpha_j(x)^{-1})^{-1}$$
over $U_{ij}$. This equality implies that there exists a map $\xi_{ij}:U_{ij}\to \comp^\times$ such that
\begin{equation}\label{e_times_l}
f_{ij}(x)^{-1}\alpha_i(x)g_{ij}(x)\alpha_j(x)^{-1}=\xi_{ij}(x)1
\end{equation}
or, equivalently,
$$f_{ij}(x)=\alpha_i(x)\xi_{ij}(x)^{-1}g_{ij}(x)\alpha_j(x)^{-1},$$
where $\alpha_i(x)$ is regarded here as an invertible matrix (by the Skolem-Noether theorem).

We now only need to show that $\{\xi_{ij}\}$ is a (twisted) cocycle. Multiplying equation \eqref{e_times_l} by the one corresponding to $\xi_{jk}$ and using the twistings for $\mathbb{E}$ and $\mathbb{F}$ (we omit any reference to $x\in U_{ijk}$ for simplicity) we obtain
$$\alpha_i \lambda_{ijk}g_{ik}\alpha_k^{-1}=\xi_{ij}\xi_{jk}\mu_{ijk}f_{ik};$$
rearranging the last equation we must have
$$\xi_{ij}\xi_{jk}=\lambda_{ijk}\mu_{ijk}^{-1}\xi_{ik},$$
as desired.
\end{proof}

Let now $\operatorname{B}(M)/\sim$ be the set of labels over $M$ subject to the identification
$$a\sim b \Longleftrightarrow \Gamma_{aa}\cong \Gamma_{bb}$$
and let $\operatorname{TVB}(S)$ be the set of twisted vector bundles over $S$. We can then define a map
$$\Phi :\operatorname{B}(M)/\sim \longrightarrow \operatorname{TVB}(S)/_{\mathbb{E}\sim \mathbb{L}\otimes \mathbb{E}}$$
by
$\Phi (a)=\mathbb{E}_a$, where $\mathbb{L}$ is a twisted line bundle. The results obtained in the previous paragraphs let us conclude with the following characterization of branes in terms of twisted bundles.

\begin{theorem}
The map $\Phi$ is injective.
\end{theorem}

In other words, we can regard each label (up to equivalence) over $M$ as a twisted bundle (again, up to equivalence) over the spectral cover.

Now, by theorem \ref{bij_tensor}, we have a bijection
$$\Psi :\opnm{TVB}(S)/_{\mathbb{E}\sim \mathbb{L}\otimes \mathbb{E}}\stackrel{\cong}{\longrightarrow}\opnm{Vect}(S)/_{E\sim L\otimes E},$$
and then every brane $a\in \opnm{B}(M)$ can in fact be taken as a vector bundle over $S$, up to tensoring with a line bundle.




\clearpage

{\small
%%%%%%%%%%%%%%%%%%%%%%%%%%%%%%%%%%%%%%%%%%%%%%%%%%%%%%%%
%%%%%%%%%%%%%%%%%%%%%%%%%%%%%%%%%%%%%%%%%%%%%%%%%%%%%%%%
\section{Resumen del Cap\'itulo \ref{dbtvb}}

En este cap\'itulo se describe la relaci\'on existente entre las fibraciones de Cardy (mas particularmente entre las branas globales $a\in \scr{B}(M)$) y los fibrados torcidos. Para eso, en primer lugar se demuestra que el funtor pushout de la categor\'ia de $\scr{O}_S$-m\'odulos en la catego\'ia de $\scr{T}_M$-m\'odulos es esencialmente sobreyectivo, donde $S$ es el recubrimiento espectral de $M$.\footnote{Un funtor $F:{\bf X}\to {\bf Y}$ se dice \emph{esencialmente sobreyectivo} si para cada $Y\in {\bf Y}$ existe un objeto $X\in {\bf X}$  tal que $F(X)$ es isomorfo a $Y$.} Esto permite deducir una relaci\'on entre los m\'odulos $\Gamma_{aa}$ y las \'algebras de Azumaya, lo que naturalmente conduce a los fibrados torcidos.


%%%%%%%%%%%%%%%%%%%%%%%%%%%%%%%%%
\subsection{\'Algebras Sobre $M$}

Trabajamos en general, para luego particularizar a los morfismos y \'algebras que nos interesan. Para eso, sea $\scr{A}$ un \'algebra sobre $M$, es decir un haz de $\scr{O}_M$-\'algebras no necesariamente conmutativas. Supongamos adem\'as que $\scr{A}$ es localmente libre como $\scr{O}_M$-m\'odulo y que $\iota :\scr{T}_M\to \scr{A}$ es un morfismo central (que en particular le da a $\scr{A}$ una estructura de $\scr{T}_M$-\'algebra).

En lo que sigue consideramos al espacio anillado $(S,\scr{O}_S)$ y tambi\'en a $M$ con dos estructuras: una dada por $\scr{O}_M$ y otra dada por el haz tangente. Si $\pi :S\to M$ es la proyecci\'on, recordemos que el funtor $\pi^*$ (\emph{pullback}) manda $\scr{O}_M$-m\'odulos en $\scr{O}_S$-m\'odulos (considerando $(M,\scr{O}_M)$) y $\scr{T}_M$-m\'odulos en $\scr{O}_S$-m\'odulos (para el caso de $(M,\scr{T}_M)$). En particular:
$$
\begin{aligned}
\pi^*\scr{T}_M &= \scr{O}_S\otimes_{\pi^{-1}\scr{O}_M}\pi^{-1}\scr{T}_M \\
\pi^*\scr{A}   &= \scr{O}_S\otimes_{\pi^{-1}\scr{T}_M}\pi^{-1}\scr{A}, \\
\end{aligned}
$$
y adem\'as resultan ser $\scr{O}_S$-\'algebras. Mas a\'un, considerando el morfismo
$$1\otimes \pi^{-1}\iota :\pi^*\scr{T}_M\longrightarrow \pi^*\scr{A},$$
el haz $\pi^*\scr{A}$ resulta ser una $\pi^*\scr{T}_M$-algebra.

Consideramos a $S$ como una subvariedad del fibrado cotangente $T^*M$, de la siguiente manera: los puntos sobre $x\in M$ son aplicaciones lineales $\varphi:T_xM\to \comp$ para las cuales existe un \'unico \'indice $i$ tal que $\varphi (e_k)=\delta_{ik}$. Llamaremos $e^\varphi(x)$ al idempotente en $T_xM$ para el cual $\varphi (e^\varphi (x))=1$. Definimos una secci\'on global
$$\sigma_0\in \pi^{-1}\scr{T}_M(S)$$
por $\sigma_0(\varphi ):=(\varphi ,e^\varphi_x )$, donde $e^\varphi_x$ es el g\'ermen de la secci\'on $e^\varphi$ en $x$. A partir de esta secci\'on se obtienen otras, que definimos a continuaci\'on (por simplicidad, notamos $\scr{T}$ al haz tangente, sin hacer referencia a la variedad $M$):
$$
\begin{aligned}
1\otimes \sigma_0 \in \Gamma (S;\pi^*\scr{T}) \quad , \quad 1\otimes \sigma_0 &:S\longrightarrow \bigsqcup_{\varphi \in S}\scr{O}_{S,\varphi}\otimes_{\scr{O}_{M,\pi (\varphi )}}\scr{T}_{\pi (\varphi )}, \\
\pi^{-1}\iota (\sigma_0) \in \Gamma(S;\pi^{-1}\scr{A}) \quad , \quad \pi^{-1}\iota (\sigma_0) &: S\longrightarrow \bigsqcup_{\varphi \in S}\scr{A}_{\pi (\varphi )}, \\
1\otimes \pi^{-1}\iota (\sigma_0) \in \Gamma (S;\pi^*\scr{A}) \quad , \quad 1\otimes \pi^{-1}\iota (\sigma_0) &: S\longrightarrow \bigsqcup_{\varphi \in S}\scr{O}_{S,\varphi}\otimes_{\scr{T}_{\pi (\varphi )}}\scr{A}_{\pi (\varphi )}, \\
\end{aligned}
$$
los cuales est\'an dados por las siguientes expresiones:
$$
\begin{aligned}
(1\otimes \sigma_0)_\varphi &= 1\otimes e^{\varphi}_x ,\\
\pi^{-1}\iota (\sigma_0)_\varphi &= \iota_x(e^{\varphi}_x), \\
(1\otimes \pi^{-1}\iota (\sigma_0))_\varphi &= 1\otimes \iota_x(e^{\varphi}_x), \\
\end{aligned}
$$
donde $x=\pi (\varphi )$.

Los haces $(1\otimes \sigma_0)\pi^*\scr{T}$ y $(1\otimes \pi^{-1}\iota(\sigma_0))\pi^*\scr{A}$ ser\'an notados $\scr{T}_0^*$ y $\scr{A}_0^*$ respectivamente. Para el g\'ermen $\iota_x(e_x^\varphi )$ usaremos la notaci\'on $\epsilon^\varphi_x$. A partir de ahora tambi\'en suprimimos los anillos de coeficientes de las notaciones que involucren productos tensoriales.

A continuaci\'on, damos una serie de isomorfismos importantes:

\begin{enumerate}
\item $\scr{T}_0^*\cong \scr{O}_S$ como $\scr{O}_S$-algebras.
\item $\pi_*\scr{T}_0^*\cong \scr{T}$ como $\scr{O}_M$-algebras
\item $\scr{A}\cong \pi_*\scr{A}_0^*$ como $\scr{T}$-algebras.
\end{enumerate}

A partir del \'ultimo isomorfismo se deduce el siguiente
\medskip

{\bf Teorema.} {\it El funtor $\pi_*:\tsf{Alg}_{\scr{O}_S}\to \tsf{Alg}_{\scr{T}}$ es esencialmente sobreyectivo.}
\medskip

Un desarrollo an\'alogo lleva tambi\'en al siguiente resultado.
\medskip

{\bf Teorema.} {\it El funtor $\pi_*:\tsf{Mod}_{\scr{O}_S}\to \tsf{Mod}_{\scr{T}}$ es esencialmente sobreyectivo.}
\medskip

Es importante observar que el algebra $\scr{A}$ juega el papel de $\Gamma_{aa}$. El segundo resultado considera el caso de los bim\'odulos $\Gamma_{ab}$.


%%%%%%%%%%%%%%%%%%%%%%%%%%%%%%%%%%%%%%%%%%%%%%%%%%%%%%%%%%%%%%%%%%%%%%%%
\subsection{La Correspondencia Entre las Branas y los Fibrados Torcidos}

Consideremos ahora un objeto global $a\in \scr{B}(M)$. Podemos entonces aplicar lo visto anteriormente a la $\scr{T}$-\'algebra $\Gamma_{aa}$ y deducir que existe una $\scr{O}_S$-\'algebra $\widetilde{\Gamma}_{aa}$ tal que $\pi_*\widetilde{\Gamma}_{aa}\cong \Gamma_{aa}$.
\medskip

{\bf Teorema.} {\it $\widetilde{\Gamma}_{aa}$ es un \'algebra de Azumaya sobre $S$.}
\medskip

La idea detr\'as de este resultado es simple: dado que el haz $\Gamma_{aa}$ es una suma de \'algebras de matrices, $\widetilde{\Gamma}_{aa}$ resulta un \'algebra de Azumaya ya que los sumandos se ``distribuyen'' en las hojas del recubrimiento $S$. Si adem\'as consideramos que $S$ es conexo, como vamos a suponer a partir de ahora, las dimensiones de los sumandos deben coincidir. Luego, sabemos que entonces debe existir un fibrado torcido $\mathbb{E}_a$ tal que
$$\opnm{END}(\mathbb{E}_a)\cong \widetilde{\Gamma}_{aa}.$$
En un caso como el anterior, diremos que $\mathbb{E}_a$ representa a la brana $a$.

Supongamos ahora que $a,b$ son branas tales que $\Gamma_{aa}\cong \Gamma_{bb}$. Entonces, por la conectividad de $S$ y los teoremas anteriores tenemos que
$$
\begin{aligned}
\Gamma_{aa}|_{U_i}&\cong \underline{\opnm{End}}^{\oplus n}_{\scr{O}_{U_i}}(\scr{M}^{(i)}), \\
\Gamma_{bb}|_{U_i}&\cong \underline{\opnm{End}}^{\oplus n}_{\scr{O}_{U_i}}(\scr{N}^{(i)}). \\
\end{aligned}
$$
para ciertos m\'odulos localmente libres $\scr{M}^{(i)},\scr{N}^{(i)}$. En particular, de esto se deduce que, si $\mathbb{E}$ y $\mathbb{F}$ son los fibrados torcidos que representan a las branas $a$ y $b$, entonces $\opnm{END}(\mathbb{E})$ y $\opnm{END}(\mathbb{F})$ son isomorfos. Mas a\'un, suponiendo que $\mathbb{E}$ y $\mathbb{F}$ son dos fibrados torcidos sobre $M$, entonces los fibrados de \'algebras $\opnm{END}(\mathbb{E})$ y $\opnm{END}(\mathbb{F})$ son isomorfossi y solo si existe un fibrado de l\'inea torcido $\mathbb{L}$ tal que $\mathbb{F}\cong \mathbb{E}\otimes \mathbb{L}$.

Si ahora $\opnm{B}(M)/\sim$ es el conjunto de branas sobre $M$ sujetas a la identificaci\'on $a\sim b \longleftrightarrow \Gamma_{aa}\cong \Gamma_{bb}$ y $\opnm{TVB}(S)$ es el conjunto de fibrados torcidos sobre $S$, tenemos que el mapa
$$\Phi :\operatorname{B}(M)/\sim \longrightarrow \operatorname{TVB}(S)/_{\mathbb{E}\sim \mathbb{L}\otimes \mathbb{E}}$$
dado por $\Phi (a)=\mathbb{E}_a$ es una aplicaci\'on inyectiva. Es decir, toda brana (sujeta a la identificaci\'on de ser iguales si sus m\'odulos de morfismos son isomorfos) se puede interpretar como un fibrados torcido, salvo multiplicaci\'on por un fibrado de l\'inea, tambien torcido.








}

% Chapter 6
%\chapter{The Symmetric Group and F-Manifolds}\label{coxeter}
%\input{coxeter.tex}

%\chapter{Appendix.}
%\input{appendix.tex}

%\printindex{}

%%%%%%%%%%%%%%%%%%%%%%%%%%%%%%%%%%%%%%%%%%

\bibliography{mybib}
\bibliographystyle{acm}

%%%%%%%%%%%%%%%%%%%%%%%%%%%%%%%%%%%%%%%%%%

%\end{large}

\end{document}





%%% Local Variables:
%%% mode: latex
%%% TeX-master: "master"
%%% End:
